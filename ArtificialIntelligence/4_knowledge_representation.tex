% !TEX root = main.tex

\section{知识表示与推理}
一阶逻辑(First-Order Logic,FOL)
\begin{itemize}
	\item 个体/常量(0-ary)
	\item 类型(unary)谓词:$A(x),B(x)$
	\item 关系(二元谓词):$L(x,y)$
\end{itemize}

\begin{definition}[项(term)]
每一个变量都是一个项。
若$t_1,\ldots,t_n$都为项,且$f$为$n$参数的的函数,则$f(t_1,\ldots,t_n)$是一个项。
\end{definition}
\begin{definition}[公式(formular)]
公式包括以下几种情况:
\begin{itemize}
\item 若$t_1,\ldots,t_n$都是项,且$P$是$n$元的谓词符号,则$P(t_1,\ldots,t_n)$是一个公式
\item 若$t_1,t_2$都是项,那么$(t_1=t_2)$是一个原子公式
\item 若$\alpha,\beta$都是公式,$v$是一个变量,则$\lnot\alpha,(\alpha\land\beta),(\alpha\lnot\beta),\exists v.\alpha,\forall v.\alpha$都是公式
\end{itemize}
\end{definition}
\begin{definition}[句子(sentence)]
没有自由变量的公式
\end{definition}
\begin{definition}[替换]
$\alpha[v/t]$表示$\alpha$中所有自由出现的$v$都用项$t$替代
\end{definition}
\begin{definition}[解释(interpretation)]
一个解释是一个对(pair)$\mI=\lrang{D,I}$,其中
\begin{itemize}
	\item $D$是论域,可以是任何非空集
	\item $I$是从谓词到函数符号的映射
	\item 如果$P$是一个$n$-参数的谓词符号,$I(P)$是一个在$D$上的$n$-参数的关系,即$I(P)\subset D^n$
\end{itemize}
\end{definition}
\begin{definition}[赋值(denotation)]
变量指派(assignment)$\mu$是一个从变量集合到论域$D$的映射
\[\begin{aligned}
\norm{v}_{\mI,\mu}&=\mu(v)\\
\norm{f(t_1,\ldots,t_n)}_{\mI,\mu}&=I(f)(\norm{t_1}_{\mI,\mu},\ldots,\norm{t_n}_{\mI,\mu})
\end{aligned}\]
\end{definition}
\begin{definition}[满足]
$\mI,\mu\models\alpha$读作$\mI,\mu$满足$\alpha$
\begin{itemize}
	\item $\mI,\mu\models\alpha\iff\lrang{\norm{t_1}_{\mI,\mu},\ldots,\norm{t_n}_{\mI,\mu}}\in I(P)$
	\item $\mI,\mu\models(t_1=t_2)\iff\norm{t_1}_{\mI,\mu}=\norm{t_2}_{\mI,\mu}$
\end{itemize}
\end{definition}
\begin{definition}[子句(clause)]
文字(literal)是原子公式或它的取反,一个子句是文字的析取(disjunction),如$p\lor\lnot r\lor s$,写作$(p,\lnot r,s)$。
特殊地,空子句$()$代表为假。
公式(formula)则是子句的合取(conjunction)。
\end{definition}

归结(resolution)
反驳(refutation)
\[\vdash\]
% nvDash

\begin{itemize}
	\item 消除蕴含:$A\to B\iff \lnot A\lor B$
	\item 将非向内推:德摩根定律
	\item 标准化变量:重命名变量使得每一个量词都是唯一的
	\item 消除存在量词(skloemize):引入新的函数符号,如$\forall x P(x)$改为$P(g(y))$
	\item 将所有量词带到最前面:只有全局量词,且名字均不同
	\item 析取分配到合取
	\item 压平
	\item 转化为子句:将量词全部移除
\end{itemize}

\begin{definition}[MGU]
两个公式$f$和$g$的替换$\sigma$
\begin{itemize}
	\item
	\item
\end{itemize}
\end{definition}

计算MGU的算法:不断代入新的元,使其一致

利用归结(两条文字合一变真删除)看是否能得到空子句

答案抽取(answer extraction)
\begin{itemize}
\item 将询问$\exists xP(x)$用$\exists x[P(x)\land\lnot\text{answer}(x)]$替换
(因为取非后变成$\forall x P(x)\implies \text{answer}(x)$)
\item 直到获得任意子句只包含答案的谓词
\end{itemize}
\begin{example}
对下列查询进行归结及答案查询
\begin{itemize}
	\item Whoever can read $R(x)$ is literate $L(x)$
	\item Dolphins $D(x)$ are not literate
	\item Flipper is an intelligent dolphin $I(x)$
\end{itemize}
Who is intelligent but cannot read?
\end{example}
\begin{analysis}
对语句进行形式化
\begin{center}
\begin{tabular}{lll}
$\forall x (R(x)\to L(x))$ & 1 & $(\lnot R(u),L(u))$\\
$\forall x (D(x)\to \lnot L(x))$ & 2 & $(\lnot D(v),\lnot L(v))$\\
$D(Flip)\land I(Flip)$ & 3 & $D(Flip)$\\
 & 4 & $I(Flip)$\\
Q:$\exists x (I(x)\land\lnot R(x))$ & 5 & $(\lnot I(y),R(y),answer(y))$\\\hline 
$R[4,5]/y=Flip$ & 6 & $(R(Flip),answer(Flip))$\\
$R[1,6]/u=Flip$ & 7 & $(L(Flip),answer(Flip))$\\
$R[2,7]/v=Flip$ & 8 & $(\lnot D(Flip),answer(Flip))$\\
$R[3,8]$ & 9 & $(answer(Flip))$
\end{tabular}
\end{center}
因此得到Flipper是聪明的但是不能阅读
\end{analysis}

一组子句是否可满足是NP完全的[Cook,1972]
