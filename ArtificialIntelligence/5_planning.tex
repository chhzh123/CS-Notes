% !TEX root = main.tex

\section{规划}
智能体应该能够对世界做出动作(action),而不仅仅是通过搜索解决问题或推理及知识表示。
核心是对动作的效果进行推理,并且计算什么动作能够达成特定的效果。

情景演算(Situation Calculation, SitCalc)三个基本部分
\begin{itemize}
	\item 动作(action)
	\item 情景(situations):动作序列,$do(a,s)$为动作、情景到新情景的函数映射,$S_0$为初始情景
	\[do(put(a,b),do(put(b,c),S_0))\]
	要区别情景与状态(state),如将硬币转两次,情景/动作历史不同,但状态都是一样的
	\item 流(fluent):从情景到情景的谓词或函数(动态变化过程)
	\item 条件(precondition):动作执行的前提条件
	\item 影响(effect):执行动作后改变的流。 % the fluents that change as the result of performing the action
	如下在情景$s$下执行修复动作后,$x$就不是破碎的
	\[\lnot Broken(x,do(repair(r,x)),s)\]
\end{itemize}

只陈述了执行动作的影响,而没有阐述没影响的部分

框架(frame)问题:找到一种高效的方法来确定动作的非效果(non-effects)而不是显式地将它们全部写下来,用一阶逻辑

利用归结进行情景演算

传统的规划没有完全或确定的信息,假设对于初始状态有完整的信息。
\begin{definition}[封闭世界假设(Closed-World Assumption, CWA)]
用于表示世界状态的知识库是一系列正的真实原子事实(与数据库类似)。
如$emp(A,C)$不在数据库中,则$\lnot emp(A,C)$为真
\end{definition}

STRIPS(Stanford Research Institute Problem Solver):
\begin{itemize}
	\item 世界被表示成封建世界知识库(CW-KB),一个STRIPS的动作表示成更新CW-KB的方式
	\item 一个动作生成新的KB,用以描述新的世界
\end{itemize}

在SitCalc中我们可能有不完全的信息(用一阶逻辑公式表示),而在STRIPS中,我们有完整的信息(用CW-KB)表示

例子如下:
$pickup(X)$:
\begin{itemize}
\item $Pre: {handempty, clear(X), ontable(X)}$
\item $Adds: {holding(X)}$
\item $Dels: {handempty, clear(X), ontable(X)}$
\end{itemize}