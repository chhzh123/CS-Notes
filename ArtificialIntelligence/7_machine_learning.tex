% !TEX root = main.tex

\section{机器学习}
\subsection{决策树}
假定样本集合$D$中第$k$类样本所占比例为$p_k(k=1,2,\ldots,|\mathcal{Y}|)$,则$D$的信息熵定义为
\[\Ent(D)=-\sum_{k=1}^{|\mathcal{Y}|}p_k\log_2 p_k\]

又假定离散属性$a$有$V$个可能取值$\{a^1,a^2,\ldots,a^V\}$,第$v$个分支结点包含了$D$中所在属性$a$上取值为$a^v$的样本,记为$D^v$,进而可定义信息增益
\[\mathrm{Gain}(D,a)=\Ent(D)-\sum_{v=1}^V\frac{|D^v|}{|D|}\Ent(D^v)\]
每次选择最大增益的属性进行划分,即
\[a_*=\argmax_{a\in A}\mathrm{Gain}(D,a)\]

以信息增益为准则来选择划分属性的算法即为ID3决策树。

\subsection{贝叶斯学习}
先验$\pr{H}$、似然$\pr{d\mid H}$、证据$d=\lrang{d_1,d_2,\ldots,d_n}$,有贝叶斯公式(先验推后验)
\[\pr{H\mid d}=\alpha\pr{d\mid H}\pr{H}\]

假设独立同分布$\pr{d\mid h}=\prod_j\pr{d_j\mid h}$
\begin{itemize}
	\item 贝叶斯学习:$\pr{X\mid d}=\sum_i\pr{X\mid d,h_i}\pr{h_i\mid d}=\sum_i\pr{X\mid h_i}\pr{h_i\mid d}$
	\item 极大后验(MAP)学习:$\pr{X\mid d}\approx\pr{X\mid h_{MAP}}$
	\[h_{MAP}=\argmax_{h_i}\pr{h_i\mid d}=\argmax_{h_i}\pr{h_i}\pr{d\mid h_i}\]
	\item 最大似然(ML)学习:$\pr{X\mid d}\approx\pr{X\mid h_{ML}}$
	\[h_{ML}=\argmax_{h_i}\pr{d\mid h_i}\]
\end{itemize}

基于属性条件独立性假设
\[\pr{c\mid\vx}=\frac{\pr{c}\pr{\vx\mid c}}{\pr{x}}=\frac{\pr{c}}{\pr{\vx}}\prod_{i=1}^d\pr{x_i\mid c}\]
其中$d$为属性数目,$x_i$为$\vx$在第$i$个属性上的取值。
因对所有类别来说$\pr{\vx}$相同,因此贝叶斯判定准则为
\[h_{NB}(\vx)=\argmax_{c\in\mathcal{Y}}\pr{c}\prod_{i=1}^d\pr{x_i\mid c}\]
令$D_c$表示训练集$D$中第$c$类样本组成的集合,有
\[\pr{c}=\frac{D_c}{D}\qquad\pr{x_i\mid c}=\frac{|D_{c,x_i}|}{|D_c|}\]

\subsection{聚类算法}
\begin{itemize}
	\item 硬聚类:每个样本都决定放在哪一个类别中
	\item 软聚类:每个样本都被指派每个类别的概率分布
\end{itemize}

\subsubsection{EM算法}
最大期望(Expectation-Maximization, EM)算法:用作软聚类
\begin{itemize}
	\item E步:计算期望,利用对隐藏变量的现有估计值(实际上就是扩充了当前数据,初始化先随机赋值,然后进入迭代),计算其最大似然估计值
	\item M步:最大化在E步上求得的最大似然值来计算参数的值。
\end{itemize}
M步上找到的参数估计值被用于下一个E步计算中,这个过程不断交替进行。

\subsubsection{K-means算法}
\begin{itemize}
	\item E步:对于每一个类别$i$和特征$X_j$,有
	\[pval(i,X_j)\gets\frac{\sum_{e:class(e)=i}val(e,X_j)}{|\{e:class(e)=i\}|}\]
	\item M步:对于每一个样本$e$,指派$e$给类别$i$使得
	\[\min_i\sum_{j=1}^n(pval(i,X_j)-val(e,X_j))^2\]
\end{itemize}

\subsection{神经网络}
\begin{theorem}[一致近似理论(Universal Approximator Theorem)]
具有至少一个隐层的深度神经网络可以无限逼近任意连续函数
\end{theorem}

前向后向传播过程:
\begin{itemize}
	\item 前向过程:
	\[in_j=\sum_i w_{ij}a_i\qquad a_j=g(in_j)\]
	\item 后向过程:
	\[\begin{aligned}
	\text{output:} & \Delta_j=g'(in_j)(y_j-a_j)\\
	\text{hidden:} & \Delta_i=g'(in_i)\sum_j w_{ij}\Delta_j
	\end{aligned}\]
\end{itemize}

注意以下两条求导公式
\[\begin{array}{rlrl}
\sigma(x)&=\dfrac{1}{1+\ee^{-x}} & \pd{\sigma(x)}{x}&=(1-\sigma(x))\sigma(x)\\
\tanh(x)&=\dfrac{\ee^x-\ee^{-x}}{\ee^x+\ee^{-x}} & \pd{\tanh(x)}{x}&=1-\tanh^2(x)
\end{array}\]

\subsection{强化学习}
目标:
\[\max_{\pi_\theta}\lrs{\sum_{t=1}^\infty\gamma^tr_t}\]

给定策略$\pi$:
\[\begin{aligned}
Q^\pi(s,a) &= \sum_{s'}\pr{s'\mid a,s}(R(s,a,s')+\gamma V^\pi(s'))\\
V^\pi(s) &= Q(s,\pi(s))
\end{aligned}\]

Q学习:随机选择动作$a$,观察回报$r$和下一状态$s'$
\begin{algorithm}
\caption{Q-Learning}
\begin{algorithmic}[1]
\State 随机初始化$Q[S,A]$
\State 观测当前状态$s$
\Repeat
\State 选择动作$a$并执行
\State 观察回报$r$和状态$s'$
\State 依据下述公式进行更新
\[Q[s,a]\gets Q[s,a]+\alpha(r+\gamma\max_{a'}Q[s',a']-Q[s,a])\]
\State $s\gets s'$
\Until{收敛}
\end{algorithmic}
\end{algorithm}