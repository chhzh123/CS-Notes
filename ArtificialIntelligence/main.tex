\documentclass{note}
\usepackage[cpp,table,pseudo]{mypackage}
\usepackage{footnote}
\makesavenoteenv{tabular}

\renewcommand{\thefootnote}{\fnsymbol{footnote}}
\newenvironment{bayesian}{\begin{center}\begin{tikzcd}[cells={nodes={draw=black, circle}}]}{\end{tikzcd}\end{center}}

\def\Ent{\mathrm{Ent}}

\title{人工智能笔记}
\author{陈鸿峥}
\date{{\builddatemonth\today}\protect\footnote{\text{Build \builddate\today}}} % protect!

\setcounter{secnumdepth}{5}

\begin{document}

\maketitle
\renewcommand{\thefootnote}{\arabic{footnote}}
\setcounter{footnote}{0}

\setcounter{tocdepth}{2}%设置深度
\tableofcontents

\bigskip\bigskip

本课程主要选用Stuart Russell和Peter Norvig的《人工智能---一种现代化方法》(Artificial Intelligence: A Modern Approach)一书,理论及实验作业请见\url{https://github.com/chhzh123/Assignments/tree/master/ArtificalIntelligence},内容非常详实,基本涵盖了人工智能的几大板块。

% !TEX root = main.tex

\section{计算机网络概述}
计算机网络将终端设备连接起来并可以传输数据。

\subsection{网络连接方式}
\begin{enumerate}
\item 直接连接的网络(直连网)
\begin{itemize}
\item \underline{点对点(point-to-point)网络}:包括专用介质(dedicated medium)、节点/主机
\begin{itemize}
	\item 单向(simplex):如广播、电视
	\item 半双工(half duplex):异步双向,如对讲机
	\item 全双工(full duplex):同步双向,如电话
\end{itemize}
\item \underline{多路访问(multiple access)网络}:共享介质(shared medium),会产生碰撞(collision)
\begin{itemize}
	\item 单播(unicast):一对一
	\item 多播(multicast):一对多
	\item 广播(broadcast):一对所有
\end{itemize}
\end{itemize}
\item 间接连接的网络:涉及交换机、路由器
\end{enumerate}

\subsection{因特网}
用路由器或网关(gateway)连接起来构成的网络称为互\textred{连}网络(internetwork)。

因特网/互联网(Internet)是一种互连网络,可以看作是把世界各地的广域网互连的网络,是世界上最大的特定计算机网络,采用\textemph{TCP/IP协议簇}作为通信规则。
\begin{itemize}
	\item 系统域网(System Area Network, SAN):电脑、鼠标、USB
	\item 局域网(Local Area Network, LAN):某一区域内由多台计算机互联成的计算机组,一般是方圆几千米以内,如小型实验室;常用\textemph{多路访问网络}
	\item 城域网(Metropolitan Area Network, MAN)
	\item 广域网(Wide Area Network, WAN):\textemph{因特网}
\end{itemize}

\myhline
因特网设备:
\begin{itemize}
	\item 终端系统/主机(end system):运行网络应用程序,如手机、浏览器
	\item 通信链路(communication link):光纤、铜线、无线电、卫星等
	\item 路由器(router):用于连接多个网络形成更大的网络
\end{itemize}

\myhline
因特网的组成:ISP(Internet Service Provider)
\begin{itemize}
	\item 网络边界(network edge):主机及网络程序,终端设备可以通过本地ISP或区域ISP连接上互联网
	\item 接入网络/接入网(access network):有线或无线接入,连接订阅者和服务提供商,如WiFi
	\item 网络核心/主干网(core network):顶层ISP(中国电信、中国移动、中国网通),可以连接局部提供商
\end{itemize}

\subsection{网络服务}
通信服务类型:
\begin{itemize}
	\item 可靠/不可靠:会不会丢包/收发是否完全相同,如文件(可靠)/视频(不可靠)
	\item 面向连接/无连接:需不需要建立通信线路,如电话(连接,双方都要在)/寄信、因特网(无连接,对方可能不在)
	\item 有确认/无确认:需不需要确认对方是否收包,因特网不需要
	\item 请求响应/消息流服务:有请求才有响应/一直发消息,如电视
\end{itemize}

因特网是\textemph{数据报服务},\textemph{无连接无确认(尽力服务)}。

\subsection{因特网体系结构}
因特网体系结构包括以下这\textemph{五层},而ISO/OSI(open system interconnection)网络包括七层协议\footnote{也有TCP/IP四层的说法,将物理层和数据链路层合并起来变成物理网络层}:
\begin{itemize}
	\item 应用层:提供对某些专门应用的支持,如\underline{FTP、SMTP、HTTP}
	\item (OSI)表示层(presentationn):提供数据转换服务, 如\underline{加密解密,压缩解压缩,数据格式变换}
	\item (OSI)会话层(session):简化会话实现机制,如\underline{数据流的检查点设置和回滚,多数据流同步}
	\item 传输层:将网络层获得的包在\textemph{进程之间}数据传送(端到端),如\underline{TCP、UDP}
	\item 网络层:\textemph{路由选择},实现在互联网中的数据传送(主机到主机),如\underline{IP协议、路由协议}
	\item 数据链路层:在\textemph{物理网络}中传送\textemph{包}(跳到跳\footnote{一跳(hop)/节点为一个物理设备,即数据链路层只考虑直连网的情况},节点到节点),如\underline{PPP、Ethernet}
	\item 物理层:线上的\textemph{比特}(传送原始比特流)
\end{itemize}
\par 其中\textemph{网络层以下不可靠,以上可靠};防止丢包的机制:\textemph{重发}。
\par 物理层和数据链路层又被称为\underline{物理网络},网络层和传输层被称为\underline{逻辑网络}。

\myhline
协议(protocol):在网络实体(entities)之间传送消息的规则,如消息的格式、收发消息的次序等。

每层传输的数据单元都称为\textemph{包}(packets),都属于某个协议,又被称为\textemph{协议数据单元}(protocol data unit, PDU),包括\textemph{头部/协议控制信息}(potocal control data, PCI)和\textemph{服务数据单元}(service data unit, SDU)两部分。

\begin{minipage}{0.4\linewidth}
\begin{center}
\begin{tikzcd}
\text{应用层Application}\arrow{d}{\text{消息message}}\\
\text{传输层Transport}\arrow{d}{\text{数据段segment}}\\
\text{网络层Network}\arrow{d}{\text{数据报datagram}}\\
\text{链路层Data-link}\arrow{d}{\text{帧frame}}\\
\text{物理层Physical}
\end{tikzcd}
\end{center}
\end{minipage}
\begin{minipage}{0.6\linewidth}
\begin{figure}[H]
	\centering
	\includegraphics[width=\linewidth]{fig/ipencap.png}
\end{figure}
\end{minipage}

\bigskip
下层把上层通过服务访问点(service access point, SAP)传来的SDU用PCI封装为PDU后传给对等实体(peer entity),即实现相同协议的实体。
同一个互连网络中网络层协议需要相同,链路层协议可以不同。

\myhline
\begin{figure}[H]
	\centering
	\includegraphics[width=0.8\linewidth]{fig/network-flow.PNG}
	\caption*{协议栈(stack):发送时封装(encaptulation),接收时拆封。}
\end{figure}

\myhline
\begin{figure}[H]
	\centering
	\includegraphics[width=0.5\linewidth]{fig/protocol_family.png}
	\caption*{协议簇(protocol family)}
\end{figure}

\subsection{网络性能分析}
当一个包到达时如果有空闲缓存则排队等待转发,产生延迟(delay);
如果没有空闲缓存,则丢弃该包,造成丢失(loss)。

包交换网络中的延迟主要有以下四点:
\begin{itemize}
	\item 处理(processing)延迟:查路由,存储转发(store-and-forward)的延迟会很大
	\item 排队(queueing)延迟:依赖于路由器的拥塞程度
	\item 发送/传输(transmission)延迟:\[\text{传输延迟}=\text{包长(bits)}/\text{链路带宽(bps, bit per second)}\]
	指从发送第一个包到发送最后一个包的间隔
	\item 传播(propagation)延迟:指对于一个包来说从发送到接收所需的时间
	\[\text{传播延迟}=\text{物理链路长度}/\text{信号传播速度}\]
\end{itemize}

接收延迟与传播延迟重合。
故忽略掉处理、排队延迟,
\[\text{总延迟(从第一个包被发送到最后一个包被接收的时间)}=\text{传播延迟}+\text{发送延迟}\]

\myhline
\par 往返时间(round trip time, RTT):从源主机到目的主机再返回源主机所花的时间
\par 带宽(bandwidth):一条链路或通道可达到的\textemph{最大}数据传输速率(bps)
\par 吞吐量(thoughput):一条链路或通路\textemph{实际}数据传输速率

\begin{example}
	如果一个长度为$3000$字节的文件用一个数据包从源主机通过一段链路传给了一个交换机,然后再通过第二段链路到达目的主机。
	如果在包交换机的延迟为$2ms$,两条链路上的传播延迟都是$2\times 10^8m/s$,带宽都是$1Mbps$,长度都是$6000km$。
	采用以下三种方式,问这个文件在这两台主机之间的总延迟是多少?
	\begin{enumerate}
		\item 交换机采用存储转发方式
		\item 将文件分成10个数据包,且存储转发
		\item 收到一位转发一位
	\end{enumerate}
\end{example}
\begin{analysis}
	\begin{enumerate}
	\item 因采用存储转发技术,先计算一段的延时,最后乘2。
	\begin{itemize}
	\item 一段的传输延时:$3000B\times 8/10^6bps=24$ms
	\item 一段的传播延时:$6000km/(2\times 10^8m/s)=30$ms
	\item 转发延时:$2$ms
	\end{itemize}
	总时长:$(24+30)\times 2+2=110$ms
	\item 类似1,但是总时长是一个包的传输传播转发延迟,加上剩余包的接收/传输延迟,见下表加粗部分
	\begin{center}
		\begin{tabular}{|c|c|c|c|c|c|}\hline
			包1 & \textemph{传输} & \textemph{传播} & \textemph{接收} & & \\\hline
			包2 &  & 传输 & 传播 & \textemph{接收} & \\\hline
			包3 &  &  & 传输 & 传播 & \textemph{接收} \\\hline
		\end{tabular}
	\end{center}
	\begin{itemize}
		\item 一段的传输延时:$300B\times 8/10^6bps=2.4$ms
		\item 一段的传播延时:$30$ms
		\item 转发延时:$2$ms
	\end{itemize}
	总时长:$(2.4+30)\times 2+2+2.4\times 9=88.4$ms
	\item 同1,但是只用计算一段传输延时,因为1位的转发延迟忽略。
	故总时长:$24+30\times 2=84$ms
\end{enumerate}
\end{analysis}
% !TEX root = main.tex

\section{搜索}
% Problem solving by search: formalization
% Uninformed search: Breadth-First, Uniform-Cost, Depth-First, Depth-Limited, and Iterative- Deepening
% Properties of search: completeness, optimality, time and space complexity
% Path/cycle checking

搜索主要包括无信息(uninformed)搜索和有信息搜索。
\begin{itemize}
	\item 状态空间(state space)
	\begin{itemize}
		\item 传统搜索:状态空间可见、动作确定性
		\item 非传统搜索:局部搜索、模拟退火、爬坡
	\end{itemize}
	\item 动作(action):不同状态之间的转换
	\item 初始状态(initial state)
	\item 目标/期望(goal)
\end{itemize}
% 对抗搜索(adversarial):博弈树(minimax)、$\alpha-\beta$剪枝

树搜索,边界集(frontier)是未探索的状态集合
\begin{algorithm}[H]
\caption{Tree Search}
\begin{algorithmic}[1]
\Procedure{TreeSearch}{(Frontier, Successors, Goal?)}
\If{Frontier is empty}
\State \Return failure
\EndIf
\State Curr $=$ select state from Frontier
\If{Goal?(Curr)}
\State \Return Curr
\EndIf
\State $\text{Frontier'} = (\text{Frontier} - \{\text{Curr}\}) \cup \text{Successors(Curr)}$
\State \Return TreeSearch(Frontier',Successors,Goal?)
\EndProcedure
\end{algorithmic}
\end{algorithm}

搜索需要关注的几个特性:
\begin{itemize}
	\item 完备性:若解存在,搜索是否总能找到解
	\item 最优性:是否总能找到最小代价的解
	\item 时间复杂性:最大需要被\textemph{生成或展开}\footnote{而不是探索的结点数目}的结点数
	\item 空间复杂性:最大需要被存储在内存中的结点数
\end{itemize}

\subsection{无信息搜索}
无信息搜索并不考虑关于特定搜索问题的领域特定的信息,主要包括宽度优先、一致代价、深度优先、深度受限、迭代加深五种算法。

\subsubsection{宽度优先搜索(BFS)}
将后继加入边界集的\textbf{后面},$b$为最大状态后继数目/分支因子(branching factor),$d$为最短距离解的行动数(注意是\textemph{边数},而不是层数!或者把根节点看作第$0$层也可以)
\begin{itemize}
	\item 完备性与最优性:所有短路总在长路前被探索,某一长度只有有限多条路径,最终可以检测所有长度为$d$的路径,从而找到最优解
	\item 时间复杂度:$1+b+b^2+\cdots+b^d+(b^d-1)b=O(b^{d+1})$,最差情况在最后一层的最后一个节点才探索到最优解,从而前面$b$个节点都要展开第$d+1$层
	\item 空间复杂度:$b(b^d-1)=O(b^{d+1})$,需要将边界集都存储下来,同上最后一层
\end{itemize}

\subsubsection{深度优先搜索(DFS)}
将后继加入边界集的\textbf{前面},即总是展开边界集中最深的节点
\begin{itemize}
	\item 完备性
	\begin{itemize}
		\item 无限状态空间:不能保证
		\item 有限状态空间无限路径:不能保证(可能有环)
		\item 有限状态空间+路径/重复状态剪枝:可以保证
	\end{itemize}
	\item 最优性:因完备性不能保证,故最优性也不能保证
	\item 时间复杂性:$O(b^m)$,其中$m$为状态空间的最长路的长度(若$m>>d$,则非常糟糕;但如果有大量解路径,则会快于BFS)
	\item 空间复杂性:$O(bm)$,\textbf{线性空间复杂性}是DFS最大的优点。边界集只包含当前路径的最深节点以及回溯节点(backtrack points为当前路径上节点的未探索的兄弟sibling)。
	注意这里需要记录路径上每个结点的孩子,因已被扩展。
\end{itemize}

\subsubsection{一致代价(Uniform-cost)}
一致代价搜索(Uniform cost search, UCS)\footnote{至于为什么叫Uniform,可以看\url{https://math.stackexchange.com/questions/112734/in-what-sense-is-uniform-cost-search-uniform}和\url{https://cs.stackexchange.com/questions/6072/why-is-uniform-cost-search-called-uniform-cost-search},比较合理的解释是到达同一结点的cost都被认为是相同的(寻找最优解时)。一致的算法总是选择边界集中第一个元素。}的边界集以路径开销升序排序,总是先展开最低开销的路径。
如果每一个动作都是一样的代价,则一致代价等价于BFS。
\begin{itemize}
	\item 完备性与最优性:假设所有转移都有代价$\geq\eps>0$,所有更低代价的路径都在高代价路径之前被展开,只有有限多的路径开销小于最优解的开销,故最终一定会到达最优解
	\item 时间复杂性:$O(b^{\lfloor C^\star/\eps\rfloor+1})$,对应着BFS中$d=C^\star/\eps$,其中$C^\star$为最优解的开销,最坏情况就是每一层开销都很小为$\eps$,那么需要$\lfloor C^\star/\eps\rfloor$层
	\item 空间复杂性:$O(b^{\lfloor C^\star/\eps\rfloor+1})$
\end{itemize}
\begin{figure}[H]
\centering
\includegraphics[width=0.7\linewidth]{fig/UCS.png}
\end{figure}
注意算法中几个关键点:从队列Pop出一个结点后
\begin{enumerate}
	\item \textemph{先做目标检测},如果是目标则立即返回(也就是说要扩展完结点进入下一层才会触发解返回)
	\item 将结点状态标记为\textemph{已探索(explored)},然后扩展(expand)该结点
	\item 做环检测,如果孩子\textemph{不在\underline{已探索或边界集}中},将孩子节点加入边界集
	\item 如果孩子在边界集中且有更高的路径开销,则将边界集中的结点用孩子结点进行\textemph{替换}
\end{enumerate}

\subsubsection{深度受限搜索(Depth-limited)}
只在最大深度执行DFS,因此无穷路径长不会存在问题
\begin{itemize}
	\item 完备性与最优性:不能保证,若解的深度大于$L$
	\item 时间复杂度:$O(b^L)$
	\item 空间复杂度:$O(bL)$
\end{itemize}

\subsubsection{迭代加深搜索(Iterative Deepening)}
IDS逐渐增加最大深度$L$,对每一个$L$做深度受限搜索
\begin{itemize}
	\item 完备性:可以保证
	\item 最优性:如果开销一致\footnote{若开销不一致,则可以采用代价界(cost bound)来代替:仅仅展开那些路径开销小于代价界的路径,同时要记录每一层深搜的最小代价。这种方式的搜索开销会非常大,有多少种不同路径开销就需要多少次迭代循环。},则可以保证
	\item 时间复杂性:$(d+1)b^0+db+(d-1)b^2+\cdots+b^d=O(b^d)$,第$0$层搜了$(d+1)$次,以此类推。可以看到时间复杂度是\textbf{比BFS优}的,因为最后一层的结点并未进行展开。
	\item 空间复杂性:$O(bd)$,同DFS
\end{itemize}

\subsubsection{双向搜索(Bidirectional)}
从源结点和汇结点同时采用BFS,直到两个方向的搜索汇聚到中间。
\begin{itemize}
	\item 完备性:由BFS保证
	\item 最优性:若一致代价则可保证
	\item 时间复杂性:$O(b^{d/2})$
	\item 空间复杂性:$O(b^{d/2})$
\end{itemize}

\subsubsection{环路/路径检测}
所有检测都是在\textemph{扩展时}进行:
\begin{itemize}
	\item 环路(cycle)检测:检测当前状态是否与所有已探索的(explored)状态重复(BFS)
	\item 路径(path)检测:只检测当前状态是否与该路径上的状态重复(DFS)
\end{itemize}
注意不能将环路检测运用在BFS上,因为这会破坏其空间复杂度的优势。

环路检测运用到UCS上依然\textemph{可以保证最优性}\footnote{注意这在启发式搜索中不一定成立}。
因为UCS第一次\textemph{探索}(注意不是展开)到某一状态的时候已经发现最小代价路径,因而再次探索该状态不会发现路径比原有的更小。

\subsubsection{总结}
\begin{center}
\begin{tabular}{ccccccc}\hline
& BFS & UCS & DFS & Depth-limited & IDS & Bidirectional\\\hline
完备性 & \cmark & \cmark & \xmark & \xmark & \cmark & \cmark\\
时间复杂度 & $O(b^{d+1})$ & $O(b^{\lfloor C^\star/\eps\rfloor+1})$ & $O(b^m)$ & $O(b^l)$ & $O(b^d)$ & $O(b^{d/2})$\\
空间复杂度 & $O(b^{d+1})$ & $O(b^{\lfloor C^\star/\eps\rfloor+1})$ & $O(bm)$ & $O(bl)$ & $O(bd)$ & $O(b^{d/2})$\\ 
最优性 & \cmark & \cmark & \xmark & \xmark & \cmark & \cmark\\\hline
\end{tabular}
\end{center}

\begin{example}
$N$个传教士和$N$个食人族要过河,他们都在河的左岸。
现在只有一条船能够运载$K$个人,要把他们都运往右岸。
要满足无论何时何地,传教士的数目都得大于等于食人族的数目,或者传教士数目为0。
\end{example}
\begin{analysis}
考虑对问题形式化为搜索问题
\begin{itemize}
	\item 状态$(M,C,B)$,其中$M$为左岸传教士数目,$C$为左岸食人族数目,$B=1$指船在左岸
	\item 动作$(m,c)$指运$m$个传教士和$c$个食人族到对岸
	\item 先决条件:传教士数目和食人族数目满足限制
	\item 效果:$(M,C,1)\stackrel{(m,c)}{\implies}(M-m,C-c,0)$\\
	$(M,C,0)\stackrel{(m,c)}{\implies}(M+m,C+c,1)$
\end{itemize}
\end{analysis}
% !TEX root = main.tex

\subsection{有信息搜索}
在无信息搜索中,我们从不估计边界集中最有期望(promising)获得最优解的结点,而是无区别地选择当前边界集中第一个结点。
然而事实上,针对不同问题我们是有对结点的先验知识(apriori knowledge)的,即从当前结点到目标结点的开销有多大。
而这就是有信息搜索(informed),或者称为启发式搜索(heuristics)。

关键在于领域特定启发式函数$h(n)$的设计,它估计了\textbf{从结点$n$到目标结点的开销(cost)}。
注意满足目标状态的结点$h(n)=0$。

\subsubsection{贪心最优搜索(Greedy Best-First Search)}
直接使用$h(n)$对边界集进行排序,但这会导致贪心地选择\textbf{看上去}离目标结点开销最小的路径。

如果存在环路,贪心最优搜索是不完备的,会陷入死循环。
\begin{center}
\begin{tikzcd}
 & s\arrow[ld]\arrow[rd] &\\
n_1\arrow[bend right,d] & & n_3\arrow[d]\\
n_2\arrow[bend right,u] & & \text{Goal}
\end{tikzcd}
\end{center}

\subsubsection{A$^\star$搜索}
综合考虑当前已走的开销和未来估计的开销。
定义一个估值函数
\[f(n)=g(n)+h(n)\]
其中$g(n)$为路径到节点$n$的代价,$h(n)$为从$n$到目标节点的代价,采用$f(n)$对边界集内的节点进行排序。

$f(n)$需要满足下列两个性质。
\begin{definition}[可采纳的(admissibility)]
假设所有代价$c(n1\to n2)\geq\eps>0$,令$h^\star(n)$为从节点$n$到目标节点$\infty$的最优解代价\footnote{如果没有路径则$h^\star(n)=\infty$},若
\[\forall n:\;h(n)\leq h^\star(n)\]
则称$h(n)$是可采纳的。
即一个可采纳的启发式函数总是\textemph{低估}了当前结点到目标结点的真实开销(这样才能保证最优解不被排除)。
因此对于任何目标结点$g$都有$h(g)=0$。
\end{definition}
\begin{definition}[一致性(consistency)/单调性(monotonicity)]
若对于所有的结点$n_1$和$n_2$,$h(n)$满足(三角不等式)
\[h(n_1)\leq c(n_1\to n_2)+h(n_2)\]
则称$h(n)$是单调的。
\end{definition}

\begin{theorem}
一致性蕴含可采纳性
\end{theorem}
\begin{analysis}
分类讨论
\begin{itemize}
	\item 当结点$n$没有到目标结点的路径,则$h(n)\leq h^\star(n)=\infty$恒成立
	\item 令$n=n_1\to n_2\to\cdots\to n_k$为从结点$n$到目标结点的最优路径,则可以用数学归纳法证明$\forall i:\;h(n_i)\leq h^\star(n_i)$,如下从后往前推
	\[h(n_{i})\leq c(n_i\to n_{i+1})+h(n_{i+1})\leq c(n_i\to n_{i+1}) + h^\star(n_{i+1})=h^\star(n_i)\]
\end{itemize}
\end{analysis}

\begin{theorem}
可采纳性蕴含最优性
\end{theorem}
\begin{analysis}
假设最优解有开销$C^\star$,则任何最优解一定会在开销大于$C^\star$的路径之前被展开(有限条路径)。

反证若路径$p$的代价$c(p)$大于最优解路径代价$c(p^\star)$且$p$在$p^\star$之前被扩展,则一定存在一个节点$n$在$p^\star$上且仍在边界集中,因此
\[c(p)\leq f(p)\leq f(n)=g(n)+h(n)\leq g(n)+h^\star(n)=c(p^\star)\]
矛盾。故在最优解展开之前的路径一定有开销$\leq C^\star$,最终我们一定会检测到最优解,而且次优解不会在最优解之前被检测。
\end{analysis}

做环检测可能导致找不到最优解。
可采纳性不能保证最优解,但\textemph{单调性环检测保证最优解}。

单调性有以下几个性质
\begin{proposition}
路径上的$f$一定是非递减的
\end{proposition}
\begin{analysis}
\[\begin{aligned}
f(n)&=g(n)+h(n)\\
&\leq g(n)+c(n\to n')+h(n')\\
&= g(n')+h(n')\\
&= f(n')
\end{aligned}\]
\end{analysis}
\begin{proposition}
如果$n_2$在$n_1$之后被扩展,则$f(n_1)\leq f(n_2)$
\end{proposition}
\begin{proposition}
当$n$在任何小于$f$值得路径之前被展开
\end{proposition}
\begin{proposition}
$A^\star$算法第一次展开某个状态时,它已经找到了到达那个状态的最小开销路径。
\end{proposition}

\subsubsection{迭代加深A$^\star$(IDA)算法}
$A^\star$算法有和BFS或UCS同样的空间复杂性问题,而迭代加深$A^\star$算法同样解决空间复杂度的问题。
与迭代加深类似,但IDA$^\star$则用$f=g+h$作为截断阈值。
每一轮迭代中选择上一轮中超过截断阈值最小的$f$。

\subsubsection{构造启发式函数}
常常需要考虑一个更加简单的问题(松弛问题),然后让$h(n)$为到达一个简单问题解的开销。
比如考虑$A$和$B$之间没有屏障,$A$和$B$相邻等。

\begin{example}
现有积木若干,积木可以放在桌子上,也可以放在另一块积木上面。
有两种操作:
\begin{enumerate}
	\item \verb'move(x,y)':把积木x放到积木y上面,前提是积木x和积木y上面都没有其他积木
	\item \verb'moveToTable(x)':把积木x放到桌子上,前提是积木x上面无其他积木,且积木x不在桌子上
\end{enumerate}
设计一个可采纳的启发式函数$h(n)$
\end{example}
\begin{analysis}
$h(n)=h_1(n)+h_2(n)$,其中$h_1(n)$为在目标状态积木的数目,$h_2(n)$为符合上下关系的积木数目($A$在$B$下,且$A$在目标状态)
\end{analysis}
\begin{theorem}
在松弛问题中的最优解也是原问题中的一个可满足的启发式函数。
\end{theorem}
\begin{definition}[支配]
如果$h_2$支配(dominate)$h_1$且除了目标结点都可满足,则$h_1(n)\leq h_2(n)$
\end{definition}
% !TEX root = main.tex

\subsection{博弈树搜索}
% Game tree search: MiniMax, alpha-beta pruning

博弈的一些前提
\begin{itemize}
	\item 两个博弈玩家
	\item 离散值:游戏和决策都可以映射到离散空间
	\item 有限的:只有有限的状态和可能的决策
	\item 零和博弈:完全竞争,即如果一个玩家胜利,则另外一个失去同样数量的收益
	\item 确定性的:没有牵涉到概率性事件,如色子、抛硬币等
	\item 完美信息博弈:状态的所有方面都可以被完全观察,即没有隐藏的卡牌
\end{itemize}

剪刀石头布是简单的一次性(one-shot)博弈
\begin{itemize}
	\item 一次移动
	\item 在博弈论中称为策略或范式博弈(strategic/normal form)
\end{itemize}

但很多游戏是牵涉到多步操作的
\begin{itemize}
	\item 轮回(turn-taking)游戏,如棋类
	\item 在博弈论中称为扩展形式博弈(extensive form)
\end{itemize}

两个玩家$A$(最大化己方收益)和$B$(最小化对方收益)
\begin{itemize}
	\item 状态集合$\mathcal{S}$
	\item 初始状态$I\in\mathcal{S}$
	\item 终止位置$T\subset\mathcal{S}$
	\item 后继:下一可能状态的集合
	\item 效益(utility)/收益(payoff)函数$V:T\mapsto\rr$,表明终止状态对$A$玩家有多好,对$B$玩家有多坏(都站在$A$角度给出)
\end{itemize}

\subsubsection{Minimax策略}
\textemph{自己选max,对方选min}
\begin{itemize}
	\item 构建整棵博弈树,然后将终止/叶子结点标上收益
	\item 回溯整棵树,然后将每个结点都标记上收益
\[U(n)=
\begin{cases}
\min\{U(c):\;c \text{ is a child of } n\} & n \text{ is a Min node}\\
\max\{U(c):\;c \text{ is a child of } n\} & n \text{ is a Max node}
\end{cases}\]
\end{itemize}
\begin{figure}[H]
\centering
\includegraphics[width=0.6\linewidth]{fig/game-tree.png}
\end{figure}

用DFS可以遍历整棵树,同时保持线性的空间复杂度,每次回溯时更新结点为min/max即可

\subsubsection{Alpha-Beta剪枝}
注意$\alpha$-$\beta$剪枝只要有一祖先大于/小于后代节点的值就可以进行剪枝,即要看\textemph{所有祖先做决定}。
\begin{itemize}
	\item 只要当前Max结点的值$\geq$祖先某一Min结点的值,就可以在该Max结点上做$\alpha$剪枝
	\item 只要当前Min结点的值$\leq$祖先某一Max结点的值,就可以在该Min结点上做$\beta$剪枝
\end{itemize}
即\textemph{当前Min小等祖先Max,当前Max大等祖先Min可剪枝}。
\begin{algorithm}[H]
\caption{Alpha-Beta Pruning}
\begin{algorithmic}[1]
\Procedure{AlphaBeta}{n,Player,alpha,beta}
\If{n is TERMINAL}
\State \Return V(n)\Comment{Return terminal states utility}
\EndIf
\State ChildList = n.Successors(Player)
\If{Player == MAX}
\For{c in ChildList}
\State alpha = max(alpha, AlphaBeta(c,MIN,alpha,beta))
\If{beta <= alpha}
\State break
\EndIf
\EndFor
\State \Return alpha
\Else\Comment{Player == MIN}
\For{c in ChildList}
\State beta = min(beta, AlphaBeta(c,MAX,alpha,beta))
\If{beta <= alpha}
\State break
\EndIf
\EndFor
\State \Return beta
\EndIf
\EndProcedure\Comment{Initial call: AlphaBeta(START-NODE,Player,$-\infty$,$+\infty$)}
\end{algorithmic}
\end{algorithm}

\begin{example}
如下采用$\alpha-\beta$剪枝进行搜索,方形为Max结点,圆形为Min结点。
\begin{figure}[H]
\centering
\includegraphics[width=0.9\linewidth]{fig/T01-3.png}
\end{figure}
\end{example}

可以证明,如果原始情况需要访问$O(b^D)$个结点,则经过$\alpha$-$\beta$剪枝后只需访问$O(b^{D/2})$个结点。

\subsubsection{其他补充}
但在现实生活的游戏中,即使采用了$\alpha$-$\beta$剪枝,博弈树也太过庞大。
如棋类的分支因子大致是35,深度为10的树已经到$2.7\times 10^{14}$个结点。
因此不能将整棵博弈树展开,需要采用一些启发式方式进行估计。

评价函数(evaluation)的一些需求:
\begin{itemize}
	\item 对于终止状态,评价函数的序应与真实的收益函数相同
	\item 对于非终止状态,评价函数则应该与真实的胜率相关联
	\item 计算时间不能花太长
	\item 通常取多个特征,然后进行加权求和(先验知识)
\end{itemize}

在线(online)/实时(real-time)搜索
\begin{itemize}
	\item 没有办法展开全部的边界集,因此限制展开的大小(在没找到去目标的真实路径就做出决定/直接选一条路就开始走)
	\item 在这种情况下,评价函数不仅仅引导搜索,更是提交真实的动作
	\item 虽然找不到最优解,但是求解时间大大缩减
\end{itemize}
% !TEX root = main.tex

\section{限制可满足性问题}
限制可满足性问题(Constraint Satisfaction Problem, CSP)指每一个状态都可以用一组特征值向量表示的问题。
\begin{itemize}
	\item $k$个特征/变量的集合
	\item 每一个变量都有一个包含有限值的定义域
	\item 一个状态可以通过给每一个变量赋值得到
\end{itemize}
% !TEX root = main.tex

\section{知识表示与推理}
一阶逻辑(First-Order Logic,FOL)
\begin{itemize}
	\item 个体/常量(0-ary)
	\item 类型(unary)谓词:$A(x),B(x)$
	\item 关系(二元谓词):$L(x,y)$
\end{itemize}

\begin{definition}[项(term)]
每一个变量都是一个项。
若$t_1,\ldots,t_n$都为项,且$f$为$n$参数的的函数,则$f(t_1,\ldots,t_n)$是一个项。
\end{definition}
\begin{definition}[公式(formular)]
公式包括以下几种情况:
\begin{itemize}
\item 若$t_1,\ldots,t_n$都是项,且$P$是$n$元的谓词符号,则$P(t_1,\ldots,t_n)$是一个公式
\item 若$t_1,t_2$都是项,那么$(t_1=t_2)$是一个原子公式
\item 若$\alpha,\beta$都是公式,$v$是一个变量,则$\lnot\alpha,(\alpha\land\beta),(\alpha\lnot\beta),\exists v.\alpha,\forall v.\alpha$都是公式
\end{itemize}
\end{definition}
\begin{definition}[句子(sentence)]
没有自由变量的公式
\end{definition}
\begin{definition}[替换]
$\alpha[v/t]$表示$\alpha$中所有自由出现的$v$都用项$t$替代
\end{definition}
\begin{definition}[解释(interpretation)]
一个解释是一个对(pair)$\mI=\lrang{D,I}$,其中
\begin{itemize}
	\item $D$是论域,可以是任何非空集
	\item $I$是从谓词到函数符号的映射
	\item 如果$P$是一个$n$-参数的谓词符号,$I(P)$是一个在$D$上的$n$-参数的关系,即$I(P)\subset D^n$
\end{itemize}
\end{definition}
\begin{definition}[赋值(denotation)]
变量指派(assignment)$\mu$是一个从变量集合到论域$D$的映射
\[\begin{aligned}
\norm{v}_{\mI,\mu}&=\mu(v)\\
\norm{f(t_1,\ldots,t_n)}_{\mI,\mu}&=I(f)(\norm{t_1}_{\mI,\mu},\ldots,\norm{t_n}_{\mI,\mu})
\end{aligned}\]
\end{definition}
\begin{definition}[满足]
$\mI,\mu\models\alpha$读作$\mI,\mu$满足$\alpha$
\begin{itemize}
	\item $\mI,\mu\models\alpha\iff\lrang{\norm{t_1}_{\mI,\mu},\ldots,\norm{t_n}_{\mI,\mu}}\in I(P)$
	\item $\mI,\mu\models(t_1=t_2)\iff\norm{t_1}_{\mI,\mu}=\norm{t_2}_{\mI,\mu}$
\end{itemize}
\end{definition}
\begin{definition}[子句(clause)]
文字(literal)是原子公式或它的取反,一个子句是文字的析取(disjunction),如$p\lor\lnot r\lor s$,写作$(p,\lnot r,s)$。
特殊地,空子句$()$代表为假。
公式(formula)则是子句的合取(conjunction)。
\end{definition}

归结(resolution)
反驳(refutation)
\[\vdash\]
% nvDash

\begin{itemize}
	\item 消除蕴含:$A\to B\iff \lnot A\lor B$
	\item 将非向内推:德摩根定律
	\item 标准化变量:重命名变量使得每一个量词都是唯一的
	\item 消除存在量词(skloemize):引入新的函数符号,如$\forall x P(x)$改为$P(g(y))$
	\item 将所有量词带到最前面:只有全局量词,且名字均不同
	\item 析取分配到合取
	\item 压平
	\item 转化为子句:将量词全部移除
\end{itemize}

\begin{definition}[MGU]
两个公式$f$和$g$的替换$\sigma$
\begin{itemize}
	\item
	\item
\end{itemize}
\end{definition}

计算MGU的算法:不断代入新的元,使其一致

利用归结(两条文字合一变真删除)看是否能得到空子句

答案抽取(answer extraction)
\begin{itemize}
\item 将询问$\exists xP(x)$用$\exists x[P(x)\land\lnot\text{answer}(x)]$替换
(因为取非后变成$\forall x P(x)\implies \text{answer}(x)$)
\item 直到获得任意子句只包含答案的谓词
\end{itemize}
\begin{example}
对下列查询进行归结及答案查询
\begin{itemize}
	\item Whoever can read $R(x)$ is literate $L(x)$
	\item Dolphins $D(x)$ are not literate
	\item Flipper is an intelligent dolphin $I(x)$
\end{itemize}
Who is intelligent but cannot read?
\end{example}
\begin{analysis}
对语句进行形式化
\begin{center}
\begin{tabular}{lll}
$\forall x (R(x)\to L(x))$ & 1 & $(\lnot R(u),L(u))$\\
$\forall x (D(x)\to \lnot L(x))$ & 2 & $(\lnot D(v),\lnot L(v))$\\
$D(Flip)\land I(Flip)$ & 3 & $D(Flip)$\\
 & 4 & $I(Flip)$\\
Q:$\exists x (I(x)\land\lnot R(x))$ & 5 & $(\lnot I(y),R(y),answer(y))$\\\hline 
$R[4,5]/y=Flip$ & 6 & $(R(Flip),answer(Flip))$\\
$R[1,6]/u=Flip$ & 7 & $(L(Flip),answer(Flip))$\\
$R[2,7]/v=Flip$ & 8 & $(\lnot D(Flip),answer(Flip))$\\
$R[3,8]$ & 9 & $(answer(Flip))$
\end{tabular}
\end{center}
因此得到Flipper是聪明的但是不能阅读
\end{analysis}

一组子句是否可满足是NP完全的[Cook,1972]

% !TEX root = main.tex

\section{规划}
智能体应该能够对世界做出动作(action),而不仅仅是通过搜索解决问题或推理及知识表示。
核心是对动作的效果进行推理,并且计算什么动作能够达成特定的效果。

情景演算(Situation Calculation, SitCalc)三个基本部分
\begin{itemize}
	\item 动作(action)
	\item 情景(situations):动作序列,$do(a,s)$为动作、情景到新情景的函数映射,$S_0$为初始情景
	\[do(put(a,b),do(put(b,c),S_0))\]
	要区别情景与状态(state),如将硬币转两次,情景/动作历史不同,但状态都是一样的
	\item 流(fluent):从情景到情景的谓词或函数(动态变化过程)
	\item 条件(precondition):动作执行的前提条件
	\item 影响(effect):执行动作后改变的流。 % the fluents that change as the result of performing the action
	如下在情景$s$下执行修复动作后,$x$就不是破碎的
	\[\lnot Broken(x,do(repair(r,x)),s)\]
\end{itemize}

只陈述了执行动作的影响,而没有阐述没影响的部分

框架(frame)问题:找到一种高效的方法来确定动作的非效果(non-effects)而不是显式地将它们全部写下来,用一阶逻辑

利用归结进行情景演算

传统的规划没有完全或确定的信息,假设对于初始状态有完整的信息。
\begin{definition}[封闭世界假设(Closed-World Assumption, CWA)]
用于表示世界状态的知识库是一系列正的真实原子事实(与数据库类似)。
如$emp(A,C)$不在数据库中,则$\lnot emp(A,C)$为真
\end{definition}

STRIPS(Stanford Research Institute Problem Solver):
\begin{itemize}
	\item 世界被表示成封建世界知识库(CW-KB),一个STRIPS的动作表示成更新CW-KB的方式
	\item 一个动作生成新的KB,用以描述新的世界
\end{itemize}

在SitCalc中我们可能有不完全的信息(用一阶逻辑公式表示),而在STRIPS中,我们有完整的信息(用CW-KB)表示

例子如下:
$pickup(X)$:
\begin{itemize}
\item $Pre: {handempty, clear(X), ontable(X)}$
\item $Adds: {holding(X)}$
\item $Dels: {handempty, clear(X), ontable(X)}$
\end{itemize}
% !TEX root = main.tex

\section{不确定性规划}
% Bayesian networks: graphs + tables, inference
% Variable elimination algorithm
% Use D-separation to determine independence

\subsection{基础知识}
一组变量$V_1,\ldots,V_n$以及其对应的有限域$\dom[V_i]$,很容易导致指数的计算复杂度。
\begin{theorem}[全概率公式]
$\{B\}_{i=1}^k$为全集$U$的一个划分,则
\[\begin{aligned}
\pr{A}&=\pr{A\cap B_1}+\cdots+\pr{A\cap B_k}\\
&=\pr{A\mid B_1}\pr{B_1}+\cdots+\pr{A\mid B_k}\pr{B_k}
\end{aligned}\]
\end{theorem}
\begin{theorem}[条件独立]
若
\[\pr{B\mid A\cap C}=\pr{B\mid A}\]
,则在给定$A$的情况下$B$条件独立于$C$($C$没有给$A$增加知识)。
若对于所有$x\in\dom[X],y\in\dom[Y],z\in\dom[Z]$,
\[\pr{X=x\land Y=y\mid Z=z}=\pr{X=x\mid Z=z}\pr{Y=y\mid Z=z}\]
则在给定$Z=z$下,$X=x$和$Y=y$条件独立。
\end{theorem}
\begin{proposition}[独立性性质]
\begin{itemize}
	\item 若$A$和$B$独立,则$\pr{A\cap B}=\pr{A}\cdot\pr{B}$
	\item 给定$A$,$B$和$C$条件独立,则
	\[\pr{B\cap C\mid A}=\pr{B\mid A}\pr{C\mid A}\]
\end{itemize}
\end{proposition}
\begin{theorem}[贝叶斯公式]
条件概率定义
\[\pr{X\mid Y}=\pr{XY}/\pr{Y}\implies \pr{XY}=\pr{X\mid Y}\pr{Y}\]
注意与贝叶斯公式区分
\[\pr{Y\mid X}=\frac{\pr{XY}}{\pr{X}}=\frac{\pr{X\mid Y}\pr{Y}}{\pr{X}}\]
\end{theorem}
\begin{theorem}[链式法则]
\[\pr{A_1\cap A_2\cap\cdots\cap A_n}=\pr{A_1\mid A_2\cap\cdots\cap A_n}\pr{A_2\mid A_3\cdots\cap A_n}\pr{A_{n-1}\mid A_n}\pr{A_n}\]
% P(123) = P(1|23)P(2|3)P(3) = P(123)/P(23) P(23)/P(3) P(3)
\end{theorem}

\subsection{贝叶斯网络}
\subsubsection{基础知识}
\begin{bayesian}
E \arrow[r] & C\arrow[r] & A \arrow[r] & B\arrow[r] & H
\end{bayesian}
有以下概率公式成立
\begin{itemize}
	\item $\pr{H\mid B,A,E,C}=\pr{H\mid B}$
	\item 由链式法则和独立性假设
	\[\begin{aligned}
		\pr{HBACE}&=\pr{H\mid BACE}\pr{B\mid ACE}\pr{A\mid CE}\pr{C\mid E}\pr{E}\\
		\pr{HBACE}&=\pr{H\mid B}\pr{B\mid A}\pr{A\mid C}\pr{C\mid E}\pr{E}
	\end{aligned}\]
	\item 通用公式
	\[\pr{X_1X_2\cdots X_n}=\pr{X_n\mid Par(X_n)}\cdots\pr{X_1\mid Par(X_1)}\]
	\item 单点概率(由全概率公式和条件概率定义)
	\[\begin{aligned}
		\pr{a}&=\sum_{c_i\in\dom[C]}\pr{a\mid c_i}\pr{c_i}\\
		&=\sum_{c_i\in\dom[C]}\pr{a\mid c_i}\sum_{e_i\in\dom[E]}\pr{c_i\mid e_i}\pr{e_i}
	\end{aligned}\]
\end{itemize}

因此每个结点只需存一个条件概率表(conditional probability table, CPT)。

\begin{definition}[D分隔(separation)]
若一组变量$E$阻隔(block)了$X$到$Y$的所有\textcolor{red}{无向路径}$P$,则称$E$ D-分隔了$X$和$Y$,且有给定证据$E$下,$X$和$Y$条件独立。
$E$阻隔(block)了路径$P$当且仅当在路径上存在某点$Z$使得下面任一成立。
\begin{figure}[H]
\centering
\includegraphics[width=0.8\linewidth]{fig/blocking.png}
\end{figure}
注意第3种情况,证据集中没有给交叉结点及其子结点,则父亲节点都独立。
\end{definition}
\begin{example}
考虑如下贝叶斯网络
\begin{bayesian}
A\arrow[dr] & & B\arrow[dl] \arrow[dr] & \\
 & C\arrow[dl] \arrow[dr] & & D\\
E & & F &
\end{bayesian}
在此例中$\pr{c\mid a,b,\lnot d,\lnot e,\lnot f}\ne\pr{c\mid a,b}$。
\end{example}
\begin{example}
\begin{figure}[H]
\centering
\includegraphics[width=0.8\linewidth]{fig/d-separation_example.png}
\end{figure}
注意第3种情况的适用条件,由于$H$在证据集中,所以并不满足第3种情况。
在第7个例子中,$AGHBE$没有被阻隔,故给定$FDH$,$A,E$也不独立。
\end{example}

\subsubsection{贝叶斯推断}
推断过程如下:
\[\begin{aligned}
\pr{a\mid d,e}&=\pr{a,d,e}/\pr{d,e}\\
&=\pr{a,d,e}/\sum_{A}\pr{a,d,e}\\
\pr{a,d,e}&=\sum_{B,C}\pr{a,B,C,d,e}
\end{aligned}\]
故只需计算$\pr{a,d,e}$。

采用动态规划的思想,存储子项,减少计算量。

记因子为某些变量的函数,如$\pr{C\mid A)}=f(A,C)$
\begin{itemize}
	\item 乘积:$h(X,Y,Z)=f(X,Y)\times g(Y,Z)$
	\item 求和:$h(Y)=\sum_{x\in\dom[X]}f(x,Y)$
	\item 因子限定:$h(Y)=f(a,Y)$
\end{itemize}

\begin{myalgorithm}[变量消除(Variable Elimination, VE)]
给定贝叶斯网络,条件概率表$F$,询问$Q$,证据$E$,其余变量为$Z$,计算$\pr{Q\mid E}$。
\begin{enumerate}
	\item 对于$f\in F$中每一变量,将其替换为$f_{E=e}$(因子限定)
	\item 对于每一$Z_j\in Z$,按给定$Z_j$顺序,并按照以下步骤消除:
	\begin{itemize}
		\item $f_1,f_2,\ldots,f_k$为含有$Z_j$的因子
		\item 计算新的因子$g_j=\sum_{Z_j}f_1\times f_2\times\cdots\times f_k$
		\item 将$f_i$从$F$中移除,并将新的因子$g_j$添加到$F$中
	\end{itemize}
	\item 剩下的因子只包含询问$Q$中的变量,则计算它们的乘积归一化得到$\pr{Q\mid E}$
\end{enumerate}
\end{myalgorithm}

可以采用桶消除(bucket elimination)算法,每次将新生成的因子放在第一个可被应用的桶中。
下图展示的是超图(hypergraph),最大超边(hyperedge)的大小即为最大的CPT表项/VE算法的复杂度。
\begin{figure}[H]
\centering
\includegraphics[width=0.8\linewidth]{fig/bucket_elimination.png}
\end{figure}

多树(polytree):单连通(singly connected)的贝叶斯网络,即在任意两个结点间只有一条路径。

最小填充(min-fill)启发式:总是先消除产生最小因子大小的变量,这种方法可以使得在线性时间内求解多树。

\begin{definition}[相关性(relevance)]
给定证据$E$询问$Q$,则有以下几种情况:
\begin{itemize}
	\item $Q$自身当然是相关的
	\item 若结点$Z$相关,则它的父母也相关
	\item 若$e\in E$是一个相关结点的\textcolor{red}{后代},则$E$也是相关的
\end{itemize}
\end{definition}
\begin{example}
\begin{figure}[H]
\centering
\includegraphics[width=0.8\linewidth]{fig/relevance_example.png}
\end{figure}
但询问$P(F\mid H)$,则$D,E,G$是不相关的。
而且这种算法会过度估计相关的变量。
\end{example}
% !TEX root = main.tex

\section{机器学习}
\subsection{决策树}
假定样本集合$D$中第$k$类样本所占比例为$p_k(k=1,2,\ldots,|\mathcal{Y}|)$,则$D$的信息熵定义为
\[\Ent(D)=-\sum_{k=1}^{|\mathcal{Y}|}p_k\log_2 p_k\]

又假定离散属性$a$有$V$个可能取值$\{a^1,a^2,\ldots,a^V\}$,第$v$个分支结点包含了$D$中所在属性$a$上取值为$a^v$的样本,记为$D^v$,进而可定义信息增益
\[\mathrm{Gain}(D,a)=\Ent(D)-\sum_{v=1}^V\frac{|D^v|}{|D|}\Ent(D^v)\]
每次选择最大增益的属性进行划分,即
\[a_*=\argmax_{a\in A}\mathrm{Gain}(D,a)\]

以信息增益为准则来选择划分属性的算法即为ID3决策树。

\subsection{贝叶斯学习}
先验$\pr{H}$、似然$\pr{d\mid H}$、证据$d=\lrang{d_1,d_2,\ldots,d_n}$,有贝叶斯公式(先验推后验)
\[\pr{H\mid d}=\alpha\pr{d\mid H}\pr{H}\]

假设独立同分布$\pr{d\mid h}=\prod_j\pr{d_j\mid h}$
\begin{itemize}
	\item 贝叶斯学习:$\pr{X\mid d}=\sum_i\pr{X\mid d,h_i}\pr{h_i\mid d}=\sum_i\pr{X\mid h_i}\pr{h_i\mid d}$
	\item 极大后验(MAP)学习:$\pr{X\mid d}\approx\pr{X\mid h_{MAP}}$
	\[h_{MAP}=\argmax_{h_i}\pr{h_i\mid d}=\argmax_{h_i}\pr{h_i}\pr{d\mid h_i}\]
	\item 最大似然(ML)学习:$\pr{X\mid d}\approx\pr{X\mid h_{ML}}$
	\[h_{ML}=\argmax_{h_i}\pr{d\mid h_i}\]
\end{itemize}

基于属性条件独立性假设
\[\pr{c\mid\vx}=\frac{\pr{c}\pr{\vx\mid c}}{\pr{x}}=\frac{\pr{c}}{\pr{\vx}}\prod_{i=1}^d\pr{x_i\mid c}\]
其中$d$为属性数目,$x_i$为$\vx$在第$i$个属性上的取值。
因对所有类别来说$\pr{\vx}$相同,因此贝叶斯判定准则为
\[h_{NB}(\vx)=\argmax_{c\in\mathcal{Y}}\pr{c}\prod_{i=1}^d\pr{x_i\mid c}\]
令$D_c$表示训练集$D$中第$c$类样本组成的集合,有
\[\pr{c}=\frac{D_c}{D}\qquad\pr{x_i\mid c}=\frac{|D_{c,x_i}|}{|D_c|}\]

\subsection{聚类算法}
\begin{itemize}
	\item 硬聚类:每个样本都决定放在哪一个类别中
	\item 软聚类:每个样本都被指派每个类别的概率分布
\end{itemize}

K-means算法
\begin{itemize}
	\item E步:对于每一个类别$i$和特征$X_j$,有
	\[pval(i,X_j)\gets\frac{\sum_{e:class(e)=i}val(e,X_j)}{|\{e:class(e)=i\}|}\]
	\item M步:对于每一个样本$e$,指派$e$给类别$i$使得
	\[\min_i\sum_{j=1}^n(pval(i,X_j)-val(e,X_j))^2\]
\end{itemize}

\subsection{神经网络}
\begin{theorem}[一致近似理论(Universal Approximator Theorem)]
具有至少一个隐层的深度神经网络可以无限逼近任意连续函数
\end{theorem}

前向后向传播过程:
\begin{itemize}
	\item 前向过程:
	\[in_j=\sum_i w_{ij}a_i\qquad a_j=g(in_j)\]
	\item 后向过程:
	\[\begin{aligned}
	\text{output:} & \Delta_j=g'(in_j)(y_j-a_j)\\
	\text{hidden:} & \Delta_i=g'(in_i)\sum_j w_{ij}\Delta_j
	\end{aligned}\]
\end{itemize}

注意以下两条求导公式
\[\begin{array}{rlrl}
\sigma(x)&=\dfrac{1}{1+\ee^{-x}} & \pd{\sigma(x)}{x}&=(1-\sigma(x))\sigma(x)\\
\tanh(x)&=\dfrac{\ee^x-\ee^{-x}}{\ee^x+\ee^{-x}} & \pd{\tanh(x)}{x}&=1-\tanh^2(x)
\end{array}\]

\subsection{强化学习}
目标:
\[\max_{\pi_\theta}\lrs{\sum_{t=1}^\infty\gamma^tr_t}\]

给定策略$\pi$:
\[\begin{aligned}
Q^\pi(s,a) &= \sum_{s'}\pr{s'\mid a,s}(R(s,a,s')+\gamma V^\pi(s'))\\
V^\pi(s) &= Q(s,\pi(s))
\end{aligned}\]

Q学习:随机选择动作$a$,观察回报$r$和下一状态$s'$
\[Q[s,a]\gets Q[s,a]+\alpha(r+\gamma\max_{a'}Q[s',a']-Q[s,a])\]

\end{document}