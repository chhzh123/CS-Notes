\documentclass[11pt,UTF8]{ctexart}
\usepackage{setspace}
\usepackage{geometry}
\usepackage{amsthm,amsmath,amsfonts,amssymb}%\because\therefore
\usepackage{mathtools}%underset
\usepackage{cancel}
\usepackage{extarrows}
\usepackage{enumerate}
\usepackage{tikz,tikz-cd,forest}
\usepackage{float}
\usepackage{gensymb}
\usepackage{url}
\usepackage[unicode=true,%本行非常重要 支持中文目录hyperref CJKbookmarks对二级目录没用
	colorlinks,
	linkcolor=black,
	anchorcolor=black,
	citecolor=black,
	CJKbookmarks=false]{hyperref}
\usepackage{datetime}
\newdateformat{builddatemonth}{\THEYEAR.\twodigit{\THEMONTH}}
\newdateformat{builddate}{\THEYEAR\twodigit{\THEMONTH}\twodigit{\THEDAY}}

\newtheorem{theorem}{定理}
\newtheorem{definition}{定义}
\newtheorem{example}{例}%*去除编号
\newtheorem*{analysis}{分析}
\newtheorem{corollary}{推论}
\newtheorem*{corollary_convex}{推论}
\newtheorem{exercise}{练习}
\geometry{top=20mm,bottom=20mm,left=20mm,right=20mm}
\pagestyle{plain}%删除页眉

\def\zz{\mathbb{Z}}
\def\rr{\mathbb{R}}
\def\diff{\,\mathrm{d}}
\def\mmid{\enspace\Big|\enspace}
\def\eps{\varepsilon}
\def\disp{\displaystyle}
\newcommand{\ssum}[1]{\displaystyle\sum_{n=1}^{\infty}#1}
\newcommand{\ssumz}[1]{\displaystyle\sum_{n=0}^{\infty}#1}
\newcommand{\ssumk}[1]{\displaystyle\sum_{k=1}^{\infty}#1}
\newcommand{\ssumkn}[1]{\displaystyle\sum_{k=1}^{n}#1}
\newcommand{\limtoinf}[1]{\displaystyle\lim_{n\to\infty}#1}
\newcommand{\intab}[3]{\displaystyle\int_{#1}^{#2}{#3}\,\diff x}
\newcommand{\intabu}[4]{\displaystyle\int_{#1}^{#2}{#3}\,\diff #4}
\newcommand{\inpro}[1]{\langle #1\rangle}
\newcommand{\lrp}[1]{\left(#1\right)}
\newcommand{\lra}[1]{\left|#1\right|}
\newcommand{\nums}[1]{\{#1\}}
\newcommand{\floor}[1]{\lfloor#1\rfloor}
\newcommand{\ol}[1]{\mathop{\overline{#1}}}%\mathop important!!!

\usetikzlibrary{graphs}
\usetikzlibrary{automata}
\usepackage{bigstrut}
\usepackage{titling}
\usepackage{float}

\renewcommand{\thefootnote}{\fnsymbol{footnote}}

\lstset{language=c++,basicstyle=\small,escapechar=`}

\setlength{\droptitle}{-55pt}%减少标题与页眉距离
\title{C结构体大小计算总结}
\author{陈鸿峥}
\postauthor{\end{tabular}\par\end{center}\vspace{-40pt}}
\date{}

\begin{document}

\maketitle
\section{基本概念}
\label{sec:introduction}
以下几个单位需要懂得区分,字的概念可以暂时不理。
\begin{itemize}
	\itemsep -3pt
	\item 位(bit/b):计算机处理、存储、传输信息的最小单位
	\item 字节(Byte/B) $1\text{ Byte}=8\text{ bit}$:现代计算机主存按\textbf{字节编址},字节是最小可寻址单位
	\item 字(Word):表示被处理信息的单位,用来度量数据类型的宽度\footnote{字长是指CPU中\textbf{数据通路的宽度},也指计算机一次能处理的二进制的长度,等于CPU内部总线的宽度或运算器的位数或通用寄存器的宽度;字和字长的宽度可以一样,也可以不同,通常是字节的整数倍}
\end{itemize}
\par 一台32位的电脑,一个字等于4个字节,字长为32位。
\par C语言的\verb'sizeof'返回的是数据类型的\textbf{字节}大小,如\verb'sizeof(int)=4',而不是32,不要搞混了。

\section{结构体大小的计算}
结构体的大小不是简单的内部数据大小之后,而是要依据一些基本原则进行存储:
\begin{enumerate}
	\itemsep -3pt
	\item 数据成员对齐规则:结构体的\textbf{第一个}数据成员放在偏移量(offset)为0的地方,之后每个数据成员存储的起始位置要从\textbf{该}成员大小的整数倍开始
	\item 结构体作为成员:如果一个结构里有某些结构体成员,则结构体成员要从其内部最大元素大小的整数倍地址开始存储
	\item 收尾工作:结构体的总大小,必须是其内部最大成员的整数倍,不足的要补齐
\end{enumerate}

上次的例子解释有误,在此重新解释。
\begin{example}
\begin{lstlisting}
struct point
{
    int index; // [0]...[3]
    double x; // [8]...[15]
    double y; // [16]...[24]
};
\end{lstlisting}
\end{example}
\begin{analysis}
\begin{itemize}
	\itemsep -3pt
	\item index占4字节。
	\item x由原则1从double大小(8)的整数倍(8)开始存储,占8字节;index的后四字节补齐。
	\item y占8字节。
	\item 故一共24字节,且不因32位机和64位机而异。
\end{itemize}
\end{analysis}

\begin{example}
\label{eg:2}
\begin{lstlisting}
struct data
{
    int id;         // [0]...[3]
    double weight;  // [8]...[15]
    float height;   // [16]...[19]
};
\end{lstlisting}
\end{example}
\begin{analysis}
\begin{itemize}
	\itemsep -3pt
	\item id占4字节。
	\item weight由原则1从double大小(8)的整数倍(8)开始存储,占8字节。
	\item height占接下来4字节。
	\item 最后由原则3,总大小要为最大成员double的整数倍(24),补齐最后4个字节,总共24字节。
\end{itemize}
\end{analysis}

\begin{example}
\begin{lstlisting}
struct data2
{
    char name[2];   // [0],[1]
    int id;         // [4]...[7]
    double score;   // [8]...[15]
    short grade;    // [16],[17]
    struct data d;  // [24]......[47]
    int last;       // [48]...[51]
};
\end{lstlisting}
\end{example}
\begin{analysis}
\begin{itemize}
	\itemsep -3pt
	\item char占1字节,数组则$\times$2。
	\item 由原则1,id从int大小(4)的整数倍(4)开始存储,占4字节。
	\item 由原则1,score从double大小(8)的整数倍(8)开始存储,占8字节。
	\item short占2字节。
	\item 由原则2,struct要从内部元素最大的一个(double)的整数倍(24)开始存放,由例\ref{eg:2}占24字节。
	\item int占接下来4字节。
	\item 由原则3,总大小要为最大成员double(单独考虑数组和结构体内元素)的整数倍,故总共56字节。
\end{itemize}
\end{analysis}

\begin{example}
\begin{lstlisting}
struct s1
{   
    char c1; // [0]
    int i;   // [4]...[7]
    char c2; // [8]
};

struct s2
{  
    char c1; // [0]
    char c2; // [1]
    int i;   // [4]...[7]
};
\end{lstlisting}
\end{example}
\begin{analysis}
同理之前的分析,s1大小为12,s2大小为8,由本例中可以看出结构体内数据的顺序也是会影响结构体的大小的。
\end{analysis}

\section{位域}
位域用来\textbf{显性地}声明一个数据的存储\textbf{位}大小,声明方法如下:
\begin{center}
\verb'<type><variable>:<bit size>'
\end{center}
如\verb'int a: 1'表示只用1位存储int类型。

涉及到位域(bit field)的问题则会更加复杂,但这很大程度上与编译器的实现有关。
如按照百度百科上给出的VC的准则是:
\begin{enumerate}
	\itemsep -3pt
	\item 如果相邻位域字段的类型相同,且其位宽之和小于类型的大小,则后面的字段将紧邻前一个字段存储,直到不能容纳为止
	\item 如果相邻位域字段的类型相同,但其位宽之和大于类型的大小,则后面的字段将从新的存储单元开始,其偏移量为该类型大小的整数倍
	\item 如果相邻的位域字段的类型不同,则各编译器的具体实现有差异,VC6采取不压缩方式(不同位域字段存放在不同的位域类型字节中),Dev-C++和GCC都采取压缩方式
\end{enumerate}
\begin{example}
\begin{lstlisting}
// A structure without forced alignment 
struct test1 
{ 
   unsigned int x: 5; 
   unsigned int y: 8; 
}; 
  
// A structure with forced alignment 
struct test2 
{ 
   unsigned int x: 5; //[0]...[3]
   unsigned int: 0; 
   unsigned int y: 8; //[4]...[8]
};
\end{lstlisting}
\end{example}
\begin{analysis}
前者相邻字段类型相同,紧邻存储,共13位,向上补齐为4字节(32位);
后者多了一个0位域,用来强制对齐边界,即x存完5位后强制对齐为32位,而后从第4个字节开始存储y,故一共8个字节。
\end{analysis}

\begin{example}
\begin{lstlisting}
struct test
{
    int a : 20;          // [0]...[2]
    char b : 3;          // [3]
    char c : 0;          // new [4]
    unsigned char c : 5; // [4],[5]
};
\end{lstlisting}
\end{example}
\begin{analysis}
\begin{itemize}
	\itemsep -3pt
	\item a占20位,向上补齐即3字节(24位)
	\item int和char不同类型,不压缩,新开一个字节存储
	\item 0位域相当于新开一个字节,用来强制对齐边界
	\item c占5位,向上补齐即2字节(8位)
	\item 由原则3,int的整数倍,故一共8字节
\end{itemize}
\end{analysis}

\begin{example}
\begin{lstlisting}
struct test
{
    int a : 1;
    int b : 2;
    inc c : 4;
    ind d : 4;
};
\end{lstlisting}
\end{example}
\begin{analysis}
a,b,c,d都可以被压缩存储,共占11位,向上对齐为4字节(32位)
\end{analysis}

关于位域的其他小知识
\begin{itemize}
	\itemsep -3pt
	\item 由第\ref{sec:introduction}部分提到的,现在的机器都采用字节编址,而位域的加入,使得数据的地址可能不是字节的整数倍,因而也就不能用指针进行访问,如\verb'&test.a'是被禁止的。
	\item 对于超出位域大小的赋值,编译器往往会报error或warning,如a字段仅1位,赋值\verb'a=2'会被警告
	\item 对于超出位域大小的声明,编译器同样会报error或warning,如\verb'int a : 33'是不被允许的
	\item 位域不能被声明为静态变量,也不能是数组
\end{itemize}

\section{总结}
上面所展示的内容很多都是与编译器具体实施相关的,记住基本原则,然后不要太较真就好。
遇到不确定的情况,最好自己写程序验证一下,在自己机器上跑跑就知道结果了。
面向32位机器的编译可以在编译时添加\verb'-m32'指令,面向64位机器则添加\verb'-m64',这样就可以在一台机子上同时验证两种情况。

\section{参考资料}
\begin{enumerate}
	\itemsep -3pt
	\item c/c++ struct的大小以及sizeof用法, \url{https://www.cnblogs.com/dingxiaoqiang/p/8059329.html}
	\item Bit field, \url{https://en.cppreference.com/w/cpp/language/bit_field}
	\item 位域, \url{https://baike.baidu.com/item/%E4%BD%8D%E5%9F%9F/9215688?fr=aladdin}
	\item Bit Fields in C, \url{https://www.geeksforgeeks.org/bit-fields-c/}
\end{enumerate}


\end{document}