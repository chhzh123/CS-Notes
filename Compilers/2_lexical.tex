% !TEX root = main.tex

\section{词法分析}
分离词法分析和语法分析可以简化这两个任务,同时提升编译器的性能与兼容性。

\subsection{基本定义}
\begin{definition}
令牌(token)是一个\underline{令牌名字}与\underline{可选属性值}构成的对;模式(pattern)描述了每个词素(lexeme)要遵循什么规则;而词素(最小意义单位)则是源程序中一连串满足模式的字母,作为令牌的实例化。
\end{definition}
\begin{example}
考虑C语句\\
 \qquad\qquad\verb'printf("Total = %d\n", score);'\\
其中\verb'printf'和\verb'score'是匹配(match)上令牌\textbf{id}模式的词素,而\verb'"Total = %d\n"'是匹配上字面值\textbf{literal}的词素。
\end{example}
简单来讲,令牌是一个更大的概念,是同类词素的集合。
比如一个令牌\textbf{comparison}的样例词素可以有\texttt{<=}和\texttt{!=}。

\begin{definition}[字母表与语言]
字母表(alphabet)$\Sigma$是有限符号(symbol)的集合,如ASCII就是一个字母表。
字符串(string)$s$是从字母表中抽取的有限符号的序列,$|s|$为字符串长度,$\epsilon$为空串。
语言(language)是字符串的可数集合。
\end{definition}
\begin{example}
字母表$\Sigma=\{0,1\}$,则$\{001,1001\}$和$\{\}$都是定义在$\Sigma$上的语言。
\end{example}

语言是一种集合,故集合运算也适用于语言。
\begin{table}[H]
\centering
\begin{tabular}{|c|c|}\hline
并集(union) & $L\cup M$\\\hline
连接(concatenation)/交集 & $LM$\\\hline
柯林闭包(Kleene closure) & $L^*=\cup_{i=0}^\infty L^i$\\\hline
正闭包(positive) & $L^+=\cup_{i=1}^\infty L^i$\\\hline
\end{tabular}
\end{table}

\subsection{正则表达式}
\begin{definition}[正则表达式(regular expression, regex)]
正则表达式$r$定义了语言$L(r)$,以递归形式定义:
\begin{enumerate}
	\item 奠基:
	\begin{itemize}
		\item $\epsilon$是正则表达式,即$L(\epsilon)=\{\epsilon\}$
		\item $a\in\Sigma$是正则表达式,即$L(\va)=\{a\}$(这里用斜体代表符号,粗体代表符号对应的正则表达式)
	\end{itemize}
	\item 推论:若$r$和$s$都是正则表达式给出了语言$L(r)$和$L(s)$,则
	\begin{itemize}
		\item $(r)|(s)$是正则表达式,表示$L(r)\cup L(s)$
		\item $(r)(s)$是正则表达式,表示$L(r)L(s)$
		\item $(r)^*$是正则表达式,表示$(L(r))^*$
		\item $(r)$是正则表达式,表示$L(r)$
	\end{itemize}
\end{enumerate}
正则表达式表示的语言叫做正规集。
\end{definition}

有以下运算规定:
\begin{itemize}
	\item 一元运算符${}^*$有最高优先级,左结合
	\item 连接优先级次之,左结合
	\item $|$优先级\textbf{最低},左结合
\end{itemize}

\begin{definition}[正则定义]
$d_i\to r_i$,其中$d_i$都是名字,且各不相同。
每个$r_i$是$\Sigma\cup\{d_1,\ldots,d_{i-1}\}$中符号上的正则表达式。
\end{definition}
\begin{example}
比如C语言的标识符可记为
\[\begin{aligned}
letter\_ &\to A|B|\cdots|Z|a|b|\cdots|z|\_\\
digit &\to 0|1|\cdots|9\\
id &\to letter\_(letter\_|digit)^*
\end{aligned}\]
\end{example}

正则表达式的拓展\footnote{更多可参见\href{https://regex101.com/}{Regex101}}:
\begin{itemize}
	\item $r^+$代表一个或多个
	\item $r?$代表零或一个
	\item $[a-z]$字母类
\end{itemize}