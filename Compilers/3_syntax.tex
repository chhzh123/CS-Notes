% !TEX root = main.tex

\section{语法分析}
\subsection{上下文无关法}
语法分析需要解决:从词法分析中获得的每个属性字(token)在语句中承担什么角色,同时检查语句是否符合程序语言的语法。

很多语言并非是正则的,比如匹配的括号串$\{(^i)^i\mid i\geq 0\}$,原因是FA不能记住其访问某一状态的次数,因此需要有更加强大的语言。

\begin{definition}[上下文无关法(context-free gramma, CFG)]
包括四部分
\begin{itemize}
	\item 终端符号(terminal)的集合$T$(即token名字)
	\item 非终端符号的集合$N$
	\item 唯一的开始符号$S\in N$
	\item 若干以下形式的产生式(production)
	\[X\to Y_1Y_2\ldots Y_n\]
	其中$X\in N$且$Y_i\in T\cup N\cup\{\epsilon\}$。
	多个左侧相同的产生式右侧可用$\mid$合并。
\end{itemize}
\end{definition}
\begin{definition}[推导(derivation)]
从开始符号开始,每一步推导就是用一个产生式的右方取代左端的非终端符号。
\end{definition}

CFG定义语言的能力比正则表达式强很大原因是它引入了\textcolor{red}{\textbf{递归}}的因素。

\begin{example}
用上下文无关文法定义下列语言:
\begin{itemize}
	\item $L=\{0^n1^n\mid n\geq 1\}$:$E\to 0E1\mid 01$
	\item 只含有$0$和$1$的回文串:$S\to 0S0\mid 1S1\mid 0\mid 1\mid \epsilon$
	\item 只含有$($和$)$的匹配括号串:$E\to (E)\mid EE\mid \epsilon$
\end{itemize}
\end{example}

\begin{itemize}
	\item 最左推导(left-most):每步推导都替换最左侧的非终端符号
	\[E\xRightarrow{lm}
	-E\xRightarrow{lm}
	-(E)\xRightarrow{lm}
	-(E+E)\xRightarrow{lm}
	-(id+E)\xRightarrow{lm}
	-(id+id)\]
	\item 最右推导(right-most):每步推导都替换最右侧的非终端符号
	\[E\xRightarrow{rm}
	-E\xRightarrow{rm}
	-(E)\xRightarrow{rm}
	-(E+E)\xRightarrow{rm}
	-(E+id)\xRightarrow{rm}
	-(id+id)\]
\end{itemize}

\begin{definition}[二义性]
如果对于一个文法,存在一个句子,对这个句子可以构造两棵不同的分析树,那么我们称这个文法为二义的。
\end{definition}
看语法分析树的叶子结点能不能连成句子。

\begin{example}
对于文法$E\to E+E\mid E*E\mid -E\mid (E)\mid id$及句子$id+id*id$,有以下两种推导:

\begin{minipage}{0.5\linewidth}
\[\begin{aligned}
E &\implies E+E\\
&\implies id+E\\
&\implies id+E*E\\
&\implies id+id*E\\
&\implies id+id*id
\end{aligned}\]
\end{minipage}
\begin{minipage}{0.5\linewidth}
\[\begin{aligned}
E &\implies E*E\\
&\implies E+E*E\\
&\implies id+E*E\\
&\implies id+id*E\\
&\implies id+id*id
\end{aligned}\]
\end{minipage}
\end{example}

文法二义性可通过引入更多的产生式来消除。
\begin{example}
$E\to E+E\mid E*E\mid (E)\mid id$是有二义的,因为不知道应该先算加法还是乘法。
可将其改为
\[\begin{aligned}
E &\to E+T\mid T\\
T &\to T*F\mid F\\
F &\to (E)\mid id
\end{aligned}\]
其中$E$为Expression,$T$为Term,$F$为Facotr,即可消除二义性(必然得先算乘法)。
相当于先算$F$,再算$T$,最后算$E$,强行添加了括号/优先级。
\end{example}
\begin{example}
悬挂的if-else:\verb'if E1 then if E2 then E3 else E4',可以令\verb'else'匹配最近的\verb'then'。
\[\begin{array}{rll}
E &\to MIF &\text{//所有的then都被匹配}\\
  &\mid UIF &\text{//仅有一些then}\\
MIF &\to \text{if } E \text{ then } MIF \text{ else } MIF &\\
  &\mid OTHER &\\
UIF &\to \text{if } E \text{ then } E &\\
  &\mid \text{if } E \text{ then } MIF \text{ else } UIF &
\end{array}\]
\end{example}

并不是所有上下文无关文法都可以做到无二义,也无法判断一个上下文无关文法是否是二义的。

\subsection{NFA转CFG}
\begin{enumerate}
	\item 对于NFA的每一状态$i$,创建非终态$A_i$
	\item 若状态$i$在输入$a$上有转换边到状态$j$,则添加生成式$A_i\to aA_j$;
	若状态$i$在输入$\epsilon$上转换到状态$j$,则添加生成时$A_i\to A_j$
	\item 若$i$是接受状态,则添加$A_i\to\epsilon$
	\item 若$i$是初始状态,则令$A_i$为语法的初始符号
\end{enumerate}

\begin{definition}[右线性文法]
如果每个产生式都属于下列形式之一
\[A\to aB\qquad A\to a\qquad A\to\epsilon\]
则这样的文法称为右线性文法
\end{definition}
\begin{definition}[左线性文法]
如果每个产生式都属于下列形式之一
\[A\to Ba\qquad A\to a\qquad A\to\epsilon\]
则这样的文法称为左线性文法
\end{definition}
\begin{theorem}
正则表达式/NFA/DFA与左/右线性文法得表达能力是等价的
\end{theorem}

在处理程序时,上下文无法文法存在局限性,无法解决诸如以下问题:
\begin{itemize}
\item 变量先声明,再使用
\item 调用函数时,实参个数和形参个数一致
\end{itemize}
都得留到语义分析阶段才解决。

\subsection{递归下降}
递归下降语法翻译即从顶层的非终端符号$E$开始,顺序尝试$E$的所有规则,不断回溯遍历,完整例子可见\href{http://web.stanford.edu/class/cs143/lectures/lecture06.pdf}{此文档},但需要先消除左递归。

\begin{definition}[左递归]
对于非终端符号$A$有生成式$A\to A\alpha$,则该文法是左递归的。
\end{definition}
\begin{example}
消除左递归的方法:先把单元素拎出来放左侧,然后把所有递归移至右侧
\[A\to A\alpha_1\mid \cdots\mid A\alpha_m\mid\beta_1\mid\cdots\beta_n
\implies
\begin{aligned}
A &\to\beta_1 A'\mid\cdots\mid\beta_n A'\\
A' &\to\alpha_1 A'\mid\cdots\mid\alpha_m A'\mid\textcolor{red}{\epsilon}
\end{aligned}\]
如果是多级左递归,则需要先将上级生成式代入到中间级生成式中,再做消除
\[\begin{aligned}
S &\to Aa\mid b\\
A &\to Ac\mid Sd\mid \epsilon
\end{aligned}\]
将下式改写为$A \to Ac\mid Aad\mid bd\mid \epsilon$,进而可消除左递归
\[\begin{aligned}
S &\to Aa\mid b\\
A &\to bdA'\mid A'\\
A' &\to cA'\mid adA'\mid \epsilon
\end{aligned}\]
\end{example}

\begin{definition}[提取左因子(left-factoring)]
将生成式右侧左部相同的因子部分提取出来,找最长前缀
\[A\to\alpha\beta_1\mid\alpha\beta_2\mid\cdots\alpha\beta_n\mid\gamma\]
$\gamma$代表所有不以$\alpha$开始的生成式,提取左因子则得到(将$\alpha$拿出来)
\[\begin{aligned}
A &\to \alpha A'\mid\gamma\\
A' &\to \beta_1\mid\beta_2\mid\cdots\beta_n
\end{aligned}\]
直至没有生成式有相同前缀
\end{definition}

\begin{algorithm}[H]
\centering
\caption{递归下降(top-down parser)}
\begin{algorithmic}[1]
\State 选择$A$生成式$A\to X_1X_2\cdots X_k$
\For{$i=1$ to $k$}
\If{$X_i$是非终端符号}
\State 调用$X_i()$
\Else\If{$X_i$等于当前的输入符号$a$}
\State 读取下一输入符号
\Else
\State $error()$
\EndIf
\EndIf
\EndFor
\end{algorithmic}
\end{algorithm}

\begin{definition}[FIRST集与FOLLOW集]
$FIRST(\alpha)$集为从$\alpha$中推导出来的字符串第一个终端符号的集合,若$\alpha\to\epsilon$,则$\epsilon\in FIRST(\alpha)$;若$A\to c\gamma$,则$c\in FIRST(A)$。
$FOLLOW(A)$集为可以出现在$A$右侧的终端符号的集合。
若$A$是最右端的符号,则字符串结束符号$\$\in FOLLOW(A)$。
\end{definition}

\begin{myalgorithm}
计算$FIRST(X)$集
\begin{enumerate}
	\item 如果$X$是终端符号,则$FIRST(X)=\{X\}$
	\item 如果$X$是非终端符号,且$X\to Y_1Y_2\cdots Y_k$。
\begin{itemize}
	\item 若$Y_1\cdots Y_{i-1}\to\epsilon$,则将$a\in FIRST(Y_i)$放入$FIRST(X)$。
	\item 若$\epsilon\in FIRST(Y_j),j=1,2,\ldots,k$,则将$\epsilon$放入$FIRST(X)$中。
\end{itemize}
	\item 若$X\to\epsilon$是生成式,将$\epsilon$放入$FIRST(X)$中
\end{enumerate}
\end{myalgorithm}

\begin{myalgorithm}
计算$FOLLOW(A)$集
\begin{enumerate}
	\item 将$\$$放入$FOLLOW(S)$,其中$S$是开始符号
	\item 如果有生成式$A\to\alpha B\beta$,那么$\forall a\in FIRST(\beta),a \ne\epsilon:\;a\in FOLLOW(B)$
	\item 如果有生成式$A\to\alpha B$,或生成式$A\to\alpha B\beta$,且$\epsilon\in FIRST(\beta)$,则$\forall a\in FOLLOW(A):\;a\in FOLLOW(B)$
\end{enumerate}
\end{myalgorithm}

简而言之,$FOLLOW$集看下一符号的$FIRST$,\textbf{如果$\epsilon$在下一符号的$FIRST$集中,则看生成式左端的$FOLLOW$集}。

另一种方式:
\begin{enumerate}
	\item \$$\in FOLLOW(S)$
	\item $\forall A\to\alpha X\beta:\;FIRST(\beta)-\{\epsilon\}\subset FOLLOW(X)$
	\item $\forall A\to\alpha X\beta,\epsilon\in FIRST(\beta):\;FOLLOW(A)\subset FOLLOW(X)$
\end{enumerate}

\begin{figure}[H]
\centering
\begin{tabular}{c}
\includegraphics[width=0.8\linewidth]{fig/follow_eg1.jpg}\\
\includegraphics[width=0.8\linewidth]{fig/follow_eg2.jpg}
\end{tabular}
\end{figure}

\begin{myalgorithm}[构造预测语法表]
对于每一生成式$A\to\alpha$,
\begin{enumerate}
	\item 对于每一终端符号$a\in FIRST(\textcolor{red}{\alpha})$,将$A\to\alpha$添加到$M[A,a]$中。
	\item 若$\epsilon\in FIRST(\textcolor{red}{\alpha})$,则对$b\in FOLLOW(A)$,将$A\to\alpha$添加到$M[A,b]$中($\beta$可以为\$)。
\end{enumerate}
注意是生成式右端。
\end{myalgorithm}

\begin{example}
考虑下面语法
\[\begin{array}{lrl}
(1) & E &\to T E'\\
(2) & E' &\to + T E'\mid\epsilon\\
(3) & T &\to F T'\\
(4) & T' &\to * F T'\mid\epsilon\\
(5) & F &\to (E) \mid id
\end{array}\]
计算$FIRST$集
\begin{itemize}
	\item 从终端符号多的开始(5),$FIRST(F)=\{(,id\}$
	\item 向上找含$F$的生成式(3),$FIRST(T)=FIRST(F)$
	\item 向上找含$T$的生成式(1),$FIRST(E)=FIRST(T)$
	\item $FIRST(E')=\{+,\epsilon\}$
	\item $FIRST(T')=\{*,\epsilon\}$
\end{itemize}
计算$FOLLOW$集
\begin{itemize}
	\item 从起始符号开始(1),注意(5)也有$E$,故$FOLLOW(E)=\{),\$\}$
	\item $E'$只出现在$E$的生成式末尾,因此$FOLLOW(E')=FOLLOW(E)=\{),\$\}$
	\item $FOLLOW(T)\subset FIRST(E')=\{+,\epsilon\}$,由于$\epsilon\in FIRST(E')$,故$FOLLOW(T)\subset FOLLOW(E)=\{),\$\}$,即$FOLLOW(T)=\{+,),\$\}$
	\item $FOLLOW(F)\subset FIRST(T')=\{*,\epsilon\}$,由于$\epsilon\in FIRST(T')$,故$FOLLOW(F)\subset FOLLOW(T)=\{+,),\$\}$,即$FOLLOW(F)=\{*,+,),\$\}$
	\item $T'$只出现在$T$的生成式末尾,因此$FOLLOW(T')=FOLLOW(T)=\{*,+,),\$\}$
\end{itemize}
对应有语法预测表
\begin{center}
\begin{tabular}{|c|c|c|c|c|c|c|}\hline
 & $id$ & $+$ & $*$ & $($ & $)$ & \$\\\hline
$E$ & $E\to TE'$ & & & $E\to TE'$ & &\\\hline
$E'$ & & $E'\to +TE'$ & & & $E'\to\epsilon$ & $E'\to\epsilon$\\\hline
$T$ & $T\to FT'$ & & & $T\to FT'$ & &\\\hline
$T'$ & & $T'\to\epsilon$ & $T'\to*FT'$ & & $T'\to\epsilon$ & $T'\to\epsilon$\\\hline
$F$ & $F\to id$ & & & $F\to(E)$ & &\\\hline
\end{tabular}
\end{center}
\end{example}

\subsection{自顶向下分析}
\begin{definition}[LL(1)文法\protect\footnote{第一个L代表输入字符串从左边开始扫描,第二个L代表得到的推导是最左推导,(1)代表向前看1个输入符号(或单词)}]
语法$G$是LL(1)文法当且仅当对于任意$A\to\alpha\mid\beta$为$G$两个不同的生成式,满足
\begin{enumerate}
	\item $\alpha$和$\beta$不会同时推导出由同一终端符号$a$开始的字符串
	\item $\alpha$和$\beta$中至多一个能获得空字串
	\item 若$\beta\to\epsilon$,则$\alpha$不能推出任何以$FOLLOW(A)$中终端符号开始的字符串;
	同样地,若$\alpha\to\epsilon$,则$\beta$不能推出任何以$FOLLOW(A)$中终端符号开始的字符串
\end{enumerate}
前两个条件等价于$FIRST(\alpha)\cap FIRST(\beta)=\varnothing$,第三个条件等价于若$\epsilon\in FIRST(\beta)$,则$FIRST(\alpha)\cap FOLLOW(A)=\varnothing$,反之同理
\end{definition}
通常没有左递归、无歧义的语法可以是LL(1)。

递归下降在每一步都会有多种生成式的选择,这会导致大量的回溯。
而在LL(1)文法中,每一步都只有一种生成式的选择,避免了回溯。

左因子分解(left-factoring)将生成式的\textbf{共同前缀}分解出来。
\begin{example}
考虑以下文法
\[\begin{aligned}
E &\to T + E\mid T\\
T &\to int \mid int * T\mid (E)
\end{aligned}\]
共同前缀分解后即得
\[\begin{aligned}
E &\to TX\\
X &\to +E\mid \epsilon\\
T &\to int\;Y\mid (E)\\
Y &\to *T\mid \epsilon
\end{aligned}\]
有LL(1)语法表,其中最左列为最左非终端符号,最上行为下一输入符号,表格内容为使用的右端生成式。
\begin{center}
\begin{tabular}{|c|c|c|c|c|c|c|}\hline
 & $int$ & $*$ & $+$ & $($ & $)$ & \$\\\hline
$E$ & $TX$ & & & $TX$ & & \\\hline
$X$ &  &  & $+E$ & & $\epsilon$ & $\epsilon$\\\hline
$T$ & $int\; Y$ & & & $(E)$ & &\\\hline
$Y$ &  & $*T$ & $\epsilon$ & & $\epsilon$ & $\epsilon$\\\hline
\end{tabular}
\end{center}
\end{example}

\begin{example}
经典的二义if-else语法(已提取左因子)
\[\begin{aligned}
S &\to iEtSS'\mid a\\
S' &\to eS\mid\epsilon\\
E &\to b
\end{aligned}\]
可以求得
\[\begin{aligned}
FIRST(S) &= \{i,a\}\\
FIRST(S') &= \{\epsilon,e\}\\
FIRST(E) &= b\\
FOLLOW(S) &= \{\$\} + FIRST(S')-\{\epsilon\}=\{\$,e\}\\
FOLLOW(S') &= \{\$\} + FOLLOW(S) = \{\$,\textcolor{red}{e}\}\\
FOLLOW(E) &= \{\$,t\}
\end{aligned}\]
因为$\epsilon\in FIRST(S'\to\epsilon)$里,而$FIRST(S'\to eS)\cap FOLLOW(S')=\{e\}\ne\varnothing$,所以不是LL(1)文法。
\end{example}

基于表的预测语法分析,用栈实现。
\begin{algorithm}
\caption{Table-Driven Predictive Parsing}
\begin{algorithmic}[1]
\State \verb'ip=0'
\State \verb'X=stack.top()'
\While {$X\ne\$$}
\If {X == w[ip]}
\State \verb'stack.pop(); ip++;'
\Else \If {X is a terminal or M[X,a]=$\varnothing$}
\State Error()
\Else
\State Output production \verb'M[X,a]'$=X\to Y_1Y_2\cdots Y_k$
\State stack.pop()
\State push $Y_k,Y_{k-1},\ldots,Y_1$ onto the stack
\EndIf
\EndIf
\State \verb'X=stack.top()'
\EndWhile
\If {w[ip] != '\$'}
\State Error()
\EndIf
\end{algorithmic}
\end{algorithm}

\begin{example}
考虑以下文法:
\[\begin{aligned}
S &\to (L) \mid a\\
L &\to L, S \mid S
\end{aligned}\]
\begin{enumerate}
	\item 消除文法的左递归.
	\item 构造文法的LL(1)分析表.
	\item 对于句子$(a, (a, a))$,给出语法分析的详细过程(参照课本228页的图4.21).
\end{enumerate}
\end{example}
\begin{analysis}
\begin{enumerate}
	\item 如下
	\[\begin{aligned}
	S &\to (L)\mid a\\
	L &\to SL'\\
	L' &\to , SL'\mid\epsilon
	\end{aligned}\]
	\item 先求出FIRST集和FOLLOW集(由于文法中存在逗号,故将字符用引号括起来以示区分)
	\[\begin{array}{rlrl}
	FIRST(S) &= \{'(','a'\} & FOLLOW(S) &= \{'\$',')'\}\\
	FIRST(L) &= \{'(','a'\} & FOLLOW(L) &= \{')'\}\\
	FIRST(L') &= \{',',\epsilon\} & FOLLOW(L') &= \{')'\}
	\end{array}\]
	LL(1)分析表如下,其中第一列为非终端符号,第一行为输入符号.
	\begin{center}
	\begin{tabular}{|c|c|c|c|c|c|}\hline
	 & $($ & $)$ & $a$ & $,$ & \$\\\hline
	S & $S\to(L)$ & & $S\to a$ & &\\\hline
	L & $L\to SL'$ & & $L\to SL'$ & &\\\hline
	L' & & $L'\to\epsilon$ & & $L'\to,SL'$ & \\\hline
	\end{tabular}
	\end{center}
	\item 语法分析过程如下
	\begin{center}
	\begin{tabular}{|l|r|r|l|}\hline
	Matched & Stack & Input & Action\\\hline
	 & $S\$$ & $(a,(a,a))\$$ & \\\hline
	 & $(L)\$$ & $(a,(a,a))\$$ & output $S\to (L)$\\\hline
	$($ & $L)\$$ & $a,(a,a))\$$ & \\\hline
	$($ & $SL')\$$ & $a,(a,a))\$$ & output $L\to SL'$\\\hline
	$($ & $aL')\$$ & $a,(a,a))\$$ & output $S\to a$\\\hline
	$(a$ & $L')\$$ & $,(a,a))\$$ & \\\hline
	$(a$ & $,SL')\$$ & $,(a,a))\$$ & output $L'\to ,SL'$\\\hline
	$(a,$ & $SL')\$$ & $(a,a))\$$ & \\\hline
	$(a,$ & $(L)L')\$$ & $(a,a))\$$ & output $S\to (L)$\\\hline
	$(a,($ & $L)L')\$$ & $a,a))\$$ & \\\hline
	$(a,($ & $SL')L')\$$ & $a,a))\$$ & output $L\to SL'$\\\hline
	$(a,($ & $aL')L')\$$ & $a,a))\$$ & output $S\to a$\\\hline
	$(a,(a$ & $L')L')\$$ & $,a))\$$ & \\\hline
	$(a,(a$ & $,SL')L')\$$ & $,a))\$$ & output $L'\to,SL'$\\\hline
	$(a,(a,$ & $SL')L')\$$ & $a))\$$ & \\\hline
	$(a,(a,$ & $aL')L')\$$ & $a))\$$ & output $S\to a$\\\hline
	$(a,(a,a$ & $L')L')\$$ & $))\$$ & \\\hline
	$(a,(a,a$ & $)L')\$$ & $))\$$ & output $L'\to\epsilon$\\\hline
	$(a,(a,a)$ & $L')\$$ & $)\$$ & \\\hline
	$(a,(a,a)$ & $)\$$ & $)\$$ & output $L'\to\epsilon$\\\hline
	$(a,(a,a))$ & $\$$ & $\$$ & \\\hline
	\end{tabular}
	\end{center}
\end{enumerate}
\end{analysis}

\subsection{自底向上分析}
自底向上的语法分析采用两种动作:
\begin{itemize}
\item 移进(shift):将\verb'|'向右移动一格
\[ABC\mid xyz\implies ABCx\mid yz\]
\item 规约(reduce):在字符串右侧逆向应用生成式
\[Cbxy\mid ijk\implies CbA\mid ijk\]
\end{itemize}

\begin{figure}[H]
\centering
\includegraphics[width=0.8\linewidth]{fig/shift-reduce.jpg}
\end{figure}
移进将终端符号移入栈中,规约将生成式的右端符号弹出,将生成式的左端\textbf{非终端}符号推入。

\begin{definition}[句柄(handle)]
A handle is a string that can be \textbf{reduced} and also allows further reductions back to the start symbol.
可以理解为当前正在处理的token,用于\textbf{规约}而不是移进。
\end{definition}

\begin{definition}[活前缀(viable prefix)]
$\alpha$是活前缀若存在$\omega$使得$\alpha\mid\omega$是移进-规约语法分析器的状态。
\end{definition}

LR分析是最通用的\textbf{非回溯}移进-规约语法解析方法,难点在于构建分析表太过麻烦,但Yacc等工具可辅助构建。

\subsubsection{LR(0)语法}
为构建规范(canoical)LR(0)项,定义增量语法$G'$有生成式$S'\to S$,其中$S$为原语法$G$的开始符号。
这个生成式用于告知parser接受(accept)输入并停止解析,即接受仅发生在要对$S'\to S$进行规约的时候。

\begin{definition}[CLOSURE]
若$I$为语法$G$项的集合,则$CLOSURE(I)$为从$I$中构造出项的集合:
\begin{enumerate}
	\item 初始时,在$I$中的每一项都会被加到$CLOSURE(I)$中
	\item 若$A\to\alpha\cdot B\beta$在$CLOSURE(I)$中且$B\to\beta$是一个生成式,则将项$B\to\cdot\gamma$加入到$CLOSURE(I)$中;重复使用此规则,直至没有新项可以被加入
\end{enumerate}
\end{definition}
\begin{definition}[GOTO]
$I$是项的集合,$X$为输入符号,$GOTO(I,X)$为所有项$[A\to\alpha X\cdot\beta]$的闭包使得$[A\to\alpha\cdot X\beta]$在$I$中
\end{definition}

\begin{definition}[核项(kernel item)]
初始项$S'\to\cdot S$及所有$\cdot$不在左端的项称为核项,
除了初始项外所有$\cdot$在左端的项称为非核项(即那些新加入闭包的项)
\end{definition}

LR(0)语法:
\begin{itemize}
	\item 栈包含$\alpha$,下一输入是$t$,DFA在输入$\alpha$上终止在状态$s$
	\item 当$s$包含$X\to\beta$的项时进行规约(没有得移进就规约,自动机无对应符号出边)
	\item 当$s$包含$X\to\beta.t\omega$的项时移进
\end{itemize}

LR(0)可能存在以下两种冲突:
\begin{itemize}
	\item 规约-规约冲突:$X\to\beta.$且$Y\to\omega.$
	\item 移进-规约冲突:$X\to\beta.$且$Y\to\omega.t\delta$
\end{itemize}

\subsubsection{SLR分析}
SLR(simple left-to-right scan)\footnote{或SLR(1)分析,通常省略(1)。用的是LR(0)项,但是在语法分析时才前看1个输入符号.}用启发式算法提升了LR(0)移进规约的效率,减少冲突。
\begin{itemize}
	\item 栈包含$\alpha$,下一输入是$t$,DFA在输入$\alpha$上停在状态$s$
	\item 当$s$包含$X\to\beta$的项且\textcolor{red}{$t\in FOLLOW(X)$}时进行$X\to\beta$规约
	\item 当$s$包含$X\to\beta.t\omega$的项时移进
\end{itemize}

\begin{myalgorithm}[SLR(1)分析表]
构造$\mathcal{C}=\{I_0,I_1,\ldots,I_n\}$为LR(0)项的集合$G'$
\begin{enumerate}
	\item 若$[A\to\alpha\cdot a \beta]\in I_i$且$GOTO(I_i,a)=I_j$,则设$ACTION[i,a]$为移进$j$,
	\item 若$[A\to\alpha\cdot]\in I_i$,则$\forall a\in FOLLOW(A)$,设$ACTION[i,a]$为规约$A\to\alpha$
	\item 若$[S'\to S\cdot]\in I_i$,则设$ACTION[i,\$]$为接受(ACC)
\end{enumerate}
若上述有冲突的动作,则该文法不是SLR(1)的。

依照分析表可以得到语法分析的算法
\begin{itemize}
	\item 若$ACTION[s,a]$为移进$t$,则将$t$\textbf{推入}栈中
	\item 若$ACTION[s,a]$为\textbf{规约$A\to\beta$},则将\textcolor{red}{$|\beta|$个}符号从栈顶\textbf{弹出},令$t$为\textbf{栈顶符号},\textbf{将$GOTO[t,A]$推入栈中},输出规约$A\to\beta$
	\item 若$ACTION[s,a]=acc$,则语法解析结束
\end{itemize}
\end{myalgorithm}

\begin{example}
考虑以下文法:
\[\begin{array}{rrll}
(1) & E &\to & E+T\\
(2) & E &\to & T\\
(3) & T &\to & TF\\
(4) & T &\to & F\\
(5) & F &\to & F^*\\
(6) & F &\to & a\\
(7) & F &\to & b
\end{array}\]
\begin{enumerate}
	\item 写出每个非终端符号的FIRST集和FOLLOW集.
	\item 构造识别这一文法所有活前缀(viable prefixes)的LR(0)自动机(参照课本4.6.2节图4.31).
	\item 构造这一文法的SLR分析表(参照课本4.6.3节图4.37).
	\item 给出SLR分析器识别输入串$a+ab^*$的过程(参照课本4.6.4节图4.38)
\end{enumerate}
\end{example}
\begin{analysis}
\begin{enumerate}
	\item $FIRST$集和$FOLLOW$集如下
	\[\begin{array}{rlrl}
	FIRST(E) &= \{a,b\} & FOLLOW(E) &=\{\$,+\}\\
	FIRST(T) &= \{a,b\} & FOLLOW(T) &=\{\$,+,a,b\}\\
	FIRST(F) &= \{a,b\} & FOLLOW(F) &=\{\$,+,*,a,b\}\\
	\end{array}\]
	\item 构造增广语法$E'\to E$,并得到LR(0)自动机如下
	\begin{figure}[H]
	\centering
	\includegraphics[width=0.8\linewidth]{fig/T06.pdf}
	\end{figure}
	\item 依据上述两问结果,可构造SLR分析表如下(s后面的数字为DFA状态编号,r后面的数字为生成式的编号)
	\begin{center}
	\begin{tabular}{|c|ccccc|ccc|}\hline
	\multirow{2}{*}{STATE} & \multicolumn{5}{c|}{ACTION} & \multicolumn{3}{c|}{GOTO}\\\cline{2-9}
	  & a  & b  & +  & *  & \$  & E & T & F \\\hline
	0 & s4 & s5 &    &    &     & 1 & 2 & 3 \\\hline
	1 &    &    & s6 &    & ACC &   &   &   \\\hline
	2 & s4 & s5 & r2 &    & r2  &   &   & 7 \\\hline
	3 & r4 & r4 & r4 & s8 & r4  &   &   &   \\\hline
	4 & r6 & r6 & r6 & r6 & r6  &   &   &   \\\hline
	5 & r7 & r7 & r7 & r7 & r7  &   &   &   \\\hline
	6 & s4 & s5 &    &    &     &   & 9 & 3 \\\hline
	7 & r3 & r3 & r3 & s8 & r3  &   &   &   \\\hline
	8 & r5 & r5 & r5 & r5 & r5  &   &   &   \\\hline
	9 & s4 & s5 & r1 &    & r1  &   &   & 7 \\\hline
	\end{tabular}
	\end{center}
	\item 依上述ACTION-GOTO表,可得以下过程
	\begin{center}
	\begin{tabular}{|r|l|l|r|l|}\hline
		& STACK & SYMBOLS & INPUT      & ACTION\\\hline
	(1) & 0     &         & $a+ab^*\$$ & $[0,a]s4$\\\hline
	(2) & 04    & $a$     & $+ab^*\$$  & $[4,a]r6\;F\to a$, $[0,F]s3$\\\hline
	(3) & 03    & $F$     & $+ab^*\$$  & $[3,+]r4\;T\to F$, $[0,T]s2$\\\hline
	(4) & 02    & $T$     & $+ab^*\$$  & $[2,+]r2\;E\to T$, $[0,E]s1$\\\hline
	(5) & 01    & $E$     & $+ab^*\$$  & $[1,+]s6$\\\hline
	(6) & 016    & $E+$     & $ab^*\$$  & $[6,a]s4$\\\hline
	(7) & 0164    & $E+a$     & $b^*\$$  & $[4,b]r6\;F\to a$, $[6,F]s3$\\\hline
	(8) & 0163    & $E+F$     & $b^*\$$  & $[3,b]r4\;T\to F$, $[6,T]s9$\\\hline
	(9) & 0169    & $E+T$     & $b^*\$$  & $[9,b]s5$\\\hline
	(10) & 01695  & $E+Tb$    & ${}^*\$$ & $[5,*]r7\;F\to b$, $[9,F]s7$\\\hline
	(11) & 01697  & $E+TF$    & ${}^*\$$ & $[7,*]s8$\\\hline
	(12) & 016978 & $E+TF^*$    & $\$$ & $[8,\$]r5\;F\to F^*$, $[\textcolor{red}{9},F]s7$\\\hline
	(13) & 01697  & $E+TF$    & $\$$ & $[7,\$]r3\;T\to TF$, $[\textcolor{red}{6},T]s9$\\\hline
	(14) & 0169   & $E+T$     & $\$$ & $[9,\$]r1\;E\to E+T$, $[\textcolor{red}{0},E]s1$\\\hline
	(15) & 01     & $E$     & $\$$ & $[1,\$]ACC$\\\hline
	\end{tabular}
	\end{center}
\end{enumerate}
\end{analysis}

\begin{example}
下面的文法并非SLR(1)
\[\begin{aligned}
S &\to L=R\mid R\\
L &\to *R\mid id\\
R &\to L
\end{aligned}\]
对于$I_2$项:
\[\begin{aligned}
S &\to L\cdot=R\\
R &\to L\cdot
\end{aligned}\]
有$ACTION[2,=]$是移进,但$FOLLOW(R)=\{\$,=\}$又会导致在$=$上进行规约,故移进-规约冲突,该文法不是SLR(1)的
\end{example}

\subsubsection{LR(1)语法}
\begin{definition}[LR(1)项]
$[A\to\alpha\cdot\beta,a]$,其中$A\to\alpha\cdot\beta$为该项的核(core),$A\to\alpha\beta$为生成式,$a$是终端符号或\$,1指项中第二个元素的长度,$a$也被称为前看(lookahead)。
只有当LR(1)项有$[A\to\alpha\cdot,a]$的形式,\textbf{且下一输入符号为$a$时,才会用$A\to\alpha$进行规约}(这在构造解析表时会用到)。
$a$总会是$FOLLOW(A)$的子集,但往往是真子集。
\end{definition}

\begin{myalgorithm}[计算$CLOSURE(I)$]
对于每一$[A\to\alpha\cdot B\beta,a]\in I$,每一$G'$中的生成式$B\to\gamma$,$\forall b\in FIRST(\textcolor{red}{\beta a})$,将$[B\to\cdot\gamma,b]$加入$I$中。

初始化$\mathcal{C}=\{CLOSURE(\{[S'\to\cdot S,\$]\})\}$。
\end{myalgorithm}
\begin{example}
考虑下面的增量语法
\[\begin{array}{rrl}
(1) & S' &\to S\\
(2) & S &\to C C\\
(3) & C &\to c C\mid d
\end{array}\]
有$FIRST(C)=\{c,d\}$,可以构造得下面的LR(1)自动机。
以$[S\to\cdot CC,\$]$为例,考虑$FIRST(C\$)=\{c,d\}$,故闭包会新增四项$[C\to\cdot cC,c]$、$[C\to\cdot cC,d]$、$[C\to\cdot d,c]$、$[C\to\cdot d,d]$。
\begin{figure}[H]
\centering
\includegraphics[width=0.8\linewidth]{fig/LR1-eg.jpg}
\end{figure}
类似地有规范LR(1)解析表
\begin{center}
\begin{tabular}{|c|c|c|c|c|c|}\hline
  & c  & d  & \$  & S & C \\\hline
0 & s3 & s4 &     & 1 & 2 \\\hline
1 &    &    & acc &   &   \\\hline
2 & s6 & s7 &     &   & 5 \\\hline
3 & s3 & s4 &     &   & 8 \\\hline
4 & r3 & r3 &     &   &   \\\hline
5 &    &    & r1  &   &   \\\hline
6 & s6 & s7 &     &   & 9 \\\hline
7 &    &    & r3  &   &   \\\hline
8 & r2 & r2 &     &   &   \\\hline
9 &    &    & r2  &   &   \\\hline
\end{tabular}
\end{center}
\end{example}

LR(0)=SLR(1)的表项较少,但LR(1)的表项就指数级上涨。

\subsubsection{LALR分析}
LALR(Lookahead LR)在实践中很常用,表大小通常远小于规范LR表。
核心思想是将LR(1)中具有\textbf{相同核}的表项合并。

\begin{example}
将上面例子中的$I_3$和$I_6$合并得到
\[\begin{aligned}
C &\to c\cdot C, c/d/\$\\
C &\to \cdot cC, c/d/\$\\
C &\to \cdot d, c/d/\$
\end{aligned}\]
$I_4$和$I_7$合并得到
\[\begin{aligned}
C &\to d\cdot, c/d/\$
\end{aligned}\]
$I_8$和$I_9$合并得到
\[\begin{aligned}
C &\to cC\cdot, c/d/\$
\end{aligned}\]
进而有LALR解析表
\begin{center}
\begin{tabular}{|c|c|c|c|c|c|}\hline
   & c   & d   & \$  & S & C \\\hline
0  & s36 & s47 &     & 1 & 2 \\\hline
1  &     &     & acc &   &   \\\hline
2  & s36 & s47 &     &   & 5 \\\hline
36 & s36 & s47 &     &   & 89\\\hline
47 & r3  & r3  & r3  &   &   \\\hline
5  &     &     & r1  &   &   \\\hline
89 & r2  & r2  & r2  &   &   \\\hline
\end{tabular}
\end{center}
\end{example}

合并LR(1)项不会导致新的移进-规约冲突,但可能会产生新的\textbf{规约-规约冲突}。

\begin{example}
证明下列文法
\[\begin{aligned}
S &\to Aa \mid bAc \mid dc \mid bda\\
A &\to d
\end{aligned}\]
是LALR(1)文法但不是SLR(1)文法.
\end{example}
\begin{analysis}
构造增广文法$S'\to\cdot S$,$FIRST(\epsilon\$)=\$$,$FIRST(a\$)=a$,可以得到$I_0$。
\begin{figure}[H]
\centering
\includegraphics[width=0.8\linewidth]{fig/T07-1.pdf}
\end{figure}
由上图知,没有相同核心(core)的状态,因此不需要合并,从而LALR分析表不冲突,该文法是LALR(1)文法。

又有$FOLLOW(A)=\{a,c\}$,考虑图中的状态$I4$,当输入符号为$c$时,$c\in FOLLOW(A)$,既有移进$S\to d\cdot c$,又有归约$S\to d\cdot$,因此SLR分析表有冲突,该文法不是SLR(1)文法。
\end{analysis}
\begin{example}
证明下列文法
\[\begin{aligned}
S &\to Aa \mid bAc \mid Bc \mid bBa\\
A &\to d\\
B &\to d
\end{aligned}\]
是LR(1)文法但不是LALR(1)文法.
\end{example}
\begin{analysis}
构造增广文法$S'\to\cdot S$,$FIRST(a\$)=a$,$FIRST(c\$)=c$,可以得到状态$I_0$。
\begin{figure}[H]
\centering
\includegraphics[width=0.8\linewidth]{fig/T07-2.pdf}
\end{figure}
由上面的DFA知LR分析表没有冲突,因此该文法是LR(1)文法。

但如果将图中相同核心的状态$I4$和$I9$合并,会有
\[\begin{aligned}
A &\to d\cdot,a/c\\
B &\to d\cdot,a/c
\end{aligned}\]
即出现了规约-规约冲突,因此该文法不是LALR(1)文法。
\end{analysis}

可以在解析表(parsing table)层面来解决语法的二义性,即选择移进/规约的特定操作。

\subsection{语法制导翻译}
抽象语法树(Abstract Syntax Trees, AST)是将原本语法树中冗余的成分给去除,比如左右括号原本都是各自一个结点,但在AST中不会呈现。

语法制导翻译(syntax-directed translation)给语法符号提供了属性(attribute),给生成式提供了动作(action)。
\begin{example}
对下列语法进行求值
\[E\to int\mid E+E\mid (E)\]
有语法制导定义
\begin{center}
\begin{tabular}{ll}
$E\to int$ & $E.val=int.val$\\
$E\to E_1+E_2$ & $E.val=E_1.val+E_2.val$\\
$E\to (E_1)$ & $E.val=E_1.val$
\end{tabular}
\end{center}
\end{example}

可以在AST上标注(annotate)两种属性:
\begin{itemize}
\item 继承属性(inherited):从语法树的父亲或兄弟中计算得到
\item 综合属性(synthesized):从后代计算得到
\end{itemize}

\forestset{
sn edges/.style={for tree={edge={-}}}
}

\begin{example}
考虑以下语法制导定义:
\begin{center}
\begin{tabular}{|l|l|}\hline
\multicolumn{1}{|c|}{语法规则} & \multicolumn{1}{c|}{语义规则}\\\hline
$S \to ABCD$ & $S.val = A.val + B.val + C.val + D.val$\\\hline
$A \to gBa$ & $A.val = B.val * 5$\\\hline
$B \to B_1b$ & $B.val = B_1.val * 2$\\\hline
$B \to b$ & $B.val = 2$\\\hline
$C \to C_1c$ & $C.val = C_1.val * 3$\\\hline
$C \to c$ & $C.val = 3$\\\hline
$D \to d$ & $D.val = 1$\\\hline
\end{tabular}
\end{center}
对于输入串$gbbabbccd$构造带注释的分析树(annotated parse tree).
\end{example}
\begin{analysis}
带注释的分析树如下
\begin{center}
\begin{forest}
sn edges
[{$S.val=20+4+9+1=34$}
	[{$A.val=4*5=20$}
		[{$\mathbf{g}$}]
		[{$B.val=2*2=4$}
			[{$B.val=2$}
				[{$\mathbf{b}$}]
			]
			[{$\mathbf{b}$}]
		]
		[{$\mathbf{a}$}]
	]
	[{$B.val=2*2=4$}
		[{$B.val=2$}
			[{$\mathbf{b}$}]
		]
		[{$\mathbf{b}$}]
	]
	[{$C.val=3*3=9$}
		[{$C.val=3$}
			[{$\mathbf{c}$}]
		]
		[{$\mathbf{c}$}]
	]
	[{$D.val=1$}
		[{$\mathbf{d}$}]
	]
]
\end{forest}
\end{center}
\end{analysis}

\begin{example}
下列文法定义了二进制整数的语法规则
\[\begin{aligned}
N &\to SL\\
L &\to LB\mid B\\
S &\to +\mid -\\
B &\to 0\mid 1
\end{aligned}\]
给出语法制导定义,求出该二进制整数的十进制值,存在综合属性\verb'val'中
\begin{center}
\begin{tabular}{|c|l|l|}\hline
序号 & 生成式 & 语义规则\\\hline
1 & $N\to SL$ & $N.val=S.sign*L.val$\\\hline
2 & $L\to L_1 B$ & $L.val=L_1.val*2+B.val$\\\hline
3 & $L\to B$ & $L.val=B.val$\\\hline
4 & $S\to +$ & $S.sign=1$\\\hline
5 & $S\to -$ & $S.sign=-1$\\\hline
6 & $B\to 0$ & $B.val=0$\\\hline
7 & $B\to 1$ & $B.val=1$\\\hline
\end{tabular}
\end{center}
\end{example}

扩展文法:
\[\begin{array}{rll}
expr &\to expr_1 + term & \{print('+')\}\\
expr &\to expr_1 - term & \{print('-')\}\\
expr &\to term & \\
term &\to 0    & \{print('0')\}\\
term &\to 1    & \{print('1')\}\\
     &\vdots   & \\
term &\to 9    & \{print('9')\}\\
\end{array}\]
大括号中的语句称为动作(action),这一系列产生式称为翻译模式(translation scheme)。

将动作视为产生式右端的一部分,则可得到扩展的语法树。
对语法分析树做先序遍历,则可以得到后缀表达式。