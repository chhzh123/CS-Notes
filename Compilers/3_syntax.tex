% !TEX root = main.tex

\section{语法分析}
\subsection{上下文无关法}
语法分析需要解决:从词法分析中获得的每个属性字(token)在语句中承担什么角色,同时检查语句是否符合程序语言的语法。
\begin{definition}[上下文无关法(context-free gramma, CFG)]
包括四部分
\begin{itemize}
	\item 终端符号(terminal)的集合$T$
	\item 非终端符号的集合$N$
	\item 唯一的开始符号$S\in N$
	\item 若干以下形式的产生式(production)
	\[X\to Y_1Y_2\ldots Y_n\]
	其中$X\in N$且$Y_i\in T\cup N\cup\{\epsilon\}$。
	多个左侧相同的产生式右侧可用$\mid$合并。
\end{itemize}
\end{definition}
\begin{definition}[推导(derivation)]
从开始符号开始,每一步推导就是用一个产生式的右方取代左端的非终端符号。
\end{definition}

CFG定义语言的能力比正则表达式强很大原因是它引入了\textbf{递归}的因素。

\begin{example}
用上下文无关文法定义下列语言:
\begin{itemize}
	\item $L=\{0^n1^n\mid n\geq 1\}$:$E\to 0E1\mid 01$
	\item 只含有$0$和$1$的回文串:$S\to 0S0\mid 1S1\mid 0\mid 1\mid \epsilon$
	\item 只含有$($和$)$的匹配括号串:$E\to (E)\mid EE\mid \epsilon$
\end{itemize}
\end{example}

\begin{itemize}
	\item 最左推导:每步推导都替换最左侧的非终端符号
	\[E\xRightarrow{lm}
	-E\xRightarrow{lm}
	-(E)\xRightarrow{lm}
	-(E+E)\xRightarrow{lm}
	-(id+E)\xRightarrow{lm}
	-(id+id)\]
	\item 最右推导:每步推导都替换最右侧的非终端符号
	\[E\implies
	-E\implies
	-(E)\implies
	-(E+E)\implies
	-(E+id)\implies
	-(id+id)\]
\end{itemize}

\begin{definition}[二义性]
如果对于一个文法,存在一个句子,对这个句子可以构造两棵不同的分析树,那么我们称这个文法为二义的。
\end{definition}
看语法分析树的叶子结点能不能连成句子。

\begin{example}
对于文法$E\to E+E\mid E*E\mid -E\mid (E)\mid id$及句子$id+id*id$,有以下两种推导:

\begin{minipage}{0.5\linewidth}
\[\begin{aligned}
E &\implies E+E\\
&\implies id+E\\
&\implies id+E*E\\
&\implies id+id*E\\
&\implies id+id*id\\
\end{aligned}\]
\end{minipage}
\begin{minipage}{0.5\linewidth}
\[\begin{aligned}
E &\implies E+E\\
&\implies id+E\\
&\implies id+E*E\\
&\implies id+id*E\\
&\implies id+id*id\\
\end{aligned}\]
\end{minipage}
\end{example}

文法二义性的消除可通过引入更多的产生式。
\begin{example}
$E\to E+E\mid E*E\mid (E)\mid id$是有二义的,因为不知道应该先算加法还是乘法。
可将其改为
\[\begin{aligned}
E &\to E+T\mid T\\
T &\to T*F\mid F\\
F &\to (E)\mid id
\end{aligned}\]
其中$E$为Expression,$T$为Term,$F$为Facotr,即可消除二义性(必然得先算乘法)。
\end{example}

并不是所有上下文无关文法都可以做到无二义,也无法判断一个上下文无关文法是否是二义的。

\subsection{NFA转CFG}
\begin{enumerate}
	\item 对于NFA的每一状态$i$,创建非终态$A_i$
	\item 若状态$i$在输入$a$上有转换边到状态$j$,则添加生成式$A_i\to aA_j$;
	若状态$i$在输入$\epsilon$上转换到状态$j$,则添加生成时$A_i\to A_j$
	\item 若$i$是接受状态,则添加$A_i\to\epsilon$
	\item 若$i$是初始状态,则令$A_i$为语法的初始符号
\end{enumerate}

\begin{definition}[右线性文法]
如果每个产生式都属于下列形式之一
\[A\to aB\qquad A\to a\qquad A\to\epsilon\]
则这样的文法称为右线性文法
\end{definition}
\begin{definition}[左线性文法]
如果每个产生式都属于下列形式之一
\[A\to Ba\qquad A\to a\qquad A\to\epsilon\]
则这样的文法称为左线性文法
\end{definition}

在处理程序时,上下文无法文法存在局限性,无法解决诸如以下问题:
\begin{itemize}
\item 变量先声明,再使用
\item 调用函数时,实参个数和形参个数一致
\end{itemize}
都得留到语义分析阶段才解决。

\subsection{递归下降}
\begin{definition}[左递归]
对于非终端符号$A$有生成式$A\to A\alpha$,则该文法是左递归的。
\end{definition}
消除左递归的方法:
\[A\to A\alpha_1\mid \cdots\mid A\alpha_m\mid\beta_1\mid\cdots\beta_n
\implies
\begin{aligned}
A &\to\beta_1 A'\mid\cdots\mid\beta_n A'\\
A' &\to\alpha_1 A'\mid\cdots\mid\alpha_m A'\epsilon
\end{aligned}\]

\begin{definition}[FIRST集与FOLLOW集]
$FIRST(\alpha)$集为从$\alpha$中推导出来的字符串第一个终端符号的集合,若$\alpha\to\epsilon$,则$\epsilon\in FIRST(\alpha)$;若$A\to c\gamma$,则$c\in FIRST(A)$。
$FOLLOW(A)$集为可以出现在$A$右侧的终端符号的集合。
若$A$是最右端的符号,则字符串结束符号$\$\in FOLLOW(A)$。
\end{definition}

LL(1)文法
\begin{itemize}
	\item 第一个L:输入字符串从左边开始扫描
	\item 第二个L:得到的推导是最左推导
	\item (1):向前看1个输入符号(或单词)
\end{itemize}

基于表的预测语法分析
\begin{algorithm}
\caption{Table-Driven Predictive Parsing}
\begin{algorithmic}[1]
\State \verb'ip=0'
\State \verb'X=stack.top()'
\While {$X\ne\$$}
\If {X == w[ip]}
\State \verb'stack.pop(); ip++;'
\Else \If {X is a terminal or M[X,a]=$\varnothing$}
\State Error()
\Else
\State Output production \verb'M[X,a]'$=X\to Y_1Y_2\cdots Y_k$
\State stack.pop()
\State push $Y_k,Y_{k-1},\ldots,Y_1$ onto the stack
\EndIf
\EndIf
\State \verb'X=stack.top()'
\EndWhile
\If {w[ip] != '\$'}
\State Error()
\EndIf
\end{algorithmic}
\end{algorithm}