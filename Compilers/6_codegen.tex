% !TEX root = main.tex

\section{代码生成及优化}
\subsection{代码生成}
\begin{itemize}
	\item 指令选择:选择最适合目标机器的指令来实现IR
	\item 寄存器分配和指派
	\item 指令调度
\end{itemize}

\begin{definition}[基本块(basic block)]
单一入口单一出口。
成为leader的指令:
\begin{enumerate}
	\item 第一条三地址指令
	\item 条件或无条件跳转指令的目标
	\item 条件或无条件跳转指令的下一指令
\end{enumerate}
\end{definition}

\subsection{代码优化}
\begin{itemize}
	\item 窥孔优化(peephole):基于滑动窗口,最小粒度
\begin{lstlisting}[language=c++]
x = x + 0 // eliminated
x = x * 1 // eliminated
y = x * 2 // y = x << 1
LD R0, a
ST a, R0 // eliminated
\end{lstlisting}
	\item 局部优化:在基本块内的优化
\begin{itemize}
	\item 公共子表达式删除
	\item 常量/拷贝传递
	\item 荣誉操作消除
\end{itemize}
	\item 循环优化:在循环内的优化
	\item 全局优化:最粗粒度的优化
\end{itemize}

\begin{definition}[循环(loop)]
只有唯一入口/头的强连通子图
\end{definition}