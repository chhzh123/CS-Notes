% !TEX root = main.tex

\section{数据的表示与存储}
\subsection{基本指标}
\begin{itemize}
	\item 位(bit/b):计算机处理、存储、传输信息的最小单位
	\item 字节(Byte/B) $1\text{ Byte}=8\text{ bit}$:现代计算机主存按字节编制,字节是最小可寻址单位
	\item 字(Word):表示被处理信息的单位,用来度量数据类型的宽度\footnote{字长是指CPU中\textbf{数据通路的宽度},也指计算机一次能处理的二进制的长度,等于CPU内部总线的宽度或运算器的位数或通用寄存器的宽度;字和字长的宽度可以一样,也可以不同,通常是字节的整数倍}
\end{itemize}
\par 一台32位的电脑,一个字等于4个字节,字长为32位;若字长为16位,则一个字等于2字节。
注意4字节相当于8位的16进制编码。

表示计算机通信带宽时
\begin{center}
\begin{tabular}{ccccccc}\hline
KB(yte) & MB & GB & TB & PB & EB & ZB\\\hline
$10^3$ & $10^6$ & $10^9$ & $10^{12}$ & $10^{15}$ & $10^{18}$ & $10^{21}$\\\hline
\end{tabular}
\end{center}

表示计算机存储容量时
\begin{center}
\begin{tabular}{ccccccc}\hline
KiB(yte) & MiB & GiB & TiB\\\hline
$2^{10}$ & $2^{20}$ & $2^{30}$ & $2^{40}$\\\hline
\end{tabular}
\end{center}

\subsection{进制}
\[\text{进位计数制(system)}=\sum\text{位权}\cdot\text{基数}^\text{指数}\]
\par在计算机中主要考虑以下几种进制系统:二进制(binary)、八进制(octonary)、十进制(decimal)、十六进制(hexadecimal)。
\begin{itemize}
	\item 十进制转二进制:整数部分除以2取余,小数部分乘2取整
	\item 二进制转八进制:从整数最低位开始,三位三位统计
	\item 二进制转十六进制:从整数最低位开始,四位四位统计
\end{itemize}

\subsection{符号数}
\begin{enumerate}
\item 二进制(真值$\to$机器数)
\begin{itemize}
	\item 符号数值(sign-magnitude)形式(原码):首位0为正数,1为负数,将符号位一起考虑有以下表示
	\[A=\begin{cases}
	A & A\in[0,2^{n-1})\\
	2^{n-1}-A & A\in(-2^{n-1},0]
	\end{cases}\]
	\item 反码(1's complement)$\ssim A$:除符号位不变,其他位取反;同理小数
	\[\ssim A=\begin{cases}
	A & A\in[0,2^{n-1})\\
	(2^n-1)+A & A\in (-2^{n-1},0]
	\end{cases}\]
	\begin{analysis}
	反码是全1的补数
	\[\ssim A=(2^n-1)-A=(11\ldots 1)_2-A_2\]
	即在$\mod 2^{n}-1$意义下的运算
	\end{analysis}
	\item 补码(2's complement)$[A]_c$:反码+1,按照原来十进制转二进制方法即可得对应有符号十进制数,由于没有正负0,故表示的数多了一位,补码的补码为原码;同理小数
	\[[A]_c=\begin{cases}
	A & A\in[0,2^{n-1})\\
	2^n+A & A\in \textcolor{red}{[-2^{n-1},0)}
	\end{cases}\]
	\begin{analysis}
	补码的设计非常关键,理解补码的由来对于后面的四则运算有着很大帮助。
	之所以要有补码,是因为希望能做到\textbf{减去一个数等于加上某个数},而这在模$2^n$的意义下即可实现。\\
	那么就有
	\[[A]_c=2^n-A=((2^n-1)-A)+1=\ssim A+1\]
	即在$\mod 2^n$意义下的运算,以4位二进制为例
	\[(5)_{10}=(0101)_2\implies (5)_c=2^4-5=11=(10000)_2-(0101)_2=(1011)_2\]
	由$[A]_c$求$[-A]_c$要连同符号位一起取反加1
	\end{analysis}
	\item 移码(bias)$[A]_b$:补码的符号位取反,引入目的是保证浮点数的机器零
	\[[A]_b=A+2^{n-1}\,,A\in(-2^{n-1},2^{n-1})\]
	\begin{analysis}
	相当于把正数移到负数的部分,负数移到正数的部分\\
	注意区别移码的定义($2^{n-1}$)和具体浮点数阶码($2^{n-1}-1$)的实施\footnote{原因可见\url{https://blog.angularindepth.com/the-mechanics-behind-exponent-bias-in-floating-point-9b3185083528}}
	\end{analysis}
\end{itemize}
\item 十进制:
\begin{itemize}
	\item ASCII码
	\item BCD码:四位表示一位十进制数
\end{itemize}
\end{enumerate}

\subsection{小数表示}
\subsubsection{定点小数}
首位符号位(同样有原码补码表示之分\footnote{16位机器,定点补码小数可表示$-1\thicksim (1-2^{-15})$})
\begin{itemize}
	\item 定点整数:小数点固定在最低位右边,$0\leq|x|\leq 2^n-1$
	\item 定点小数:小数点固定在\textbf{数值部分}最高位的左边,$0\leq|x|\leq 1-2^{-n}$
\end{itemize}

\subsubsection{浮点数(IEEE 754)}
\begin{center}
\begin{tabular}{|c|c|c|}\hline
符号S,1 & 阶码E,8,移码 & 尾数F,23,原码
\\\hline
\end{tabular}
\end{center}
其中,移码偏置常数为127(单精度)、1023(双精度),用以简化比较。
\par 规格化数,类似于科学记数法,即令小数点前面必为1,隐含表示,以增加数据表示精度。
\par 故浮点数可表示为
\[(-1)^S\times1.F\times 2^{E-127}\]
\begin{example}
\[1\;0110\;1001\;0001=1.0110\;1001\;0001\times 2^{(12)_{10}}\]
\par 指数:$12+127=139\to 1000\;1011$
\par 尾数:$011\;0100\;1000\;1000\;0000\;0000$ 左对齐,因为有小数点
\begin{center}
\begin{tabular}[htbp]{|c|c|c|}
\hline
符号S & 指数E(exponent) & 尾数F(mantissa)\\\hline
$0$ & $1000\;1011$ & $011\;0100\;1000\;1000\;0000\;0000$\\\hline
1位 & 8位 & 23位\\\hline
\end{tabular}
\end{center}
\end{example}
\par 用两侧的极端值用来表示特殊值
\begin{center}
\begin{tabular}{ccc}
阶码(移码) & 尾数 & 数据类型\\\hline
$1\thicksim 254$ & 任何值 & 规格化数\\
0 & 0 & 0\\
0 & 非零数 & 非规格化数\\
255 & 0 & $+\infty/-\infty$\\
255 & 非零数 & NAN, Not A Number
\end{tabular}
\end{center}
\par 浮点数单精度可表示范围$[10^{-38},10^{+38}]$,双精度$[10^{-308},10^{+308}]$

\subsection{数据存储方式}
数据在计算机中都是分段存储,故有以下两种存储方式:
\begin{itemize}
	\item 大端方式(Big Endian):最高有效位(MSB)所在地址为数的地址,MIPS,Photoshop、JPEG
	\item 小端方式(Little Endian):最低有效位(LSB)所在地址为数的地址,x86,GIF、RTF
\end{itemize}
\par 如果遇到存储方式不同的两台机器,则要发生字节交换,即对大端小端进行互换。

下面是C语言中数据类型大小,都以字节为单位。
\begin{center}
\begin{tabular}{|c|c|c|}\hline
类型 & 数据长度\\\hline
char & 1 \\\hline
short & 2 \\\hline
int & 4 \\\hline
long & 4/8 \\\hline
float & 4 \\\hline
double & 8 \\\hline
\end{tabular}
\end{center}
\par 具体存储中会遇到数据边界对齐的问题,特别是结构体(struct)的实现。
按字地址对齐(4的倍数,二进制后两位为0),有利于减少访存次数。

\subsection{数据纠错}
冗余校验思想,通过增添校验位来实现纠错。
\begin{itemize}
	\item 奇偶校验码:$P=b_{n-1}\oplus b_{n-2}\oplus\cdots\oplus b_0\oplus 1$与结果的$P'$再取异或,为1则奇数位错\\
	只能发现奇数位出错,且不具有纠错能力
	\item 海明码
	\item 循环码
\end{itemize}