% !TEX root = main.tex

\section{计算机网络概述}
计算机网络是自主计算机的的互连集合。

\subsection{网络连接方式}
直接连接的网络
\begin{itemize}
\item 点对点(point-to-point)网络:包括专用介质(dedicated medium)、节点/主机
\begin{itemize}
	\item 单向(simplex)
	\item 半双工(half duplex)
	\item 全双工(full duplex)
\end{itemize}
\item 多路访问(multiple access)网络:共享介质(shared medium)、广播、碰撞(collision)
\begin{itemize}
	\item 单播(unicast)
	\item 多播(multicast)
	\item 广播(broadcast)
\end{itemize}
\end{itemize}

间接连接的网络
\begin{itemize}
	\item 中间节点、路由器(router)
	\item 包(packet)
	\item 存储转发(store and forward)
	\item 路由选择(routing)
	\item 路由表(routing table)
	\item 目的地(destination)、下一跳(next hop)
\end{itemize}

\subsection{因特网}
网络互连:用路由器(或网关gateway)连接起来构成的网络称为互连网络(internetwork)。

因特网/互联网(Internet)是一种互连网络
\begin{itemize}
	\item 系统域网(System Area Network, SAN)
	\item 局域网(Local Area Network, LAN)
	\item 城域网(Metropolitan Area Network, MAN)
	\item 广域网(Wide Area Network, WAN)
\end{itemize}

因特网:
\begin{itemize}
	\item 终端系统/主机:运行网络应用程序
	\item 通信链路(communication link):光纤、铜线、无线电、卫星等
	\item 路由器(router)
\end{itemize}

因特网的结构:ISP(Internet Service Provider)
\begin{itemize}
	\item 顶层ISP:主干网(中国电信、中国移动、中国网通)
	\item 区域ISP:可以私自互联
	\item 本地ISP
\end{itemize}

\subsection{网络服务}
网络提供的服务:
\begin{itemize}
	\item 可靠/不可靠:会不会丢包
	\item 面向连接/无连接:需要连通
	\item 有确认/无确认
	\item 数据报服务:无连接无确认(因特网)
	\item 请求相应和消息流服务
\end{itemize}

因特网体系结构:
\begin{itemize}
	\item 应用层:提供对某些专门应用的支持,如FTP、SMTP、HTTP
	\item 传输层:将网络层获得的包在\textbf{进程之间}数据传送(端到端),如TCP、UDP
	\item 网络层:\textbf{路由选择},实现在互联网中的数据传送(主机到主机),IP、路由协议
	\item 数据链路层:在物理网络中传送包(跳到跳,节点到节点),PPP、Ethernet
	\item 物理层:线上的\textbf{比特}(传送原始比特流)
\end{itemize}
网络层以下不可靠,以上可靠;防止丢包的机制:重发

协议(protocol):在网络实体(entities)之间传送消息的规则,如消息的格式、收发消息的次序等

协议栈:发送时封装(encaptulation),接收时拆封。
每层传输的数据单元都称为包(packets),都属于某个协议,又被称为协议数据单元(protocol data unit, PDU)=协议控制信息(potocal control data, PCI)+服务数据单元(SDU)
\begin{center}
\begin{tikzcd}
\text{应用层Application}\arrow{d}{\text{消息message}}\\
\text{传输层Transport}\arrow{d}{\text{数据段segment}}\\
\text{网络层Network}\arrow{d}{\text{数据报datagram}}\\
\text{链路层Link}\arrow{d}{\text{帧frame}}\\
\text{物理层Physical}
\end{tikzcd}
\end{center}
不同协议则添加不同头部。
路由做得事情是拆一层封装,然后重新加一层。
同一个互联网络中网络层协议需要相同,链路层协议可以不同。

% 服务访问点(service access point, SAP)

ISO/OSI(open system interconnection)网络七层协议,在应用层和传输层中间添加两层:
\begin{itemize}
	\item 表示层(presentationn):提供数据转换服务, 例如,加密解密,压缩解压缩,数据格式变换
	\item 会话层(session):简化会话实现机制,例如,数据流的检查点设置和回滚以及多数据流同步
\end{itemize}

对等实体:实现相同协议

\subsection{网络性能分析}
当一个包到达时如果有空闲缓存则排队等待转发,产生延迟(delay);
如果没有空闲缓存,则丢弃该包,造成丢失(loss)。

包交换(packet-switching)网络中的延迟
\begin{itemize}
	\item 处理(processing)延迟:查路由
	\item 排队(queueing)延迟
	\item 发送/传输(transmission)延迟:包长(bits)/链路带宽(bps, bit per second);指从发送第一个包到发送最后一个包的间隔
	\item 传播(propagation)延迟:指对于一个包来说从发送到接收所需的时间
\end{itemize}
接收延迟与传播延迟重合。
故总延迟(从第一个包被发送到最后一个包被接收的时间)=传播延迟+发送延迟。

往返时间(round trip time, RTT):从源主机到目的主机再返回源主机所花的时间

\begin{itemize}
	\item 带宽(bandwidth):一条链路或通道可达到的最大数据传输速率(bps)
	\item 吞吐量(thoughput):一条链路或通路实际数据传输速率
\end{itemize}