% !TEX root = main.tex

\section{物理层}
\subsection{通信系统}
信息能够被解释为数据(msg/data),用符号(sign)记录,用信号(signal)传递(transmit),用熵(entropy)测量

\begin{itemize}
	\item 信号:光、电、广播
	\item 模拟信号:连续取值
	\item 数字信号/跳变信号:离散取值
	\item 模拟传输:模拟信号、放大器(amplifier)
	\item 数字传输:数字信号、中继器(repeater)
\end{itemize}

\subsection{编码方式}
载波信号(carrier)一般采用正弦波信号:角频率$\omega$、频率$f$、周期$T$、振幅$A$、相位($\varphi$)
\begin{itemize}
	\item 频移键控(frequency-shift keying, FSK):通过不同频率表示不同信息0/1
	\item 幅移键控(amplitude-shift keying, ASK)
	\item 相移键控(phase-shift keying, PSK)
	\item 正交调幅(QAM-quadrature amplitude modulation):用不同的振幅/频率表示不同的多位信息$000\thicksim 111$
\end{itemize}

\begin{enumerate}
\item 单极编码(unipolar):0V即0,$+E$V为1,但是会产生
\begin{itemize}
	\item 时钟漂移:不同的时钟会有差别,一定要有跳变
	\item 基线漂移:线很长会有(积累很多电荷,以为是1),一定要有变化/正负
\end{itemize}
\item 不归零编码/双极编码(non-return-to-zero/bipolar, NRZ) :$-E$为0,$+E$为1,解决基线漂移问题(平衡01);全是0或全是1,还是没法区分
\item 不归零编码反转(Inverted, NRZI):\textbf{差分}码波形,相邻码元的电位改变表示1,而电位不改变表示0;也可以反过来。 该表示方法与码元本身电位或极性无关,而仅与相邻码元的电位变化有关
\item 曼彻斯特(Manxhester)编码:从相邻时刻的中间起$降-E\thicksim +E$,$0\to 10, 1\to 01$,可克服时钟漂移和基线漂移;频率高,传输有问题,对传输介质要求高
\item 差分曼彻斯特编码:在每一位开始时间如果跳变则为0,否则为1
\item 4B/5B编码:用5比特代表4比特,多一位冗余;每个编码没有多于1个前导零和多于2个末端零
\end{enumerate}

\subsection{物理介质}
非屏蔽双绞线(unshielded twisted pair, UTP):四对线(绿绿白、橙橙白、蓝蓝白、棕棕白)
屏蔽双绞线(STP)
同轴电缆(coaxial cable)
光导纤维(optical fiber)
\begin{itemize}
	\item 单模光纤
	\item 多模光纤
\end{itemize}

无线介质

多路复用(multiplexing)
\begin{itemize}
	\item 时分多路复用(time division multiplexing, TDM)
	\item 频分多路复用(frequency, FDM)
	\item 波分多路复用(wavelength)
	\item 码分多路复用(code)
\end{itemize}