% !TEX root = main.tex

\section{数据链路层}
数据链路层把数据包,即\textbf{帧(frame)},从一个节点通过链路(直连网络或\textbf{物理网络})传给相邻另一个节点(主机和路由器)

数据链路层的功能如下:
\begin{itemize}
	\item 成帧(framing)
	\item 差错检测(error detect):比特错,纠错
	\item 差错控制(error control):丢包、重复、错序、流控制(flow control)
	\item 介质访问控制(medium access control):多路访问,碰撞(collision)
\end{itemize}

针对点对点和多路访问网络分别制定了两个子层:
\begin{itemize}
	\item 逻辑链路控制(Logic Link Control, LLC)子层:提供可靠数据传输
	\begin{itemize}
		\item LLC1提供无确认无连接服务
		\item LLC2提供有确认面向连接的服务,实现滑动窗口协议
		\item LLC3提供有确认无连接的服务
	\end{itemize}
	\item 介质访问控制(Media Access Control, MAC)子层:专门用来处理多路访问网络中的冲突(点对点网络没有冲突就不用)
\end{itemize}
注意数据链路层、网络层错了就错了,不提供纠正服务,由上层纠正

\subsection{逻辑链路控制子层}
\subsubsection{差错检测}
在数据报后加校验码(头部加序号),通过链路传输看是否有数据报/校验码错误
\begin{enumerate}
\item 奇偶校验:若接收方收到奇数个1,则有出错
\begin{itemize}
	\item 一维偶校验:只能检错;最后补一位使得全部为偶数个1,如010补为010|1,而101补为101|0
	\item 二维偶校验:检错+纠错一位;横纵同时偶校验
\end{itemize}
\item 校验和(checksum):将所有数据加起来\\
由于需要使用加法器,校验和一般不用于数据链路层,而是更高层,例如IP层和传输层
\item 循环冗余校验码(Cyclic Redundancy Check, CRC):
补充n位后除以一个n+1位的除数,模2除法(按位异或,做减法时没有借位)\\
接收方连带校验码一起除,余数为0则没错\\
链路层常用CRC,因为检错率很高,且容易实现(触发器+异或门)
\end{enumerate}

\subsubsection{可靠数据传输}
超时则自动重发请求(Automatic Repeat reQuest, ARQ):每发送一帧都启动一个超时定时器,如果它的确认帧(ACK)在其超时时间内到达就删除该定时器,否则,重传该帧并重启定时器

主要的协议如下:
\begin{itemize}
	\item 停等协议(stop-and-wait):ARQ协议,只有收到前一个数据帧的确认帧才可以发送下一个数据帧\\
	三种出错情况:
	\begin{itemize}
		\item 数据帧丢失
		\item 确认帧丢失
		\item 超时:收到ACK表明接收方一定收到,可以发送新的数据帧,重传的也一定要发ACK
	\end{itemize}
	效率/吞吐量十分低,信道空闲时间长
	\item 滑动窗口协议(sliding window):ARQ协议,不需等待前面发送的帧的确认帧返回,就可以连续发送下一个,其个数不能超过发送窗口大小(sending window size, SWS)(连续发送数据帧可用序号范围,用于流控制:控制发送速度,否则会发生溢出(overlow),后面覆盖前面的)\\
	这里的确认帧是指在此之前的帧都已收到(直连网中间没有节点,后面收到前面一定收到;只要出错纠正不了直接丢弃)
	\begin{itemize}
	\item 回退N协议(go back N):同滑动窗口连续发送,但某个ACK没收到则重传在此ACK之后的所有帧(超时重传),丢3则ACK4发2\\
	发送窗口需要缓存SWS个帧,以便重传;发送窗口中序号最小的为sendbase
	\item 选择性重传(selective repeat):通过发送否定性确认帧(negative acknowledgement, NAK)要求重传该帧;如3丢失,4发送NAK=3,5发送ACK=2,重传3\\
	接收窗口(receiving window size, RWS)表示接收缓冲区大小(RWS$\leq$SWS,最好是等于,尽量减少重传帧;但序号少的话导致重复),用于确定应该保存哪些帧,用序号范围表示\\
	超时时间应该设长,确保帧的2次来回;没有后续帧也会超时重传;无论窗口内窗口外收到都要发确认\\
	\end{itemize}
\end{itemize}
注意不管哪一个重传机制,序号可以重复使用,因此有最小序号问题。
\begin{example}
	序号8个,SWS=RWS=4,345670123456,5丢失
\end{example}
\begin{analysis}
	回退N:346705670123456\\
	选择性重传:346705123456
\end{analysis}

提高滑动窗口协议的效率:
\begin{itemize}
	\item 选择性确认(selective acknowledgement):接受方把已收到的帧的序号告诉发送方(收到不用重传告诉发送方哪些已收到,则只用重传某帧)
	\item 捎带确认(piggybacking):通信双方\textbf{全双工}方式工作,接收方在发数据给对方时顺便把确认号也告诉对方(两个滑动窗口,两边都要发数据),需要结合下面一起使用
	\item 延迟确认(delayed acknowledgement):接收方收到一帧后并不立即发送确认帧,而是等待一段时间再发送
\end{itemize}

标志位成帧,区分包的起始和终结;同时需要转义

链路层的实现:在网络接口卡(network interface card, NIC)及其驱动程序上实现,路由器在接口模块上实现

\subsection{介质访问控制子层}
\subsubsection{简介}
\begin{enumerate}
\item \textbf{PPP协议(point-to-point):点到点网络}
\begin{itemize}
	\item 根据HDLC(high-level data link control)协议进行设计,主要用于串行电缆、电话线(MODEM)等串行链路
	\item 提供连接认证、传输加密和压缩功能,为网络层协议提供服务
	\item 没有纠错功能,也没有流控制和确保有序的功能
\end{itemize}

\item \textbf{以太网:多路访问网络}
\begin{itemize}
\item 纯ALOHA:想发送就发送,超时未收到确认则发生冲突
\item 分槽ALOHA:将时间分为长度相同的时槽,每个站点只在时槽开始时发送。\\
信道空,立即以概率$p$发送,以概率$1-p$延迟一个时间槽;信道忙,延迟一个时间槽。
\item 载波监听CSMA(Carrier Sense Multiple Access):发送前先监听信道
\begin{itemize}
	\item 信道空,立即发送;信道忙,持续监听(1-persistent CSMA,以太网)
	\item 信道空,发送;信道忙,延迟一段随机长度时间(non-persistent CSMA)
	\item 信道空,立即以概率$p$发送,以概率$1-p$延迟一个时间槽;信道忙,延迟一个时间槽(p-persistent CSMA,分槽ALOHA)
\end{itemize}
\end{itemize}
\end{enumerate}

\subsubsection{以太网MAC层协议}
载波监听CSMA/CD(with collision detection)
\begin{enumerate}
	\item 发送数据帧之前先监听信道。如果信道空闲,立即发送。
	如果信道忙,则持续监听,直到信道空闲,立即发送。
	\item 边发送边检测冲突。如果发送完毕都没有检测到冲突,则发送成功。
	\item 如果检测到冲突,则停止发送,并发送32位干扰位(jamming signal)以加强冲突信号。
	采用二进制指数退避算法\textbf{随机延迟}一段时间后,转(1)。
\end{enumerate}

二进制指数退避算法(binary exponential backoff)
\begin{itemize}
	\item 时间片$\tau$的长度为512b时间,10Mbps的以太网为51.2$\mu s$
	\item 帧间空隙(interframe gap)为96b
	\item 每次从$2^j-1$个时间片随机选择一个
\end{itemize}

802.3的MAC帧格式
\begin{center}
前导字符(8B)+目的地址(6B)+源地址(6B)+类型/长度(2B)+有效载荷/填充位+帧校验序列(4B)
\end{center}
\begin{itemize}
\item 源地址:一般为发送者的单播地址
\item 目的地址(6B):一般为接收者的单播地址
\end{itemize}

\begin{itemize}
	\item 单播/网卡/烧录地址:全球唯一,每个网卡/接口一个
	\item 多播地址:字节0第0位为1,地址非全1
	\item 广播地址:48位全为1
\end{itemize}

\subsubsection{透明网桥}
网桥/交换机:二层(从下往上数)交换机数据链路、物理层

校园网大多交换机,路由少

扩展/桥接局域网:每一个局域网(LAN)都是一个网段

MAC地址表,如
\begin{center}
\begin{tabular}{|c|c|}\hline
	B & P1\\\hline
	D & P2\\\hline
\end{tabular}
\end{center}
\begin{itemize}
	\item 表里查到则转发(forward)
	\item 表里没有则扩散/泛洪(flood):由端口P1,扩散到其他所有端口P2,P3(多播、广播一定扩散)
	扩散不回传
	% 只要广播都要收下了,操作系统一定要处理
	收到的部分不会往回转
	\item 从某一条路发来则不能传回去,过滤/丢弃(filter)
\end{itemize}

生存期(Time to live, TTL):单位为秒,每次发送都会重置,对于不活跃的表项自动删掉(减少表的大小,查找速度更快)

自学习:利用\textbf{源地址}学习,如信息从A-P1来,则记为A-P1,同时设好TTL
如果收到的帧有错则直接丢弃,根本不会学习
更新记录,重置超时计时器

广播风暴:回路,容错,防止一个网桥崩溃就全局崩溃
\begin{center}
	\begin{tikzcd}
		& A\arrow{ld}\arrow{rd} &\\
		B_1\arrow{rd} & & B_2\arrow{ld}\\
		& C &
	\end{tikzcd}
\end{center}

冲突域:中间可以加集线器

\subsection{生成树协议}
IEEE 802.1w RSTP(Rapid Spanning Tree Protocol)


将所有的LAN和网桥都抽象为结点,避免冲突即构造一棵生成树(注意不是最小生成树)

先确定根网桥,即BID(Bridge ID)最小的

每个网段(需要集线器)依赖于连通的网桥,每个网桥都把自己到根的距离发出去(竞选/配置消息)

网桥之间的开销为1,选一条最短路径

网桥只在根端口和指定端口之间转发\textbf{数据}帧

主机端口出去不加帧

扩散自己BID,最后只剩下根网桥认为自己是根
取得优胜的,作为制定网桥;相同距离时,BID小的优胜;端口号小的优胜

只有从根端口过来的才扩散配置消息
其他端口来的不扩散
这样不会形成回路

只能在根端口和指定端口转发,不可通过阻塞端口

断了/失效了则变成无穷大,其他网桥可成为指定网桥

防止广播风暴,又能自动修复损害网桥(通过冗余方式),增加可靠性

\subsection{虚拟局域网}
虚拟局域网(Virtual LAN, VLAN)将原来的局域网分割成多个相互隔离的局域网,只在具有相同颜色的端口间转发。

扩散到帧内制定的端口或\textbf{干道端口}

查MAC地址表转发

如果从干道收到的帧没有VLAN ID,则认为是本征(native)VLAN,默认为VLAN1。

多生成树协议:管理员规定哪些VLAN为一组,构成多生成树,其余的用公公生成树
\begin{itemize}
	\item 公共生成树(common spanning tree, CST)
	\item 多生成树(Multiple spanning tree protocol, MSTP)
\end{itemize}

\subsection{交换机}
交换机是一个把多个网段连接起来的设备,也称为\textbf{多端口网桥}(switch=bridge)。
注意输入输出端口一样。

交换结构(fabrics)
\begin{itemize}
	\item 共享总线式交换机:同样有冲突问题
	\item 纵横式(crossbar)
\end{itemize}

交换机转发方法
\begin{itemize}
\item 存储转发(store and forward):交换机收到整个帧后转发\\
目的MAC地址(6B),转也要依照CSMA/CD来转发
\item 直通(cut through):收到一个转一个,出现碎片\\
\item 无碎片(fragment free):都没冲突才开始转发,\textbf{最小帧保证}\\
交换机不用收到整个帧而是收到64B(冲突窗口)后自动转发
\item 适应性交换(adaptive switching):自动在上面三种方式中选择
\end{itemize}
全双工模式:因为没有冲突,CSMA/CD算法可以被关闭
自动翻转(auto-MDIX):
大部分交换机可以自动选择连接方式,交叉线或直通线

令牌环网(token ring):通过在站点之间传递令牌防止冲突,具有优先权,take turns protocol
以太网没有确认机制,没有优先权
必然有特殊的站点(监控站点)产生令牌帧,选举出监控站点(MAC地址最小)
保有令牌帧的时间是相同的

源路由网桥:用来连令牌环网,将路径记到头部,下一次就不用查