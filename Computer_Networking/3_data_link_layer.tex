% !TEX root = main.tex

\section{数据链路层}
数据链路层把数据包(packet)从一个节点通过链路(直连网络或物理网络)传给相邻另一个节点

\subsection{简介}
基本术语如下:
\begin{itemize}
	\item 节点(node):主机和路由器
	\item 链路(link):连接相邻节点的通道,有线链路、无线链路、局域网
	\item 帧(frame):第二层数据包
\end{itemize}

功能
\begin{itemize}
	\item 成帧(framing)
	\item 差错监测(error detect):比特错,纠错
	\item 差错控制(error control):丢包、重复、错序、流控制(flow control)
	\item 介质访问控制(medium access control):多路访问,碰撞(collision)
\end{itemize}

\subsection{差错检测}
在数据报后加校验码(头部加序号),通过链路传输看是否有数据报/校验码错误

奇偶校验
\begin{itemize}
	\item 一维偶校验:只能检错;最后补一位使得全部为偶数个1,如010补为010|1,而101补为101|0
	\item 二维偶校验:检错+纠错一位;横纵同时偶校验
\end{itemize}
若接收方收到奇数个1,则有出错

校验和(checksum):将所有数据加起来\\
由于需要使用加法器,校验和一般不用于数据链路层,而是更高层,例如IP层和传输层

循环冗余校验码(Cyclic Redundancy Check, CRC):
补充n位后除以一个n+1位的除数,模2除法(按位异或,做减法时没有借位)\\
接收方连带校验码一起除,余数为0则没错\\
链路层常用CRC,因为检错率很高,且容易实现(触发器+异或门)

\subsection{可靠数据传输}
发送方传送数据帧,接收方回传确认帧(ACK)

超时则自动重发请求(Automatic Repeat reQuest, ARQ):每发送一帧都启动一个超时定时器,哪个帧超时将重传该帧,并重启定时器

\begin{itemize}
	\item 停等协议(stop-and-wait):ARQ协议,只有收到前一个数据帧的确认帧才可以发送下一个数据帧
	\item 滑动窗口协议(sliding window):ARQ协议,不需等待前面发送的帧的确认帧返回,就可以连续发送下一个,其个数不能超过发送窗口大小(sending window size, SWS)(连续发送数据帧可用序号范围,用于流控制:控制发送速度,否则会发生溢出(overlow),后面覆盖前面的)\\
	这里的确认帧是指在此之前的帧都已收到
	\item 回退N协议(go back N):同滑动窗口连续发送,但某个ACK没收到则重传在此ACK之后的所有帧(超时重传),丢3则4发2
	\item 选择性重传(selective repeat):否定性确认帧(negative acknowledgement, NAK),要求重传某一帧;如3丢失,4发送NAK=3,5发送ACK=2\\
	接收窗口(receiving window size, RWS)表示接收缓冲区大小,用于确定应该保存哪些帧,用序号范围表示
\end{itemize}

% 序号可以重复使用
% 吞吐量取决于协议

提高滑动窗口协议的效率:
\begin{itemize}
	\item 选择性确认(selective acknowledgement):接受方把已收到的帧的序号告诉发送方
	\item 捎带确认(piggybacking):通信双方全双工方式工作,接收方在发数据给对方时顺便把确认号也告诉对方
	\item 延迟确认(delayed acknowledgement):接收方收到一帧后并不立即发送确认帧,而是等待一段时间再发送
\end{itemize}

PPP协议(point-to-point):点到点网络的数据链路层协议,主要用于串行电缆、电话线(MODEM)等串行链路

链路层的实现:在网络接口卡(network interface card, NIC)及其驱动程序上实现,路由器在接口模块上实现