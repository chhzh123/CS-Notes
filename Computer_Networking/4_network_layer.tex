% !TEX root = main.tex

\section{网络层}
\subsection{IP数据报}
\subsubsection{数据传输技术}
\begin{itemize}
\item 电路交换(circuit switching):实际接通一条物理线路,时分多路复用,电话;频分多路复用,电视;一直占用,不管有无数据交互
\item 包交换/分组交换(packet switching):统计多路复用,按需分配;可能引起网络拥塞,适合发送突发数据
\begin{itemize}
	\item 虚电路:需建立连接才可以传输数据(仿照电话系统,因特网之前),好处在于保留带宽
	\begin{itemize}
	    \item 交换式(要交换才建立连接):建立虚电路(VC)表,虚电路标识符(VCI),类似于电话
	    \item 永久式(建立后一直保持):由管理员维护
	\end{itemize}
	\item 数据报(datagram):不需建立连接,因特网,\textbf{不预留带宽}
\end{itemize}
\end{itemize}

一般网络的服务模型:Aynchronous Transfer Mode, ATM
\begin{center}
\begin{tabular}{|c|c|c|c|c|c|c|}\hline
网络结构 & 服务模型 & 带宽 & 不丢包 & 有序 & 及时 & 拥塞反馈\\\hline
ATM & 恒定位速率 & 固定速率 & 是 & 是 & 是 & 无拥塞\\\hline
ATM & 可变位速率 & 确保速率 & 是 & 是 & 是 & 无拥塞\\\hline
ATM & 可用位速率 & 最小保证 & 否 & 是 & 否 & 是\\\hline
ATM & 未指定位速率 & 无 & 否 & 是 & 否 & 否\\\hline
因特网 & 尽力服务 & 无 & 否 & 否 & 否 & 否\\\hline
\end{tabular}
\end{center}

IP协议是因特网的网络层协议
\begin{itemize}
\item 可路由的(routable):全局地址,按层分配
\item 尽力服务(best effort):无连接无确认的数据报服务
\item IP协议可以运行在\textbf{任何}网络上,不仅仅是因特网
\end{itemize}

\subsubsection{IP数据报格式}
\begin{itemize}
\item 4个字节一个字,头部最多$(2^4-1)*4=60$B,除选项20B,IPv4选项最多40B,太少了
\item 生存期(TTL)限制在因特网上的停留时间,实际限制为经过的路由器数目,即跳数(hop count),超过则自动清除,防止兜圈,每次经过路由器减1\\
TTL初值默认设置为网络直径的两倍,Windows默认64\\
长了就有捷径(cut-through),因此发展到现在因特网的直径依然在32左右
\item IP数据报一定要封装成帧,通过物理层传输,每次都要修改源和目的地址
\item IP数据报服务类型(type of quality, ToQ),但路由器都没有实现
\end{itemize}

IP数据报的分段和重组
\begin{itemize}
\item 一个物理网络的最大传输单元(maximum transmission unit, MTU)是该网络可以运载的最大有效载荷,即数据帧的数据部分的最大长度\\
如:以太网(DIXv2)的MTU为1500, FDDI和令牌环的MTU分别为4353和4482
\item 只要发出去一定会封装成帧(注意要加头部),帧最长就是MTU,因而要分成多段再分
\item 如果一个数据报的大小大于要承载它的网络的MTU,路由器需要先对该数据报进行分段(fragment)
\item 源主机每次发送IP数据报时都会把标识(Identification)字段加1。
\item 分段时用标识的值保持不变,并且用偏移量字段(offset)指出该片段的数据部分相对原来数据报的偏移量(以8字节为单位),给出原来片段的次序
\item MF(More Fragment), DF(Don't Fragment)
\item 小于MTU-20B,边界,一定要能被8整除,尽可能大(8字节,一定要除掉)
\item IPv6中间不能分段
\item 1400B=512B+512B+376B
\item Path MTU discovery:找到路径上最小的MTU,发现路径上最小MTU
\item 选项最后一定对齐到边界
\item 生存期和头部校验(检验和)会变,其他不变
\end{itemize}

\subsection{IP地址}
48位的MAC地址和32位的IP地址都是全局的(全球分配),但是IP地址空间分层,是可路由的

IP地址可划分为两个部分:
\begin{itemize}
	\item 网络号/网络前缀/网络标识:确定拥有该IP地址的主机位于哪个网络
	\item 主机号:确定属于该网络的哪台主机
\end{itemize}

有类网:ABC单播,D多播,E保留,地址范围如下(点分十进制)
\begin{itemize}
	\item 0 $\thicksim$ 127
	\item 128 $\thicksim$ 191
	\item 192 $\thicksim$ 223
	\item 224 $\thicksim$ 239
	\item 240 $\thicksim$ 255
\end{itemize}

解决IPv4地址不够用的问题
\begin{itemize}
	\item 将一个有类网可以划分为多个相同大小的子网(subnet)\\
用子网掩码(subnet mask)划分边界:主机号全0,剩下的部分(网络号和子网号)全是1\\
子网掩码与IP地址\textbf{相与},若相等则在同个子网中
	\item 变长子网掩码(Variable-length subnet mask, VLSM):允许把一个有类网划分为多个不同大小的子网,类似变长指令集\\
解决主机数目不均匀的问题,如100、50、25、10,则不能等距划分子网\\
用长度来表示子网掩码,如/26代表255.255.255.192
	\item 无类域间路由选择协议(classless inter-domain routing, CIDR):将多个有类网合并为一个更大的网络,称为超网(supernet)\\
可以显著减少路由表中路由的数量,称为路由聚合(route aggregation)
	\item 网络地址转换(network address translation, NAT):\textbf{最节约地址的方法},将内部地址映射为外部地址的技术(可以扩展6w多倍),将私有地址映射为全局地址\\
NAT将内部源地址转换为外部地址\\
NAPT将端口号也加入NAT的映射中
\end{itemize}

地址解析协议(address resolution protocol, ARP)可以\underline{将IP地址映射为MAC地址}\footnote{也有将也有MAC映射为IP地址的协议}
\begin{itemize}
	\item ARP请求广播帧(谁的IP地址是XXX),ARP响应单播帧(返回MAC地址),IP地址与MAC地址的端口号相同
	\item 没有超时重传机制,超时没有收到响应则丢弃引发ARP查询的IP分组
	\item 源主机获得的映射结果缓存在ARP表中$\lrang{\text{IP address},\text{MAC address},\text{TTL}}$,TTL一般为2到20分钟
	\item 当收到ARP请求,目的主机会缓存源主机的映射,其他主机如果已缓存该映射,则会重置TTL
	\item 也可直接将映射加入ARP缓存,称为静态ARP映射,不会因超时而删除
	\item 源硬件地址和协议地址、目标协议地址都知道,但\textbf{目的硬件地址}不知
\end{itemize}
\begin{figure}[H]
	\centering
	\includegraphics[width=0.7\linewidth]{fig/ARP.PNG}
\end{figure}

DHCP协议(Dynamic Host Configuration Protocol)用于主机在加入网络时\textbf{动态租用}IP地址,用UDP,四个步骤如下
\begin{itemize}
	\item DHCP发现(discover)
	\item DHCP提供(offer)
	\item DHCP请求(request)
	\item DHCP确认(ACK)
\end{itemize}

因特网控制消息协议(Internet Control Message Protocol, ICMP)用于主机或路由器发布网络级别的控制消息,主要是出错/丢包后将信息发回给源主机(TTL减到0、不可达),如回响请求和答复消息(ping)、不可达消息、时间超时消息(原IP头部+原IP数据部份的头64B)、重定位消息

% Path MTU Discovery 用于寻找路径上最小MTU
% Traceroute 用于获得整条路径上的路由器IP地址
% 交换机端口镜像Port Mirroring, also known as SPAN (Switched Port Analyzer) 实现监听
% DHCP Relay(中继) 多个局域网共享一个DHCP服务器 https://www.netmanias.com/en/post/blog/6004/dhcp-network-protocol/what-is-a-dhcp-relay-agent

\subsection{路由协议}
有类网的路由选择算法:
利用数据包中的\textbf{目的地址}得到\textbf{目的网络号},然后查询\textbf{路由表}(routing table)/转发表(forwarding table)
\begin{itemize}
	\item 如果查询的结果为\textbf{直连网},则\textbf{下一跳(next hop)为空},直接把数据包从查出的接口转发到目的主机
	\item 否则,如果查询得到\textbf{下一跳}(路由器),则把数据包转发给下一跳
	\item 如果没有查到任何匹配项,则把数据包转发给\textbf{默认路由器}(也算查到)
	\item 如果没有设置默认路由,则\textbf{丢弃}该数据包
\end{itemize}

无类网的路由选择:
无类网的路由表里有子网掩码
\begin{itemize}
	\item 匹配方法: 目的IP地址 \& 子网掩码 $==$ 子网号
	\item 最长匹配原则(The longest match rule): 当有多条路由都匹配时选择子网掩码最长(1的长度)的路由,因为更详细
	\item 从IP数据报中获取目的地址,利用目的地址\textbf{查路由表}(同有类网)
	\item 直连网将数据报直接\textbf{封装成帧}发送,不需要目的地址(PPP)
	\item 以太网同样要封装成帧,从\textbf{路由表中查出下一跳}的IP地址,通过\textbf{ARP协议}获得\textbf{目的MAC地址},加入帧内
	\item 如没有下一跳,则直接取\textbf{IP数据报中的目的IP地址}(已经到达了)
	\item 发送时要遵守以太网协议(CSMA/CD)
	\item 每到一个路由器都将帧拆出来,再重新封装,目的和源MAC地址(上一跳路由器MAC地址)全要发生变化
\end{itemize}

路由表
\begin{itemize}
	\item 127.0.0.1是内部的,其他是外部的需要经过防火墙
	\item 接口不一样,因此要在路由器里写两项,一项内部一项外部
	\item 往外发,默认路由会匹配
	\item 选择跃点数(metric)小的一项,如果跃点数也相同,则两个接口都会发送
\end{itemize}

路由表可以由管理员手工建立,也可以由路由/路由选择协议(routing protocols)自动建立。
路由协议即自动建立路由表,包括网络号、子网号、下一跳、接口、开销等。
所建立的路由分别称为\textbf{静态路由}和\textbf{动态路由}。
默认路由和直连路由都是静态路由。

建路由表是记\textbf{最短路径}的下一跳。

整个因特网实际上由很多机构进行管理。每个机构管理自己的网络,它们有权决定采用什么协议和网络控制策略。这样在\textbf{同一个机构}管理下的网络称为一个\textbf{自治系统}(autonomous systems, AS)。因特网实际上是由很多自治系统构成的。
\begin{itemize}
	\item 用于在AS\textbf{内部}(Intra-AS)建立动态路由的路由协议称为\textbf{内部网关协议}(Interior Gateway Protocols, IGP)。例如,\underline{RIP协议}和\underline{OSPF协议}。一个AS通常运行单一IGP。
	\item 用于在AS\textbf{之间}(Inter-AS)建立动态路由的路由协议称为\textbf{外部网关协议}(Exterior Gateway Protocol, EGP)。例如,\underline{BGP协议}。
	\item 运行同一个IGP协议的连通区域也称为路由选择域(routing domain)。一个AS可以运行多个IGP协议,形成多个路由选择域。
\end{itemize}

加了网关相当于加了默认路由。

网络层在入口位置有防火墙,未查路由表就丢弃了。

路由算法(Routing algorithm):路由协议里用的算法, 由于两个路由器之间都有开销,可以建立一个图,找最短路径
\begin{itemize}
	\item 链接状态(link state):Dijstra
	\item 距离向量(distance vector):BellmanFord
\end{itemize}

\subsubsection{RIP协议}
路由信息协议(Route Information Protocol, RIP):距离向量算法的路由协议(问路),工作原理是\underline{采用邻居的路由表构造自己的路由表}。
\begin{itemize}
	\item 每\textbf{30秒}\footnote{太频繁会占用带宽}RIP路由器把它的整个路由表发送给邻居。具体实现时每个邻居会错开发送,30秒的时间也会随机变化一点。
	\item 初始时每个RIP路由器只有到直连网的路由,它们的距离为1。
	\item 到目的网络的距离以跳为单位。最大距离为15。距离16表示无穷大,即目的网络不可达。
\end{itemize}

具体算法:
当收到邻居发来的路由表(update packet),路由器将更新它的路由表$\lrang{\text{目的网络},\text{开销},\text{下一跳}}$:
\begin{enumerate}
	\item 收到路由的距离全部加1(即一跳的距离)
	\item 利用上述路由修改路由表:
	\begin{itemize}
		\item 把路由表中不存在的路由加入路由表
		\item 如果比路由表中的路由的距离更小,则更新该路由的距离为新距离,把下一跳改为邻居。如果原来的更大,则\textbf{也要进行改动},因为原来的路可能断了。即路由的下一跳送来的新路由,则必须修改距离。
	\end{itemize}
	\item 如果路由存在,就要重置失效定时器
\end{enumerate}
RIP路由表的每一项都有TTL(Time-To-Live),用失效定时器(invalid timer)计时,超时则让该路由失效

RIP协议存在的问题
\begin{itemize}
	\item 慢收敛:最短时间接近$0$(这样看更新时刻),最长时间$30(m-1)$,平均时间$15(m-1)$
	\item 计数到无穷:N1-R1通路断了,R1收到R2的路由表,更改自己的
\end{itemize}

RIP协议的技术
\begin{itemize}
	\item 水平分割(split horizon)技术:从一个接口学来的路由不会从该接口发回去;依然会计数到无穷,三角形R1断了,R1先发,R2后发
	\item 毒性反转(poison reverse)技术:当一条路由变为无效之后,路由器并不立即将它从路由表中删除,而是将其距离改为用16后广播给邻居,使邻居所拥有的该路由立即失效,而不是等待TTL到期后删除,以迅速消除路由环路,这种方法称为毒性反转,距离为16的路由称为毒化路由(poisoned route)
	\item 抑制技术(hold down):距离被改为无穷大的路由在一段短时间(180秒)内其距离不允许被修改
	\item 触发更新(triggered update):一旦出现路由变化将立即把变化的路由发送给邻居。原有的30秒发一次完整的路由表依然不变
\end{itemize}

\subsubsection{OSPF协议}
开放最短路径优先协议(Open Shortest Path First, OSPF)采用链路状态路由算法,可能是在大型企业中使用最广泛的内部网关协议:
\begin{itemize}
	\item 利用最短路径算法,如Dijkstra,求出一个节点(源节点)到所有其它节点的最短路径
	\item 利用这些最短路经上的下一个节点作为下一跳得到源节点的转发表(路由表)
\end{itemize}

OSPF协议的简单描述:
\begin{itemize}
\item 周期性地收集链路状态,并扩散给AS中的所有路由器
\item 用收到的链路状态建立整个AS的拓扑结构图
\item 利用Dijkstra算法计算到AS中所有网络的最短路径
\item 利用这些路径上的下一跳建立路由表
\end{itemize}