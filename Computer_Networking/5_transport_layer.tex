% !TEX root = main.tex

\section{传输层} % transport layer
传输层协议称为\textbf{端到端}或\textbf{进程到进程}的协议。 因特网的传输层可以为两个进程在\textbf{不可靠的网络层}上建立一条\textbf{可靠的逻辑链路},可以提供\textbf{字节流}传输服务,并且可以进行\textbf{流控制}和\textbf{拥塞控制}。

因特网的传输层有两个协议: UDP和TCP。
UDP协议提供不可靠的尽力服务, TCP协议提供可靠的字节流服务。
协议号:TCP为6,ICMP=1;UDP为17,IGMP=2

TCP/UDP通过数据段(segment)中的目的端口号(2B)确定将收到的数据段交给上层哪个进程。
\begin{itemize}
    \item 知名端口:0-1023,为提供知名网络服务的系统进程所用。    例如: 80-HTTP, 21-ftp Control, 20-ftp Data, 23-telnet,25-SMTP, 110-POP3,53-DNS
    \item 注册端口:1024-49151。在IANA注册的专用端口号,为企业软件所用。
    \item 动态端口:49152-65535,没有规定用途的端口号,一般用户可以随意使用。也称为私    用或暂用端口号。
\end{itemize}

\subsection{UDP协议}
用户数据报协议(User Datagram Protocol, UDP)只提供\textbf{无连接的不可靠的尽力服务}。发送给接收进程的数据有可能丢失,也有可能错序。

接收进程每次接收一个完整的数据报,如果进程设置的接收缓冲区不够大,收到的数据报将被截断。

数据报的内容
\begin{itemize}
    \item 总长度:整个UDP报文长度
    \item 源端口号和目的端口号:用于关联发送进程和接收进程
    \item 校验和:由伪IP头、 UDP头(校验和为0)和UDP数据形成。其中,伪IP头的协议号
    为17。如果发送方把校验和设置为0,接收方会忽略校验和。 UDP长度就是UDP头部的总长度。
\end{itemize}


\subsection{TCP协议}
传输控制协议(Transmission Control Protocol, TCP)为进程之间提供\textbf{面向连接的可靠的}数据传送服务。TCP为\textbf{全双工}协议。TCP提供\textbf{流控制}机制,即控制发送方的发送速度,使发送的数据不会淹没接收方。作为因特网的主要数据发源地,TCP还提供\textbf{拥塞控制}功能。

\begin{itemize}
\item 一个TCP连接提供可靠的字节流服务。字节流服务表示没有消息边界。例如,多次发送的数据可以放在一个数据段中传送且\textbf{不标识边界}。
\item 每个数据段的数据部分的最大长度(字节)不能超过MSS(Maximum Segment Size)。
\item 每个TCP连接可以由四元组唯一标识:\underline{源IP地址、源端口号、目的IP地址、目的端口号}
\item 客户端通过查路由表知道\textbf{IP地址},\textbf{端口号}自动选一个未用的
\end{itemize}

TCP数据报格式
\begin{itemize}
\item 头部长度以四个字节为单位
\item 校验和由伪IP头、 TCP头和TCP数据部分形成。其形成方法与UDP协议类似。
\item 紧急指针用于指出带外数据(out-of-band)的边界。标志URG为1时有效
\end{itemize}

TCP协议工作过程
\begin{center}
    \begin{tikzcd}
        \text{建立连接}\arrow{r} & \text{传送数据} & \text{释放连接}
    \end{tikzcd}
\end{center}
\begin{itemize}
    \item 建立连接:非对称活动,服务器一直在等,客户向服务器呼叫
    \item 传送数据:全双工方式
    \item 释放连接:对称活动,可由任何一方发起
\end{itemize}

三次握手建立连接
\begin{itemize}
    \item SYN,Seq\#=x
    \item (Cli)SYN+ACK,Seq\#=y; Ack\#=x+1
    \item ACK,Ack\#=y+1
\end{itemize}
这里确认号含义与数据链路层不同,这里是期待接收的序号

四次握手关闭连接
\begin{itemize}
    \item FIN,Seq\#=x
    \item (Cli)ACK,Ack\#=x+1
    \item (Cli)FIN,Seq\#=y
    \item ACK,Ack\#=y+1
\end{itemize}
可以合并中间两次握手(ACK和FIN)
