% !TEX root = main.tex

\section{应用层}

\section{其他内容}
\subsection{无线局域网}
无线局域网(WiFi)802.11

不同标准MAC层一样,改进的是物理层(发、编码、怎么收)

无线信道特点:信号衰减快

无固定设施(自组织):独立基本服务集(Basic Service Set, BSS),不需要路由器,可以几台PC直接连

有固定设施:有个接入点(Access Point, AP),有路由器,连成基本服务集;
扩展的服务集(ESS),完全无缝连接,自动迁移到下一个AP

网络名就是SSID

MAC层一般用分布协调功能(DCF)
\begin{itemize}
\item 原子操作:发完数据等待28$\mu$s后才发ACK
\item CSMA/CA(Carrier Avoidance)算法:尽可能少冲突,尽可能少重传;即使空闲也要随机等一段时间
\end{itemize}

翻译网桥:一种帧转换为另一种帧

5GHz WiFi干扰小

跟以太网不同,WiFi在物理层是有格式的,远距离会换编码,降速进行传输

MIMO技术:多发射天线和多接收天线形成多个空间数据流

\subsection{网络管理}
SNMP协议:Simple Network Management Protocol

\subsection{广域网}
ADSL不是用语音传,是用数据传的

现在都不用电话线连

\subsection{软件定义网络}
软件定义网络(Software Defined Network, SDN):原来路由器交换机都是固定电路,但现在可编程,未来网络发展方向

\begin{center}
\begin{tikzcd}
\text{应用}\arrow{d}\\
\text{控制器}\arrow{d}\\
\text{转发器}
\end{tikzcd}
\end{center}

控制器相当于一个电脑,用来修改下面(几十台)转发器的路由表

最大的好处在于\textbf{虚拟化}

\subsection{多媒体网络}
插值解决丢包问题,因为是连续变化的

传语音数据是用RTP协议传的,基于UDP协议

SIP协议用来做网络会议

H.232协议则更加完整

令牌桶:没传的时候放令牌,有传的时候用光令牌

MPLS加标签实现VPN,沿着同一条路径走过去