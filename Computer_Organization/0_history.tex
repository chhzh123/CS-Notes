% !TEX root = main.tex

\section{计算机历史}
\subsection{计算模型}
\begin{itemize}
	\item 图灵机(1936)
	\item 冯诺依曼体系结构(1945)\footnote{非冯诺依曼体系结构:并行计算、量子计算、生物计算} --- 存储程序原理(\textbf{运算器}为中心)\\
	计算机采用二进制表示机器指令和数据,按照程序指令\textbf{顺序}执行
\begin{center}
\begin{tikzcd}
& & \text{存储器}\arrow{d} & & \\
\quad\arrow{r} & \text{输入设备}\arrow{r} & \text{运算器}\arrow{r}\arrow{d}\arrow{u} & \text{输出设备}\arrow{r} & \quad\\
& & \text{控制器}\arrow{u}\arrow{lu}\arrow{ru}\arrow[bend left]{uu} & &
\end{tikzcd}
\end{center}
而现在由于计算不是瓶颈,存储访问成为了瓶颈,故现代微机以\textbf{存储器}为中心
\begin{center}
\begin{tikzcd}
& & \text{运算器}\arrow{d} & & \\
\quad\arrow{r} & \text{输入设备}\arrow{r} & \text{存储器}\arrow{r}\arrow{d}\arrow{u} & \text{输出设备}\arrow{r} & \quad\\
& & \text{控制器}\arrow{u}\arrow{lu}\arrow{ru}\arrow[bend left]{uu} & &
\end{tikzcd}
\end{center}
\end{itemize}

\subsection{物理器件与大规模集成}
\begin{center}
\begin{tabular}{|c|c|c|c|}
\hline
% 年份 & 姓名 & 事件 & 备注 \\
1904 & 弗莱明(Fleming) & 二极管 & \\\hline
1907 & 德福雷斯特(De Forest) & 三极管 & \\\hline
1938 & 香农(Shannon) & 布尔代数与二值电子器件 & 奠定数字电路基石 \\\hline
1946 & & 第一台通用计算机ENIAC & \\\hline
1947 & \begin{tabular}{c}布莱顿(Brattain)\\
巴丁(Bardeen)\end{tabular} & 点接触晶体管 & \\\hline
1949 & 肖克利(Shockley) & 结型晶体管(1949) & 1956诺贝尔奖\\\hline
1950 & & 二进制和存储程序EDVAC & 实现冯诺依曼设想 \\\hline
1958 & Jack Kilby & 集成电路 & 2000诺贝尔奖 \\\hline
1965 & Moore & 摩尔定律 & \begin{tabular}{c}
在价格不变的情况下,每18个月芯片上\\
晶体管数目翻倍,性能也提升一倍
\end{tabular}\\\hline
1971 & Intel Co & 第一款微处理器4004 & \\\hline
\end{tabular}
\end{center}

\subsection{计算机的发展}
\subsubsection{单处理器(1971-2002)}
\begin{proposition}[安迪-比尔定律]
Andy gives, Bill takes away. 安迪是原Intel CEO,比尔是原微软CEO,硬件厂商靠软件开发商用光自己提供的硬件资源得以生存
\end{proposition}
性能提升主要手段
\begin{itemize}
	\item 提升工作主频(KHz$\to$GHz):生产工艺不断进步,流水线技术
	\item 发掘指令级并行(ILP)
\end{itemize}
但遇到频率墙和功耗墙
\[\text{功耗(power)}\propto 1/2\times\text{CMOS电容}\times\text{电压}^2\times\text{转换(01)频率}\]
\par
2004年,Intel放弃4GHz Pentium4芯片开发,因无法解决散热问题,通过加快主频提升处理器性能的路走到尽头

\subsubsection{多核处理器}
采用多核处理器不过是将硬件的问题丢到软件\footnote{“向多核的转变并不是因为我们在软件或体系结构技术上取得了中大突破而带来的。相反,这种转变是当单处理器体系结构发展遇到了难以克服的巨大障碍时,我们被迫作出的一种选择。”---Kurt Keutzer (UCB), \emph{The Landscape of Parallel Computing Research: A View from Berkeley}}
\begin{theorem}[阿姆达尔(Amdahl)定律]
\label{thm:amdahl}
\[\text{改进后的执行时间}=\text{受改进影响部分的执行时间}/\text{改进提高的倍数}+\text{不受影响的执行时间}\]
\[S_A=\frac{1}{s+(1-s)/N},\]
\end{theorem}
对计算机系统的某个部分采用并行优化措施后所获得的计算机性能的提高是有上限的,上限由串行部分所占的比例决定
\begin{theorem}[古斯塔夫森(Gustafson)定律]
\[S_G=(s'+p'\times N)/(s'+p')=N+(1-N)\times s',\]
其中,$s'$和$p'$为程序串行部分与可并行化部分在并行系统上执行的时间占总时间的比例,$N$为处理器数量,简便起见设总时间$s'+p'=1$
\end{theorem}
打破Amdahl定律\textbf{问题规模不变}的假设,任何足够大的任务都可以被有效地并行化,只要问题规模可扩展,并行所带来的加速比就可以扩展