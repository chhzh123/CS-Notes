% !TEX root = main.tex

\section{导论}
\subsection{基本概念}
表示计算机通信带宽时
\begin{center}
\begin{tabular}{ccccccc}\hline
KB(yte) & MB & GB & TB & PB & EB & ZB\\\hline
$10^3$ & $10^6$ & $10^9$ & $10^{12}$ & $10^{15}$ & $10^{18}$ & $10^{21}$\\\hline
\end{tabular}
\end{center}
表示计算机存储二进制时
\begin{center}
\begin{tabular}{ccccccc}\hline
KiB(yte) & MiB & GiB & TiB\\\hline
$2^{10}$ & $2^{20}$ & $2^{30}$ & $2^{40}$\\\hline
\end{tabular}
\end{center}
\begin{itemize}
	\item 位(bit/b):计算机处理、存储、传输信息的最小单位
	\item 字节(Byte/B) $1\text{ Byte}=8\text{ bit}$:现代计算机主存按字节编制,字节是最小可寻址单位
	\item 字(Word):表示被处理信息的单位,用来度量数据类型的宽度\footnote{字长是指CPU中数据通路的宽度,等于CPU内部总线的宽度或运算器的位数或通用寄存器的宽度;字和字长的宽度可以一样,也可以不同,通常是字节的整数倍}
\end{itemize}
\par 一台32位的电脑,一个字等于4个字节,字长为32位;若字长为16位,则一个字等于2字节.

\subsection{程序到电子信号}
\begin{center}
\begin{tikzcd}
\text{高级语言}\quad\arrow{r}{\text{预编译、编译}} & \quad\text{汇编语言}\arrow{r}{\text{汇编}} & \text{机器语言}\arrow{r}{\text{链接}} & \text{可执行文件}\arrow{d}{\text{加载}}\\
& & \text{电路上的电信号}\quad & \quad\text{二进制机器指令流(存储器)}\arrow[swap]{l}{\text{CPU取指译码}}
\end{tikzcd}
\end{center}
计算机内部工作过程:逐条执行加载到内存中的二进制机器指令流的过程
指令执行分为两个阶段,周期性重复性进行:
\begin{itemize}
	\item 取指阶段:CPU从内存中读取指令,程序计数器(PC)保存要被要被取出的\textbf{下一条}指令的地址,除非遇跳转指令,否则都加一个增量\footnote{程序计数器(Program Counter)是一个实际存在的寄存器吗? - Belleve的回答 - 知乎 \url{https://www.zhihu.com/question/22609253/answer/21965180} PC每次增加\textbf{一条指令的长度/寻址粒度},在MIPS中一条指令长4字节,寻址粒度1字节,故每次PC加4;而x86体系指令长度不定,每次增加量会变化}
	\item 执行阶段:对取出的指令译码后执行
\end{itemize}

\subsection{功能部件}
\subsubsection{中央处理器(CPU)}
核心组成:
\begin{itemize}
	\item 控制单元(Control unit):对指令进行译码,产生控制信号
	\item 数据通路(Datapath):完成指令执行,ALU+寄存器
\end{itemize}
寄存器分类:
\begin{itemize}
	\item 通用寄存器(GRS, General Register Set):存放操作数和中间结果
	\item 程序计数器(PC, Program Counter):存放下条要执行的指令
	\item 指令寄存器(IR, Instruction Register):存放当前指令
\end{itemize}

\subsubsection{存储器}
层次化结构(Hierarchies)
\begin{itemize}
	\item 内存(Primary):高速缓存(Cache)、主存(MM, Main Memory)
	\item 外存(Secondary):磁盘、光盘、闪存等
\end{itemize}

\subsection{计算机结构的八个想法}
\begin{enumerate}
	\item 摩尔(Moore)定律:集成电路资源每$18-24$个月翻倍
	\item 抽象(abstraction):简化设计
	\item 加速常用操作(Make common case fast):见定理\ref{thm:amdahl}
	\item 并行(parallelism)
	\item 流水线(pipelining)
	\item 预测(prediction)
	\item 内存等级制(hierarchy)
	\item 冗余实现可靠性(redundancy):检测故障及解决
\end{enumerate}

\subsection{性能评价}
\label{subsec:performance}
\[\text{计算机的性能(Performance)}=1/\text{执行时间(Execution time)}\]
按照单位(量纲)进行换算即可
\[\begin{aligned}
\text{CPU执行时间(s)}&=\text{执行程序所需CPU时钟周期(cyc)}\times\text{时钟周期s/cyc)}\\
&=\text{指令数目(ins)}\times\text{CPI(cyc/ins)}\times\text{时钟周期(s/cyc)}
\end{aligned}\]
程序性能对执行事件的影响:
\begin{itemize}
	\item 算法、编程语言、编译器都对指令数和CPI产生直接影响
	\item 指令集体系结构则同时对指令数、CPI、时钟频率产生影响
\end{itemize}