% !TEX root = main.tex

\section{指令系统}
IS处在软件和硬件的交界面上,能同时被硬件设计者和系统程序员看到\\
从硬件设计者角度看
\begin{itemize}
	\item IS为CPU提供功能需求
	\item IS设计目标:易于硬件逻辑设计
\end{itemize}
从系统程序员角度看
\begin{itemize}
	\item 通过IS使用硬件资源
	\item IS设计目标:易于编写编译器
\end{itemize}
IS设计的好坏决定了计算机的性能和成本

\subsection{概述}
\begin{center}
\begin{tabular}{|m{8cm}|m{8cm}|}
\hline
\multicolumn{1}{|c|}{复杂指令集计算机} & \multicolumn{1}{c|}{精简指令集计算机} \\\hline
CISC, Complex Instruction Set Computer & RISC, Reduce Instruction Set Computer\\\hline
出现较早,大而全 & 小而精 \\\hline
指令周期长,专用寄存器,微程序控制,难编译优化生成高效目标代码,效率低(二八定律,简单指令使用频率高) & 指令周期短,\textbf{大量通用寄存器},组合逻辑电路控制,优化编译系统,简单寻址方式\\\hline
变长指令字 & 定长指令字 \\\hline
借鉴思想 & 现在多用 \\\hline
x86 & ARM, MIPS, SPARC \\\hline
\end{tabular}
\end{center}

\subsection{指令格式}
指令一般由\textbf{操作码}和\textbf{地址码}(包括操作数和寻址方式)决定

\subsubsection{操作码设计}
\begin{itemize}
	\item 关注程序代码长度时:变长指令字、变长操作码
	\item 关注性能时:定长指令字、定长操作码
\end{itemize}
扩展操作码:使用频率高的指令用短的操作码,频率低的指令用长的操作码
\begin{itemize}
	\item 零地址指令:空操作、停机、堆栈
	\item 一地址指令:取反、取负、累加器
	\item 二地址指令(最常用):分别存放双目运算中两个源操作数地址,并将其中一个地址作为结果地址
	\item 三地址指令(RISC):双目运算中两个源操作数地址和一个结果地址
\end{itemize}
\begin{example}
36位长的指令系统,设计一个扩展操作码,使之能表示以下指令
\begin{itemize}
	\item 7条两个14位地址和一个5位地址的指令
	\item 600条一个14位地址和一个5位地址的指令
	\item 100条无地址指令
\end{itemize}
\end{example}
\begin{analysis}
各指令的范围如下
\begin{itemize}
\item 三地址指令:$36=3(op)+14(ad1)+14(ad2)+5(ad3)$\\
$000\thicksim 110\,((0)_{10}\thicksim(6)_{10})$共7条,$111$为扩展码
\item 二地址指令:$36=3+14(op)+14(ad1)+5(ad2)$\\
$00,0000,0000,0000\thicksim 00,0010,0101,0111\,((0)_{10}\thicksim(599)_{10})$共600条,$111,00,0010,0101,1000$为拓展码
\item 零地址指令:$36=3+14(op)+12+7$\\
最后7位$000,0000\thicksim 110,0011\,((0)_{10}\thicksim(99)_{10})$共100条
\end{itemize}
\end{analysis}

\subsubsection{寻址方式}
通常特指操作数寻址(对应的是指令寻址,PC增值和跳转),目的是扩大访存范围,提高访问数据的灵活性和有效性
\begin{center}
\begin{tabular}{|m{3cm}|m{8cm}|m{4cm}|}
\hline
立即数寻址 & 直接给出操作数本身,无需访存快速,操作数大小受地址字段长度限制,大量使用 & MOV AX, 1000H\\\hline
存储器直接寻址 & 操作数在存储器中,直接给出操作数在存储器中的地址,寻址空间受指令地址字段长度限制,较少使用 & MOV AX, [1000H]\\\hline
存储器间接寻址 & 存储器中的内容是操作数的地址,需二次寻址 & \\\hline
寄存器直接寻址 & 直接给出寄存器编号,无需访存速度快,地址范围有限,可用通用寄存器较少,使用最多,提高性能常用手段 & MOV AX, BX\\\hline
寄存器间接寻址 & 寄存器中的内容是操作数的地址,二次寻址 & MOV AX, [BX]\\\hline
相对寻址(偏移) & 相对\textbf{当前指令(PC)}位移量为A的单元,跳转指令 & EA=(PC)+A\\\hline
基址寻址(偏移) & 相对基址(B)位移量为A的单元,\textbf{OS页面(重定位)},面向系统,程序逻辑空间与存储器物理空间的无关性 & EA=(B)+A \\\hline
变址寻址(偏移) & 相对形式地址A(数组基址)位移量为(I)的单元,X为数组元素大小,面向用户 & EA=(I)+A, I=(I)$\pm$X \\\hline
堆栈寻址 & 从寄存器到堆栈或反过来,指令短 & EA=栈顶(SP)\\\hline
复合寻址 & 间接寻址+相对/变址寻址 & 间接相对EA=(PC)+A, 相对间接EA=((PC)+A)\\\hline
\end{tabular}
* (X)代表X地址/寄存器内的内容,如((X))代表寄存器间接寻址
\end{center}
寻址方式的确定
\begin{itemize}
	\item 操作码中给定寻址方式:MIPS
	\item 专门寻址方式:x86(0-1字节)
\end{itemize}
\begin{itemize}
	\item 指令总数取决于操作码位数
	\item 寄存器决定了编码位数
	\item 地址寄存器(MAR)的位数取决于主存地址空间大小
	\item 数据寄存器(MDR)取决于机器字长
\end{itemize}
注意看是按\textbf{字}编址还是按\textbf{字节}编址\\
但从80年代开始,几乎所有机器都采用字节编址(byte addressing)

\subsection{数据表示}
\subsubsection{进制(system)}
\[\text{进位计数制}=\text{基数}+\text{位权}\]
二进制(binary)、八进制(octonary)、十进制(decimal)、十六进制(hexadecimal)
\begin{itemize}
	\item 十进制转二进制:整数部分除以2取余,小数部分乘2取整
	\item 二进制转八进制:从整数最低位开始,三位三位统计
	\item 二进制转十六进制:从整数最低位开始,四位四位统计
\end{itemize}

\subsubsection{符号数}
\begin{enumerate}
\item 二进制(真值$\to$机器数)
\begin{itemize}
	\item 符号数值(sign-magnitude)形式(原码):首位0为正数,1为负数,将符号位一起考虑有以下表示
	\[A=\begin{cases}
	A & A\in[0,2^{n-1})\\
	2^{n-1}-A & A\in(-2^{n-1},0]
	\end{cases}\]
	\item 反码(1's complement)$\ssim A$:除符号位不变,其他位取反;同理小数
	\[\ssim A=\begin{cases}
	A & A\in[0,2^{n-1})\\
	(2^n-1)+A & A\in (-2^{n-1},0]
	\end{cases}\]
	\begin{analysis}
	反码是全1的补数
	\[\ssim A=(2^n-1)-A=(11\ldots 1)_2-A_2\]
	即在$\mod 2^{n}-1$意义下的运算
	\end{analysis}
	\item 补码(2's complement)$[A]_c$:反码+1,按照原来十进制转二进制方法即可得对应有符号十进制数,由于没有正负0,故表示的数多了一位,补码的补码为原码;同理小数
	\[[A]_c=\begin{cases}
	A & A\in[0,2^{n-1})\\
	2^n+A & A\in \textcolor{red}{[-2^{n-1},0)}
	\end{cases}\]
	\begin{analysis}
	补码的设计非常关键,理解补码的由来对于后面的四则运算有着很大帮助。
	之所以要有补码,是因为希望能做到\textbf{减去一个数等于加上某个数},而这在模$2^n$的意义下即可实现。\\
	那么就有
	\[[A]_c=2^n-A=((2^n-1)-A)+1=\ssim A+1\]
	即在$\mod 2^n$意义下的运算,以4位二进制为例
	\[(5)_{10}=(0101)_2\implies (5)_c=2^4-5=11=(10000)_2-(0101)_2=(1011)_2\]
	由$[A]_c$求$[-A]_c$要连同符号位一起取反加1
	\end{analysis}
	\item 移码(bias)$[A]_b$:补码的符号位取反,引入目的是保证浮点数的机器零
	\[[A]_b=A+2^{n-1}\,,A\in(-2^{n-1},2^{n-1})\]
	\begin{analysis}
	相当于把正数移到负数的部分,负数移到正数的部分\\
	注意区别移码的定义($2^{n-1}$)和具体浮点数阶码($2^{n-1}-1$)的实施\footnote{原因可见\url{https://blog.angularindepth.com/the-mechanics-behind-exponent-bias-in-floating-point-9b3185083528}}
	\end{analysis}
\end{itemize}
\item 十进制:
\begin{itemize}
	\item ASCII码
	\item BCD码:四位表示一位十进制数
\end{itemize}
\end{enumerate}

\subsubsection{小数表示}
\begin{itemize}
\item 定点数,首位符号位
\begin{itemize}
	\item 定点整数:小数点固定在最低位右边,$0\leq|x|\leq 2^n-1$
	\item 定点小数:小数点固定在\textbf{数值部分}最高位的左边,$0\leq|x|\leq 1-2^{-n}$
\end{itemize}
\item 浮点数(IEEE 754)
\begin{center}
\begin{tabular}{|c|c|c|}\hline
符号S,1 & 阶码E,8,移码 & 尾数F,23,原码
\\\hline
\end{tabular}
\end{center}
规格化数,即令小数点前面必为1,隐含表示\\
移码偏置常数为127(单精度)、1023(双精度),作用为简化比较
\[(-1)^S\times1.F\times 2^{E-127}\]
\end{itemize}
\begin{example}
\[1\;0110\;1001\;0001=1.0110\;1001\;0001\times 2^{(12)_{10}}\]
\par 指数:$12+127=139\to 1000\;1011$
\par 尾数:$011\;0100\;1000\;1000\;0000\;0000$ 左对齐,因为有小数点
\begin{center}
\begin{tabular}[htbp]{|c|c|c|}
\hline
符号S & 指数E(exponent) & 尾数F(mantissa)\\\hline
$0$ & $1000\;1011$ & $011\;0100\;1000\;1000\;0000\;0000$\\\hline
1位 & 8位 & 23位\\\hline
\end{tabular}
\end{center}
\end{example}
特殊值表示
\begin{center}
\begin{tabular}{ccc}
阶码(移码) & 尾数 & 数据类型\\\hline
$1\thicksim 254$ & 任何值 & 规格化数\\
0 & 0 & 0\\
0 & 非零数 & 非规格化数\\
255 & 0 & $+\infty/-\infty$\\
255 & 非零数 & NAN, Not A Number
\end{tabular}
\end{center}
单精度可表示范围$[10^{-38},10^{+38}]$,双精度$[10^{-308},10^{+308}]$

\subsubsection{C语言数据类型}
C语言中数据类型大小以字节为单位
\begin{center}
\begin{tabular}{|c|c|c|}\hline
声明 & 数据长度(32位机,Byte)\\\hline
char & 1 \\\hline
short & 2 \\\hline
int & 4 \\\hline
long & 4 \\\hline
float & 4 \\\hline
double & 8 \\\hline
\end{tabular}
\end{center}


\subsection{数据存储}
\begin{itemize}
	\item 大端方式(Big Endian):最高有效位(MSB)所在地址为数的地址,MIPS,Photoshop、JPEG
	\item 小端方式(Little Endian):最低有效位(LSB)所在地址为数的地址,x86,GIF、RTF
\end{itemize}
字节交换:大端小端互换\\
数据边界对齐:减少访存次数,按字地址对齐(4的倍数,二进制后两位为0)


\subsection{数据纠错}
冗余校验思想,增添校验位
\begin{itemize}
	\item 奇偶校验码:$P=b_{n-1}\oplus b_{n-2}\oplus\cdots\oplus b_0\oplus 1$与结果的$P'$再取异或,为1则奇数位错\\
	只能发现奇数位出错,且不具有纠错能力
	\item 海明码
	\item 循环码
\end{itemize}


\subsection{MIPS指令系统}
所有指令都是32位宽,按字地址对齐
\begin{table}[htbp]
\caption{三种指令格式}
\begin{tabular}{|c|c|c|c|}
\hline
R-Type & 用于寄存器 sub rd,rs,rt & 寄存器寻址 & $\begin{aligned}32&=6(op)+5(rs)+5(rt)\\
&+5(rd)+5(shamt)+6(funct)\end{aligned}$ \\\hline
I-Type & \begin{tabular}{ll}
运算指令: & ori rt,rs,imm16\\
存储指令: & lw rt,rs,imm16\\
条件分支: & beq rs,rt,imm16
\end{tabular} & 立即数、基址(PC)寻址 & $\begin{aligned}32&=6(op)+5(rs)\\&+5(rt)+16(\text{immediate})\end{aligned}$ \\\hline
J-Type & 无条件跳转 j target & 按字对齐的伪直接寻址 & $32=6(op)+26(\text{target address})$ \\\hline
\end{tabular}
\end{table}
注:J-Type中伪直接寻址含义:PC高4位拼上26位直接目标地址,最后添2个0(相当于乘4)为32位目标地址\\
指令字段含义
\begin{itemize}
	\item 操作码(op)
	\item 第一个源操作数寄存器(rs),5位32个
	\item 第二个源操作数寄存器(rt)
	\item 结果寄存器(rd)
	\item 移位指令的位移量(shamt)
\end{itemize}
操作码的不同编码定义了不同的含义,若操作码相同时,再用不同功能码区分


\subsection{MIPS语法表}
\begin{table}
\centering
\caption{通用寄存器}
\begin{tabular}{|c|c|c|}
\hline
寄存器 & 名称 & 用途\\\hline
\$0 & \$zero & 常量0\\\hline
\$1 & \$at & 保留给汇编器\\\hline
\$2-\$3 & \$v0-\$v1 & 函数调用返回值\\\hline
\$4-\$7 & \$a0-\$a3 & 函数调用参数\\\hline
\$8-\$15 & \$t0-\$t7 & 临时变量\\\hline
\$16-\$23 & \$s0-\$s7 & 保存(saved)\\\hline
\$24-\$25 & \$t8-\$t9 & 其他临时变量\\\hline
\$26-\$27 & \$k0-\$k1 & 为OS保留\\\hline
\$28 & \$gp & 全局指针(Global Pointer)\\\hline
\$29 & \$sp & 堆栈指针(Stack Pointer)\\\hline
\$30 & \$fp & 帧指针(Frame Pointer)\\\hline
\$31 & \$ra & 函数调用返回地址(return address)\\\hline
\end{tabular}
\end{table}
三个特殊寄存器:HI、LO、PC\par
存储器数据指定:
\begin{itemize}
	\item 31个32位通用寄存器(GPR),零寄存器$r_0=0$
	\item 32个32位浮点寄存器($f_0-f_{31}$)
	\item 3个特殊寄存器:HI、LO、PC
	\item 32位机$\to$可访问空间 $2^{32}bytes=4GB$(按字节编址)
	\item 大端方式
	\item 只可通过load/store指令访问存储器(不像x86算术运算也可直接访存)
	\item 访存地址通过一个32位寄存器内容加16位偏移量(有符号补码)得到
	\item 数据按边界对齐
\end{itemize}
函数调用:
\begin{itemize}
	\item 若过程调用参数多余4个,返回值超过2个,则需要保存到寄存器的栈区中
	\item 上方是调用程序压栈(pushed by caller),保存在调用程序帧中,返回地址上方连续区域
	\item 被调程序使用(used by callee),相对于帧指针(\%ebp)的访问
	\item 栈向下增长至堆,堆向上增长至栈
\end{itemize}

\begin{table}[htbp]
\centering
\caption{MIPS指令}
\begin{tabular}{|c|l|l|l|}
\hline
\multirow{6}*{算术} & 加法    & add \$s1,\$s2,\$s3 & \$s1=\$s2+\$s3 \\
\cline{2-4}      & 立即数加法 & addi \$s1,\$s2,20 & \$s1=\$s2+20 \\
\cline{2-4}      & 乘法    & mult \$s2,\$s3 & HI,LO=\$s2 \$s3 \\
\cline{2-4}      & 除法    & div \$s2,\$s3 & LO=\$s2/\$s3,HI=\$s2 mod \$s3 \\
\cline{2-4}      & 从高位移出 & mfhi \$s1 & \$s1=HI \\
\cline{2-4}      & 从低位移出 & mflo \$s1 & \$s1=LO \\
\hline
\multirow{3}[6]{*}{逻辑} & 与     & and \$s1,\$s2,\$s3 & \$s1=\$s2\&\$s3 \\
\cline{2-4}      & 立即与   & andi \$s1,\$s2,20 & \$s1=\$s2\&20 \\
\cline{2-4}      & 左移    & sll \$s1,\$s2,20 & \$s1=\$s2$<<$20 \\
\hline
\multirow{4}[8]{*}{传输} & 存字    & sw \$s1,500(\$s2) & (\$s2+500)=\$s1 \\
\cline{2-4}      & 存半字   & sh \$s1,500(\$s2) & (\$s2+500)=\$s2 \\
\cline{2-4}      & 存位    & sb \$s1,500(\$s2) & (\$s2+500)=\$s3 \\
\cline{2-4}      & 读字    & lw \$s1,500(\$s2) & \$s1=(\$s2+500) \\
\hline
\multirow{3}[6]{*}{条件分支} & 分支等于  & beq \$s1,\$s2,25 & if (\$s1==\$s2) goto PC+4+100 \\
\cline{2-4}      & 分支不等  & bne \$s1,\$s2,25 & if (\$s1!=\$s2) goto PC+4+100 \\
\cline{2-4}      & 分支小于  & slt \$s1,\$s2,\$s3 & if (\$s2$<$\$s3) \$s1=1;else \$s1=0 \\
\hline
\multirow{3}[6]{*}{跳转} & 跳转    & j 10000 & goto 10000 \\
\cline{2-4}      & switch/函数返回 & jr \$ra & goto \$ra \\
\cline{2-4}      & 函数调用  & jal 10000 & \$ra=PC+4;goto 10000 \\
\hline
\end{tabular}%
\end{table}