% !TEX root = main.tex

\section{计算机的运算}
\subsection{基本运算}
\subsubsection{位运算}
位运算针对二进制数,逻辑运算针对表达式的值
\begin{center}
\begin{tabular}{|c|c|c|}
\hline
\multirow{2}*{无符号数} & 逻辑左移 & 高位移出,低位补0\\
\cline{2-3} & 逻辑右移 & 低位移出,高位补0\\
\hline
\multirow{2}*{有符号数} & 算术左移 & 高位移出,低位补0\\
\cline{2-3} & 算术右移 & 低位移出,高位补\textbf{符号位}\\
\hline
\end{tabular}
\end{center}
移位符号\verb'<<'和\verb'>>'不区分算术还是逻辑移位,只由参与运算的数值决定\\
算术左移溢出判断:若移出的位不等于新的符号位,即$CF\oplus SF=1$,则溢出

\subsubsection{位扩展和位截断}
\begin{itemize}
	\item 位扩展:如\verb'float'变\verb'double';数据存入寄存器时也要扩展,\verb'lb'
	\begin{itemize}
		\item 无符号数:0扩展,即前面补0
		\item 有符号整数:符号扩展,即前面补符号
	\end{itemize}
	\item 位截断:如\verb'double'变\verb'int';强行丢弃长数的高位,可能溢出或数据不正确
\end{itemize}

\subsubsection{加减法}
\begin{center}
\begin{tabular}{|c|c|c|}\hline
原码 & 补码 & 移码\\\hline
符号与数值单独运算 & 符号与数值一起运算 & 符号与数值一起运算\\\hline
同号相加进位溢出 & \begin{tabular}{c}变形两位补码\\01正溢出,10负溢出\end{tabular} & 两加数和和数符号都相同则溢出\\\hline
$A\pm B=A\pm B$
&$\begin{aligned}
[A+B]_c&=[A]_c+[B]_c\\
[A-B]_c&=[A]_c+[-B]_c
\end{aligned}$
&$\begin{aligned}
[A]_b+[B]_b&=[A+B]_{\textcolor{red}{c}}\\
[A]_b-[B]_b&=[A]_b+[-[B]_b]_c=[A-B]_{\textcolor{red}{c}}
\end{aligned}$\\\hline
浮点数尾数 & 定点数 & 浮点数阶码\\\hline
\end{tabular}
\end{center}
% 符号位位1,结果为补码

注意进位和溢出的区别,在模$2^n$意义下,加负数进位相当于回到原点;而只有同号运算才可能溢出

C不考虑溢出的异常处理
\begin{itemize}
	\item unsigned整型溢出:模运算
	\item signed整型溢出:undefined behavior
	\item MIPS上的C编译器会选用无符号的算术运算指令,如\verb'addu'、\verb'addiu'、\verb'subu'
\end{itemize}

全加器实现
\[\begin{aligned}
S_i&=A_i\oplus B_i\oplus C_{i-1}\\
C_i&=(A_i\oplus B_i)C_{i-1}+A_iB_i
\end{aligned}\]

并行加法器:
\begin{itemize}
	\item 若$A_iB_i=1$,则$C_{\text{out}}=1$与$C_{\text{in}}$无关(进位生成)
	\item 若$A_i+B_i=1$,则$C_{\text{out}}=1$与$C_{\text{in}}$相同(进位传递)
\end{itemize}

组内并行,组间串或并行

BCLA(成组先行进位)芯片


\subsubsection{乘法}

\subsubsection{除法}