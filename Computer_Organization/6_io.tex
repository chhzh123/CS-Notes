% !TEX root = main.tex

\section{输入输出系统}
\subsection{IO接口}
IO接口是主机与IO设备之间数据交换的界面,它屏蔽了IO设备的差异,提供了一致的访问界面。
功能如下
\begin{enumerate}
	\item 数据格式(串并)转换和电平变换
	\item 数据缓存(速度不匹配)
\end{enumerate}
IO端口:IO接口中的各类寄存器\\
寄存器编址
\begin{itemize}
	\item 统一编址,与主存统一编址,可以用访存指令去访问外设中的存储器
	\item 独立编址,对IO端口单独编号,需要专门的IO指令
\end{itemize}
% IO设备的连接

\subsection{磁盘存储器}
磁盘组织:磁道、扇区、柱面\\
道(track)密度:垂直于磁道方向上(半径方向)单位长度磁介质所容纳的磁道数\\
位(bit)密度:单位长度磁道上所能记录的二进制信息位数
面(surface)密度:单位面积上记录的二进制信息位数=道密度$\times$位密度\\
存取时间$t$=寻道时间$t_s$+旋转等待时间$t_w$+数据传输时间($t_{WR}$)\\
最大旋转延迟=1/磁盘转速*60s/1min\\
平均旋转延迟=最大旋转延迟/2

\subsection{闪存存储器}
电可擦写、可编程只读存储器(EEPROM)
\begin{itemize}
	\item NOR:随机,可以直接按字节访问,主要用于存储程序代码(code)
	\item NAND:块级IO访问,主要用于存储数据(data)
\end{itemize}
读快、写慢(块擦)\\
通过损耗均衡(wear leveling)减少块的磨损\\
固态盘SSD即为闪存

% 光存储器

\subsection{IO控制方式}
\subsubsection{程序查询方式}
IO完全由CPU指令控制,数据传输再CPU的寄存器与外设及其接口的数据缓冲寄存器之间进行,IO不直接访问内存
\par 完全串行工作

\subsubsection{程序中断/中断驱动方式}
由外设主动通知CPU,可以处理异常事件
\begin{enumerate}
	\item 当外设准备好时,向CPU发中断请求
	\item CPU响应后,中止线性程序执行,转入“中断服务程序”进行输入/输出操作
	\item 中断服务程序执行完,CPU返回程序断点继续执行,外设和CPU并行工作
\end{enumerate}
中断响应的条件
\begin{itemize}
	\item CPU处于开中断状态
	\item 在一条指令\textbf{执行完}(区别“异常”是在指令执行过程中)
	\item 至少要有一个未被屏蔽的中断请求
\end{itemize}
中断响应过程
\begin{itemize}
	\item 关中断
	\item 保护断点和程序状态
	\item 识别中断源
\end{itemize}
中断判优:在同时出现的若干个中断请求中找出级别最高的,以便进行相应的中断服务
\begin{itemize}
	\item 软件判优:轮询法
	\item 硬件判优
	\begin{itemize}
		\item 串行判优:链式查询
		\item 并行判优:独立请求
	\end{itemize}
\end{itemize}
优先级
\begin{itemize}
	\item 中断响应的优先级由硬件排队线路决定
	\item 中断处理优先级由软件设置屏蔽码决定
\end{itemize}
缺点
\begin{itemize}
	\item 对IO请求相应慢
	\item 数据传送速度慢
\end{itemize}

\subsubsection{直接存储器(DMA)访问方式}
直接存储器存取(Direct Memory Access):
\begin{itemize}
	\item 独立于处理器、能在高速外设和主存之间直接传送数据
	\item 由专门硬件(DMA控制器)控制总线进行传输
	\item 高速设备(磁盘光盘等),成批数据交换,且单位数据间的时间间隔较短
\end{itemize}
\begin{itemize}
\item 采用“请求--响应”方式
\begin{itemize}
	\item 每当高速设备准备好数据,就进行一次“DMA请求”,DMA控制器接收到DMA请求后,申请总线使用权
	\item DMA控制器的总线使用优先级比CPU高
\end{itemize}
\item 与中断控制方式结合使用
\begin{itemize}
	\item DMA传送前, “寻道”“旋转” 等操作结束时, 通过“中断” 告知CPU
	\item DMA控制器控制总线传送数据时, CPU执行其他程序
	\item DMA传送结束时, 要通过“DMA结束中断” 告知CPU
\end{itemize}
\end{itemize}
\par CPU对DMA请求的响应时间可以发生在每个总线周期结束时
\par DMA请求的优先级高于中断请求
IO数据一致性:OS维护
\begin{itemize}
	\item IO读操作:cache置为无效
	\item IO写操作:cache flush(刷新),强迫cache中被更新的数据写回内存
\end{itemize}

\subsubsection{RAID盘阵}
廉价磁盘冗余阵列(Redundant Array of Inexpensive Disks)
增加容量
并行工作提高数据传输速度
冗余进行错误恢复提高系统可靠性