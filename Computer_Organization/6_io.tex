% !TEX root = main.tex

\section{输入输出系统}
\subsection{IO接口}
IO接口是主机与IO设备之间数据交换的界面,它屏蔽了IO设备的差异,提供了一致的访问界面。
功能如下
\begin{enumerate}
	\item 数据格式(串并)转换和电平变换
	\item 数据缓存(速度不匹配)
\end{enumerate}
IO端口:IO接口中的各类寄存器\\
寄存器编址
\begin{itemize}
	\item 统一编址,与主存统一编址,可以用访存指令去访问外设中的存储器
	\item 独立编址,对IO端口单独编号,需要专门的IO指令
\end{itemize}
% IO设备的连接 6.2
接口芯片有使能端CS

\subsection{磁盘存储器}
磁盘组织:磁道、扇区(访问信息最小单位)、柱面\\
道(track)密度:垂直于磁道方向上(半径方向)单位长度磁介质所容纳的磁道数\\
位(bit)密度:单位长度磁道上所能记录的二进制信息位数\\
面(surface)密度:单位面积上记录的二进制信息位数=道密度$\times$位密度\\
存取时间$t$=寻道时间$t_s$+旋转等待时间$t_w$+数据传输时间($t_{WR}$)\\
最大旋转延迟=1/磁盘转速*60s/1min\\
平均旋转延迟=最大旋转延迟/2(旋转半圈时间)\\
硬盘容量=柱面$\times$磁头数$\times$每磁道扇区数$\times$每扇区字节数\\
转速单位为rpm或r/min
\begin{example}
设磁盘平均寻道时间为10.5ms,磁盘的转速为3000转/分,每个扇区的字节数为512,每条磁道的容量为3072字节,则磁盘存储器读写一个扇区平均访问时间为?
\end{example}
\begin{analysis}
转速50转/分,则每转半圈为$t_w=0.01s$\\
数据传输时间(看扇区数据量/一s扫过的数据量)$t_{WR}=512B/(3072B\times 50)=3.33ms$\\
故$t=10.5ms+10ms+3.33ms=23.83ms$
\end{analysis}

\subsection{闪存存储器}
电可擦写、可编程只读存储器(EEPROM)
\begin{itemize}
	\item NOR:随机,可以直接按字节访问,主要用于存储程序代码(code)
	\item NAND:块级IO访问,主要用于存储数据(data)
\end{itemize}
读快、写慢(块擦)\\
通过损耗均衡(wear leveling)减少块的磨损\\
固态盘SSD即为闪存

\subsection{光存储器}
光存储器(Optical Disk Memory, ODM):采用激光手段对存储介质进行读写操作
\begin{itemize}
	\item 工作原理:将激光聚焦成极细光束在存储介质上存储信息,根据激光束和反射光的强弱不同,实现信息读写
	\item 优点:光盘记录密度高,单片存储容量大,非接触式读写,易于更换保管,对环境条件没有苛求
	\item 缺点:光盘机寻道时间长,可擦写性能不如磁盘快
\end{itemize}
按记录介质分类
\begin{itemize}
	\item 形变型光盘:不可逆,如CD-ROM、VCD、DVD-ROM
	\item 相变型光盘(Phase Change Disc, PCD):不可逆、可逆
	\item 磁光型光盘(Magneto Optical Disc, MDO):可多次读写
\end{itemize}
蓝光存储密度高

\subsection{RAID盘阵}
计算机系统总体性能的提高很不匹配:处理器和主存性能改进快,但辅存性能改进慢
\par 廉价磁盘冗余阵列(Redundant Array of Inexpensive Disks, RAID)
\begin{itemize}
	\item 多个廉价磁盘增加容量(类似多体交叉)
	\item 并行工作提高数据传输速度
	\item 冗余进行错误恢复,进而提高系统可靠性
\end{itemize}
\begin{enumerate}
	\item RAID0:条带化;存储容量、读写速度,但可靠性不够
	\item RAID1:镜像盘1+1冗余;有可靠性,但无存储容量、读写速度
	\item RAID2:条带化+海明校验码;冗余信息开销大,读性能高,写要同时写数据盘和校验盘,性能低
	\item RAID3:奇偶校验法生成单个冗余盘,条区增大为KB,提高吞吐率;适用于大量顺序数据访问的应用场合,如医学图像处理等;某个磁盘损坏时,可以通过其它磁盘重新生成;校验时,须将所有未写数据盘的原数据全部读出,再与写入盘的新数据异或形成新校验数据,并将其写入校验盘
	\item RAID4:独立存取技术, 使用更大条区,各个盘相互独立访问,并发I/O,共享的校验盘;适合于较小的数据访问,允许并发地发生多个独立访问;多个并发写时,唯一的共享校验盘成为性能瓶颈,应用不广泛;校验时:对小块写,只须将要写数据盘和校验盘的原数据读出,再与写入盘的新数据异或形成新校验数据,并将其写入校验盘
	\item RAID5:奇偶校验块分布在各个磁盘中,各个盘相互独立的并发访问;兼顾存储性能、 数据可靠性和存储成本,广泛应用于文件和应用服务器、数据库服务器、万维网、邮件服务器等,可以对单盘失效进行恢复
	\item RAID6:两个独立的奇偶校验系统,数据可靠性非常高,可以容两块磁盘同时出错;校验开销更大、 写性能较差、复杂的控制方式,限制了RAID6的广泛应用
	\item RAID0+1:条带化+镜像,单一磁盘坏了就全坏
	\item RAID1+0:镜像+条带化,如果一个磁盘坏了,镜像盘重建即可
\end{enumerate}

\subsection{IO控制方式}
\subsubsection{程序查询(Polling)方式}
IO完全由CPU指令控制,数据传输再CPU的寄存器与外设及其接口的数据缓冲寄存器之间进行,IO不直接访问内存
\begin{itemize}
	\item 查询期间,可以一直不断查询(原地踏步),也可以定时查询(要保证数据不丢失)
	\item 完全串行工作或部分串行,适用于慢速设备(否则数据会被冲掉)
	\item 开销极大,CPU完全在等待外设完成
\end{itemize}

\subsubsection{程序中断(Interrupt)/中断驱动方式}
由外设主动通知CPU,可以处理异常事件
\begin{enumerate}
	\item 中断请求:当外设准备好时,向CPU发中断请求
	\item 中断相应:CPU响应后,中止线性程序执行,转入“中断服务程序”进行输入/输出操作
	\item 中断处理:中断服务程序执行完,CPU返回程序断点继续执行,外设和CPU并行工作
\end{enumerate}
区别中断和异常:
\begin{itemize}
	\item 中断是在一条指令执行结束后开始查询有无中断请求,有的话立即响应,所以中断请求一定是在当前指令执行完时响应中断
	\item 异常发生在指令执行过程中,所以异常请求不能等到指令执行完才进行异常处理
\end{itemize}
中断响应的条件
\begin{itemize}
	\item CPU处于开中断状态
	\item 在一条指令\textbf{执行完}(区别“异常”是在指令执行过程中)
	\item 至少要有一个未被屏蔽的中断请求
\end{itemize}
中断响应过程
\begin{itemize}
	\item 关中断
	\item 保护断点和程序状态
	\item 识别中断源
\end{itemize}
中断判优:在同时出现的若干个中断请求中找出级别最高的,以便进行相应的中断服务
\begin{itemize}
	\item 软件判优:轮询法
	\item 硬件判优
	\begin{itemize}
		\item 串行判优:链式查询
		\item 并行判优:独立请求
	\end{itemize}
\end{itemize}
优先级
\begin{itemize}
	\item 中断响应的优先级由硬件排队线路决定
	\item 中断处理优先级由软件设置屏蔽码决定
\end{itemize}
缺点
\begin{itemize}
	\item 对IO请求相应慢
	\item 数据传送速度慢
\end{itemize}
% 中断地址指主程序的断点地址,即中断结束后的返回地址

\subsubsection{直接存储器(DMA)访问方式}
直接存储器存取(Direct Memory Access):
\begin{itemize}
	\item 独立于处理器、能在高速外设和主存之间直接传送数据
	\item 由专门硬件(DMA控制器)控制总线进行传输
	\item 高速设备(磁盘光盘等),成批数据交换,且单位数据间的时间间隔较短
\end{itemize}
\par 每次传送数据都要占用一个存储周期
\begin{itemize}
\item 采用“请求--响应”方式
\begin{itemize}
	\item 每当高速设备准备好数据,就进行一次“DMA请求”,DMA控制器接收到DMA请求后,申请总线使用权
	\item DMA控制器的总线使用优先级比CPU高
\end{itemize}
\item 与中断控制方式结合使用
\begin{itemize}
	\item DMA传送前,“寻道”“旋转”等操作结束时,通过“中断”告知CPU
	\item DMA控制器控制总线传送数据时,CPU执行其他程序
	\item DMA传送结束时,要通过“DMA结束中断”告知CPU
\end{itemize}
\end{itemize}
\par CPU对中断请求的相应只能在\textbf{每条指令执行完}后,对DMA请求的响应时间可以发生在每个\textbf{总线/存取周期结束}时
\par DMA请求的优先级高于中断请求
DMA数据传送方式(由于可能出现CPU争用现象)
\begin{itemize}
	\item CPU停止法(成组传送):CPU脱离总线,停止访问主存;可通过在DMA接口中引入缓冲器减少CPU等待时间
	\item 周期挪用/窃取法(单字传送):CPU让出一个总线事务周期
	\item 交替分时访问法:将存储周期分为两个时间片(透明DMA方式)
\end{itemize}
IO数据一致性:OS维护
\begin{itemize}
	\item IO读操作:cache置为无效
	\item IO写操作:cache flush(刷新),强迫cache中被更新的数据写回内存
\end{itemize}

\subsubsection{其他}
\begin{itemize}
	\item 通道方式:I/O模块具有独立I/O处理指令,具备执行专门I/O程序的能力。CPU只需在主存中事先组织好I/O程序,发出相应的I/O命令即可,其对I/O的干预极少
	\item 处理机方式:I/O模块是一个专门的I/O处理机,拥有独自的存储器和指令集,CPU几乎不参与I/O
\end{itemize}

\subsection{串行接口}
串行通信是将数据的各个位一位一位地,通过单条\textbf{1位}宽的传输线按顺序分时传送\\
优点:传输距离长,抗干扰强,费用低\\
波特率:单位时间内传送的二进制数据的位数,以位/秒(b/s)表示,也称为数据位率。 它是衡量串行通信速率的重要指标。\\
收/发时钟直接决定了通信线路上数据传输的速率,对于收/发双方之间数据传输的同步有十分重要的作用。\\
信道复用
\begin{itemize}
	\item 时分多路复用(TDM, Time Division Multiplexing):将一条物理传输线路按时间分成若干时间片轮换地为多个信号所占用,每个时间片由复用的一个信号占用
	\item 频分多路复用(FDM, Frequency Division Multiplexing):利用频率调制原理,将要同时传送的多个信号进行频谱搬移,使它们互不重叠地占据信道频带的不同频率段,然后经发送器从同一信道上同时或不同时地发送出去
\end{itemize}
异步串行通信协议
\begin{itemize}
	\item 起始位:标识数据帧的开始。
	\item 数据位:要传输的实际数据(5-8位),先发送最低位
	\item 校验位:提高正确性,通常采用简单的奇偶校验
	\item 停止位:标识数据帧的结束
\end{itemize}