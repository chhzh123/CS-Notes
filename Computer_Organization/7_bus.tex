% !TEX root = main.tex

\section{总线}
\subsection{总线概述}
总线:在多个部件之间实现互连,用于分时共享方式传输公共信息的一组数据通路
\par常见IO总线:ISA、USB、PCI、VL-BUS
\par现行PC机主要系统总线是PCI和ISA
\par三总线结构:处理机总线、PCI总线、ISA总线
\par总线周期:通过总线完成一次内存或IO设备读写操作所需时间
\par总线性能影响因素
\begin{itemize}
	\item 长度
	\item 连接元件数目
\end{itemize}
要在存储器与总线相连部分加三态门,防止随时读取数据(高阻态)
\par 总线的分类
\begin{itemize}
	\item 内部总线:寄存器、ALU
	\item 系统总线:CPU、MM、IO控制器
	\begin{itemize}
		\item 控制线:决定总线功能强弱、适应性(控制IO、读写等)
		\item 数据线:决定一次能传送数据的位数/能力
		\item 地址线:决定最大寻址空间
	\end{itemize}
	\item 通信/IO/外部总线:主机、IO设备
\end{itemize}
* 在系统总线的数据线上,不传输握手信号
\par 总线传送控制/定时方式:总线在双方交换数据的过程中需要时间上配合关系的控制,实质是一种协议或规则
\begin{itemize}
	\item 同步方式:统一时钟信号,电路简单,适合高速设备;时钟偏移,总线长度不能很长,按最慢的设置(CPU内部总线)
	\item 异步方式:比同步方式慢,总线频带窄,总线传输周期长,需要握手,用于不同存取速度设备(IO总线)
	\begin{itemize}
		\item 不互锁:发送接收完就不理对方状态
		\item 半互锁:发送方要得到接收方的相应才可进行其他事情
		\item 全互锁:来回握手,接收方再得到发送方确认才可进行其他事情
	\end{itemize}
	\item 半同步方式:wait/ready信号是单向的,不是互锁的;适用于系统工作速度不高,但又包含了许多工作速度差异较大的各类设备的简单系统
	\item 分离方式:总线读周期分成两个子周期(寻址+数据传送),在两子周期之间,退出总线,从设备准备数据;适用于有很多主模块(如多个处理器或多个DMA设备)的系统
\end{itemize}
总线主设备
\begin{itemize}
	\item 总线使用请求信号$\to$总线使用应答信号$\to$使用总线
	\item 中断请求信号$\to$中断相应信号和中断向量$\to$等待CPU处理
	\item DMA请求信号$\to$DMA应答信号$\to$数据交换$\to$撤销DMA请求信号
\end{itemize}
总线操作五个步骤
\begin{itemize}
	\item 传输请求
	\item 总线仲裁
	\item 部件寻址
	\item 数据传输
	\item 总线释放
\end{itemize}

\subsection{总线设计}
高性能(宽通路,分离数据地址线)与低成本(窄通路,复用数据地址线)之间的权衡
\par 复用:不同信号在同一总线上分时传送
\par总线控制器:对存储空间进行分配、启动等,仲裁哪个主设备获得总线使用权
\[\text{总线带宽(数据传输率)}=\text{总线宽度}/8b\times\text{总线时钟频率}\]
\par注意区别总线周期(通过总线完成一次完整数据传输所需要的时间)与系统周期
\par 总线的操作可分为传输请求、总线仲裁、部件寻址、数据传输、总线释放
\begin{itemize}
	\item 主设备:能申请并获得总线控制权的设备
	\item 从设备:被主设备访问的设备
\end{itemize}
\par总线仲裁(arbiter):总线主设备请求并获得总线控制权的过程
\begin{itemize}
	\item 分布仲裁:自举裁决、冲突检测
	\item 集中仲裁:菊花链(串行逐一访问)、独立请求并行判优
\end{itemize}
仲裁后,获得总线控制权的设备建立“总线忙”信号,用完即撤销
\par 数据传输方式
\begin{itemize}
	\item 基本:一个地址一个数据
	\item 成组/猝发(burst):提高总线数据传输率,一个地址多个数据
\end{itemize}