% !TEX root = main.tex

\section{总线}
% 高阻态
总线速度影响因素
\begin{itemize}
	\item 长度
	\item 连接元件数目
\end{itemize}
总线的分类
\begin{itemize}
	\item 芯片内总线
	\item 系统总线
	\begin{itemize}
		\item 控制线:强弱、适应性
		\item 数据线:一次能传送数据的位数/能力
		\item 地址线:最大寻址空间
	\end{itemize}
	\item 通信总线
\end{itemize}
总线主设备
\begin{itemize}
	\item 总线使用请求信号$\to$总线使用应答信号$\to$
	\item 中断请求信号$\to$中断相应信号和中断向量$\to$等待CPU处理
	\item DMA请求信号$\to$DMA应答信号$\to$数据交换$\to$撤销DMA请求信号
\end{itemize}
总线控制器
\[\text{总线带宽}=\text{总线宽度}/8b\times\text{总线时钟频率}\]
总线带宽/数据传输率
\par总线仲裁(arbiter):总线主设备请求并获得总线控制权的过程
\begin{itemize}
	\item 串行
	\item 定时方式
	\begin{itemize}
		\item 同步:时钟偏移,长度不能很长,按最慢的设置
		\item 异步:请求应答时间长
	\end{itemize}
\end{itemize}