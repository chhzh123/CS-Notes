\documentclass{myslide}
\usepackage[slide]{mypackage}

\title{Computer System Architecture}
\subtitle{--- Lectures from Dannie Sanchez, MIT}

% \author[chhzh123]{陈鸿峥}
\date[Dec 1, 2018]{December, 2018}

\keywords{}

\begin{document}

\begin{frame}
\titlepage
\end{frame}

\begin{frame}
\tableofcontents[subsectionstyle=show]
\end{frame}

\section{课程简介}
\begin{frame}
\sectionpage
\end{frame}

\begin{frame}{课程简介}
Daniel Sanchez (MIT)
\begin{itemize}
	\item 6.004 Computation Structures
	\item 6.823 Computer System Architecture
\end{itemize}
\end{frame}

\begin{frame}{每节基本内容}
6.004 Computation Structures\\
重在工程实践,理论部分比较少
\begin{itemize}
	\only<1>{
	\item L01: 计算机层次结构、现代硬件设计流、数电与模电简介
	\item L02: 组合电路(真值表、布尔代数、基本门电路)
	\item L03: 原码加法、半加器、全加器、BSV初步
	\item L04: 数的表示(补码)、补码加减法、原码乘法、前瞻加法器
	\item L05: BSV实现加法器、乘法器
	\item L06: 时序电路(触发器、模4计数器实现、gcd实现)、有限状态机简介
	}
	\only<2>{
	\item L07: BSV IO接口
	\item L08: BSV复杂组合电路设计
	\item L09: 冯诺依曼模型、指令集(RISC-V)(讲了不同类型指令,但是没讲数据流!)
	\item L10: 改写HLL程序为RISC-V(包括过程调用、堆栈)
	\item L11: 指令译码(单周期CPU)
	\item L12: 2周期CPU
	}
	\only<3>{
	\item L13: 内存层次结构(局部性、cache直接映射与全相连)、SRAM与DRAM
	\item L14: cache的映射、替换策略(写回写穿写分配写不分配)、cache的实施
	\item L15: 流水线简介
	\item L16: 流水线中的冒险
	\item L17: 流水线实施
	\item L18: 流水线CPU实施
	}
	\only<4>{
	\item L19: 虚拟化、异常
	\item L20: 虚拟内存
	\item L21: IO设备
	}
\end{itemize}
\end{frame}

\begin{frame}{每节基本内容}
6.823 Computer System Architecture\\
谈及现代技术,但都讲得不够细;章节之间连续性不强,更像一个seminar,每次一个专题
\begin{itemize}
	\only<1>{
	\item L01: 计算机历史、指令集发展
	\item L02: MIPS、数据通路
	\item L03: 内存层次结构、cache映射替换策略
	\item L04: 内存管理(内存碎片、页表)
	\item L05: 现代虚存系统
	\item L06: 指令流水与冒险
	}
	\only<2>{
	\item L07: 指令冒险解决方案
	\item L08: 复杂流水简介
	\item L09: 复杂流水(OOO)与异常
	\item L10: 分支预测
	\item L11: 预测执行
	\item L12: 现代内存的操作
	}
	\only<3>{
	\item L13: 多线程
	\item L14: cache一致性
	\item L15: directory-based cache coherence
	\item L16: 内存一致性模型
	\item L17: 片上网络(拓扑结构与流控制)
	\item L18: 片上网络(布局布线)
	}
	\only<4>{
	\item L19: reliable architectures
	\item L20: 微指令与VLIW
	\item L21: 向量机
	\item L22: GPU
	\item L23: 事务内存
	\item L24: 虚拟机
	}
\end{itemize}
\end{frame}

\section{课程设置}
\begin{frame}
\sectionpage
\end{frame}

\begin{frame}{课程设置}
数字电路:
\begin{itemize}
	\item 数字与模拟
	\item 布尔代数(SOP、POS、Karnaugh图)
	\item 组合电路(比较器、译码器、编码器、选择器、分配器;加法器)
	\item 时序电路(锁存器、触发器;分频计数器)
	\item 电路设计(有限状态机、驱动转移方程)
\end{itemize}
\end{frame}

\begin{frame}{课程设置}
计算机组成原理:
\begin{itemize}
	\item 计算模型与计算机的发展
	\item 总体框架、数的表示(进制、原反补移码、浮点数)
	\item 指令集(CISC与RISC、指令格式)%ILP
	\item 中央处理器(CPU)
	\begin{itemize}
		\item 单周期数据通路
		\item 多周期数据通路
		\item 流水线、冒险的出现与解决
	\end{itemize}
	\item 存储器层次结构
\end{itemize}
略过Computer Arithmetic
\end{frame}

\end{document}