% !TEX root = main.tex

\section{恢复系统} % Chap 16
\subsection{故障分类}
\begin{itemize}
	\item 事务故障(failure)
	\begin{itemize}
		\item 逻辑错误:事务由于某些内部条件而无法继续正常执行,如非法输入。
		\item 系统错误:系统进入一种不良状态(如死锁),事务无法正常运行,但是之后某个时间点能够重新执行。
	\end{itemize}
	\item 系统崩溃(crash):硬件故障,或数据库软件/操作系统的漏洞,导致易失性存储器内容丢失,事务停止。
	\item 磁盘故障(failure):数据传送过程中由于磁头损坏或故障造成的磁盘块上内容丢失。
\end{itemize}

\subsection{恢复与原子性}
使用日志来重做(redo)和撤销(undo)事务。

事务回滚操作:
\begin{enumerate}
	\item 从后往前扫描日志,对于所发现的$T_i$的每个形如$<T_i,X_j,V_1,V_2>$的日志记录:
	\begin{enumerate}
		\item 值$V_1$被写到数据项$X_j$中,且
		\item 往日志中写一个特殊的只读日志记录$<T_i,X_j,V_1>$,其中$V_1$是在本次回滚中数据项$X_j$恢复成的值。这种日志称作补偿日志记录(compensation log record)。
	\end{enumerate}
	\item 一旦发现$<T_i,start>$的日志记录,就停止从后往前扫描,并往日志中写一个$<T_i,abort>$的日志记录
\end{enumerate}