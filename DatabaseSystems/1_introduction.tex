% !TEX root = main.tex

本课程采用书目Avi Silberschatz, Henry F. Korth, S. Sudarshan, \emph{Database System Concepts (6th ed)}\footnote{\url{http://www.db-book.com}}。

\section{数据库系统概述}
早期的数据库直接建立在文件系统上,但这会导致:
\begin{itemize}
	\item 数据冗余与不一致
	\item 访问数据非常麻烦
	\item 完整性问题:难以添加限制(如年龄为非负整数)
	\item 更新的原子性
	\item 多用户的并发访问
	\item 安全性问题:权限
\end{itemize}

查询过程:解释编译+求值(evaluation)

\section{关系型数据库}
纵向为属性(attributes/columns),横向为元组(tuples/rows)

注意关系都是无序的,元组可以以任意顺序存储

\begin{itemize}
	\item Schema:\verb'instructor(ID,name,dept_name,salary)'
	\item Instance:局部数据
	\item 键值(keys)$R$
	\item 超键(superkey)$K\subset R$
	\item 候选键(candidate key)$K$为原子/不可分割/最小键
\end{itemize}

关系代数(relational algebra)
\begin{center}
\begin{tabular}{|c|c|c|}
\hline
选择 & $\sigma$ &
\begin{tabular}{l}
挑选出符合一定性质的元组\\
$\sigma_{\text{Sub="Phy"}\land\text{age}>30}(\text{teachers})$
\end{tabular}\\\hline
投影 & $\Pi$ & 
\begin{tabular}{l}
只选出对应属性\\
$\Pi_{\text{ID,name,salary}}(\text{teachers})$
\end{tabular}\\\hline
笛卡尔积 & $\times$ & 将两个关系整合(简单并置,需要进一步筛选)\\\hline
合并 & $r\Join_\theta s=\sigma_\theta(r\times s)$ & \\\hline
并集 & $\cup$ & 数目应相同,属性可兼容\\\hline
交集 & $\cap$ & \\\hline
差集 & $-$ & \\\hline
赋值 & $\gets$ & \\\hline
重命名 & $\rho_x(E)$ & 给$E$的返回值赋名为$x$\\\hline
\end{tabular}
\end{center}