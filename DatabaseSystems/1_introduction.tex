% !TEX root = main.tex

\section{数据库系统概述}
早期的数据库直接建立在文件系统上,但这会导致:
\begin{itemize}
	\item 数据冗余与不一致
	\item 访问数据非常麻烦
	\item 完整性问题:难以添加限制(如年龄为非负整数)
	\item 更新的原子性
	\item 多用户的并发访问
	\item 安全性问题:权限
\end{itemize}

查询过程:解释编译+求值(evaluation)

\section{关系型数据库}
纵向为属性(attributes/columns),横向为元组(tuples/rows)

注意关系都是无序的,元组可以以任意顺序存储

\begin{itemize}
	\item Schema:\verb'instructor(ID,name,dept_name,salary)'
	\item Instance:局部数据
	\item 键值(keys)$R$
	\item 超键(superkey)$K\subset R$
	\item 候选键(candidate key)$K$为原子/不可分割/最小键
\end{itemize}

关系代数(relational algebra)
\begin{itemize}
	\item 选择$\sigma$:挑选出符合一定性质的元组
	\[\sigma_{\text{Sub="Phy"}\land\text{age}>30}(\text{teachers})\]
	\item 投影$\Pi$:只选出对应属性
	\[\Pi_{\text{ID,name,salary}}(\text{teachers})\]
	\item 笛卡尔积$\times$:将两个关系整合(简单并置,需要进一步筛选)
	\item 合并$r\Join_\theta s=\sigma_\theta(r\times s)$
	\item 并集$\cup$:数目应相同,属性可兼容
	\item 交集$\cap$
	\item 差集$-$
	\item 赋值$\gets$
	\item 重命名$\rho_x(E)$:给$E$的返回值赋名为$x$
\end{itemize}