% !TEX root = main.tex

\section{SQL简介} % Chap 3
SQL即结构化查询语言(Structured Query Language, SQL)

\subsection{数据定义语言(DDL)}
数据类型
\begin{itemize}
	\item \verb'char(n)':固定长度字符串
	\item \verb'varchar(n)':可变长度字符串,用得最多
	\item \verb'int'
	\item \verb'smallint'
	\item \verb'numeric(p,d)':定点数,共$p$位(包括符号位),其中$d$位在小数点右侧
	\item \verb'real', \verb'double precision'
	\item \verb'float(n)':精度至少为$n$位的浮点数
\end{itemize}
\begin{lstlisting}[language=sql]
create table instructor (
	ID char(5),
	name varchar(20) not null,
	dept_name varchar(20),
	salary numeric(8,2) default 0,
	primary key (ID), -- integrity constraints
	foreign key (dept_name) references department);
\end{lstlisting}

\subsection{数据查询语言}
基本的形式如下所示
\begin{lstlisting}[language=sql]
select A1, A2, ..., An
from r1, r2, ..., rm
where P
\end{lstlisting}
其中$A_i$为属性(attribute)、$R_i$为一个关系(relation)/一个表、$P$是谓词(preicate),返回满足关系的数组。

\begin{itemize}
	\item 自然连接(natural join):将两个关系的各个元组进行连接,并且只考虑那些在两个关系模式中\textbf{都出现}的属性上取值相同的元组对(相当于取\textbf{交})
	\item 连接(join):只考虑$A$这个属性进行连接
\end{itemize}
\begin{lstlisting}[language=sql]
select name, title
from (instructor natural join teachers) join course using (course_id);
-- only join two relations with the same course_id
\end{lstlisting}

字符串运算:模式匹配
\begin{itemize}
	\item \verb'upper(s)'、\verb'lower(s)'
	\item \verb'%':匹配任意字串
	\item \verb'_':匹配任意一个字符
	\item \verb'like':以这些字符串进行匹配
\end{itemize}

\begin{lstlisting}[language=sql]
select distinct dept_name
from instructor

select * -- all attributes
from instructor

select '437' as FOO

select ID, name, salary/12 as monthly_salary

select name
from instructor
where dept_name = 'Comp. Sci.' and salary > 80000

select *
from instructor, teaches -- Cartesian product

select distinct T.name
from instructor as T, instructor as S -- rename, Cartesian product
where T.salary > S.salary and S.dept_name = 'Comp. Sci.'
-- salary between 90000 and 100000

/**
percent ( % ). The % character matches any substring.
underscore ( _ ). The _ character matches any character.
**/
select name
from instructor
where name like '%dar%' matches any string containing "dar" as a substring

select distinct name
from instructor
order by name -- sorted

select *
from instructor
order by salary desc, name asc;

(select course_id from section where sem = 'Fall' and year = 2017)
union -- intersect, except
(select course_id from section where sem = 'Spring' and year = 2018)

select name
from instructor
where salary is null

-- aggregrate functions
-- arg, min, max, sum, count
select dept_name, avg (salary) as avg_salary
from instructor
where dept_name= 'Comp. Sci.'
group by dept_name;

select count (distinct ID)
from teaches
where semester = 'Spring' and year = 2018;

-- group constraints
-- find the average salary in each dept
select dept_name, avg(salary) as avg_salary
from instructor
group by dept_name -- can only be contained by select clause
having avg(salary) > 42000; -- effected after grouping
\end{lstlisting}

三值逻辑,添加了Unknown
\begin{center}
\begin{tabular}{|c|c|c|}\hline
AND & OR & NOT\\\hline
\begin{tabular}{l}
$T \land U = U$\\
$F \land U = F$\\
$U \land U = U$
\end{tabular} & 
\begin{tabular}{l}
$T \lor U = T$\\
$F \lor U = U$\\
$U \lor U = U$
\end{tabular} &
$\lnot U = U$\\\hline
\end{tabular}
\end{center}

SQL在谓词中使用\verb'null'测试空值,因而为找出\verb'instructor'关系中\verb'salary'为空值的所有教师,可写
\begin{lstlisting}
select name
from instructor
where salary is null
\end{lstlisting}

集合成员资格通过\verb'in'来判断,同时有\verb'some'和\verb'all'的关系测试,下面给出了一些嵌套子查询的例子。
\begin{lstlisting}
select name
from instructor
where salary > some (select salary
					from instructor
					where dept_name = 'Biology')

select S.ID, S.name
from student as S
where not exists ((select course_id
				from course
				where dept_name = 'Biology')
				except
				(select T.course_id
				from takes as T
				where S.ID = T.ID))

select T.course_id
from course as T
where 1 >= (select count(R.course_id)
			from section as R
			where T.course_id = R.course_id and R.year = 2009);
-- there is no unique in MySQL

with max_budget(value) as
	(select max(budget)
	from department)
select budget
from department, max_budget
where department.budget = max_budget.value;

select dept_name,
	(select count(*)
	from instructor
	where department.dept_name = instructor.dept_name)
	as num_instructors
from department; -- scalar subquery
\end{lstlisting}

数据库的修改操作如下
\begin{lstlisting}
delete from instructor
where dept_name = 'Finance';

insert into course(course_id, title)
	values ('CS-437','Database Systems');

update instructor
set salary = salary * 1.05
where salary < 70000;

update instructor
set salary = case
			when salary <= 1000 then salary * 1.05
			when salary <= 2000 then salary * 2.05
			else salary * 1.03
		end
\end{lstlisting}