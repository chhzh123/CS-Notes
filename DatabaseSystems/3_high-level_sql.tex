% !TEX root = main.tex

\section{进阶SQL} % Chap 4
\subsection{内外连接}
\verb'on'条件可以提供比自然连接更为丰富的连接条件
\begin{lstlisting}
select *
from students join takes on student.ID = takes.ID; -- condition
\end{lstlisting}

直接使用\verb'natural join'可能导致元组丢失,比如有一些学生没有选修任何课程,则其在student关系中不会与takes关系中任何元组配对。
因此有\textbf{外连接}:
\begin{itemize}
	\item 左外连接(left outer join):只保留出现在左外连接运算之前(左边)关系中的元组,若右侧关系中没有对应属性则用null代替
	\item 右外连接(right outer join)
	\item 全外连接(full outer join):左外连接和右外连接的组合
\end{itemize}

从而可以很容易查询出“所有课程一门也没有选修的学生”:
\begin{lstlisting}
select ID
from student natural left outer join takes
where course_id is null;
\end{lstlisting}

为了将常规连接和外连接区分开,SQL中将常规连接称为内连接(inner join),默认的都是内连接。

\subsection{视图}
让用户看到整个逻辑模型显然是不合适的,出于安全考虑,可能需要向用户隐藏特定的数据。
本来通过select可以把需要的数据计算并存储下来,但一旦底层数据发生变化,查询的结果就不再匹配。
因此,为了解决这样的问题,SQL提供一种虚关系,称为视图(view),只有在使用的时候才会被计算。
\begin{lstlisting}
create view faculty as
select ID, name, dept_name
from instructor;
\end{lstlisting}
之后便可以直接使用\verb'from'语句访问视图。

\begin{definition}[物化视图(materialized view)]
如果用于定义视图的实际关系改变,视图也会跟着修改。
\end{definition}

由于视图并不是数据库底层的关系,故一般数据库不允许对视图关系进行修改。

\subsection{事务}
SQL标准规定当一条SQL语句被执行,就隐式开始了一个事务。
下列SQL语句之一会结束一个事务:
\begin{itemize}
	\item Commit work:将事务所做的更新在数据库中持久保存。
	在事务被提交后,一个新的事务自动开始。
	\item Rollback work:撤销该事务中所有SQL语句对数据库的更新,数据库将恢复到执行该事务的第一条语句之前的状态。
	如果遇断电,回滚会在下一次重启时自动执行。
\end{itemize}

\subsection{完整性约束}
单个关系上的约束
\begin{itemize}
	\item \verb'not null':跟在属性定义后面
	\item \verb'unique (A1,A2,...,Am)':声明$A_1,\ldots,A_m$构成一个候选码
	\item \verb'check(<predicate>)':关系中每个元组都必须满足该谓词,如\verb'check(budget>0)'
\end{itemize}

参照完整性/子集依赖:外码
\begin{lstlisting}
foreign key (dept_name) references department
	on delete cascade
	on update cascade
\end{lstlisting}
会级联(cascade)删除或更新。

\subsection{授权}
\verb'grant'授权,\verb'revoke'回收
\begin{itemize}
	\item \verb'select':允许用户修改关系中任意数组
	\item \verb'insert'
	\item \verb'delete'
\end{itemize}
用户名\verb'public'包括了当前所有用户和未来用户的权限。

\begin{lstlisting}
grant <authorization list>
on <relation name/view name>
to <user/user list>

grant update (budget) on department to Amit, Satoshi;
\end{lstlisting}

可以先创建角色(role)然后给用户授予角色。