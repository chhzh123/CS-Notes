% !TEX root = main.tex

\section{形式化关系查询语言} % Chap 6
\subsection{关系代数}
\begin{center}
\begin{tabular}{|c|c|c|}
\hline
选择 & $\sigma$ &
\begin{tabular}{l}
挑选出符合一定性质的元组\\
$\sigma_{\text{Sub="Phy"}\land\text{age}>30}(\text{teachers})$
\end{tabular}\\\hline
投影 & $\Pi$ & 
\begin{tabular}{l}
只选出对应属性\\
$\Pi_{\text{ID,name,salary}}(\text{teachers})$
\end{tabular}\\\hline
笛卡尔积 & $\times$ & 将两个关系整合(简单并置,需要进一步筛选)\\\hline
自然连接 & $r\Join s=\prod_{R\cup S}(\sigma_{r.A_1=s.A_1\land\cdots\land r.A_n=s.A_n})(r\times s)$ & \\\hline
$\theta$连接 & $r\Join_\theta s=\sigma_\theta(r\times s)$ & \\\hline
外连接 & $\leftouterjoin, \rightouterjoin, \fullouterjoin$ & \\\hline
并集 & $\cup$ & 数目应相同,属性可兼容\\\hline
交集 & $\cap$ & \\\hline
差集 & $-$ & \\\hline
赋值 & $\gets$ & \\\hline
重命名 & $\rho_x(E)$ & 给$E$的返回值赋名为$x$\\\hline
\end{tabular}
\end{center}

扩展的关系代数运算:
\begin{itemize}
	\item 广义投影:$\prod_{F_1,F_2,\ldots,F_n}(E)$,其中$F_i$中每一个都是涉及常量及$E$的模式中属性的算术表达式
	\item 聚类:$\mathcal{G}$,如count, min, max,考虑group by可以写成下列形式
	\[{}_{dept\_name}\mathcal{G}_{\textbf{average}(salary)}(instructor)\]
\end{itemize}

\begin{definition}[聚集运算]
聚集运算$\mathcal{G}$的通常形式如下:
\[{}_{G_1,G_2,\ldots,G_n}\mathcal{G}_{F_1(A_1),\ldots,F_n(A_n)}(E)\]
其中$E$是任意关系代数表达式,$G_1,\ldots,G_n$是用于分组的一系列属性;
每个$F_i$是一个聚集函数,每个$A_i$是一个属性名。
$E$结果中的元组将会以如下方式分为若干组:
\begin{itemize}
	\item 同组中所有元组在$G_1,\ldots,G_n$上的取值相同
	\item 不同组中的元组在$G_1.\ldots,G_n$上的取值不同
\end{itemize}
\end{definition}

\subsection{其他关系演算}
\subsubsection{元组关系演算}
元组关系演算是非过程化的查询语言(有点像声明式语言),它只描述所需信息,而不给出获得该信息的具体过程。
\[t\in\text{instructor}\land\exists s\in\text{department}(t[\text{dept\_name}]=s[\text{dept\_name}])\]
前者$t$为自由变量,后者$s$为受限变量。

\subsubsection{域关系演算}
\[\{<x_1,x_2,\ldots,x_n>\mid P(x_1,x_2,\ldots,x_n)\}\]
其中$<x_1,\ldots,x_n>\in r$,每一个$x_i$为域变量,$r$为$n$个属性上的关系。