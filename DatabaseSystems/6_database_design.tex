% !TEX root = main.tex

\section{关系数据库设计} % Chap 8
大的模式会存在大量冗余,小的模式会导致信息丢失(有损分解)。

\subsection{范式理论}
定义属性集$\alpha$,关系模式为$r(R)$。
关系模式是一个属性集,但不是所有属性集都是模式。
超码则用$K$来表示。
\begin{definition}[超码]
$R$的子集$K$是$r(R)$的超码的条件是:在关系$r(R)$的任意合法实例中,对于$r$实例中的元组对$t_1$和$t_2$总满足,若$t_1\ne t_2$,则$t_1[K]\ne t_2[K]$,即$K$唯一标识一条元组。(而\textemph{函数依赖则是唯一标识某些属性})
\end{definition}

设计需要满足一定的范式(normal form),核心目的是\textcolor{red}{减少冗余}:
% https://www.geeksforgeeks.org/normal-forms-in-dbms/
\begin{itemize}
	\item 第一范式(1NF):全部属性都是\textemph{单值属性}(原子性)

	\item 第二范式(2NF):关系模式$R\in 1NF$,且每一个\textemph{非主属性\textbf{完全依赖}于$R$的主码},而不是只依赖于其中一部分属性(即其中一部分属性的值即可确定该属性的值,部分依赖)
	\begin{example}
	一个学生-课程关系模式如下:
	\begin{center}
	\begin{tabular}{|c|c|c|}\hline
		Stud\_No & Course\_No & Course\_Fee\\\hline
		1 & C1 & 1000\\
		2 & C2 & 1500\\
		1 & C4 & 2000\\
		4 & C3 & 1000\\
		4 & C1 & 1000\\
		2 & C5 & 2000\\\hline
	\end{tabular}
	\end{center}
	如此例中Stud\_No和Course\_No是主码,但是实际上只由Course\_No就已经可以决定Course\_Fee了,因此这个例子不满足2NF。
	\end{example}

	\item 第三范式(3NF):关系模式$R\in 2NF$,且对于$F^+$中所有形如$\alpha\to\beta$的函数依赖,至少下列一项成立
	\begin{itemize}
		\item $\alpha\to\beta$是一个平凡的函数依赖
		\item $\alpha$是$R$的一个超码
		\item $\beta-\alpha$中的每个属性$A$都含于$R$的一个候选码中
	\end{itemize}
	(非主属性依赖于主码但不能通过另一非主属性进行依赖,即\textemph{不存在传递依赖})
	\begin{example}
	一个学生-国家关系模式如下:
	\begin{center}
	\begin{tabular}{ccccc}\hline
	Stud No & Stud Name & Stud State & Stud Country & Stud Age\\\hline
	1 & Alice & s1 & c & 18\\
	2 & Alice & s2 & c & 19\\
	3 & Bob & s2 & c & 21\\\hline
	\end{tabular}
	\end{center}
	有传递依赖Stud No$\to$Stud State$\to$Stud Country,因此不满足3NF。
	\end{example}

	\item 巴斯-科德/BC范式(Boyce-Codd NF, 3.5NF):关系模式$R\in 3NF$,对于$F^+$中所有形如$\alpha\to\beta$的函数依赖,至少有以下一项成立:
	\begin{itemize}
		\item $\alpha\to\beta$是平凡的函数依赖(即$\beta\subset\alpha$)
		\item $\alpha$是模式$R$的一个\textemph{超码}(即$\alpha$可以整条元组)
	\end{itemize}
\end{itemize}

\subsection{函数依赖}
\begin{definition}[函数依赖(functional dependency)]
设$R$为关系模式,$\alpha\subset R,\beta\subset R$,函数依赖$\alpha\to\beta$在$R$上满足,当且仅当对于任意合法的关系$r(R)$,任何两个关于$r$的数对$t_1$和$t_2$,如果满足属性$\alpha$,那么它们也满足属性$\beta$,即
\[t_1[\alpha]=t_2[\alpha]\implies t_1[\beta]=t_2[\beta]\]
实际上就是\textemph{函数单射}的概念,属性$\alpha$可以\textemph{唯一确定}属性$\beta$的值。
若函数依赖$K\to R$在$r(R)$上成立(注意这里$R$相当于全部属性,或者写成$K\to (R-K)$),则$K$是$r(R)$的一个\textemph{超码}。
\end{definition}
\begin{example}
考虑下例$r(A,B)$的关系
\begin{center}
\begin{tabular}{|c|c|}\hline
A & B\\\hline
1 & 4\\\hline
1 & 5\\\hline
3 & 7\\\hline
\end{tabular}
\end{center}
关系$A\to B$不成立,但关系$B\to A$成立。
\end{example}

\begin{definition}[平凡]
在所有关系中都满足的函数依赖则是平凡的函数依赖,比如$A\to A$。
一般地,若$\beta\subset\alpha$,则$\alpha\to\beta$的函数依赖是平凡的。
\end{definition}

\subsubsection{闭包}
\begin{definition}[逻辑蕴含与闭包]
若关系模式$r(R)$的每一个满足$F$的实例也满足$f$,则$R$上的函数依赖$f$被$r$上的函数依赖集$F$逻辑蕴含。
$F$的闭包是被$F$逻辑蕴含的所有函数依赖的集合,记作$F^+$。
设$\alpha$为属性集,将函数依赖集$F$下被$\alpha$函数确定的所有属性的集合称为$F$下$\alpha$的闭包。
\end{definition}

\begin{theorem}[逻辑蕴含公理]
下面的前三条为最基本的公理(Armstrong),可以找出给定$F$的所有$F^+$
\begin{itemize}
	\item 自反律(reflexivity):若$\alpha$为一属性集且$\beta\subset\alpha$,则$\alpha\to\beta$
	\item 增补律(augmentation):若$\alpha\to\beta$成立且$\gamma$为一属性集,则$\gamma\alpha\to\gamma\beta$成立
	\item 传递律(transitivity):若$\alpha\to\beta$和$\beta\to\gamma$成立,则$\alpha\to\gamma$成立
	\item 合并律(union):若$\alpha\to\beta$和$\alpha\to\gamma$成立,则$\alpha\to\beta\gamma$成立
	\item 分解律(decomposition):若$\alpha\to\beta\gamma$成立,则$\alpha\to\beta$和$\alpha\to\gamma$成立
	\item 伪传递律(pseudotransitivity):若$\alpha\to\beta$和$\gamma\beta\to\delta$成立,则$\alpha\gamma\to\delta$成立
\end{itemize}
\end{theorem}

\begin{algorithm}
\caption{计算$F^+$}
\begin{algorithmic}[1]
\State $F^+:=F$
\Repeat
\State 对$F^+$的函数依赖应用自反律、增补律和传递律,将结果加入$F^+$
\Until{result 不变}
\end{algorithmic}
\end{algorithm}

\begin{algorithm}[H]
\caption{计算$F$下属性$\alpha$的闭包$\alpha^+$}
\begin{algorithmic}[1]
\State result$:=\alpha$
\Repeat
\For{每一个函数依赖$\beta\to\gamma\in F$}
\State 若$\beta\subset$ result,则result $:=$ result $\cup\gamma$
\EndFor
\Until{result 不变}
\end{algorithmic}
\end{algorithm}

属性闭包算法有多种用途:
\begin{itemize}
	\item 用于\textemph{判断$\alpha$是否为超码},计算$\alpha^+$,检查$\alpha^+$是否包含$R$中所有属性
	\item 通过检查是否$\beta\subset\alpha^+$,我们可以检查$\alpha\to\beta$是否成立(即是否属于$F^+$),也就是用属性闭包计算$\alpha^+$,看是否包含$\beta$
	\item 另一种计算$F^+$的方法:$\forall\gamma\subset R$,找出$\gamma^+$;$\forall S\subset\gamma^+$,输出函数依赖$\gamma\to S$
\end{itemize}

\subsubsection{正则覆盖}
\begin{definition}[无关(extraneous)]
如果去除函数依赖中的一个属性不改变该函数依赖集的闭包,则称该属性是无关的,即\textemph{去掉该属性依然可以推得其函数依赖成立},形式化定义即
\begin{itemize}
	\item 若$A\in\alpha$且$F$逻辑蕴含$(F-\{\alpha\to\beta\})\cup\{(\alpha-A)\to\beta\}$,则属性$A$在$\alpha$中无关
	\item 若$A\in\beta$且函数依赖集$(F-\{\alpha\to\beta\})\cup\{\alpha\to(\beta-A)\}$逻辑蕴含$F$,则属性$A$在$\beta$中无关
\end{itemize}
\end{definition}

检验属性$A$是否无关的方法如下:
\begin{itemize}
	\item 若$A\in\beta$,为检验$A$是否无关,考虑集合
	\[F'=(F-\{\alpha\to\beta\})\cup\{\alpha\to(\beta-A)\}\]
	\textemph{检验$\alpha\to A$是否能由$F'$推出},
	计算$F'$下的$\alpha^+$,若$\alpha^+$包含$A$,则$A$在$\beta$中无关。
	\item 若$A\in\alpha$,为检验$A$是否无关,令$\gamma=\alpha-\{A\}$,\textemph{检验$\gamma\to\beta$是否可以由$F$推出},
	计算$F$下的$\gamma^+$,若$\gamma^+$包含$\beta$中所有属性,则$A$在$\alpha$中无关。
\end{itemize}
简言之,计算$A\subset\alpha_{F'}^+$和$\beta\subset\gamma_{F}^+$,但这很麻烦,通常直接通过逻辑蕴含判断。

\begin{definition}[正则覆盖(canonical cover)]
$F$的正则覆盖$F_c$是一个依赖集,使得$F_c$和$F$双向逻辑蕴含,且
\begin{itemize}
	\item $F_c$中任何函数依赖都不含无关属性
	\item $F_c$中函数依赖的左半部都是唯一的,即不存在$\alpha_1\to\beta_1$和$\alpha_2\to\beta_2$满足$\alpha_1=\alpha_2$
\end{itemize}
\end{definition}
\begin{algorithm}
\caption{计算正则覆盖}
\begin{algorithmic}[1]
\State $F_c=F$
\Repeat
\State 用合并律将$F_c$中$\alpha_1\to\beta_1$和$\alpha_1\to\beta_2$替换为$\alpha_1\to\beta_1\beta_2$
\State 对$F_c$每一个函数依赖移除无关属性
\Until{$F_c$不变}
\end{algorithmic}
\end{algorithm}

\subsubsection{无损分解与保持依赖的分解}
\begin{definition}[无损分解]
用$r(R_1)$和$r(R-2)$代替$r(R)$没有信息损失,则为无损分解。
\[\Pi_{R_1}(r)\Join\Pi_{R_2}(r)=r\]
\end{definition}

\begin{definition}[保持依赖的分解]
$F$是模式$R$上一个函数依赖集,$R_i$为$R$的一个分解,$F$在$R_i$上限定(restriction)是$F^+$中所有只包含$R_i$中属性的函数依赖的集合$F_i$。
令$F'=\bigcup_i F_i$,具有$F'^+=F^+$的分解为保持依赖的分解。
\end{definition}
% TODO: 保持依赖性验证

\subsection{分解算法}
\subsubsection{BCNF分解}
简化判定方法:
\begin{itemize}
	\item 检查非平凡函数依赖$\alpha\to\beta$,计算$\alpha^+$,验证它是否包含$R$中所有属性,即验证它是否为$R$的超码;不需检查$F^+$中所有函数依赖(但分解后就不能这么做了)
	\item 对$R_i$中属性每个子集$\alpha$,确保$F$下$\alpha^+$要么不含$R_i-\alpha$的任何属性,要么包含$R_i$所有属性。若$R_i$上有某个属性集$\alpha$违反条件,即下述函数依赖会出现在$F^+$中,说明违反BCNF
	\[\alpha\to(\alpha^+-\alpha)\cap R_i\]
\end{itemize}
\begin{algorithm}
\caption{BCNF分解}
\begin{algorithmic}[1]
\State 初始化result $:=\{R\}$
\State 计算$F^+$
\State 若result中存在模式$R_i$不属于BCNF,则将其分解为\textemph{$(R_i-\beta)$和$(\alpha,\beta)$},其中$\alpha\to\beta$为$R_i$上成立的非平凡函数依赖,满足$\alpha\to R_i$不属于$F^+$,且$\alpha\cap\beta=\varnothing$
\end{algorithmic}
\end{algorithm}

\subsubsection{3NF分解}
\begin{algorithm}
\caption{3NF分解}
\begin{algorithmic}[1]
\State 令$F_c$为$F$的\textemph{正则覆盖}
\State 对于每一个$F_c$中的函数依赖$\alpha\to\beta$,$R_i=\alpha\beta$
\State 若$R_i$都不包含$R$的候选码,则新增$R_{i+1}$为$R$的候选码
\State 若$R_j\in R_k$,则删除$R_j$,直至不再有可删除的$R_j$
\end{algorithmic}
\end{algorithm}

\subsubsection{总结}
应用函数以来进行数据库设计的目标是:
\begin{enumerate}
	\item BCNF
	\item 无损
	\item 保持依赖
\end{enumerate}

\subsection{多值依赖}
函数依赖规定了某些元组不能出现在关系中。
如果$A\to B$,则不能有两个元组在$A$上的值相同,而在$B$上值不同。
多值依赖不排除某些元组的存在,而要求某种形式的其他元组存在于关系中。
函数依赖称为相等产生(equality-generating)依赖,多值依赖称为元组产生依赖。
\begin{definition}[多值依赖(multivalued dependency)]
设$R$为关系模式,$\alpha\subset R,\beta\subset R$,多值依赖在$R$上满足$\alpha\to\to\beta$,% \twoheadrightarrow
当且仅当对于所有数对$t_1$和$t_2$,使得$t_1[\alpha]=t_2[\alpha]$,都存在数对$t_3$和$t_4$使得
\[\begin{aligned}
\displaystyle t_{1}[\alpha ]&= t_{2}[\alpha ]=t_{3}[\alpha ]=t_{4}[\alpha ]\\
\displaystyle t_{3}[\beta ]&= t_{1}[\beta ]\\
\displaystyle t_{3}[R-\beta ]&= t_{2}[R-\beta ]\\
\displaystyle t_{4}[\beta ]&= t_{2}[\beta ]\\
\displaystyle t_{4}[R-\beta ]&= t_{1}[R-\beta ]
\end{aligned}\]
简而言之,若${\displaystyle (a,b,c)}$和${\displaystyle (a,d,e)}$在$R$中,则${\displaystyle (a,b,e)}$和${\displaystyle (a,d,c)}$在$R$中。
即两个属性互相独立,但是都依赖于第三个属性。
\begin{center}
\begin{tikzcd}
 & b\arrow[rdd]\arrow[r] & c\\
a\arrow[ru]\arrow[rd] & &\\
 & d\arrow[ruu]\arrow[r] & e
\end{tikzcd}
\end{center}
典型例子是$(\text{课程},\text{课本},\text{老师})$,一门课有两本课本,两个老师,课本和老师之间不存在函数依赖,但都依赖于课程。
\end{definition}

令$D$表示函数依赖和多值依赖的集合,则$D^+$是由$D$逻辑蕴含的所有函数依赖和多值依赖的集合。
对于多值依赖有以下规则
\begin{itemize}
	\item 每个函数依赖都是一个多值依赖,即若$\alpha\to\beta$,则$\alpha\to\to\beta$
	\item 若$\alpha\to\to\beta$,则$\alpha\to\to R-\alpha-\beta$
\end{itemize}

第四范式(4NF):函数依赖和多值依赖集为$D$的关系模式$r(R)$属于第四范式的条件是对$D^+$中所有形如$\alpha\to\to\beta$的多值依赖$\alpha,\beta\subset R$,至少有以下之一成立:
\begin{itemize}
	\item $\alpha\to\to\beta$是一个平凡的多值依赖
	\item $\alpha$是$R$的一个超码(这里已经蕴含了$4NF\in BCNF$)
\end{itemize}