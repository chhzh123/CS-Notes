% !TEX root = main.tex

\section{关系数据库设计} % Chap 8
\subsection{范式理论}
设计需要满足一定的范式(normal form)
% https://www.geeksforgeeks.org/normal-forms-in-dbms/
\begin{itemize}
	\item 第一范式(1NF):全部属性都是单值属性

	\item 第二范式(2NF):关系模式$R\in 1NF$,且每一个非主属性\textbf{完全依赖}于$R$的主码,而不是只依赖于其中一部分属性(即其中一部分属性的值即可确定该属性的值,部分依赖)
	\begin{example}
	一个学生-课程关系模式如下:
	\begin{center}
	\begin{tabular}{|c|c|c|}\hline
		Stud\_No & Course\_No & Course\_Fee\\\hline
		1 & C1 & 1000\\
		2 & C2 & 1500\\
		1 & C4 & 2000\\
		4 & C3 & 1000\\
		4 & C1 & 1000\\
		2 & C5 & 2000\\\hline
	\end{tabular}
	\end{center}
	如此例中Stud\_No和Course\_No是主码,但是实际上只由Course\_No就已经可以决定Course\_Fee了,因此这个例子不满足2NF。
	\end{example}

	\item 第三范式(3NF):关系模式$R\in 2NF$,且对于$F^+$中所有形如$\alpha\to\beta$的函数依赖,至少下列一项成立
	\begin{itemize}
		\item $\alpha\to\beta$是一个平凡的函数依赖
		\item $\alpha$是$R$的一个超码
		\item $\beta-\alpha$中的每个属性$A$都含于$R$的一个候选码中
	\end{itemize}
	(即不存在传递依赖)
	\begin{example}
	一个学生-国家关系模式如下:
	\begin{tabular}{ccccc}\hline
	Stud No & Stud Name & Stud State & Stud Country & Stud Age\\\hline
	1 & Alice & s1 & c & 18\\
	2 & Alice & s2 & c & 19\\
	3 & Bob & s2 & c & 21\\\hline
	\end{tabular}
	有传递依赖Stud No$\to$Stud State$\to$Stud Country,因此不满足3NF。
	\end{example}

	\item BC范式(Boyce-Codd NF, 3.5NF):关系模式$R\in 3NF$,且对于每一个非平凡的函数依赖左侧都是一个超码
\end{itemize}

\subsection{函数依赖}
\begin{definition}[函数依赖(functional dependency)]
设$R$为关系模式,$\alpha\subset R,\beta\subset R$,函数依赖$\alpha\to\beta$在$R$上满足,当且仅当对于任意合法的关系$r(R)$,任何两个关于$r$的数对$t_1$和$t_2$,如果满足属性$\alpha$,那么它们也满足属性$\beta$,即
\[t_1[\alpha]=t_2[\alpha]\implies t_1[\beta]=t_2[\beta]\]
实际上就是函数单射的概念,属性$\alpha$可以唯一确定属性$\beta$的值。
若$\beta\subset\alpha$,则称函数依赖是平凡的。
\end{definition}
考虑下例$r(A,B)$的关系
\begin{center}
\begin{tabular}{|c|c|}\hline
A & B\\\hline
1 & 4\\\hline
1 & 5\\\hline
3 & 7\\\hline
\end{tabular}
\end{center}
关系$A\to B$不成立,但关系$B\to A$成立。

\begin{theorem}[逻辑蕴含公理]
下面的前三条为最基本的公理(Armstrong),可以找出给定$F$的所有$F^+$
\begin{itemize}
	\item 自反律(reflexivity):若$\alpha$为一属性集且$\beta\subset\alpha$,则$\alpha\to\beta$
	\item 增补律(augmentation):若$\alpha\to\beta$成立且$\gamma$为一属性集,则$\gamma\alpha\to\gamma\beta$成立
	\item 传递律(transitivity):若$\alpha\to\beta$和$\beta\to\gamma$成立,则$\alpha\to\gamma$成立
	\item 合并律(union):若$\alpha\to\beta$和$\alpha\to\gamma$成立,则$\alpha\to\beta\gamma$成立
	\item 分解律(decomposition):若$\alpha\to\beta\gamma$成立,则$\alpha\to\beta$和$\alpha\to\gamma$成立
	\item 伪传递律(pseudotransitivity):若$\alpha\to\beta$和$\gamma\beta\to\delta$成立,则$\alpha\gamma\to\delta$成立
\end{itemize}
\end{theorem}

\begin{algorithm}
\caption*{计算$F$下$\alpha$的闭包$\alpha^+$}
\begin{algorithmic}[1]
\State result$:=\alpha$
\Repeat
\For{每一个函数依赖$\beta\to\gamma\in F$}
\State 若$\beta\subset$ result,则result $:=$ result $\cup\gamma$
\EndFor
\Until{result 不变}
\end{algorithmic}
\end{algorithm}

\begin{definition}[多值依赖(multivalued dependency)]
设$R$为关系模式,$\alpha\subset R,\beta\subset R$,多值依赖在$R$上满足$\alpha\twoheadrightarrow\beta$,
当且仅当对于所有数对$t_1$和$t_2$,使得$t_1[\alpha]=t_2[\alpha]$,都存在数对$t_3$和$t_4$使得
\[\begin{aligned}
\displaystyle t_{1}[\alpha ]&= t_{2}[\alpha ]=t_{3}[\alpha ]=t_{4}[\alpha ]\\
\displaystyle t_{3}[\beta ]&= t_{1}[\beta ]\\
\displaystyle t_{3}[R-\beta ]&= t_{2}[R-\beta ]\\
\displaystyle t_{4}[\beta ]&= t_{2}[\beta ]\\
\displaystyle t_{4}[R-\beta ]&= t_{1}[R-\beta ]
\end{aligned}\]
简而言之,设${\displaystyle (x,y,z)}$有对应着属性${\displaystyle \alpha ,}{\displaystyle \beta ,}{\displaystyle R-\alpha -\beta }$,则对于任意$R$中的元素${\displaystyle (a,b,c)}$和${\displaystyle (a,d,e)}$,都有${\displaystyle (a,b,e)}$和${\displaystyle (a,d,c)}$在$R$中。
\end{definition}

第四范式(4NF):函数依赖和多值依赖集为$D$的关系模式$r(R)$属于第四范式的条件是对$D^+$中所有形如$\alpha\to\to\beta$的多值依赖$\alpha,\beta\subset R$,至少有以下之一成立:
\begin{itemize}
	\item $\alpha\to\to\beta$是一个平凡的多值依赖
	\item $\alpha$是$R$的一个超码
\end{itemize}