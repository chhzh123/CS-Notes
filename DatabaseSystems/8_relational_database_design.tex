% !TEX root = main.tex

\section{关系型数据库设计}
数据库设计的范式(normal form)
\begin{itemize}
	\item 第一范式:原子性,即所有元素都是不可分的单元
	\item 
\end{itemize}

\begin{definition}[函数依赖(functional dependency)]
设$R$为关系模式,$\alpha\subset R,\beta\subset R$,函数依赖$\alpha\to\beta$在$R$上满足,当且仅当对于任意合法的关系$r(R)$,任何两个关于$r$的数对$t_1$和$t_2$,如果满足属性$\alpha$,那么它们也满足属性$\beta$,即
\[t_1[\alpha]=t_2[\alpha]\implies t_1[\beta]=t_2[\beta]\]
实际上就是函数单射的概念。
\end{definition}
考虑下例$r(A,B)$的关系
\begin{center}
\begin{tabular}{|c|c|}\hline
A & B\\\hline
1 & 4\\\hline
1 & 5\\\hline
3 & 7\\\hline
\end{tabular}
\end{center}
关系$A\to B$不成立,但关系$B\to A$成立。

\begin{definition}[多值依赖(multivalued dependency)]
设$R$为关系模式,$\alpha\subset R,\beta\subset R$,多值依赖在$R$上满足$\alpha\twoheadrightarrow\beta$,
当且仅当对于所有数对$t_1$和$t_2$,使得$t_1[\alpha]=t_2[\alpha]$,都存在数对$t_3$和$t_4$使得
\[\begin{aligned}
\displaystyle t_{1}[\alpha ]&= t_{2}[\alpha ]=t_{3}[\alpha ]=t_{4}[\alpha ]\\
\displaystyle t_{3}[\beta ]&= t_{1}[\beta ]\\
\displaystyle t_{3}[R-\beta ]&= t_{2}[R-\beta ]\\
\displaystyle t_{4}[\beta ]&= t_{2}[\beta ]\\
\displaystyle t_{4}[R-\beta ]&= t_{1}[R-\beta ]
\end{aligned}\]
简而言之,设${\displaystyle (x,y,z)}$有对应着属性${\displaystyle \alpha ,}{\displaystyle \beta ,}{\displaystyle R-\alpha -\beta }$,则对于任意$R$中的元素${\displaystyle (a,b,c)}$和${\displaystyle (a,d,e)}$,都有${\displaystyle (a,b,e)}$和${\displaystyle (a,d,c)}$在$R$中
\end{definition}