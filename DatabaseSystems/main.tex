\documentclass{note}
\usepackage[cpp,table,pseudo]{mypackage}
\usepackage{footnote}
\makesavenoteenv{tabular}

\def\ojoin{\setbox0=\hbox{$\bowtie$}%
  \rule[-.02ex]{.25em}{.4pt}\llap{\rule[\ht0]{.25em}{.4pt}}}
\def\leftouterjoin{\mathbin{\ojoin\mkern-5.8mu\bowtie}}
\def\rightouterjoin{\mathbin{\bowtie\mkern-5.8mu\ojoin}}
\def\fullouterjoin{\mathbin{\ojoin\mkern-5.8mu\bowtie\mkern-5.8mu\ojoin}}

\renewcommand{\thefootnote}{\fnsymbol{footnote}}
\lstset{language=sql}

\title{数据库系统原理笔记}
\author{陈鸿峥}
\date{{\builddatemonth\today}\protect\footnote{\text{Build \builddate\today}}} % protect!

\begin{document}

\maketitle
\renewcommand{\thefootnote}{\arabic{footnote}}
\setcounter{footnote}{0}

\setcounter{tocdepth}{2}%设置深度
\tableofcontents

\bigskip\bigskip

% !TEX root = main.tex

\section{计算机网络概述}
计算机网络将终端设备连接起来并可以传输数据。

\subsection{网络连接方式}
\begin{enumerate}
\item 直接连接的网络(直连网)
\begin{itemize}
\item \underline{点对点(point-to-point)网络}:包括专用介质(dedicated medium)、节点/主机
\begin{itemize}
	\item 单向(simplex):如广播、电视
	\item 半双工(half duplex):异步双向,如对讲机
	\item 全双工(full duplex):同步双向,如电话
\end{itemize}
\item \underline{多路访问(multiple access)网络}:共享介质(shared medium),会产生碰撞(collision)
\begin{itemize}
	\item 单播(unicast):一对一
	\item 多播(multicast):一对多
	\item 广播(broadcast):一对所有
\end{itemize}
\end{itemize}
\item 间接连接的网络:涉及交换机、路由器
\end{enumerate}

\subsection{因特网}
用路由器或网关(gateway)连接起来构成的网络称为互\textred{连}网络(internetwork)。

因特网/互联网(Internet)是一种互连网络,可以看作是把世界各地的广域网互连的网络,是世界上最大的特定计算机网络,采用\textemph{TCP/IP协议簇}作为通信规则。
\begin{itemize}
	\item 系统域网(System Area Network, SAN):电脑、鼠标、USB
	\item 局域网(Local Area Network, LAN):某一区域内由多台计算机互联成的计算机组,一般是方圆几千米以内,如小型实验室;常用\textemph{多路访问网络}
	\item 城域网(Metropolitan Area Network, MAN)
	\item 广域网(Wide Area Network, WAN):\textemph{因特网}
\end{itemize}

\myhline
因特网设备:
\begin{itemize}
	\item 终端系统/主机(end system):运行网络应用程序,如手机、浏览器
	\item 通信链路(communication link):光纤、铜线、无线电、卫星等
	\item 路由器(router):用于连接多个网络形成更大的网络
\end{itemize}

\myhline
因特网的组成:ISP(Internet Service Provider)
\begin{itemize}
	\item 网络边界(network edge):主机及网络程序,终端设备可以通过本地ISP或区域ISP连接上互联网
	\item 接入网络/接入网(access network):有线或无线接入,连接订阅者和服务提供商,如WiFi
	\item 网络核心/主干网(core network):顶层ISP(中国电信、中国移动、中国网通),可以连接局部提供商
\end{itemize}

\subsection{网络服务}
通信服务类型:
\begin{itemize}
	\item 可靠/不可靠:会不会丢包/收发是否完全相同,如文件(可靠)/视频(不可靠)
	\item 面向连接/无连接:需不需要建立通信线路,如电话(连接,双方都要在)/寄信、因特网(无连接,对方可能不在)
	\item 有确认/无确认:需不需要确认对方是否收包,因特网不需要
	\item 请求响应/消息流服务:有请求才有响应/一直发消息,如电视
\end{itemize}

因特网是\textemph{数据报服务},\textemph{无连接无确认(尽力服务)}。

\subsection{因特网体系结构}
因特网体系结构包括以下这\textemph{五层},而ISO/OSI(open system interconnection)网络包括七层协议\footnote{也有TCP/IP四层的说法,将物理层和数据链路层合并起来变成物理网络层}:
\begin{itemize}
	\item 应用层:提供对某些专门应用的支持,如\underline{FTP、SMTP、HTTP}
	\item (OSI)表示层(presentationn):提供数据转换服务, 如\underline{加密解密,压缩解压缩,数据格式变换}
	\item (OSI)会话层(session):简化会话实现机制,如\underline{数据流的检查点设置和回滚,多数据流同步}
	\item 传输层:将网络层获得的包在\textemph{进程之间}数据传送(端到端),如\underline{TCP、UDP}
	\item 网络层:\textemph{路由选择},实现在互联网中的数据传送(主机到主机),如\underline{IP协议、路由协议}
	\item 数据链路层:在\textemph{物理网络}中传送\textemph{包}(跳到跳\footnote{一跳(hop)/节点为一个物理设备,即数据链路层只考虑直连网的情况},节点到节点),如\underline{PPP、Ethernet}
	\item 物理层:线上的\textemph{比特}(传送原始比特流)
\end{itemize}
\par 其中\textemph{网络层以下不可靠,以上可靠};防止丢包的机制:\textemph{重发}。
\par 物理层和数据链路层又被称为\underline{物理网络},网络层和传输层被称为\underline{逻辑网络}。

\myhline
协议(protocol):在网络实体(entities)之间传送消息的规则,如消息的格式、收发消息的次序等。

每层传输的数据单元都称为\textemph{包}(packets),都属于某个协议,又被称为\textemph{协议数据单元}(protocol data unit, PDU),包括\textemph{头部/协议控制信息}(potocal control data, PCI)和\textemph{服务数据单元}(service data unit, SDU)两部分。

\begin{minipage}{0.4\linewidth}
\begin{center}
\begin{tikzcd}
\text{应用层Application}\arrow{d}{\text{消息message}}\\
\text{传输层Transport}\arrow{d}{\text{数据段segment}}\\
\text{网络层Network}\arrow{d}{\text{数据报datagram}}\\
\text{链路层Data-link}\arrow{d}{\text{帧frame}}\\
\text{物理层Physical}
\end{tikzcd}
\end{center}
\end{minipage}
\begin{minipage}{0.6\linewidth}
\begin{figure}[H]
	\centering
	\includegraphics[width=\linewidth]{fig/ipencap.png}
\end{figure}
\end{minipage}

\bigskip
下层把上层通过服务访问点(service access point, SAP)传来的SDU用PCI封装为PDU后传给对等实体(peer entity),即实现相同协议的实体。
同一个互连网络中网络层协议需要相同,链路层协议可以不同。

\myhline
\begin{figure}[H]
	\centering
	\includegraphics[width=0.8\linewidth]{fig/network-flow.PNG}
	\caption*{协议栈(stack):发送时封装(encaptulation),接收时拆封。}
\end{figure}

\myhline
\begin{figure}[H]
	\centering
	\includegraphics[width=0.5\linewidth]{fig/protocol_family.png}
	\caption*{协议簇(protocol family)}
\end{figure}

\subsection{网络性能分析}
当一个包到达时如果有空闲缓存则排队等待转发,产生延迟(delay);
如果没有空闲缓存,则丢弃该包,造成丢失(loss)。

包交换网络中的延迟主要有以下四点:
\begin{itemize}
	\item 处理(processing)延迟:查路由,存储转发(store-and-forward)的延迟会很大
	\item 排队(queueing)延迟:依赖于路由器的拥塞程度
	\item 发送/传输(transmission)延迟:\[\text{传输延迟}=\text{包长(bits)}/\text{链路带宽(bps, bit per second)}\]
	指从发送第一个包到发送最后一个包的间隔
	\item 传播(propagation)延迟:指对于一个包来说从发送到接收所需的时间
	\[\text{传播延迟}=\text{物理链路长度}/\text{信号传播速度}\]
\end{itemize}

接收延迟与传播延迟重合。
故忽略掉处理、排队延迟,
\[\text{总延迟(从第一个包被发送到最后一个包被接收的时间)}=\text{传播延迟}+\text{发送延迟}\]

\myhline
\par 往返时间(round trip time, RTT):从源主机到目的主机再返回源主机所花的时间
\par 带宽(bandwidth):一条链路或通道可达到的\textemph{最大}数据传输速率(bps)
\par 吞吐量(thoughput):一条链路或通路\textemph{实际}数据传输速率

\begin{example}
	如果一个长度为$3000$字节的文件用一个数据包从源主机通过一段链路传给了一个交换机,然后再通过第二段链路到达目的主机。
	如果在包交换机的延迟为$2ms$,两条链路上的传播延迟都是$2\times 10^8m/s$,带宽都是$1Mbps$,长度都是$6000km$。
	采用以下三种方式,问这个文件在这两台主机之间的总延迟是多少?
	\begin{enumerate}
		\item 交换机采用存储转发方式
		\item 将文件分成10个数据包,且存储转发
		\item 收到一位转发一位
	\end{enumerate}
\end{example}
\begin{analysis}
	\begin{enumerate}
	\item 因采用存储转发技术,先计算一段的延时,最后乘2。
	\begin{itemize}
	\item 一段的传输延时:$3000B\times 8/10^6bps=24$ms
	\item 一段的传播延时:$6000km/(2\times 10^8m/s)=30$ms
	\item 转发延时:$2$ms
	\end{itemize}
	总时长:$(24+30)\times 2+2=110$ms
	\item 类似1,但是总时长是一个包的传输传播转发延迟,加上剩余包的接收/传输延迟,见下表加粗部分
	\begin{center}
		\begin{tabular}{|c|c|c|c|c|c|}\hline
			包1 & \textemph{传输} & \textemph{传播} & \textemph{接收} & & \\\hline
			包2 &  & 传输 & 传播 & \textemph{接收} & \\\hline
			包3 &  &  & 传输 & 传播 & \textemph{接收} \\\hline
		\end{tabular}
	\end{center}
	\begin{itemize}
		\item 一段的传输延时:$300B\times 8/10^6bps=2.4$ms
		\item 一段的传播延时:$30$ms
		\item 转发延时:$2$ms
	\end{itemize}
	总时长:$(2.4+30)\times 2+2+2.4\times 9=88.4$ms
	\item 同1,但是只用计算一段传输延时,因为1位的转发延迟忽略。
	故总时长:$24+30\times 2=84$ms
\end{enumerate}
\end{analysis} % Chap 1, 2
% !TEX root = main.tex

\section{SQL简介}
结构化查询语言(Structured Query Language, SQL)

数据定义语言(Data Definition Language, DDL)
\begin{lstlisting}[language=sql]
create table instructor (
	ID char(5),
	name varchar(20),
	dept_name varchar(20),
	salary numeric(8,2),
	primary key (ID),
	foreign key (dept_name) references department);
\end{lstlisting}

查询语言
\begin{lstlisting}[language=sql]
select A1, A2, ..., An
from r1, r2, ..., rm
where P
\end{lstlisting}
其中$A_i$为属性(attribute)、$R_i$为一个关系(relation)/一个表、$P$是谓词(preicate)

\begin{lstlisting}[language=sql]
select distinct dept_name
from instructor

select all dept_name
from instructor

select *
from instructor

select '437' as FOO

select ID, name, salary/12 as monthly_salary

select name
from instructor
where dept_name = 'Comp. Sci.' and salary > 80000

select *
from instructor, teaches -- Cartesian product

select distinct T.name
from instructor as T, instructor as S -- rename
where T.salary > S.salary and S.dept_name = 'Comp. Sci.'

/**
percent ( % ). The % character matches any substring.
underscore ( _ ). The _ character matches any character.
**/
select name
from instructor
where name like '%dar%' matches any string containing "dar" as a substring

select distinct name
from instructor
order by name

select name
from instructor
where salary between 90000 and 100000 -- both contain

(select course_id from section where sem = 'Fall' and year = 2017)
union -- intersect, except
(select course_id from section where sem = 'Spring' and year = 2018)

select name
from instructor
where salary is null

-- arg, min, max, sum, count
select avg (salary)
from instructor
where dept_name= 'Comp. Sci.';

select count (distinct ID)
from teaches
where semester = 'Spring' and year = 2018;
\end{lstlisting}

三值逻辑,添加了Unknown
\begin{center}
\begin{tabular}{|c|c|c|}\hline
AND & OR & NOT\\\hline
\begin{tabular}{l}
$T \land U = U$\\
$F \land U = F$\\
$U \land U = U$
\end{tabular} & 
\begin{tabular}{l}
$T \lor U = T$\\
$F \lor U = U$\\
$U \lor U = U$
\end{tabular} &
$\lnot U = U$\\\hline
\end{tabular}
\end{center} % Chap 3
% !TEX root = main.tex

\section{进阶SQL} % Chap 4
\subsection{内外连接}
\verb'on'条件可以提供比自然连接更为丰富的连接条件
\begin{lstlisting}
select *
from students join takes on student.ID = takes.ID; -- condition
\end{lstlisting}

直接使用\verb'natural join'可能导致元组丢失,比如有一些学生没有选修任何课程,则其在student关系中不会与takes关系中任何元组配对。
因此有\textbf{外连接}:
\begin{itemize}
	\item 左外连接(left outer join):只保留出现在左外连接运算之前(左边)关系中的元组,若右侧关系中没有对应属性则用null代替
	\item 右外连接(right outer join)
	\item 全外连接(full outer join):左外连接和右外连接的组合
\end{itemize}

从而可以很容易查询出“所有课程一门也没有选修的学生”:
\begin{lstlisting}
select ID
from student natural left outer join takes
where course_id is null;
\end{lstlisting}

为了将常规连接和外连接区分开,SQL中将常规连接称为内连接(inner join),默认的都是内连接。

\subsection{视图}
让用户看到整个逻辑模型显然是不合适的,出于安全考虑,可能需要向用户隐藏特定的数据。
本来通过select可以把需要的数据计算并存储下来,但一旦底层数据发生变化,查询的结果就不再匹配。
因此,为了解决这样的问题,SQL提供一种虚关系,称为视图(view),只有在使用的时候才会被计算。
\begin{lstlisting}
create view faculty as
select ID, name, dept_name
from instructor;
\end{lstlisting}
之后便可以直接使用\verb'from'语句访问视图。

\begin{definition}[物化视图(materialized view)]
如果用于定义视图的实际关系改变,视图也会跟着修改。
\end{definition}

由于视图并不是数据库底层的关系,故一般数据库不允许对视图关系进行修改。

\subsection{事务}
SQL标准规定当一条SQL语句被执行,就隐式开始了一个事务。
下列SQL语句之一会结束一个事务:
\begin{itemize}
	\item Commit work:将事务所做的更新在数据库中持久保存。
	在事务被提交后,一个新的事务自动开始。
	\item Rollback work:撤销该事务中所有SQL语句对数据库的更新,数据库将恢复到执行该事务的第一条语句之前的状态。
	如果遇断电,回滚会在下一次重启时自动执行。
\end{itemize}

\subsection{完整性约束}
单个关系上的约束
\begin{itemize}
	\item \verb'not null':跟在属性定义后面
	\item \verb'unique (A1,A2,...,Am)':声明$A_1,\ldots,A_m$构成一个候选码
	\item \verb'check(<predicate>)':关系中每个元组都必须满足该谓词,如\verb'check(budget>0)'
\end{itemize}

参照完整性/子集依赖:外码
\begin{lstlisting}
foreign key (dept_name) references department
	on delete cascade
	on update cascade
\end{lstlisting}
会级联(cascade)删除或更新。

\subsection{授权}
\verb'grant'授权,\verb'revoke'回收
\begin{itemize}
	\item \verb'select':允许用户修改关系中任意数组
	\item \verb'insert'
	\item \verb'delete'
\end{itemize}
用户名\verb'public'包括了当前所有用户和未来用户的权限。

\begin{lstlisting}
grant <authorization list>
on <relation name/view name>
to <user/user list>

grant update (budget) on department to Amit, Satoshi;
\end{lstlisting}

可以先创建角色(role)然后给用户授予角色。 % Chap 4
% !TEX root = main.tex

\section{形式化关系查询语言} % Chap 6
\subsection{关系代数}
\begin{center}
\begin{tabular}{|c|c|c|}
\hline
选择 & $\sigma$ &
\begin{tabular}{l}
挑选出符合一定性质的元组\\
$\sigma_{\text{Sub="Phy"}\land\text{age}>30}(\text{teachers})$
\end{tabular}\\\hline
投影 & $\Pi$ & 
\begin{tabular}{l}
只选出对应属性\\
$\Pi_{\text{ID,name,salary}}(\text{teachers})$
\end{tabular}\\\hline
笛卡尔积 & $\times$ & 将两个关系整合(简单并置,需要进一步筛选)\\\hline
自然连接 & $r\Join s=\prod_{R\cup S}(\sigma_{r.A_1=s.A_1\land\cdots\land r.A_n=s.A_n})(r\times s)$ & \\\hline
$\theta$连接 & $r\Join_\theta s=\sigma_\theta(r\times s)$ & \\\hline
外连接 & $\leftouterjoin, \rightouterjoin, \fullouterjoin$ & \\\hline
并集 & $\cup$ & 数目应相同,属性可兼容\\\hline
交集 & $\cap$ & \\\hline
差集 & $-$ & \\\hline
赋值 & $\gets$ & \\\hline
重命名 & $\rho_x(E)$ & 给$E$的返回值赋名为$x$\\\hline
\end{tabular}
\end{center}

扩展的关系代数运算:
\begin{itemize}
	\item 广义投影:$\prod_{F_1,F_2,\ldots,F_n}(E)$,其中$F_i$中每一个都是涉及常量及$E$的模式中属性的算术表达式
	\item 聚类:$\mathcal{G}$,如count, min, max,考虑group by可以写成下列形式
	\[{}_{dept\_name}\mathcal{G}_{\textbf{average}(salary)}(instructor)\]
\end{itemize}

\begin{definition}[聚集运算]
聚集运算$\mathcal{G}$的通常形式如下:
\[{}_{G_1,G_2,\ldots,G_n}\mathcal{G}_{F_1(A_1),\ldots,F_n(A_n)}(E)\]
其中$E$是任意关系代数表达式,$G_1,\ldots,G_n$是用于分组的一系列属性;
每个$F_i$是一个聚集函数,每个$A_i$是一个属性名。
$E$结果中的元组将会以如下方式分为若干组:
\begin{itemize}
	\item 同组中所有元组在$G_1,\ldots,G_n$上的取值相同
	\item 不同组中的元组在$G_1.\ldots,G_n$上的取值不同
\end{itemize}
\end{definition}

\subsection{其他关系演算}
\subsubsection{元组关系演算}
元组关系演算是非过程化的查询语言(有点像声明式语言),它只描述所需信息,而不给出获得该信息的具体过程。
查询表达为
\[\{t\mid P(t)\}\]
即使谓词$P$为真的元组$t$的集合,$t[A]$表示元组$t$在属性$A$上的值,用$t\in r$表示元组$t$在关系$r$中。

如工资大于80000美元的所有教师的ID为(注意这里涉及到属性选择)
\[\{t\mid\exists s\in instructor(t[ID]=s[ID]\land s[salary]>80000\}\]

其实与命题逻辑类似
\[t\in\text{instructor}\land\exists s\in\text{department}(t[\text{dept\_name}]=s[\text{dept\_name}])\]
前者$t$为自由变量,后者$s$为受限变量。

元组关系演算可能产生一个无限的关系
\[\{t\mid\lnot(t\in instructor)\}\]
这样的元组有无限多个且不在数据库中,因此这样写是\textbf{不安全的}。

因此引入元组关系公式$P$的域的概念,如$\dom(t\in instructor\land t[salary]>80000)$是包括80000和出现在instructor中的所有值集合。
若表达式$\{t\in P(t)\}$结果所有值均来自$\dom(P)$,则说表达式$\{t\in P(t)\}$是安全的。

\subsubsection{域关系演算}
\[\{<x_1,x_2,\ldots,x_n>\mid P(x_1,x_2,\ldots,x_n)\}\]
其中$<x_1,\ldots,x_n>\in r$,每一个$x_i$为域变量,$r$为$n$个属性上的关系。

与元组关系演算最大的区别在于,域关系演算更加细粒度。
比如找出工资大于80000美元的所有教师的ID:
\[\{<i>\mid\exists n,d,s(<i,n,d,s>\in instructor\land s>80000)\}\]

同样有安全性问题,可类似定义。 % Chap 6
% !TEX root = main.tex

\section{E-R模型} % Chap 7
在设计一个数据库模式时,要避免以下两个缺陷:
\begin{itemize}
	\item 冗余:可能导致不同关系表中的信息没有及时更新
	\item 不完整:只有对应开课的实体而没有对应课程的实体
\end{itemize}

\subsection{实体-关系模型}
\begin{definition}[实体]
实体是现实世界中可区别于所有其他对象的一个事物或对象。
每个实体有一组性质/属性(attribute),其中一些性质的值可以唯一标识一个实体。
实体集则是相同类型具有相同属性或性质的一个实体集合。
实体集是一个抽象概念,而实体集的外延(extension)则是指属于实体集实体的实际集合。
\end{definition}

大学中实际教师的集合构成了实体集instructor的外延。
实体集不必互不相交,如定义大学里所有人的实体集(person)。
一个person实体可以是instructior实体,可以是student实体,可以既是instructor实体又是student实体,也可以都不是。

\begin{definition}[联系]
联系(relationship)是指多个实体间的相互关联。
联系集是相同类型联系的集合。
\[\{(e_1,e_2,\ldots,e_n)\}\mid e_1\in E_1, e_2\in E_2, \ldots, e_n\in E_n\]
联系也可以具有描述性属性(如下右图)。
\begin{figure}[H]
\centering
\begin{tabular}{cc}
\includegraphics[width=0.5\linewidth]{fig/relationship_set.png}
\includegraphics[width=0.5\linewidth]{fig/relationship_set(description).png}
\end{tabular}
\end{figure}
实体集之间的关联称为参与(participation),若$E$中每个实体都参与到联系集$R$中的至少一个联系中,则称参与是全部的(total)的,否则是部分的(partial)。
E-R模式中的联系实例则为具体命名实体间的关联。
实体在联系中扮演的功能称为实体的角色(role)。
参与联系集的实体集数目称为联系集的度(degree),二元联系集的度为2,三元联系集度为3。
\end{definition}

\begin{definition}[属性]
每个属性都有一个可取值的集合,称为该属性的域(domain),或值集(value set)。
属性可以分为简单和复合(composite)属性,复合属性可以再划分为更小的部分,如名字分为名和姓。
也可能是单值或多值属性,比如老师可以有多个电话号码,这是多值属性。
还有派生(derived)属性,可以通过其他属性计算得出。
\end{definition}
通常如果一个属性在两个实体集中出现,且这两个实体集存在关联,则该属性是冗余的,需要被移除。

\begin{definition}[映射基数(mapping cardinality)/基数比率]
一个实体通过一个联系集能关联的实体个数,包括了一对一、一对多(many)、多对一、多对多这几种情况。
one代表至多一个,many代表零个或多个。
\begin{figure}[H]
\centering
\includegraphics[width=\linewidth]{fig/mapping_radix.png}
\end{figure}
\end{definition}
也可以采用更为复杂的映射基数$l..h$的形式表示,其中$l$表示最小映射基数,$h$为最大映射基数。
最小值为$1$代表这个实体集在该联系集中全部参与,最大值为$1$表示这个实体至多参与一个联系,最大值为$*$则没有限制。

\begin{definition}[强弱实体集]
没有足够属性形成主码的实体集称为弱实体集,有主码的实体集则称为强实体集。
弱实体集必须与另一个称作标识(identifying)或属主(owner)实体集的实体集关联才有意义。
弱实体集的分辨符(discriminator)/部分码是使得能够区分弱实体集中实体的方法,用虚下划线标明。
\end{definition}

E-R图,菱形代表联系,双横线代表全部参与。
\begin{figure}[H]
\centering
\includegraphics[width=0.8\linewidth]{fig/university_E-R.png}
\end{figure}
\begin{figure}[H]
\centering
\includegraphics[width=0.5\linewidth]{fig/E-R-figure.png}
\end{figure}

一个常见的错误是将一个实体集的主码作为另一个实体集的属性(关系被隐藏了),而不是使用联系。

\subsection{扩展的E-R属性}
\begin{definition}[特化和泛化]
在实体集内部进行分组的过程称为特化(specialization),如将实体集person划分为employee和student。
概化(generalization)则是自底向上的反过程。
高层与低层实体集也可被分别称为超类和子类,同样有继承(inheritance)的特性。
如果一个实体集作为低层实体集参与到多个联系中,则称这个实体集具有多继承,且产生的结构称为格(lattice)。
\end{definition}

\begin{definition}[聚集]
聚集是一种抽象,其中联系集和跟它们相关的实体集一起被看作高层实体集,并且可以参与联系。
\end{definition} % Chap 7
% !TEX root = main.tex

\section{关系数据库设计} % Chap 8
大的模式会存在大量冗余,小的模式会导致信息丢失(有损分解)。

\subsection{范式理论}
定义属性集$\alpha$,关系模式为$r(R)$。

设计需要满足一定的范式(normal form),核心目的是\textcolor{red}{减少冗余}:
% https://www.geeksforgeeks.org/normal-forms-in-dbms/
\begin{itemize}
	\item 第一范式(1NF):全部属性都是\textemph{单值属性}(原子性)

	\item 第二范式(2NF):关系模式$R\in 1NF$,且每一个\textemph{非主属性\textbf{完全依赖}于$R$的主码},而不是只依赖于其中一部分属性(即其中一部分属性的值即可确定该属性的值,部分依赖)
	\begin{example}
	一个学生-课程关系模式如下:
	\begin{center}
	\begin{tabular}{|c|c|c|}\hline
		Stud\_No & Course\_No & Course\_Fee\\\hline
		1 & C1 & 1000\\
		2 & C2 & 1500\\
		1 & C4 & 2000\\
		4 & C3 & 1000\\
		4 & C1 & 1000\\
		2 & C5 & 2000\\\hline
	\end{tabular}
	\end{center}
	如此例中Stud\_No和Course\_No是主码,但是实际上只由Course\_No就已经可以决定Course\_Fee了,因此这个例子不满足2NF。
	\end{example}

	\item 第三范式(3NF):关系模式$R\in 2NF$,且对于$F^+$中所有形如$\alpha\to\beta$的函数依赖,至少下列一项成立
	\begin{itemize}
		\item $\alpha\to\beta$是一个平凡的函数依赖
		\item $\alpha$是$R$的一个超码
		\item $\beta-\alpha$中的每个属性$A$都含于$R$的一个候选码中
	\end{itemize}
	(非主属性依赖于主码但不能通过另一非主属性进行依赖,即\textemph{不存在传递依赖})
	\begin{example}
	一个学生-国家关系模式如下:
	\begin{center}
	\begin{tabular}{ccccc}\hline
	Stud No & Stud Name & Stud State & Stud Country & Stud Age\\\hline
	1 & Alice & s1 & c & 18\\
	2 & Alice & s2 & c & 19\\
	3 & Bob & s2 & c & 21\\\hline
	\end{tabular}
	\end{center}
	有传递依赖Stud No$\to$Stud State$\to$Stud Country,因此不满足3NF。
	\end{example}

	\item 巴斯-科德/BC范式(Boyce-Codd NF, 3.5NF):关系模式$R\in 3NF$,对于$F^+$中所有形如$\alpha\to\beta$的函数依赖,至少有以下一项成立:
	\begin{itemize}
		\item $\alpha\to\beta$是平凡的函数依赖(即$\beta\subset\alpha$)
		\item $\alpha$是模式$R$的一个超码
	\end{itemize}
\end{itemize}

\subsection{函数依赖}
\begin{definition}[函数依赖(functional dependency)]
设$R$为关系模式,$\alpha\subset R,\beta\subset R$,函数依赖$\alpha\to\beta$在$R$上满足,当且仅当对于任意合法的关系$r(R)$,任何两个关于$r$的数对$t_1$和$t_2$,如果满足属性$\alpha$,那么它们也满足属性$\beta$,即
\[t_1[\alpha]=t_2[\alpha]\implies t_1[\beta]=t_2[\beta]\]
实际上就是\textemph{函数单射}的概念,属性$\alpha$可以\textemph{唯一确定}属性$\beta$的值,
若函数依赖$K\to R$在$r(R)$上成立,则$K$是$r(R)$的一个\textemph{超码}。
\end{definition}
\begin{example}
考虑下例$r(A,B)$的关系
\begin{center}
\begin{tabular}{|c|c|}\hline
A & B\\\hline
1 & 4\\\hline
1 & 5\\\hline
3 & 7\\\hline
\end{tabular}
\end{center}
关系$A\to B$不成立,但关系$B\to A$成立。
\end{example}

\begin{definition}[平凡]
在所有关系中都满足的函数依赖则是平凡的函数依赖,比如$A\to A$。
一般地,若$\beta\subset\alpha$,则$\alpha\to\beta$的函数依赖是平凡的。
\end{definition}

\subsubsection{闭包}
\begin{definition}[逻辑蕴含与闭包]
若关系模式$r(R)$的每一个满足$F$的实例也满足$f$,则$R$上的函数依赖$f$被$r$上的函数依赖集$F$逻辑蕴含。
$F$的闭包是被$F$逻辑蕴含的所有函数依赖的集合,记作$F^+$。
设$\alpha$为属性集,将函数依赖集$F$下被$\alpha$函数确定的所有属性的集合称为$F$下$\alpha$的闭包。
\end{definition}

\begin{theorem}[逻辑蕴含公理]
下面的前三条为最基本的公理(Armstrong),可以找出给定$F$的所有$F^+$
\begin{itemize}
	\item 自反律(reflexivity):若$\alpha$为一属性集且$\beta\subset\alpha$,则$\alpha\to\beta$
	\item 增补律(augmentation):若$\alpha\to\beta$成立且$\gamma$为一属性集,则$\gamma\alpha\to\gamma\beta$成立
	\item 传递律(transitivity):若$\alpha\to\beta$和$\beta\to\gamma$成立,则$\alpha\to\gamma$成立
	\item 合并律(union):若$\alpha\to\beta$和$\alpha\to\gamma$成立,则$\alpha\to\beta\gamma$成立
	\item 分解律(decomposition):若$\alpha\to\beta\gamma$成立,则$\alpha\to\beta$和$\alpha\to\gamma$成立
	\item 伪传递律(pseudotransitivity):若$\alpha\to\beta$和$\gamma\beta\to\delta$成立,则$\alpha\gamma\to\delta$成立
\end{itemize}
\end{theorem}
计算$F^+$的算法即不断应用自反律、增补律和传递律。

\begin{algorithm}
\caption{计算$F$下$\alpha$的闭包$\alpha^+$}
\begin{algorithmic}[1]
\State result$:=\alpha$
\Repeat
\For{每一个函数依赖$\beta\to\gamma\in F$}
\State 若$\beta\subset$ result,则result $:=$ result $\cup\gamma$
\EndFor
\Until{result 不变}
\end{algorithmic}
\end{algorithm}

\subsubsection{正则覆盖}
\begin{definition}[无关(extraneous)]
如果去除函数依赖中的一个属性不改变该函数依赖集的闭包,则称该属性是无关的,即\textemph{去掉该属性依然可以推得其函数依赖成立}。
\end{definition}

检验属性$A$是否无关的方法如下:
\begin{itemize}
	\item 若$A\in\beta$,为检验$A$是否无关,考虑集合
	\[F'=(F-\{\alpha\to\beta\})\cup\{\alpha\to(\beta-A)\}\]
	检验$\alpha\to A$是否能由$F'$推出,计算$F'$下的$\alpha^+$,若$\alpha^+$包含$A$,则$A$在$\beta$中无关。
	\item 若$A\in\alpha$,为检验$A$是否无关,令$\gamma=\alpha-\{A\}$,检验$\gamma\to\beta$是否可以由$F$推出,
	计算$F$下的$\gamma^+$,若$\gamma^+$包含$\beta$中所有属性,则$A$在$\alpha$中无关。
\end{itemize}

\begin{definition}[正则覆盖(canonical cover)]
$F$的正则覆盖$F_c$是一个依赖集,使得$F_c$和$F$双向逻辑蕴含,且
\begin{itemize}
	\item $F_c$中任何函数依赖都不含无关属性
	\item $F_c$中函数依赖的左半部都是唯一的,即不存在$\alpha_1\to\beta_1$和$\alpha_2\to\beta_2$满足$\alpha_1=\alpha_2$
\end{itemize}
\end{definition}
\begin{algorithm}
\caption{计算正则覆盖}
\begin{algorithmic}[1]
\State $F_c=F$
\Repeat
\State 用合并律将$F_c$中$\alpha_1\to\beta_1$和$\alpha_1\to\beta_2$并为$\alpha_1\to\beta_1\beta_2$
\State 对$F_c$每一个函数依赖移除无关属性
\Until{$F_c$不变}
\end{algorithmic}
\end{algorithm}

\subsubsection{无损分解与保持依赖的分解}
\begin{definition}[无损分解]
用$r(R_1)$和$r(R-2)$代替$r(R)$没有信息损失,则为无损分解。
\[\Pi_{R_1}(r)\Join\Pi_{R_2}(r)=r\]
\end{definition}

\begin{definition}[保持依赖的分解]
$F$是模式$R$上一个函数依赖集,$R_i$为$R$的一个分解,$F$在$R_i$上限定(restriction)是$F^+$中所有只包含$R_i$中属性的函数依赖的集合$F_i$。
令$F'=\bigcup_i F_i$,具有$F'^+=F^+$的分解为保持依赖的分解。
\end{definition}
% TODO: 保持依赖性验证

\subsection{分解算法}
\subsubsection{BCNF分解}
\begin{enumerate}
	\item 初始化result $:=\{R\}$
	\item 计算$F^+$
	\item 若result中存在模式$R_i$不属于BCNF,则根据下式进行分解
	\[\text{result} := (\text{result}-R_i)\cup(R_i-\beta)\cup(\alpha,\beta)\]
	其中$\alpha\to\beta$为$R_i$上成立的非平凡函数依赖,满足$\alpha\to R_i$不属于$F^+$,且$\alpha\cap\beta=\varnothing$
\end{enumerate}

\subsubsection{3NF分解}
\begin{enumerate}
	\item 令$F_c$为$F$的正则覆盖
	\item 对于每一个$F_c$中的函数依赖$\alpha\to\beta$,$R_i=\alpha\beta$
	\item 若$R_j\in R_k$,则删除$R_j$
	\item 直至不再有可删除的$R_j$
\end{enumerate}

\subsubsection{总结}
应用函数以来进行数据库设计的目标是:
\begin{enumerate}
	\item BCNF
	\item 无损
	\item 保持依赖
\end{enumerate}

\subsection{多值依赖}
函数依赖规定了某些元组不能出现在关系中。
如果$A\to B$,则不能有两个元组在$A$上的值相同,而在$B$上值不同。
多值依赖不排除某些元组的存在,而要求某种形式的其他元组存在于关系中。
函数依赖称为相等产生(equality-generating)依赖,多值依赖称为元组产生依赖。
\begin{definition}[多值依赖(multivalued dependency)]
设$R$为关系模式,$\alpha\subset R,\beta\subset R$,多值依赖在$R$上满足$\alpha\to\to\beta$,% \twoheadrightarrow
当且仅当对于所有数对$t_1$和$t_2$,使得$t_1[\alpha]=t_2[\alpha]$,都存在数对$t_3$和$t_4$使得
\[\begin{aligned}
\displaystyle t_{1}[\alpha ]&= t_{2}[\alpha ]=t_{3}[\alpha ]=t_{4}[\alpha ]\\
\displaystyle t_{3}[\beta ]&= t_{1}[\beta ]\\
\displaystyle t_{3}[R-\beta ]&= t_{2}[R-\beta ]\\
\displaystyle t_{4}[\beta ]&= t_{2}[\beta ]\\
\displaystyle t_{4}[R-\beta ]&= t_{1}[R-\beta ]
\end{aligned}\]
简而言之,若${\displaystyle (a,b,c)}$和${\displaystyle (a,d,e)}$在$R$中,则${\displaystyle (a,b,e)}$和${\displaystyle (a,d,c)}$在$R$中。
即两个属性互相独立,但是都依赖于第三个属性。
\begin{center}
\begin{tikzcd}
 & b\arrow[rdd]\arrow[r] & c\\
a\arrow[ru]\arrow[rd] & &\\
 & d\arrow[ruu]\arrow[r] & e
\end{tikzcd}
\end{center}
典型例子是$(\text{课程},\text{课本},\text{老师})$,一门课有两本课本,两个老师,课本和老师之间不存在函数依赖,但都依赖于课程。
\end{definition}

令$D$表示函数依赖和多值依赖的集合,则$D^+$是由$D$逻辑蕴含的所有函数依赖和多值依赖的集合。
对于多值依赖有以下规则
\begin{itemize}
	\item 每个函数依赖都是一个多值依赖,即若$\alpha\to\beta$,则$\alpha\to\to\beta$
	\item 若$\alpha\to\to\beta$,则$\alpha\to\to R-\alpha-\beta$
\end{itemize}

第四范式(4NF):函数依赖和多值依赖集为$D$的关系模式$r(R)$属于第四范式的条件是对$D^+$中所有形如$\alpha\to\to\beta$的多值依赖$\alpha,\beta\subset R$,至少有以下之一成立:
\begin{itemize}
	\item $\alpha\to\to\beta$是一个平凡的多值依赖
	\item $\alpha$是$R$的一个超码(这里已经蕴含了$4NF\in BCNF$)
\end{itemize} % Chap 8
% !TEX root = main.tex

\section{存储与文件结构} % Chap 10
\subsection{存储器件}
物理存储:易失(volatile)、非易失

磁盘被划分为道、扇区、柱面

常见磁盘性能衡量指标:
\begin{itemize}
	\item 访问时间:
	\begin{itemize}
		\item 寻道(seek)时间:到达特定的道
		\item 旋转时间
	\end{itemize}
	\item 数据传输速率
	\item 平均失效时间(MTTF):通常3-5年
\end{itemize}

通过并行来提高性能:数据拆分(striping)
\begin{itemize}
	\item 比特级拆分:每个字节按比特分开,存储到多个磁盘上
	\item 块级拆分:将块拆分到多张磁盘,即将磁盘阵列看成一张单独的大磁盘,并给块进行逻辑编号
\end{itemize}

独立磁盘冗余阵列(Redundant Array of Independent Disk, RAID)

\begin{minipage}[H]{0.5\linewidth}
\begin{itemize}
	\item RAID-0:块级拆分但无冗余
	\item RAID-1:块级拆分+冗余,数据库常用
	\item RAID-2:纠错码(Error-Correcting-Code, ECC),可以读出其余位和纠错位重建信息
	\item RAID-3:可以检测一整个扇区是否被正确读出
	\item RAID-4:在一张独立磁盘上为其他$N$张磁盘对应的块保留一个奇偶校验块
	\item RAID-5:将数据和奇偶校验位分布到$N+1$张磁盘中,所有磁盘都能参与读请求
	\item RAID-6:P+Q冗余方案,可应对多张磁盘故障情况
\end{itemize}
\end{minipage}
\begin{minipage}[H]{0.5\linewidth}
\begin{figure}[H]
\centering
\includegraphics[width=0.5\linewidth]{fig/RAID.png}
\end{figure}
\end{minipage}

\subsection{文件结构}
\subsubsection{定长记录}
每次插入删除都涉及后续元组的移动,可以如下维护一个空闲列表(free list)
\begin{figure}[H]
\centering
\includegraphics[width=0.6\linewidth]{fig/free_lists.png}
\end{figure}

\subsubsection{变长记录}
变长记录以以下几种方式出现:
\begin{itemize}
	\item 多种记录类型在一个文件中存储
	\item 允许一个或多个字段是变长的记录类型
	\item 允许可重复字段的记录类型,如数组或多重集合
\end{itemize}
\begin{figure}[H]
\centering
\includegraphics[width=0.5\linewidth]{fig/variable-length.png}
\end{figure}

用分槽的页结构(slotted-page)来处理块中变长记录的问题
\begin{figure}[H]
\centering
\includegraphics[width=0.5\linewidth]{fig/slotted_page.png}
\end{figure}

块头包含以下信息
\begin{itemize}
	\item 块头中记录条目的个数
	\item 块中空闲空间的末尾处
	\item 由包含记录位置和大小的记录条目组成的数组
\end{itemize}
注意实际记录是从\textemph{块的尾部}往前插。

\subsection{文件中记录的组织}
\begin{itemize}
	\item 堆文件组织:一条记录可放在文件中任何地方,只要那个地方有空间存放这条记录
	\item 顺序文件组织:根据其搜索码值顺序存储
	\item 散列文件组织:散列函数
\end{itemize}

多表聚簇组织
\begin{figure}[H]
\centering
\includegraphics[width=0.5\linewidth]{fig/multitable_clustering.png}
\end{figure} % Chap 10
% !TEX root = main.tex

\section{索引与散列} % Chap 11
\subsection{基本概念}
通常索引文件包含记录/索引项,要比原文件小很多
\begin{center}
\begin{tabular}{|c|c|}\hline
search-key & pointer\\\hline
\end{tabular}
\end{center}

两种基本的索引类型:顺序(ordered)索引、散列(hash)索引

\subsection{顺序索引}
\begin{definition}[聚集索引]
在顺序索引中,包含记录的文件按照某个\textbf{搜索码}指定顺序排序,则该搜索码对应的索引称为聚集索引(clustering index),也被称为主索引(primary)。
但注意并\textbf{不是建立在主码上的索引}。
若搜索码指定的顺序与文件中记录的物理顺序不同,则称为非聚集索引(secondary)。
\end{definition}
\begin{definition}[稠密/稀疏索引]
对于每个搜索码都出现索引记录的称为稠密索引,只包含特定搜索码值的称为稀疏索引。
非聚集索引一定是稠密的。
\end{definition}

稀疏索引查找方法即找到最大搜索码键值小于$K$的,然后继续往下搜索。
\begin{figure}[H]
\centering
\includegraphics[width=0.6\linewidth]{fig/dense_sparse_indexing.png}
\end{figure}

通过建造多级稀疏索引,来减少访问时间。
\begin{figure}[H]
\centering
\includegraphics[width=0.5\linewidth]{fig/multilevel-index.png}
\end{figure}

索引更新操作:
\begin{itemize}
	\item 删除:稠密连同搜索码直接删,稀疏判大小
	\item 插入:同上直接插
\end{itemize}

\subsection{B+树索引}
当文件大起来时上述索引变得很慢,周期性更新整个文件非常麻烦,因此有B+树自动管理小的局部插入删除。

\subsubsection{基本性质}
B+树索引采用平衡树结构,树根到树叶的每条路径长度相同。
$n$-阶B树满足以下性质:
\begin{itemize}
	\item 所有叶子结点都必须在同一层
	\item 除了根结点外的\textbf{非叶子节点}都至少有$\lceil n/2\rceil$个孩子,至多$n$个孩子
	\item 叶子结点至少有\textcolor{red}{$\lceil (n-1)/2\rceil$}个关键字,最多$n-1$个关键字
	\item 特殊情况:
	\begin{itemize}
		\item 根节点不是叶子,则至少有$2$个孩子
		\item 根节点是叶子,则可以有$[0,n-1]$个值
	\end{itemize}
	\item 每个结点有$n$个指针,$n-1$个键值/搜索码\textbf{升序排序}(通常是严格不等式)
	\begin{center}
	\begin{tabular}{|c|c|c|c|c|c|c|}\hline
	$P_1$ & $K_1$ & $P_2$ & $\cdots$ & $P_{n-1}$ & $K_{n-1}$ & $P_n$\\\hline
	\end{tabular}
	\end{center}
	\item 左侧的叶子结点的搜索码值一定都小于\textbf{等于}右侧的叶子结点的搜索码值
	\item 指针与其\textbf{后面}的键值为一组,即$P_i$指向搜索码值为$K_i$的记录(叶子结点),或$P_i$指向$[K_{i-1},K_i)$(非叶子节点,注意右侧不包含);$P_n$指向下一个叶子节点
\end{itemize}

而B+树则是将所有关键字存储在叶子结点,其他结点作为索引,并且为每个叶子结点增加一个链指针。
\begin{figure}[H]
\centering
\includegraphics[width=0.6\linewidth]{fig/bp-tree.png}
\end{figure}

根节点下至少有$2\lceil n/2\rceil$个键值,下一层至少$2\lceil n/2\rceil^2$个,以此类推;
若记录中有$K$个搜索码值,则B+树树高不会高过$\lceil\log_{\lceil n/2\rceil}(K)\rceil$。

% 重复?

\subsubsection{插入}
\begin{itemize}
	\item 搜索码值已经存在于叶子结点:直接加记录
	\item 搜索码值没有出现,添加记录到主文件,插入到叶子节点;如果没有位置,需要进行分裂
	\begin{itemize}
		\item 将这$n$个$(\text{搜索码值},\text{指针})$进行排序,取前$\lceil n/2\rceil$个在原来节点,其余在新节点
		\item 令新节点为$p$,$k$为$p$中的最小值,则插入$(k,p)$到父亲中,$p$为$k$的后续指针
		\item 若父亲满了,则继续分裂并传播
	\end{itemize}
\end{itemize}
\begin{figure}[H]
\centering
\includegraphics[width=0.8\linewidth]{fig/bp-tree_insertion.png}
\end{figure}

\subsubsection{删除}
\begin{itemize}
	\item 直接移除
	\item 如果删除导致下溢(underfull),则需要合并兄弟
	\begin{itemize}
		\item 若\textemph{兄弟节点}未满,则与兄弟节点合并(注意不是同层都是兄弟)
		\item 删除$(K_{i-1},P_i)$,其中$P_i$是指向删除结点的指针,递归这个过程
	\end{itemize}
	\item 如果合并没法合到一个节点,则需要\textbf{重分配}(redistribute)指针使得两个节点都含有多于最小键值数目的项。
	将左侧分配到右侧,并重新分配指针
\end{itemize}
\begin{figure}[H]
\centering
\includegraphics[width=0.8\linewidth]{fig/bp-tree_deletion.png}
\end{figure}
\begin{figure}[H]
\centering
\includegraphics[width=0.8\linewidth]{fig/bp-tree_deletion2.png}
\end{figure}
\begin{figure}[H]
\centering
\includegraphics[width=0.8\linewidth]{fig/bp-tree_deletion3.png}
\end{figure}

\subsubsection{其他事项}
\begin{itemize}
	\item 前缀压缩:比如``abcde''和``abds''可以通过``ab''进行区分
	\item 多码索引:构成元组进行索引
\end{itemize}

\subsubsection{B+树文件组织}
\begin{figure}[H]
\centering
\includegraphics[width=0.8\linewidth]{fig/bp-tree_file_organization.png}
\end{figure}

\subsubsection{B树}
\begin{figure}[H]
\centering
\includegraphics[width=0.8\linewidth]{fig/b-tree.png}
\end{figure}

\subsection{静态散列}
散列(hash)函数应该满足:
\begin{itemize}
	\item 分布均匀:为每个桶分配同样数量的搜索码值
	\item 分布随机:不应与搜索码的任何外部可见排序相关
\end{itemize}

两种散列结构:
\begin{itemize}
	\item 闭散列:一个桶的溢出桶都用链表链接在该桶后面形成溢出链(overflow chaining)
	\item 开散列:线性探查法(probing)插入到下一个有空间的桶
\end{itemize}

下例的散列函数为ID各位数字之和对8取模
\begin{figure}[H]
\centering
\includegraphics[width=\linewidth]{fig/hash_index.png}
\end{figure}

静态散列问题:
\begin{itemize}
	\item 太小导致后期冲突多性能下降
	\item 太大则大量空间被浪费
\end{itemize}

\subsection{动态散列}
当数据库增大或缩小时,可扩充散列(extendable hashing)可通过桶的分裂或合并来适应数据库大小变化。

通过哈希函数的前$i$位确定索引
\begin{figure}[H]
\centering
\includegraphics[width=0.6\linewidth]{fig/extendable_hash.png}
\end{figure}
但缺点在于查找涉及一个附加的间接层,系统在访问桶本身之前必须先访问桶地址表。

\subsection{位图索引}
一共是元组个数$n$位,若第$i$个元组的该属性为某特定值,则设为1,否则置0。
\begin{figure}[H]
\centering
\includegraphics[width=0.9\linewidth]{fig/bitmap.png}
\end{figure} % Chap 11
% !TEX root = main.tex

\section{事务} % Chap 14
\begin{definition}[事务(transaction)]
构成单一逻辑工作单元的操作集合称为事务。
即使有故障,数据库系统也必须保证数据库的正确执行——要么执行整个事务,要么属于该事务的操作一个也不执行。
\end{definition}

事务具有以下的基本性质:
\begin{itemize}
	\item 原子性:要么执行完,要么不执行,不能执行到一半。
	\item 一致性:除了基本的数据完整性约束,还有更多的一致性约束。
	\item 隔离性:每个事务都察觉不到系统中有其他事务在并发执行,一定是完成一个再进行下一个。
	\item 持久性:一个事务成功完成对数据库的改变是永久的,即使出现系统故障。
\end{itemize}

事务的基本状态:
\begin{itemize}
	\item 活动的(active):初始状态
	\item 部分提交的(partially commited):最后一条语句执行后
	\item 失败的(failed):执行出错
	\item 中止的(aborted):事务回滚且数据库已恢复到事务开始执行前
	\item 提交的(commited):成功完成
\end{itemize}
\begin{figure}[H]
\centering
\includegraphics[width=0.5\linewidth]{fig/transaction_state.png}
\end{figure}

\begin{definition}[冲突]
当$I$和$J$是不同事务在相同数据项上的操作,并且其中至少有一个是\verb'write'指令时,则称$I$与$J$是冲突的。
\end{definition}
\begin{definition}[冲突等价(conflict equivalent)]
如果调度$S$可以经过一系列非冲突指令交换转换为$S'$,则称$S$和$S'$是冲突等价的。
若$S'$为串行调度,则$S$为冲突可串行化的(conflict serializable)。
\end{definition}

可以通过构造一个优先图(precedence graph)来判断是否冲突可串行化。
如果无环(用环检测算法),则冲突可串行化,可以通过拓扑排序得到串行化顺序。

隔离性级别:
\begin{itemize}
	\item 可串行化(serializable)
	\item 可重复读(repeatable read)
	\item 已提交读(read committed)
	\item 未提交读(read uncommitted)
\end{itemize}

隔离性级别的实现:并发控制机制
\begin{itemize}
	\item 锁
	\item 时间戳:读时间戳记录读该数据项的事务的最大/最近时间戳,写时间戳记录写入该数据项当前值的事务的时间戳。
	时间戳用来确保在访问冲突情况下,事务按照事务时间戳顺序来访问数据项。
	当不可能访问时,违例事务会中止,并且分配一个新的时间戳重新开始
	\item 多版本与快照隔离(snapshot isolation)
\end{itemize} % Chap 14
% !TEX root = main.tex

\section{并发控制} % Chap 15
\subsection{基于锁的协议}
\begin{itemize}
	\item 共享锁(shared):如果事务$T_i$获得了数据项$Q$上的共享型锁(记为$S$),则$T_i$可读但不能写$Q$
	\item 排他锁(exclusive):如果事务$T_i$获得了数据项$Q$上的排他型锁(记为$X$),则$T_i$既可读又可写$Q$
\end{itemize}

\begin{table}
\centering
\caption{锁相容性矩阵comp}
\begin{tabular}{|c|c|c|}\hline
 & $S$ & $X$\\\hline
$S$ & true & false\\\hline
$X$ & false & false\\\hline
\end{tabular}
\end{table}

令$\{T_0,\ldots,T_n\}$是参与调度$S$的一个事务集,若存在数据项$Q$,使得$T_i$在$Q$上持有$A$型锁,之后$T_j$在$Q$上持有$B$型锁,且$comp(A,B)=false$,则称在$S$中$T_i$先于$T_j$,记为$T_i\to T_j$,即在任何等价的串行调度中,$T_i$必须出现在$T_j$之前。

两阶段封锁协议:可保证冲突可串行化,但不能保证不发生死锁
\begin{enumerate}
	\item 增长(growing)阶段:事务可以获得锁,但不能释放锁
	\item 缩减(shrinking)阶段:事务可以释放锁,但不能获得新锁
\end{enumerate}

自动为事务产生适当的加锁、解锁指令:
\begin{itemize}
	\item 事务$T_i$进行$read(Q)$操作时,系统产生一条$lock-S(Q)$指令,该$read(Q)$指令紧跟其后
	\item 事务$T_i$进行$write(Q)$操作时,系统检查$T_i$是否已在$Q$上持有共享锁。
	若有,则系统发出$upgrade(Q)$指令,后接$write(Q)$指令。
	否则系统发出$lock-X(Q)$指令,后接$write(Q)$指令。
	\item 当一个事务提交或中止后,该事务持有的所有锁都被释放。
\end{itemize}

\subsection{基于图的协议}
要求所有数据项集合$D=\{d_1,d_2,\ldots,d_n\}$满足偏序$\to$:如果$d_i\to d_j$,则任何既访问$d_i$又访问$d_j$的事务必须首先访问$d_i$,然后访问$d_j$。
偏序意味着集合$D$可以视为有向无环图,称为数据库图。

在树形协议中,可用的加锁指令只有lock-X。
每个事务$T_i$对一数据项最多能加一次锁,并且遵从以下规则:
\begin{enumerate}
	\item $T_i$首次加锁可以对任何数据项进行
	\item 此后,$T_i$对数据项$Q$加锁的前提是$T_i$当前持有$Q$的父项上的锁
	\item 对数据项解锁可以随时进行
	\item 数据项被$T_i$加锁并解锁后,$T_i$不能再对该数据项加锁
\end{enumerate}
所有满足树形协议的调度都是冲突可串行化的,且保证不会发生死锁。

\subsection{基于时间戳的协议}
若事务$T_i$已赋予时间戳$TS(T_i)$,此时有一新事务$T_j$进入系统,则$TS(T_i)<TS(T_j)$,可利用\textbf{系统时钟}或者\textbf{逻辑计数器}实现。
事务的时间戳决定了串行化顺序,因此若$TS(T_i)<TS(T_j)$,则系统必须保证所产生的串行调度等价于事务$T_i$出现在事务$T_j$之前的某个串行调度。

\begin{itemize}
	\item 若$TS(T_i)\geq$W-timestamp(Q),则执行读操作
	\item 若$TS(T_i)\geq$R/W-timestamp(Q),则执行写操作
	\item 其他情况都会导致回滚
\end{itemize} % Chap 15
% !TEX root = main.tex

\section{恢复系统} % Chap 16
\subsection{故障分类}
\begin{itemize}
	\item 事务故障(failure)
	\begin{itemize}
		\item 逻辑错误:事务由于某些内部条件而无法继续正常执行,如非法输入。
		\item 系统错误:系统进入一种不良状态(如死锁),事务无法正常运行,但是之后某个时间点能够重新执行。
	\end{itemize}
	\item 系统崩溃(crash):硬件故障,或数据库软件/操作系统的漏洞,导致易失性存储器内容丢失,事务停止。
	\item 磁盘故障(failure):数据传送过程中由于磁头损坏或故障造成的磁盘块上内容丢失。
\end{itemize}

恢复算法包括两个部分:
\begin{itemize}
	\item 在正常事务执行过程中做的动作,获得之后能够从错误中恢复的信息
	\item 在数据库发生故障时执行的动作,使得恢复后可以确保原子性、一致性和持久性
\end{itemize}

\subsection{恢复与原子性}
记录以下操作(所有日志都应该在操作前被写入)
\begin{itemize}
\item 当一个事务$T_i$开始时,它会记录一个日志$<T_i\quad\text{start}>$。
\item 在每一个写操作$write(X)$之前,记录日志$<T_i,X,V_1,V_2>$,其中$V_1$是旧值,$V_2$是新值。
\item 当$T_i$完成它最后一条指令时,记录日志$<T_i\quad\text{commit}>$被写入。只有当提交日志被写入稳定存储,这个事务才能被称为已提交。
\end{itemize}

事务回滚操作:
\begin{enumerate}
	\item \textbf{从后往前}扫描日志,对于所发现的$T_i$的每个形如$<T_i,X_j,V_1,V_2>$的日志记录:
	\begin{enumerate}
		\item 值$V_1$被写到数据项$X_j$中,且
		\item 往日志中写一个特殊的只读日志记录$<T_i,X_j,V_1>$,其中$V_1$是在本次回滚中数据项$X_j$恢复成的值。这种日志称作补偿日志记录(compensation log record)。
	\end{enumerate}
	\item 一旦发现$<T_i,start>$的日志记录,就停止从后往前扫描,并往日志中写一个$<T_i,abort>$的日志记录
\end{enumerate}

使用日志来重做(redo)和撤销(undo)事务:
\begin{itemize}
	\item Redo指写入新值$V_2$
	\item Undo指写回旧值$V_1$
\end{itemize}

Redo阶段:
\begin{itemize}
	\item 找到最后一个检查点$<\text{checkpoint}\quad L>$,设置undo-list为$L$
	\item 从上面的检查点做前向扫描,
	\begin{itemize}
		\item 若$<T_i,X_j,V_1,V_2>$被发现,则重做,将$V_2$写到$X_j$
		\item 当$<T_i\quad\text{start}>$被发现,则添加$T_i$到undo-list中
		\item 当$<T_i\quad\text{commit}>$或$<T_i\quad\text{abort}>$被发现,从undo-list中移除$T_i$
	\end{itemize}
\end{itemize}

Undo阶段:从后面往前扫日志
\begin{itemize}
	\item 若$<T_i,X_j,V_1,V_2>$被发现且$T_i$在undo-list中,则
	\begin{itemize}
		\item 执行undo,将$V_1$写到$X_j$
		\item 写日志$<T_i,X_j,V_1>$
	\end{itemize}
	\item 当$<T_i\quad\text{start}>$被发现且$T_i$在undo-list中,则
	\begin{itemize}
		\item 写日志$<T_i\quad\text{abort}>$
		\item 将$T_i$从undo-list中移除
	\end{itemize}
	\item 当undo-list为空时停止,即$<T_i\quad\text{start}>$在undo-list中被每一个事务找到
\end{itemize}
\begin{figure}[H]
\centering
\includegraphics[width=0.8\linewidth]{fig/recovery.png}
\end{figure}

但每个操作都要写日志就很慢,因此采用检查点(checkpoint)进行打包。
\begin{figure}[H]
\centering
\includegraphics[width=0.6\linewidth]{fig/checkpoints_eg.png}
\end{figure}
其中,
\begin{itemize}
	\item $T_1$可被忽略,因为已经写入磁盘
	\item 检查点是$T_c$,故$T_2$和$T_3$需要Redo
	\item 而$T_4$需要undo
\end{itemize} % Chap 16

\end{document}

% 1,2,3,4,6,7,8,10,11,14,15.1-15.2,16.1-16.4