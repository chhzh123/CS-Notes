% !TEX root = main.tex

\section{彩色图像处理}
图像主要包括三方面内容:颜色、形状、纹理

\subsection{颜色特性}
彩色光的3个基本量:
\begin{itemize}
	\item 辐射率:从光源流出能量的总量,用Watt表示
	\item 光强:观察者从光源接受的能量总和,用流明表示
	\item 亮度:主观描绘子,人感觉到的
\end{itemize}

人眼中有600-700万个锥状体分别对红色(700nm)、绿色(546nm)、蓝色(435.8nm)敏感\footnote{注意这里只是人为定义波长,真正的红绿蓝是一段波长区间。}。
65\%对红光敏感、33\%对绿光敏感、2\%对蓝光敏感。

三基色原理:
红绿蓝为三种基色,组成RGB三维加性空间
\begin{figure}[H]
\centering
\includegraphics[width=0.8\linewidth]{fig/rgb_and_cmyk.png}
\end{figure}

区别颜色的特性:亮度、色调、色饱和度。
颜色通常用亮度和彩色表征,色调、和饱和度统称为彩色色度。
设$X,Y,Z$为红绿蓝的系数,则做归一化,得到颜色的唯一比例
\[\begin{aligned}
x&=\frac{X}{X+Y+Z}\\
x&+y+z=1
\end{aligned}\]

真彩色(24b):RGB各用一个字节表示,共216种安全色(各种设备都可以正常显示),剩下40个为控制字节

\subsection{颜色空间}
\begin{itemize}
	\item RGB:与人眼视觉系统密切相连
	\item CNY/CNYK:青色(cyan)、品红(magenta)、黄色(yellow)、黑色(key/black)。打印主要靠反射(减性空间),如黄色是白光将蓝色吸收掉。由于油墨很少能将颜色都吸收掉,深色效果较差,故加入一种黑色K。
	\item HIS/HSL/HSV:色度(hue)、亮度(intensity)、饱和度(saturation),亮度与色彩分离。广泛应用于计算机视觉、图像检索和视频检索。
\end{itemize}

\subsection{伪彩色处理}
根据一定的准则对灰度值赋以彩色的处理。
之所以需要伪彩色,是因为人类可以辨别上千种颜色和强度,但只能辨别二十几种\textbf{灰度}。
比如将不同灰度赋予不同颜色,得到热度图(heatmap)。

\subsection{全彩色处理}
两种处理方式:分通道处理、向量处理。

补色:两种颜色混在一起为白色(RGB补色为CMY)
\begin{figure}[H]
\centering
\includegraphics[width=0.6\linewidth]{fig/complements.png}
\end{figure}

\subsection{彩色分割}
\begin{itemize}
	\item HSI颜色空间分割直观,H色调图像方便描述彩色,S饱和度图像做模板分离感兴趣的特征区,I强度图像不携带彩色信息。如用门限产生二值图像,大于门限的像素赋为1,其他赋为0。
	\item RGB彩色空间直接,用欧式距离度量。
\end{itemize}
% 彩色空间的人脸检测

如果直接采用3个独立平面形成的合成梯度图可能导致彩色边缘检测错误,因此要采用Di Zenzo提出的方法。