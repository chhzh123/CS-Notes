% !TEX root = main.tex

\section{图像分割}
图像分割一般基于亮度值的两种基本特性(\textbf{不连续性}和\textbf{相似性})来分割。

\begin{definition}[分割]
令$R$为全图,可将分割看作$R$划分为$n$个子区域$R_1,R_2,\ldots,R_n$的过程:
\begin{enumerate}
	\item $\disp\bigcup_{i=1}^n R_i=R$
	\item $R_i$是一个连通区域
	\item $R_i\cap R_j=\varnothing$
	\item $Q(R_i)=TRUE$
	\item $Q(R_i\cup R_j)=FALSE$,对于任何$R_i$和$R_j$的邻接区域
\end{enumerate}
其中,$Q(R_k)$为定义在集合$R_k$的点上的逻辑属性
\end{definition}

\subsection{点检测与线检测}
\subsubsection{点检测}
\begin{center}
\begin{tabular}{|c|c|c|}\hline
1 & 1 & 1\\\hline
1 & -8 & 1\\\hline
1 & 1 & 1\\\hline
\end{tabular}
\end{center}
若作用算子后的图像$|R(x,y)|\geq T$,则记为$1$。

\subsubsection{线检测}
\begin{figure}[H]
\centering
\includegraphics[width=0.9\linewidth]{fig/line_detect.png}
\end{figure}

\subsection{边缘检测}
傅里叶变换无法刻画边缘,只知道高频成分,不知道高频在哪里。
一种方法是局部傅里叶变换,衍生出小波变换(就是要构造一种高通滤波器):有震荡信号的位置(小范围震荡且积分为0),可以刻画边缘。

台阶/阶梯、斜坡/斜、屋顶/Delta边缘模型如下。
\begin{figure}[H]
\centering
\includegraphics[width=0.9\linewidth]{fig/edge_model.png}
\end{figure}

如下图所示,二阶导数会增大噪声,因此做边缘检测之前应该先抑制噪声(平滑)。
\begin{figure}[H]
\centering
\includegraphics[width=0.9\linewidth]{fig/edge_detect_noise.png}
\end{figure}

常用的边缘检测算子:一阶微分-梯度,二阶微分-Laplace算子
\begin{figure}[H]
\centering
\includegraphics[width=0.9\linewidth]{fig/edge_detect_op.png}
\end{figure}

\subsubsection{Marr-Hildreth边缘检测器}
由于拉普拉斯算子的应用通常会放大图像的噪声,因此通常先平滑,再应用拉普拉斯算子。
假设$f(x,y)$为图像,$h(x,y)$为高斯平滑函数
\[h(x,y)=-\ee^{-\frac{x^2+y^2}{2\sigma^2}}\]
则只需做一步操作完成平滑及边缘检测
\[\nabla^2 [f(x,y)*h(x,y)]=f(x,y)*[\nabla^2 h(x,y)]\]
其中
\[\nabla^2 h(x,y)=\frac{2}{\sigma^2}\lrs{1-\frac{x^2+y^2}{2\sigma^2}}\ee^{-\frac{x^2+y^2}{2\sigma^2}}\]
即高斯拉普拉斯(LoG)变换。

墨西哥草帽函数。
(高斯函数的微分就是一种小波,做微分后负号与负号抵消;而且高斯函数有无穷阶导数)
\begin{figure}[H]
\centering
\includegraphics[width=0.6\linewidth]{fig/LoG.png}
\end{figure}

得到Marr-Hildreth的边缘检测算法
\begin{enumerate}
	\item 计算高斯拉普拉斯变换(LoG)
	\item 找到零交叉点(边缘正负变化的地方)
\end{enumerate}

用高斯差分DoG滤波可以近似代替LoG,
\[DoG(x,y)=\frac{1}{2\pi\sigma_1^2}\ee^{-\frac{x^2+y^2}{2\sigma_1^2}}-\frac{1}{2\pi\sigma_2^2}\ee^{-\frac{x^2+y^2}{2\sigma_2^2}},\;\sigma_1>\sigma_2\]
以$\sigma_1:\sigma_2=1.6:1$来定,则近似的$\sigma$为
\[\sigma^2=\frac{\sigma_1^2\sigma_2^2}{\sigma_1^2-\sigma_2^2}\ln\lrs{\frac{\sigma_1^2}{\sigma_2^2}}\]

\subsubsection{Canny边缘检测器}
目标:
\begin{itemize}
	\item 低错误率:边缘一个不落,一个不多
	\item 边缘点应该被很好定位:标记的边缘点与真实边缘中心之间的距离最小
	\item 单一边缘点响应:对于真实的边缘点,检测器仅返回一个点,即真实边缘周围的局部最大数应该最小
\end{itemize}

算法步骤:
\begin{enumerate}
	\item 用一个高斯滤波器平滑输入图像
	\item 计算梯度幅值图像和角度图像(Canny为二阶梯度算子)
	\[M(x,y)=\sqrt{g_x^2+g_y^2},\quad\alpha(x,y)=\arctan\lrs{\frac{g_y}{g_x}}\]
	\item 对梯度幅值图像应用非最大抑制
\begin{figure}[H]
\centering
\includegraphics[width=0.6\linewidth]{fig/canny-margin.png}
\end{figure}
\begin{enumerate}
	\item 寻找最接近$\alpha(x,y)$的方向$d$
	\item $g_N$代表细化后的边缘
	\[g_N(x,y)=\begin{cases}0 & M(x,y)\text{的值至少小于沿$d$的两个邻居之一}\\M(x,y) & \text{否则}\end{cases}\]
\end{enumerate}
	\item 用双阈值处理和连接分析来检测并连接边缘\\
	双阈值法:高阈值$T_H$和低阈值$T_L$,比率为$2:1$或$3:1$
	\[\begin{aligned}
	g_{NN}(x,y) &= g_N(x,y)\geq T_H & \text{强边缘}\\
	g_{NL}(x,y) &= g_N(x,y)\geq T_L & \text{弱边缘(可能是边缘也可能不是)+强边缘}\\
	g_{NL}(x,y) &= g_{NL}(x,y)-g_{NH}(x,y) & \text{弱边缘}
	\end{aligned}\]
	用弱边缘补齐强边缘来获得完整边缘
\begin{enumerate}
	\item 在$g_{NN}(x,y)$中定位下一个未被访问的边缘像素$p$
	\item 在$g_{NL}(x,y)$中用8连通方法连接到$p$
	\item 如果$g_{NN}(x,y)$中所有非零标记都已经访问过,则跳到(d),否则(a)
	\item 将$g_{NL}(x,y)$中未被标记为有效边缘的像素的所有像素置零。
	将$g_{NL}(x,y)$中非零像素附加到$g_{NN}(x,y)$
\end{enumerate}
\end{enumerate}

\subsection{边缘连接}
\textbf{边界是封闭的边缘。}
\[\text{边界检测}=\text{边缘检测}+\text{边缘连接}\]

\begin{itemize}
\item 边缘连接:两个端点只有在边缘强度和走向相近的情况下才能连接。
如果像素$(s,t)$在像素$(x,y)$的邻域内且满足:
\[|M(x,y)-M(s,t)|\leq E,\quad |\alpha(x,y)-\alpha(s,t)|\leq A\]
则可以将$(s,t)$与$(x,y)$连接起来。

\item 边界跟踪:依照角度搜索,用$3\times 3$区域平均值代替单像素点(避免噪声影响),称为虫
\begin{figure}[H]
\centering
\includegraphics[width=0.5\linewidth]{fig/boundary_tracing.png}
\end{figure}

\item 区域处理:用多边形拟合算法
\begin{figure}[H]
\centering
\includegraphics[width=0.5\linewidth]{fig/polygon_boundary.png}
\end{figure}
\end{itemize}

\subsection{边界检测}
用Hough变换进行边界检测:通过边界点找已知形状的目标。 % 考试必考

\subsubsection{直线检测问题}
已知一组边缘点,找一条直线,使它通过最多边缘点。

直线方程用极坐标表示
\[\rho=x\cos\theta+y\sin\theta\]
通过辅助角变换可得
\[\rho=A_0\sin(\theta+\phi_0)\]
故可映射到$\rho O\theta$空间,其中每一个点是$xOy$平面上的通过同一个点的一条线。
\begin{figure}[H]
\centering
\includegraphics[width=0.6\linewidth]{fig/hough.png}
\end{figure}

如果有一组位于由参数$\rho_0$和$\theta_0$决定的直线上的边缘点,则每个边缘点对应了$\rho,\theta$空间的一条正弦曲线。
所有这些曲线必然会交于点$(\rho_0,\theta_0)$,因为这是它们共享的一条直线的参数。
\begin{figure}[H]
\centering
\includegraphics[width=0.6\linewidth]{fig/hough2.png}
\end{figure}

故对于边缘点的直线拟合问题,即找一个使边缘点确定的正弦曲线相交最多的点$(\rho,\theta)$。

可以建立$\rho,\theta$空间的二维直方图来确定对于边缘点的最佳拟合直线参数$(\rho_0,\theta_0)$。
具体算法如下:
\begin{enumerate}
	\item 对于每个边缘点$(x,y)$。建立直线方程:
	\[\rho=x\cos(\theta)+y\sin(\theta)\]
	\item 假定$\rho,\theta$的变化范围为$\rho\in[\rho_{\min},\rho_{\max}],\theta\in[\theta_{\min},\theta_{\max}]$,建立$A(\rho,\theta)$累加器
	\item 给定$\theta$,由原始方程确定$\rho$,则$A(\rho,\theta)+=1$
	\item 对于所有的边缘点,执行上述步骤,找出最大的$A(\rho,\theta)$,即为所求
\end{enumerate}

\subsubsection{其他检测问题}
换成圆/椭圆也一样
\[(x-x_0)^2+(y-y_0)^2=r^2\]
在参数空间建立3D累加数组$A$,让$(x_0,y_0)$变化,计算$r$。


\subsection{阈值处理}
阈值处理模型
\[T=T[x,y,p(x,y),f(x,y)\]
如果
\[g(x,y)=\begin{cases}
1 & f(x,y)>T\\
0 & f(x,y)\leq T
\end{cases}\]
当$T$为适用于整个图像的常数时,则该公式给出的处理为全局阈值处理,$f(x,y)>T$的任何点为一个对象点,否则该点为背景点。
如果取决于邻域$p(x,y)$,则为局部阈值处理。
如果取决于$(x,y)$本身,则为动态阈值处理/自适应阈值处理。

背景光照不均匀时阈值处理方法:
\begin{enumerate}
	\item 直接矫正方法:用恒定灰度的平坦表面成像获得光照模式,用相反的模式与图像相乘来矫正
	\item 采用顶帽变换来获得全局阴影模式
	\item 使用可变阈值近似处理非均匀性
\end{enumerate}

\subsubsection{基本全局阈值处理}
试探法:
\begin{enumerate}
	\item 为全局阈值$T$选择一个初始估计值
	\item 用$T$分割图像,生成两组像素:$G_1$由所有灰度值大于$T$的像素组成,而$G_2$   由所有灰度值小于或等于$T$的像素组成。
	\item 对区域$G_1$和$G_2$中的所有像素计算平均灰度值$m_1$和$m_2$
	\item 计算新的门限值:$T=1/2(m_1+m_2)$
	\item 重复步骤2到4,直到逐次迭代所得的$T$值之差小于实现定义的参数$\Delta T$
\end{enumerate}

\subsubsection{Otsu最佳全局阈值处理}
将其看为一个分类问题,用贝叶斯决策。

\begin{enumerate}
\item 归一化直方图$p_i=n_i/(NM)$
\item 给定阈值$T(k)$,将其分为左右两个部分$C_1$和$C_2$
\item 求出两块的条件概率密度$p_i/P_1(k)$和$p_i/P_2(k)$
\item 定义归一化度量指标$\eta=\frac{\sigma_B^2}{\sigma_G^2}$\\
其中$\sigma_G^2$为全局方差
\[\sigma_G^2=\sum_{i=0}^{L-1}(i-m_G)^2p_i\]
$\sigma_B^2$为类间方差
\[\begin{aligned}
\sigma_B^2 &= P_1(m_1-m_G)^2+P_2(m_2-m_G)^2\\
&=P_1P_2(m_1-m_2)^2\\
&=\frac{(m_G P_1-m)^2}{P_1(1-P_1)}
\end{aligned}\]
\item 基本结论是$m_1$和$m_2$相隔越大,则$\sigma_B^2$越大;$\eta$是分割的可分性度量,$\sigma_B^2$越大,则$\eta$越大。
进而有最佳阈值
\[\sigma_B^2(k^*)=\max_{0\leq k\leq L-1}\sigma_B^2(k)\]
\end{enumerate}

\subsubsection{预处理}
用图像平滑来改善全局阈值处理:但难以处理单峰

用边缘来改进全局阈值处理:梯度算法能能够区分边缘区域与平坦区域;拉普拉斯算子可以确定给定的像素在边缘的亮一边还是暗一边。
局部阈值法:
\[s(x,y)=\begin{cases}
0 & \nabla f<T\\
+ & \nabla f\geq T, \nabla^2 f\geq 0\\
- & \nabla f\geq T, \nabla^2 f< 0
\end{cases}\]
\begin{enumerate}
	\item 计算$f(x,y)$的梯度图或者拉普拉斯绝对值图
	\item 给定一个阈值
	\item 用步骤2)的阈值对步骤1)的结果做阈值处理(通常取第$n$个百分比),产生二值图像$g_T(x,y)$(可以是梯度二值图和拉普拉斯绝对值图二值图取“或”的组合)
	\item 计算$h(x,y)=f(x,y).*g_T(x,y)$,计算$h(x,y)$的直方图
	\item 在4)直方图的基础上,用Otsu方法做图像分割
\end{enumerate}

\subsubsection{多阈值处理}
在$K$个类的情况下,同样可以定义类间方差,来计算最大的归一化度量指标

\subsubsection{可变阈值处理}
通过图像分块的阈值处理,因为块内的光照近似均匀


\subsection{基于区域的分割}
$f(x,y)$为图像,$S(x,y)$为种子阵列,种子处为1,其他为0。
$Q$为位置$(x,y)$的属性,基于8连通的区域生长算法为:
\begin{enumerate}
	\item 在$S(x,y)$中找连通分量,并将连通分量腐蚀为一个像素;
	将找到的所有这种像素标记为$1$,$S$的其他像素标记为$0$
	\item 计算
	\[f_Q(x,y)=\begin{cases}
	1 & \text{$(x,y)$处属性$Q$为真}\\
	0 & \text{其他}
	\end{cases}\]
	\item 分割后图像$g$为:把$f_Q$中与种子点8连通的所有1值点添加到$S$中的每个种子点
	\item 使用不同的区域标记出$g$中的每个连通分量
\end{enumerate}

\begin{itemize}
	\item 区域生长法
	\item 区域分离合并法
\end{itemize}