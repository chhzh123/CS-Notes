% !TEX root = main.tex

\section{特征选择}
特征选择是从原始特征中挑选出最有代表性、性能最好的特征,而特征提取则是用映射(或变换)的方法将原始特征变换维较少的新特征;降维则是通过原始特征构造新的特征,新的特征能够更好表示原始数据,且特征数量较原始特征少。

\subsection{子集搜索与子集评价}
特征选择的两个关键环节:子集搜索和子集评价

产生初始候选子集,评价候选子集的好坏,基于评价结果产生下一个候选子集。
\begin{itemize}
	\item 前向搜索:逐渐增加相关特征
	\item 后向搜索:从完整特征集合开始,逐渐减少特征
	\item 双向搜索:每一轮逐渐增加相关特征,同时减少无关特征
\end{itemize}

\subsection{特征选择方法}
\begin{itemize}
	\item 过滤型(filter):先进行特征选择,再进行建模;
	评价指标:
	\begin{itemize}
		\item 皮尔逊相关系数:系数取值区间为$[-1,1]$,$-1$完全负相关,$+1$完全正相关,$0$则没有相关
		\[r=\frac{\mathop{cov}(X,Y)}{\mathop{var}(X)\mathop{var}(Y)}\]
		计算简单,但是不能反映变量之间的非线性关系
		\item 信息增益:特征子集$A$确定了对数据集$D$的一个划分,$A$上的取值将数据集$D$分为$V$份,每一份用$D^v$表示,$\mathop{Ent}(D^v)$表示$D^v$上的信息熵
		\[\mathop{Gain}(D,A)=\mathop{Ent}(D)-\sum_{v=1}^V\frac{|D^v|}{|D|}\mathop{Ent}(D^v)\]
	\end{itemize}
	\item 封装型(wrapper):直接将学习器的性能作为特征子集的评价标准,如召回率、准确率、AUC等指标。LVW (Las Vegas Wrapper)采用随机策略进行子集搜索,并以最终分类器的误差作为特征子集评价标准
	\begin{enumerate}
		\item 在循环的每一轮随机产生一个特征子集
		\item 在随机产生的特征子集上通过交叉验证推断当前特征子集的误差
		\item 多次循环,在多个随机产生的特征子集中选择误差最小的特征子集作为最终解
	\end{enumerate}
	\item 嵌入型(embedded):将特征选择和模型训练融为一体
	\begin{itemize}
		\item 正则化方法:L1范数LASSO回归$\lambda\norm{\vw}_1$,L2范数岭回归$\lambda\norm{\vw}_2$
		\item 随机森林/决策树:平均不纯度减少,平均精确率减少
		\item 递归特征消除(Recursive Feature Eliminiation, RFE):反复构建模型选出最好/最坏的特征,将选出来的特征放到一边,然后在剩余特征上重复这个过程,遍历所有特征。
		在这个过程中特征被消除的次序就是特征的排序。
		消除的稳定性很大程度取决于模型迭代时选用的底层模型。
	\end{itemize}
	\item 无监督特征选择:Laplacian score选择那些能够最好保持数据流形结构的特征子集
\end{itemize}

数据和特征决定了机器学习的上限,而模型和算法意在逼近这个上限,特征工程就是最大限度地从原始数据中提取特征以供算法和模型使用。