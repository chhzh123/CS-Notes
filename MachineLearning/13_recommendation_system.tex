% !TEX root = main.tex

\section{推荐系统}
$X$为用户的集合,$S$为推荐项的集合,效用函数$U:X\times S\to R$,$R$是评分的集合,全序。

效用矩阵$U$往往是稀疏的,大多用户对大多数商品都没有评价、冷启动(新用户/新商品)。

\subsection{基于内容}
基于内容的(content-based)方法核心思想是推荐与该用户之前评分高的项目相似的项目。

对每一个项目(item),创建项目特写(profile),即其特征。
如电影的导演、名字、演员等。
选取重要特征的方法,如果从文本中获取可用词频-逆文档频率(term frequency-inverse document frequency, TF-IDF)模型。
\[\begin{aligned}
TF &= \frac{\text{单词在句子中出现的频率}}{\text{句子的单词总数}}\\
IDF &= \frac{\text{总的句子数目}}{\text{包含单词的句子数目}}
\end{aligned}\]
用户特写则可通过对项目特写\textbf{取平均}得到。

给定用户特写$\vx$和项目特写$\vi$,可用下式计算相似度
\[u(\vx,\vi)=\cos(\vx,\vi)=\frac{\vx\cdot\vi}{\norm{\vx}\cdot\norm{\vi}}\]

基于内容方法的优点:
\begin{itemize}
	\item 无需其他用户数据
	\item 能够匹配不同用户独特的taste
	\item 能够适应新的/非大众项目
	\item 可解释性强
\end{itemize}
缺点:
\begin{itemize}
	\item 寻找适合的特征很难
	\item 如何建立新用户的画像
	\item 过度专一,不会推荐用户内容以外的东西,无法适应用户多种兴趣
\end{itemize}

\subsection{协同过滤}
考虑用户$x,选出$$n$个与$x$的评分相似的用户,基于这些用户的评分,估计$x$的评分。
相似性可用Jaccard、Cosine、Pearson度量。
这是用户-用户(user-user)协同过滤的方式。

项-项(item-item)协同过滤定义$s_{ij}$为商品$i$和商品$j$的相似度,选择$i$的$k$个最近邻$N(i;x)$(被用户$x$打分最相近$i$的商品),估计用户$x$在商品$i$上的评分
\[r_{xi}=b_{xi}+\frac{\sum_{j\in N(i;x)}s_{ij}\cdot(r_{xj}-b_{xj})}{\sum_{j\in N(i;x)s_{ij}}}\]
其中$b_{xi}=\mu+b_x+b_i$为基线偏移量,$\mu$为评分总平均,$b_x$为用户$x$评分的标准偏差($\bar{r_x}-\mu$),$b_i$为商品$i$的评分偏差。

优点:适用于任何商品,无需特征选择
\par缺点:
\begin{itemize}
	\item 冷启动:需要有足够的用户来进行匹配
	\item 稀疏性:很难找到用户同时评分同个商品
	\item 首评分用户:无法推荐未被评分的商品
	\item 欢迎度偏差:无法满足个人独特喜好
	\item 时间复杂度:$k$个最相似用户计算量大,需要先降维或聚类
\end{itemize}

利用插值方法构造可学习权重
\[\hat{r_{xi}}=b_{xi}+\sum_{j\in N(i;x)}w_{ij}(r_{xj}-b_{xj})\]

混合方法:
\begin{itemize}
\item 使用多种推荐方式,并混合预测(用线性模型)
\item 添加基于内容的方法来做协同过滤
\begin{itemize}
	\item 商品特写用以新商品问题
	\item 人口统计特性用以解决新用户问题
\end{itemize}
\end{itemize}

\subsection{隐因子模型}
隐因子模型(latent factor model, LFM)采用SVD对矩阵进行分解
\[R\approx Q\cdot P^\T\]
$R\in\rr^{m\times n}$为评分矩阵,$m$为商品数目,$n$为用户数目;
$Q\in\rr^{m\times k}$为项-因子矩阵;
$P^\T\in\rr^{k\times n}$为因子-用户矩阵。

由于SVD能够给出最小重构误差
\[\min_{U,V,\Sigma}\sum_{ij\in A}(A_{ij}-[U\Sigma V^\T]_{ij})^2\]

目标是找到$P$和$Q$使得
\[\min_{P,Q}\sum_{(i,x)\in R}(r_{xi}-q_i\cdot p_x)^2\]
但我们目标却是使上式在未见过的测试数据上最小化,而我们能做的只是在训练集上最小化,故需要避免过拟合。
添加正则项
\[\min_{P,Q}\sum_{training}(r_{xi}-q_ip_x)^2+\lrs{\lambda_1\sum_x\norm{p_x}^2+\lambda_2\sum_i\norm{q_i}^2}\]
利用随机梯度下降(SGD)进行求解。

\subsection{冷启动}
解决冷启动的思路\footnote{参考自推荐系统冷启动 - gongyouliu的文章 - 知乎\url{https://zhuanlan.zhihu.com/p/79950668}}:
\begin{itemize}
	\item 提供非个性化的推荐(用户冷启动)
	\item 利用用户注册时提供的信息(用户冷启动、系统冷启动)
	\item 基于内容做推荐(用户冷启动、系统冷启动)
	\item 利用标的物的metadata信息做推荐(标的物冷启动)
	\item 采用快速试探策略(用户冷启动、标的物冷启动)
	\item 采用兴趣迁移策略(用户冷启动、系统冷启动)
	\item 采用基于关系传递的策略(标的物冷启动)
\end{itemize}

用户冷启动:
\begin{itemize}
	\item 利用先验数据做推荐:直接推荐新的最热门的东西
	\item 给用户多样化选择:新用户注册的信息
	\begin{itemize}
		\item 人口统计学数据:年龄、性别、地域、学历、职业
		\item 社交关系:好友推荐
		\item 用户自行填写兴趣点:内容推荐、快速试探、兴趣迁移
	\end{itemize}
\end{itemize}

物品冷启动:
\begin{itemize}
	\item 利用标的物的metadata信息做推荐
	\begin{itemize}
		\item 利用标的物跟用户行为的相似性
		\item 利用标的物跟标的物的相似性
	\end{itemize}
	\item 采用快速试探策略
	\item 采用基于关系传递的策略:A与B相似,B与C相似,则A与C相似
\end{itemize}