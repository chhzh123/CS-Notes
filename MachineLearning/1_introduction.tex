% !TEX root = main.tex

\section{简介}
机器学习理论主要是设计和分析一些让计算机可以自动“学习”的算法,数据挖掘是用人工智能、机器学习、统计学和数据库的交叉方法在相对较大型的数据集中发现模式的计算过程。 
大体上看,数据挖掘可以视为机器学习和数据库的交叉,它主要利用机器学习界提供的技术来\textbf{分析}海量数据,利用数据库界提供的技术来\textbf{管理}海量数据。

\begin{quote}
"A computer program is said to learn from experience E with respect to some class of tasks T and performance measure P, if its performance  at tasks in T, as measured by P, improves with experience E."
\hfill---Tom Mitchell
\end{quote}

\subsection{分类}
机器学习(machine learning)通常可以分为以下几类:
\begin{itemize}
	\item 监督学习(supervised learning):有标签(label)
	\begin{itemize}
		\item 回归(regression):连续值
		\item 分类(classification):离散值
	\end{itemize}
	\item 无监督学习(unsupervised learning):无标签
	\begin{itemize}
		\item 降维
		\item 聚类(clustering)
	\end{itemize}
	\item 强化学习(reinforcement learning):延后的标签
\end{itemize}

\subsection{历史}
\begin{itemize}
	\item 推理期(1950-1970):逻辑理论家(Logic Theorist)A.~Newell \& H.~Simon(1975图灵奖)
	\item 知识期(1970s):知识工程之父E.~A.~Feigenbaum(1994图灵奖)
	\item 学习期(1980s):决策树、归纳逻辑程序设计(Prolog)
\end{itemize}

机器学习的五大学派(tribe):
\begin{itemize}
	\item 符号主义(symbolist)
	\item 联结主义(connectionist)
	\item 进化主义(evolutionaries)
	\item 贝叶斯主义(bayesians)
	\item 类比主义(analogizers)
\end{itemize}
\begin{figure}[H]
\centering
\includegraphics[width=0.8\linewidth]{fig/A-Look-at-Machine-Learning-Evolution.png}
\end{figure}