% !TEX root = main.tex

\section{聚类}
\subsection{原型聚类}
给定样本集$D=\{\vx_1,\vx_2,\ldots,\vx_m\}$,Kmeans针对聚类所得簇划分$\mathcal{C}=\{C_1,C_2,\ldots,C_k\}$最小化平方误差
\[MSE=\sum_{i=1}^k\sum_{\vx\in C_i}\norm{\vx-\vmu_i}_2^2\]
其中$\vmu_i$为簇$C_i$的均值向量。
上式一定程度上刻画了簇内样本围绕簇均值向量的紧密程度,$MSE$越小则簇内样本相似度越高。

k邻近算法选择$k$个邻居,将其归为多的那个类别。
对于categorical属性,有序关系的可直接编码为数字,没有序关系的则要用独热码。
快速实现kNN算法可以用KD树,快速筛选出前$k$个距离最小的点。