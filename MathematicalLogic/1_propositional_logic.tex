% !TEX root = main.tex

\section{命题逻辑}
\subsection{自然推断}
\begin{definition}[命题(proposition)]
命题或声明式句子是指可判断为真或者假的句子。
不可被分解的(indecomposable)命题为原子命题。
\end{definition}

关于命题公式的定义在这里不再给出,注意$\to$是右结合(right-associative)的,如$p\to q\to r$等价于$p\to(q\to r)$。

\begin{definition}[自然推断(deduction)]
假设有一系列前提(premise)公式$\phi_1,\phi_2\ldots,\phi_n$,及结论$\psi$,那么推断过程可记为
\[\phi_1,\phi_2,\ldots,\phi_n\vdash\psi\]
这一表达式称为一个序列(sequent),若一个证明可以被找到则称它是合法的(valid)。
\end{definition}

推理的基本规则:
\begin{itemize}
	\item and-introduction ($\land i$):前提与前提为真
	\[\frac{\phi\qquad\psi}{\phi\land\psi}\land i\]
	\item and-elimination ($\land e_i$):前提与中子成分为真
	\[\frac{\phi\land\psi}{\phi}\land e_1\qquad \frac{\phi\land\psi}{\psi}\land e_2\]
	\item negation-introduction ($\lnot\lnot i$)
	\[\frac{\phi}{\lnot\lnot\phi}\lnot\lnot i\]
	\item negation-elimination ($\lnot\lnot e$)
	\[\frac{\lnot\lnot\phi}{\phi}\lnot\lnot e\]
	\item implication-elimination $\to e$
	\[\frac{\phi\quad \phi\to\psi}{\psi}\to e\]
	\item implies-introduction $\to i$
	\[\dfrac{\fbox{\begin{tabular}{c}$\phi$\\$\vdots$\\$\psi$\end{tabular}}}{\phi\to\psi}\to i\]
	\item or-introduction $\lor e$
	\[\frac{\phi}{\phi\lor\psi}\lor i_1\qquad
	\frac{\psi}{\phi\lor\psi}\lnot\lnot i_2\qquad
	\frac{\phi\lor\psi\quad \fbox{\begin{tabular}{c}$\phi$\\$\vdots$\\$\chi$\end{tabular}}\quad \fbox{\begin{tabular}{c}$\phi$\\$\vdots$\\$\chi$\end{tabular}}}{\chi}\lor e\]
	\item bottom/not-elimination
	\[\frac{\bot}{\phi}\bot e\qquad \frac{\phi\qquad\lnot\phi}{\bot}\lnot e\]
	\item negation
	\[\frac{\fbox{\begin{tabular}{c}$\phi$\\$\vdots$\\$\bot$\end{tabular}}}{\lnot \phi}\lnot i\]
\end{itemize}

\begin{example}
证明$p\land q,r\vdash q\land r$是合法的。
\end{example}
\begin{analysis}
推理过程如下
\begin{center}
\begin{tabular}{lll}
1 & $p\land q$ & premise\\
2 & $r$ & premise\\
3 & $q$ & $\land e_2\quad 1$\\
4 & $q\land r$ & $\land i\quad 3,2$
\end{tabular}
\end{center}
\[\frac{\frac{p\land q}{q}\land e_2\quad r}{q\land r}\land i\]
\end{analysis}

\begin{definition}[定理(theorem)]
有着合法序列$\vdash\phi$的逻辑公式$\phi$称为定理。
\end{definition}

三条进阶推理规则:
\begin{itemize}
\item 拒取式(modus tollens, MT)
\[\frac{\phi\to\psi\quad \lnot\psi}{\lnot\phi} MT\]
\item 反证法(proof by contradition, PBC)
\[\frac{\fbox{\begin{tabular}{c}$\lnot\phi$\\$\vdots$\\$\bot$\end{tabular}}}{\phi} PBC\]
\item 排中律(the law of the excluded middle, LEM)
\[\phi\lor\lnot\phi\text{必有一个为真}\]
\end{itemize}

\begin{definition}[可证明等价性(provably equivalent)]
令$\phi$和$\psi$为命题逻辑公式,$\phi$和$\psi$是可证明等价的当且仅当序列$\phi\vdash\psi$和$\psi\vdash\phi$都是合法的,或者$\phi\dashv\vdash\psi$
\end{definition}