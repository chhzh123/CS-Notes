% !TEX root = main.tex

\section{程序验证}
\begin{definition}[霍尔三元组(Hoare triple)]
\[\hoare{\phi}P\hoare{\psi}\]
代表若程序$P$在满足$\phi$的状态下运行,则执行完$P$的状态会满足$\psi$。
$\phi$称为先验条件(precondition),$\psi$称为后验条件(postcondition)。
存储变量$x$记为$l(x)$。
\end{definition}
\begin{example}
若输入$x$是正数,则求出一个数字它的平方小于$x$。\\
记程序为$P$,则Hoare三元组为
\[\hoare{x>0}P\hoare{y\cdot y<x}\]
一个可行的程序$P$可以是
\begin{lstlisting}
y = 0;
while (y * y < x) { y = y + 1; }
y = y - 1;
\end{lstlisting}
\end{example}
\begin{definition}[部分正确性(partial correctness)]
若对于\textbf{所有}满足先验条件$\phi$的状态,\textbf{只要$P$需要能停止(terminate)},经过$P$的执行,都满足后验条件$\psi$,则$\hoare{\phi}P\hoare{\psi}$满足部分正确性。
若$\models_{par}\hoare{\phi}P\hoare{\psi}$成立,则称$\models_{par}$为部分正确性关系。
全部正确性(total correctness)则是保证了$P$一定会停止。
\end{definition}
\begin{example}
考虑下面求阶乘的程序Fac1:
\begin{lstlisting}
y = 1;
z = 0;
while (z != x) {
  z = z + 1;
  y = y * z;
}
\end{lstlisting}
\begin{itemize}
\item $\models_{tot}\hoare{x \geq 0 } Fac1\hoare{y = x! }$成立,只要$x \geq 0$,Fac1一定会停止,并且有结果$y = x!$
\item $\models_{tot}\hoare{\top } Fac1\hoare{y = x! }$不保证成立,因为Fac1对于$x$的负数值不会停止
\item $\models_{par}\hoare{x \geq 0 } Fac1\hoare{y = x! }$和$\models_{par}\hoare{\top} Fac1\hoare{y = x!}$成立
\end{itemize}
\end{example}
\begin{definition}[逻辑变量(logical variable)]
逻辑变量在$\phi$或$\psi$是自由的,且不出现在$P$中
\end{definition}
\begin{example}
$\hoare{x=x_0\land x\geq 0}Sum\hoare{z=x_0(x_0+1)/2}$
\begin{lstlisting}
z = 0;
while (x != 0) {
  z = z + x;
  x = x - 1;
}
\end{lstlisting}
\end{example}

\subsection{证明论}
\begin{itemize}
	\item Composition
	\[\frac{\htriple{\phi}{C_1}{\eta}\qquad\htriple{\eta}{C_2}{\psi}}{\htriple{\phi}{C_1;C_2}{\psi}}\]
	\item Assignment
	\[\frac{}{\htriple{\psi[E/x]}{x=E}{\psi}}\]
	注意这条公式相当tricky,是在先验条件中将$x$换为$E$(即恢复$x=E$的赋值),在具体证明中往往是反过来用
	\item If-statement
	\[\frac{\htriple{\phi\land B}{C_1}{\psi}\qquad\htriple{\phi\land\lnot B}{C_2}{\psi}}{\htriple{\phi}{\text{if }B\{C_1\}\text{ else }\{C_2\}}{\psi}}\]
	\item Partial-while
	\[\frac{\htriple{\psi\land B}{C}{\psi}}{\htriple{\psi}{\text{while }B\{C\}}{\psi\land\lnot B}}\]
	\item Implied
	\[\frac{\vdash_{AR}\phi'\to\phi\qquad\htriple{\phi}{C}{\psi}\qquad\vdash_{AR}\psi\to\psi'}{\htriple{\phi'}{C}{\psi'}}\]
	基本谓词逻辑演算及算术表达都满足
\end{itemize}
\begin{example}[表格证明(proof tableaux)]
$\vdash_{par}\htriple{y<3}{y=y+1}{y<4}$
\[\begin{array}{ll}
\hoare{y<3} &\\
\hoare{y+1 < 4} & \text{Implied}\\
y = y+1 &\\
\hoare{y<4} &\text{Assignment}
\end{array}\]
\end{example}

\[\htriple{\phi}{\text{if }(B)\{C_1\}\text{ else }\{C_2\}}{\psi}\]
\begin{itemize}
	\item 将$\psi$反推$C_1$,结果记为$\phi_1$
	\item 将$\psi$反推$C_2$,结果记为$\phi_2$
	\item 令$\phi=(B\to\phi_1)\land(\lnot B\to\phi_2)$
\end{itemize}
\begin{example}
考虑程序$Succ$
\begin{lstlisting}
a = x + 1;
if (a-1 == 0) { y = 1; } else { y = a; }
\end{lstlisting}
证明$\vdash_{par}\htriple{\top}{Succ}{y=x+1}$合法
\end{example}
\begin{analysis}
可以自底向上分析
\begin{lstlisting}[language=c++]
  `\textcolor{red}{$\hoare{\top}$}'
  `\textcolor{red}{$\hoare{(x+1-1=0\to1=x+1)\land(\lnot(x+1-1=0)\to x+1=x+1)}$ Implied}'
a = x + 1;
  `\textcolor{red}{$\hoare{(a-1=0\to 1=x+1)\land(\lnot(a-1=0)\to a=x+1)}$ Assignment}'
if (a - 1 == 0) {
    `\textcolor{red}{$\hoare{1=x+1}$ If-statements}'
  y = 1;
    `\textcolor{red}{$\hoare{y=x+1}$ Assignment}'
} else {
    `\textcolor{red}{$\hoare{a=x+1}$ If-statements}'
  y = a;
    `\textcolor{red}{$\hoare{y=x+1}$ Assignment}'
}
    `\textcolor{red}{$\hoare{y=x+1}$ If-statements}'
\end{lstlisting}
\end{analysis}

\[\htriple{\phi}{\text{while }B\{C\}}{\psi}\]
需要找到合适的不变量(invariant)$\eta$使得
\begin{itemize}
	\item $\models_{AR}\phi\to\eta$
	\item $\models_{AR}\eta\land\lnot B\to\psi$
	\item $\models_{par}\hoare{\eta}$ while $(B)\{C\}\hoare{\eta\land\lnot B}$
\end{itemize}
\begin{example}
考虑程序$Fac1$
\begin{lstlisting}[language=c++]
y = 1;
z = 0;
while (z != x) { // L1
  z = z + 1;
  y = y * z;
} // L2
\end{lstlisting}
证明
\[\models_{AR}(y=1\land z=0)\to(y=z!)\qquad
\models_{AR}(y=z!\land x=z)\to(y=x!)\]
\begin{lstlisting}[language=c++]
  `\textcolor{red}{$\hoare{\top}$}'
  `\textcolor{red}{$\hoare{1=0!}$ Implied}'
y = 1;
  `\textcolor{red}{$\hoare{y=0!}$ Assignment}'
z = 0;
  `\textcolor{red}{$\hoare{y=z!}$ Assignment}'
while (z != x) {
    `\textcolor{red}{$\hoare{y=z!\land z\ne x}$ Invariant Hyp. $\land$ guard}'
    `\textcolor{red}{$\hoare{y\cdot(z+1)=(z+1)!}$ Implied}'
  z = z + 1;
    `\textcolor{red}{$\hoare{y\cdot z=z!}$ Assignment}'
  y = y * z;
    `\textcolor{red}{$\hoare{y=z!}$ Assignment}'
}
  `\textcolor{red}{$\hoare{y=z!\land\lnot(z\ne x)}$ Partial-while}'
  `\textcolor{red}{$\hoare{y=x!}$ Implied}'
\end{lstlisting}
\end{example}