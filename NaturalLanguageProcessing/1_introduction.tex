% !TEX root = main.tex

\section{简介}
自然语言处理(natrual language processing, NLP)的主要内容:机器翻译、信息检索、自动文摘、观点挖掘、问答系统、信息抽取、文档分类、文字编辑和自动校对、语音识别、文语转换、语音合成、说话人识别/认同/验证。

\begin{definition}[熵]
概率分布为$p(x)=P(X=x)$,则熵$H(X)$为
\[H(X)=-\sum_{x\in X}p(x)\log_2p(x)\]
并约定$0\log 0=0$,单位为二进制位比特(bit)
\end{definition}
\begin{definition}[联合熵]
$X,Y$为离散型随机变量$X,Y\thicksim p(x,y)$,联合熵定义为
\[H(X,Y)=-\sum_{x\in X}\sum_{y\in Y}p(x,y)\log_2 p(x,y)\]
\end{definition}
\begin{definition}[条件熵]
\[H(Y\mid X)=-\sum_{x\in Y}\sum_{y\in Y}p(x,y)\log_2 p(y\mid x)\]
\end{definition}
\begin{definition}[相对熵(KL距离)]
两个概率分布$p(x)$和$q(x)$的相对熵定义为
\[D(p||q)=\sum_{x\in X}p(x)\log\frac{p(x)}{q(x)}\]
\end{definition}
\begin{definition}[交叉熵]
若随机变量$X\thicksim p(x)$,$q(x)$用于近似$p(x)$的概率分布,则$X$与模型$q$的交叉熵定义为
\[H(X,q)=H(X)+D(p||q)=-\sum_x p(x)\log q(x)\]
其常用来衡量估计模型与真实概率分布之间的差异
\end{definition}
\begin{definition}[互信息]
如果$(X,Y)\thicksim p(x,y)$,则$X,Y$之间的互信息$I(X;Y)$定义为
\[I(X;Y)=H(X)-H(X\mid Y)=\sum_{x\in X}\sum_{y\in Y}p(x,y)\log_2\frac{p(x,y)}{p(x)p(y)}\]
互信息值越大,表示两个汉字之间的结合越紧密,越有可能成词。
\end{definition}

基于上下文分类的消歧方法:
假设多义词$w$所处的上下文语境为$C$,若$w$的多个词义记为$s_i$,则可通过计算$\argmax p(s_i\mid C)$确定$w$的词义。

由贝叶斯公式,并运用独立性假设
\[p(s_i\mid C)=\frac{p(s_i)\times p(C\mid s_i)}{p(C)}=p(s_i)\prod_{v_k\in C}p(v_k\mid s_i)\Big/ p(C)\]
因此只需求(可转换为对数加法运算)
\[\hat{s}_i=\argmax_{s_i}\lrs{p(s_i)}\prod_{v_k\in C}p(v_k\mid s_i)\]
由统计数据可得
\[\begin{aligned}
p(v_k\mid s_i)&=\frac{N(v_k,s_i)}{N(s_i)}\\
p(s_i)&=\frac{N(s_i)}{N(w)}
\end{aligned}\]
其中$N(s_i)$为训练数据中词$w$用于语义$s_i$时的次数,而$N(v_k,s_i)$为$w$用于语义$s_i$时词$v_k$出现在$w$的上下文中的次数,$N(w)$为$w$在训练数据中出现的总次数。