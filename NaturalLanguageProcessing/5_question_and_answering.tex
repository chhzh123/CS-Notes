% !TEX root = main.tex

\section{问答与对话}
\begin{definition}[问答系统]
输入:自然语言的问句,而非关键词的组合\\
输出:直接答案,而非文档集合
\end{definition}

优点:
\begin{itemize}
	\item 相对于基于知识推理的问答系统而言:不受知识库规模限制,不受领域限制,更加接近真实应用需求
	\item 相对于搜索引擎而言:问答式检索系统接受的是自然语言形式的提问,由于自然语言处理技术的应用,对用户意图的把握更加准确,呈现给用户的答案更加准确
\end{itemize}
缺点:目前问答式检索系统仅能处理有限的简单问题,如Factoid问题等

问答式检索方法:
\begin{itemize}
\item 信息检索+信息抽取
\item 信息检索+模式匹配:离线阶段获取答案模式,在线阶段首先判断当前提问属于哪一类,然后使用这类提问的所有模式来抽取候选答案
\item 信息检索+自然语言处理技术
\item 基于统计翻译模型的问答技术:把提问句看作答案句在同一语言内的一种翻译
\end{itemize}

新方法基于深度学习:把回答问题的过程看作一个黑盒,通过复杂神经网络和超大规模数据集训练出一个拟合能力强大的模型。

对话管理(状态跟踪和学习)方法:
\begin{itemize}
	\item 有限状态机(finite state machine):太受限了
	\item 基于框架的方法(frame-based):
	\begin{itemize}
		\item 使用框架的结构指导对话过程:机器根据框架进行提问,人也根据框架进行回复
		\item 问答过程就是一个槽-值填充过程:所有槽都填满了,就可以通过信息系统查询
		\item 用户可以一次性回答多个系统问题
	\end{itemize}
	\item 统计方法(MDP):状态、动作、目标
\end{itemize}