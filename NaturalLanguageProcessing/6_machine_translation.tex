% !TEX root = main.tex

\section{机器翻译}
基本翻译方法
\begin{itemize}
	\item 直接转换
	\item 基于规则的翻译方法
	\item 基于中间语言
	\item 基于语料库:基于事例、统计翻译、神经网络翻译
\end{itemize}

\subsection{基于规则的翻译方法}
\begin{enumerate}
	\item 对源语言句子进行词法分析
	\item 句法/语义分析
	\item 源语言句子结构到译文结构的转换
	\item 译文句法结构生成
	\item 源语言词汇到译文词汇转换
	\item 译文词法选择与生成
\end{enumerate}
\begin{figure}[H]
\centering
\includegraphics[width=0.6\linewidth]{fig/rule-based_mt.png}
\end{figure}

优点:保持原文结构、规范语句
\par 弱点:规则由人工编写,工作量大,主观性强,不利于系统扩充,灵活性低(对非规范语言现象无法处理)

\subsection{基于语料库的翻译方法}
\begin{figure}[H]
\centering
\includegraphics[width=0.6\linewidth]{fig/example-based_mt.png}
\end{figure}

优点:不需源语言句子符合语法规定,也不许对源语言句子做深入分析
\par 缺点:两个不同句子之间相似性难以把握,难以处理陌生语言现象,且事例库庞大起来难以搜索

\subsection{统计翻译}
\subsubsection{基本方法}
\begin{itemize}
	\item 源语言:$S=s_1^m=s_1s_2\cdots s_m$
	\item 目标语言:$T=t_1^l=t_1t_2\cdots t_n$
\end{itemize}
通过最大后验求解
\[\argmax_TP(T\mid S)=\argmax_T P(S\mid T)P(T)\]
其中$P(S\mid T)$为翻译模型(TM),确定了单词和词语如何被翻译(fidelity),从平行语料中学习;
$P(T)$为语言模型(LM),确定怎么写出好的目标语言的句子(fluency),从单语语料中学习。

三个关键问题:
\begin{itemize}
	\item 估计语言模型概率$P(T)$
	\item 估计翻译模型概率$P(S\mid T)$:关键问题是怎样定义目标语言句子中的词与源语言句子中的词之间的对应关系,基本原理是对位(alignment)模型
	\item 快速搜索最大值解
\end{itemize}

\subsubsection{基于短语的翻译模型}
基于词的翻译模型很难消除歧义,很难处理一对多、多对一问题。
以短语为基本翻译单元,遵循短语划分、短语翻译、短语调序的步骤。
\begin{definition}[对齐一致性]
$S^j$中每个词$S_k$,若$(k,k')\in A$,则$i'\leq k'\leq j'$,$T_i^j$中每个词$T_{t'}$,若$(t,t')\in A$,则$i\leq t\leq j$。
\end{definition}

关键问题是学习短语翻译规则(双语句对词语对齐、短语翻译规则抽取)、估计短语翻译概率。

\subsection{神经机器翻译}
统计机器翻译都是人工设定的模块和特征,可解释性高、模块定制化、错误追踪,但是数据稀疏、不擅长复杂结构、依赖先验知识。

分布式的语义表示是统计机器翻译到机器翻译的核心
\begin{figure}[H]
\centering
\includegraphics[width=0.6\linewidth]{fig/attention.png}
\end{figure}

BLEU(BiLingual Evaluation Understudy)评价方法:统计同时出现在系统译文和参考译文中的$n$元词个数,最后把匹配到$n$元词的数目除以系统译文的$n$元词数目,得到评测结果。