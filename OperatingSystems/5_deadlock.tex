% !TEX root = main.tex

\section{并发---死锁与饥饿}
\subsubsection{死锁}
死锁(deadlock):一个进程集合中的每个进程都在等待只能由该集合中的其他一个进程才能引发的事件(释放占有资源/进行某项操作)
\begin{figure}[H]
    \centering
    \begin{tabular}{c}
    \includegraphics[width=0.6\linewidth]{fig/deadlock.png}\\
    \includegraphics[width=0.6\linewidth]{fig/deadlock2.png}
    \end{tabular}
    \caption*{联合进程图,上图死锁,下图无死锁}
\end{figure}

\begin{center}
\begin{tabular}{|l|l|l|p{10em}|l|}
\hline
\multicolumn{1}{|p{3.065em}|}{原则} & \multicolumn{1}{p{6em}|}{资源分配策略} & \multicolumn{1}{p{10em}|}{不同的方案} & 主要优点  & \multicolumn{1}{p{10em}|}{主要缺点} \bigstrut\\
\hline
\multicolumn{1}{|l|}{\multirow{6}[6]{*}{预防}} & \multicolumn{1}{l|}{\multirow{6}[6]{*}{保守,预提交资源}} & \multicolumn{1}{l|}{\multirow{3}[2]{*}{一次性请求所有资源}} & \tabitem对执行单一突发行为的进程有效 & \multicolumn{1}{p{10em}|}{\tabitem低效} \bigstrut[t]\\
      &       &       & \tabitem不需抢占 & \multicolumn{1}{p{10em}|}{\tabitem延时进程的初始化} \\
      &       &       & \multicolumn{1}{r|}{} & \multicolumn{1}{p{10em}|}{\tabitem进程必须知道未来的资源请求} \bigstrut[b]\\
\cline{3-5}      &       & \multicolumn{1}{p{10em}|}{抢占} & \tabitem便于用于状态易保存和恢复的资源 & \multicolumn{1}{p{10em}|}{\tabitem过多的不必要抢占} \bigstrut\\
\cline{3-5}      &       & \multicolumn{1}{l|}{\multirow{2}[2]{*}{资源排序}} & \tabitem通过编译时检测可实施 & \multicolumn{1}{l|}{\multirow{2}[2]{*}{\tabitem不允许增加资源请求}} \bigstrut[t]\\
      &       &       & \tabitem由于问题已在系统设计时解决,不需运行时再计算 &  \bigstrut[b]\\
\hline
\multicolumn{1}{|l|}{\multirow{2}[2]{*}{避免}} & \multicolumn{1}{l|}{\multirow{2}[2]{*}{位于检测和预防中间}} & \multicolumn{1}{l|}{\multirow{2}[2]{*}{操纵以发现至少一条安全路径}} & \multirow{2}[2]{*}{\tabitem不需抢占} & \multicolumn{1}{p{10em}|}{\tabitem OS必须知道未来的资源请求} \bigstrut[t]\\
      &       &       & \multicolumn{1}{l|}{} & \multicolumn{1}{p{10em}|}{\tabitem进程可能被长期阻塞} \bigstrut[b]\\
\hline
\multicolumn{1}{|l|}{\multirow{2}[2]{*}{检测}} & \multicolumn{1}{l|}{\multirow{2}[2]{*}{自由}} & \multicolumn{1}{l|}{\multirow{2}[2]{*}{周期性地调用以测试死锁}} & \tabitem不会延时进程的初始化 & \multicolumn{1}{l|}{\multirow{2}[2]{*}{\tabitem丢失固有抢占}} \bigstrut[t]\\
      &       &       & \tabitem易于在线处理 &  \bigstrut[b]\\
\hline
\end{tabular}%
\end{center}

死锁定理:资源分配图中存在环路是存在死锁的充分必要条件(每个资源类中只包含一个资源实例)。其中指向资源的是\underline{申请边},离开资源的是\underline{分配边}。
\begin{figure}[H]
    \centering
    \includegraphics[width=0.6\linewidth]{fig/resource_graph.png}
\end{figure}

产生死锁必须\textbf{同时}满足以下四个条件,只要其中一条不成立,则死锁不会发生
\begin{itemize}
    \item \underline{互斥}:进程对所分配的资源排他性控制,在一段时间内该资源仅为一个进程所占有。
    \item \underline{不可抢占}(不剥夺):进程所获得的资源在未使用完毕之前,不能被其他进程强行夺走,只能主动释放
    \item \underline{占有且等待}(请求和保持):进程已经保持了至少一个资源,但又提出了新的资源请求,而该资源已经被其他进程占有。此时请求进程被阻塞,但对自己已获得的资源保持不放。
    \item \underline{循环等待}:存在一种进程资源的循环等待链,链中每一个进程已获得的资源同时被链中下一个进程所请求。
\end{itemize}

\subsection{死锁预防}
\subsubsection{破坏互斥条件}
允许多个进程同时使用资源,但不适用于绝大多数资源,适用条件:
\begin{itemize}
\item 资源的固有特性允许多个进程同时使用(如文件允许多个进程同时读)
\item 借助特殊技术允许多个进程同时使用(如打印机的SPOOLing技术)
\end{itemize}

\subsubsection{破坏占有且等待条件}
禁止已拥有资源的进程再申请其他资源,如要求所有进程在开始时一次性地申请在整个运行过程所需的全部资源;或申请资源时要先释放其占有资源后,再一次性申请所需全部资源
\begin{itemize}
    \item 优点:简单、易于实现、安全
    \item 缺点:进程延迟运行,资源严重浪费
\end{itemize}

\subsubsection{破坏不可剥夺条件}
一个已经占有了某些资源的进程,当它再提出新的资源请求而不能立即得到满足时,必须释放它已经占有的所有资源,待以后需要时再重新申请;OS可以剥夺一个进程占有的资源,分配给其他进程

适用条件:资源的状态可以很容易地保存和恢复(如CPU)
缺点:实现复杂、代价大,反复申请/释放资源、系统开销大、降低系统吞吐量

\subsubsection{破坏环路等待条件方法}
\begin{itemize}
    \item 要求每个进程任何时刻只能占有一个资源,如果要申请第二个则必须先释放第一个(不现实)
    \item 对所有资源按类型进行\textbf{线性排队},进程申请资源必须严格按资源序号递增的顺序(可避免循环等待)
\end{itemize}

缺点
\begin{itemize}
    \item 很难找到令每个人都满意的编号次序,类型序号的安排只能考虑一般作业的情况, 限制了用户简单、自主地编程
    \item 易造成资源的浪费(会不必要地拒绝对资源的访问)
    \item 可能低效(会使进程的执行速度变慢)
\end{itemize}

\subsection{死锁避免}
进程启动拒绝:考虑一个有$n$个进程和$m$种不同类型资源的系统。定义以下向量和矩阵:
\begin{center}
    \begin{tabular}{c|c}\hline
        Resource=$\vr=\bmat{R_1 & R_2 & \cdots & R_m}$ & 系统中每种资源的总量\\\hline
        Available=$\vv=\bmat{V_1 & V_2 & \cdots & V_m}$ & 未分配给进程的每种资源的总量\\\hline
        Claim=$C=\bmat{C_{11} & \cdots & C_{1m}\\\vdots &\ddots & \vdots\\C_{n1} & \cdots & C_{nm}}$ & $C_{ij}$为进程$i$对资源$j$的请求\\\hline
        Allocation=$A=\bmat{A_{11} & \cdots & A_{1m}\\\vdots &\ddots & \vdots\\A_{n1} & \cdots & A_{nm}}$ & $A_{ij}$为分配给进程$i$资源$j$的数目\\\hline
    \end{tabular}
\end{center}
有下列关系式成立:
\begin{enumerate}
    \item $\forall j:\;\disp r_j=v_j+\sum_{i=1}^n A_{ij}$。所有资源要么可用,要么已经被分配
    \item $\forall i,j:\;A_{ij}\leq C_{ij}\leq R_i$:分配小于等于最大请求,最大请求小于等于资源总量
\end{enumerate}

进而可以定义死锁避免策略:若一个新进程的资源需求会导致死锁,则拒绝启动这个进程,当且仅当
\[\forall j:\;R_j\geq C_{(n+1)j}+\sum_{i=1}^n C_{ij}\]

进程分配拒绝:银行家算法(Dijkstra)
\[\forall j:\;C_{ij}-A_{ij}\leq v_j\]
\begin{itemize}
    \item 只要系统处于安全状态(至少有一个资源分配序列不会导致死锁,即所有进程都能运行到结束),必定不会进入死锁状态
    \item 不安全状态不一定是死锁状态,但不能保证不会进入死锁状态
    \item 如果一个新进程的资源请求会导致不安全状态,则拒绝启动这个进程
\end{itemize}

优点
\begin{itemize}
    \item 比死锁预防限制少
    \item 无须死锁检测中的资源剥夺和进程重启
\end{itemize}
缺点
\begin{itemize}
    \item 必须事先声明每个进程请求最大资源
    \item 进程必须无关,没有同步要求
    \item 分配的资源数目是固定的
    \item 占有资源时进程不能退出
\end{itemize}

\begin{example}
    在如下条件下考虑银行家算法。\\
    6个进程:P0-P5\\
    4种资源:A(15单位)、B(6单位)、C(9单位)、D(10单位)\\
    时间T0时的情况,左侧为分配矩阵$A$,右侧为请求矩阵$C$
    \begin{center}
        \begin{tabular}{|c|c|c|c|c||c|c|c|c|}\hline
            & A & B & C & D & A & B & C & D\\\hline
            P0 & 2 & 0 & 2 & 1 & 9 & 5 & 5 & 5\\\hline
            P1 & 0 & 1 & 1 & 1 & 2 & 2 & 3 & 3\\\hline
            P2 & 4 & 1 & 0 & 2 & 7 & 5 & 4 & 4\\\hline
            P3 & 1 & 0 & 0 & 1 & 3 & 3 & 3 & 2\\\hline
            P4 & 1 & 1 & 0 & 0 & 5 & 2 & 2 & 1\\\hline
            P5 & 1 & 0 & 1 & 1 & 4 & 4 & 4 & 4\\\hline
            $\vv$ & 6 & 3 & 5 & 4 & & & &\\\hline
            All & 15 & 5 & 9 & 10 & & & &\\\hline
        \end{tabular}
    \end{center}
\end{example}
\begin{analysis}
    需求矩阵$C-A$
    \begin{center}
        \begin{tabular}{|c|c|c|c|c|}\hline
            P0 & 7 & 5 & 3 & 4\\\hline
            P1 & 2 & 1 & 2 & 2\\\hline
            P2 & 3 & 4 & 4 & 2\\\hline
            P3 & 2 & 3 & 3 & 1\\\hline
            P4 & 4 & 1 & 2 & 1\\\hline
            P5 & 3 & 4 & 3 & 3\\\hline
        \end{tabular}
    \end{center}
    看有没有进程所请求的资源都小于等于可用资源,用完则将当前分配资源还回可用资源中
    \begin{center}
        \begin{tabular}{|c|c|c|c|c|}\hline
            原来 & 6 & 3 & 5 & 4\\\hline
            P1 & 6 & 4 & 6 & 5\\\hline
            P2 & 10 & 5 & 6 & 7\\\hline
            P3 & 11 & 5 & 6 & 8\\\hline
            P4 & 12 & 6 & 6 & 8\\\hline
            P5 & 13 & 6 & 7 & 9\\\hline
            P0 & 15 & 6 & 9 & 10\\\hline
        \end{tabular}
    \end{center}
    如果某一个进程的请求后,需求矩阵每一行均小于等于可用资源的话,基于死锁避免原则,则该请求应该被拒绝
\end{analysis}

\subsection{死锁检测}
检测方法:
\begin{itemize}
    \item 单个资源实例:检测资源分配图中是否存在环路(依据——死锁定理)
    \item 多个资源实例:类似银行家算法的安全检查
\end{itemize}

恢复:
\begin{itemize}
    \item 剥夺法:连续剥夺资源指导不存在死锁
    \item 回退法:将每个死锁进程回滚到前面定义的某些检查点(checkpoint),并重启所有进程(死锁可能重现)
    \item 杀死进程法
\end{itemize}

\subsubsection{饥饿}
饥饿:一组进程中,某个或某些进程\textbf{无限等待}该组进程中其他进程所占用的资源

进入饥饿状态的进程可以只有一个,而由于循环等待条件而进入死锁状态的进程必须大于等于两个