% !TEX root = main.tex

\section{操作系统概述}
\subsection{概述}
操作系统核心即怎么虚拟多几个冯诺依曼计算机出来给程序用。
操作系统是控制应用程序执行的程序,是应用程序和计算机硬件间的接口(屏蔽硬件细节)。

这里先解释几个概念
\begin{itemize}
	\item 并发:两件事情\textbf{可以}同时(simultaneously)发生,没有时间限制,$t_1>t_2$,$t_1<t_2$,$t_1=t_2$都可
	\item 同步:两个事件有确定的时间限制
	\item 异步:两件事不知道何时发生
\end{itemize}

% 对称多处理器(symmetric multiple processors, SMP):共享内存(访存时间大致相同)、IO设备,所有处理器/核可执行相同功能(对称性)

\subsection{发展历史}
\begin{itemize}
	\item 串行处理(1940s):没有OS,人工调度,准备时间长
	\item 简单批处理系统(1950s):
	\begin{itemize}
		\item 使用监控程序(monitor),读入用户程序执行
		\item 提供内存保护、计时器、特权指令、中断
		\item 两种操作模式:用户态、内核态(mode)
		\item 单道程序(uniprogramming)批处理:处理器必须等到IO指令结束后才能继续
	\end{itemize}
	\item 多道程序批处理(1950s末):多个作业同时进入主存,切换运行,\textbf{充分利用处理器}
	\item 分时系统(1961):MIT CTSS(Compatible Time-Sharing System),满足用户与计算机交互的需要,\textbf{减小响应时间};多个交互作业,多个用户,把运行时间分成很短的时间片轮流分配
	\item 实时系统:专用,工业、金融、军事
\end{itemize}

现代的操作系统通常同时具有分时、实时核多道批处理的功能,因此被称为通用操作系统。
而OS也不仅是在PC机上有,网络OS、分布式OS、嵌入式OS层出不穷。

% 主要成就:进程、信息保护和安全、内存管理、调度和资源管理