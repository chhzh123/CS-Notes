% !TEX root = main.tex

\section{线程}
在没有线程概念的系统中,进程是资源分配、调度/执行的单位;而在有线程概念的系统中,线程就成了基本调度单位

线程的优点:
\begin{itemize}
    \item 创建速度快
    \item 终止所用时间少
    \item 切换时间少
    \item 通信效率高,同一进程无需调用内核,共享存储空间
\end{itemize}

用户级线程(ULT):线程管理都由应用程序完成(线程库),内核不知道线程的存在,优点:
\begin{itemize}
    \item 线程切换不需要模式切换
    \item 调度算法可以应用程序专用
    \item ULT不需要内核支持,线程库可以在任何OS上运行
\end{itemize}
缺点:
\begin{itemize}
    \item 一个线程阻塞会导致整个进程阻塞
    \item 不能利用多核和多处理器技术
\end{itemize}

内核级线程(KLT):线程管理由内核完成(提供API),调度基于线程进行,优点:
\begin{itemize}
    \item 线程阻塞不会导致进程阻塞
    \item 可以利用多核和多处理器技术
    \item 内核例程本身也可以使用多线程
\end{itemize}
缺点:
\begin{itemize}
    \item 线程切换需要进行模式切换
\end{itemize}

线程与进程之间的关系
\begin{itemize}
    \item 1:1,每个进程都有唯一线程,DOS、传统Unix
    \item M:1,一个进程多个线程,Windows NT、Linux、Mac OS、iOS
    \item 1:M,一个线程可在多个进程环境中迁移
\end{itemize}

* Linux并不区分线程和进程,采用Copy On Write (COW)方式