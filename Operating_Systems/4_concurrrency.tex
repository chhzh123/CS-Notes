% !TEX root = main.tex

\section{并发}
\subsection{基本概念}
\begin{itemize}
    \item 原子(atomic)操作:不可分割
    \item 临界区(critical section):不允许多个进程同时进入的一段访问\textbf{共享资源}的代码
    \item 死锁(deadlock):两个及以上进程,因每个进程都在等待其他进程做完某事(如释放资源),而不能继续执行
    \item 活锁(livelock):两个及以上进程,为响应其他进程中的变化,而不断改变自己的状态,但是没有做任何有用的工作
    \item 互斥(mutual exclusion):当一个进程在\textbf{临界区访问共享资源}时,不允许其他进程进入访问
    \item 竞争条件(race condition, RC):多个进程/线程读写共享数据,其结果依赖于它们执行的相对速度
    \item 饥饿(starvation):可运行的进程长期未被调度执行
\end{itemize}

核心内容
\[\text{并发}\to\text{共享}\to\text{RC问题}\to\text{互斥}\]

共享数据的最终结果取决于进程执行的相对速度

同步:有明确的时间先后限制,两个或多个进程之间的操作存在时间上的约束(也包含互斥)

\subsection{互斥}
软件方法
\begin{itemize}
    \item Dekker算法
    \item Peterson算法
\end{itemize}
硬件方法
\begin{itemize}
    \item 关中断:限制处理器交替执行各进程的能力,不能用于多核
    \item TestSet(TS)指令:比较并交换,原子指令,一个指令周期内完成,不会被中断
    \item Exchange指令(x86\verb'xchg'指令):同上,适用于单核多核,多变量多临界区,但需要忙等待(busy waiting)/自旋等待(spin waiting),可能饥饿或死锁
\end{itemize}

\subsubsection{信号量(semaphore)}
\begin{itemize}
    \item 整数:可用资源数($\geq 0$),需要初始化
    \item P操作(\verb'semWait'):信号量的值\textbf{减1}(申请一个单位的资源),若信号量变为\textbf{负数},则执行P操作进程阻塞,\textbf{让权等待}
    \item V操作(\verb'semSignal'):信号量的值\textbf{加1}(释放一个单位的资源),若信号量\textbf{不是正数}(绝对值=现被阻塞的进程数/等待队列的长度),则使一个因P操作被阻塞的进程解除阻塞(唤醒)
\end{itemize}
需要保证P操作和V操作的原子性
\begin{lstlisting}
    struct semaphore {
        int count;
        PCB *next;
    } s;
    void Init(s) {
        s.count = nr;
        s.next = NULL;
    }
    void P(semaphore s) {
        s.count--;
        if (s.count < 0)
            Block(CurruntProcess, s);
    }
    void V(semaphore s) {
        s.count++;
        if (s.count <= 0)
            WakeUp(s);
    }
\end{lstlisting}
注意P操作是\textbf{小于0},V操作\textbf{小于等于0}

信号量的优点是简单且表达能力强,用P、V操作可解决多种类型的同步/互斥问题,但不够安全,P、V操作使用不当会产生死锁。

二元信号量省空间,不能代表资源数量,要引入全局变量代表数量

信号量实现互斥
\begin{itemize}
    \item 对于每一个RC问题,设一个信号量(向系统调用向内核调用),\textbf{初始化为$1$}
    \item 所有相关进程在进入临界区\textbf{之前}对该信号量进行\textbf{P操作}
    \item 出临界区\textbf{之后}进行\textbf{V操作}
\end{itemize}

信号量实现进程同步
\begin{itemize}
    \item 对每一个同步关系都要设一个信号量,初值看具体问题
    \item P后续动作的前面
    \item V前驱动作的前面
\end{itemize}

组合动作:生产者-消费者问题


管程(monitor):通过集中管理(封装同步机制与同步策略)以保证安全

消息传递
\begin{itemize}
    \item send
    \item receive
\end{itemize}

读者-写者问题