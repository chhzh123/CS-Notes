% !TEX root = main.tex

\section{并发---互斥与同步}
\subsection{基本概念}
\begin{itemize}
    \item 原子(atomic)操作:不可分割
    \item 临界区(critical section):不允许多个进程同时进入的一段访问\textbf{共享资源}的代码
    \item 死锁(deadlock):两个及以上进程,因每个进程都在等待其他进程做完某事(如释放资源),而不能继续执行
    \item 活锁(livelock):两个及以上进程,为响应其他进程中的变化,而不断改变自己的状态,但是没有做任何有用的工作
    \item 互斥(mutual exclusion):当一个进程在\textbf{临界区访问共享资源}时,不允许其他进程进入访问
    \item 竞争条件(race condition, RC):多个进程/线程读写共享数据,其结果依赖于它们执行的相对速度
    \item 饥饿(starvation):可运行的进程长期未被调度执行
\end{itemize}

核心内容
\[\text{并发}\to\text{共享}\to\text{RC问题}\to\text{互斥}\]

共享数据的最终结果取决于进程执行的相对速度(异步性),需要保证进程的结果与相对执行速度无关

同步:有明确的时间先后限制,两个或多个进程之间的操作存在时间上的约束(也包含互斥)

\subsection{互斥}
互斥的要求
\begin{itemize}
    \item 在具有相同资源或共享对象的临界区的所有进程中,一次只允许一个进程进入临界区(强制排它)
    \item 一个在非临界区停止的进程必须不干涉其他进程(充分并发)
    \item 没有进程在临界区中时,任何需要访问临界区的进程必须能够立即进入(空闲让进)
    \item 决不允许出现一个需要访问临界区的进程被无限延迟(有限等待)
    \item 相关进程的执行速度和处理机数目没有任何要求或限制(满足异步)
    \item 当进程不能进入临界区,应该立即释放处理机,防止进程忙等待(让权等待)
\end{itemize}

\subsubsection{简单的尝试}
第一种尝试(单标志法):两个进程轮流进入临界区

\begin{minipage}{0.5\linewidth}
\begin{lstlisting}
while (turn != 0)
    /* do nothing */;
/* critical section */
turn = 1;
\end{lstlisting}
\end{minipage}
\begin{minipage}{0.5\linewidth}
\begin{lstlisting}
while (turn != 1)
    /* do nothing */;
/* critical section */
turn = 0;
\end{lstlisting}
\end{minipage}

可以保证互斥,硬性规定进入的顺序,但是
\begin{itemize}
    \item 忙等待,白白消耗CPU时间
    \item 必须轮流进入临界区,不合理,限制推进速度
    \item 若一个进程失败,则另一个将永远被阻塞
\end{itemize}
难以支持并发处理

一共四种尝试
\begin{itemize}
    \item 单标志法
    \item 双标志先检查:死锁
    \item 双标志后检查:不能保证互斥
    \item 双标志延迟礼让:活锁
\end{itemize}

\begin{example}
    忙等待效率一定比阻塞等待效率低吗
\end{example}
\begin{analysis}
    一般情况下确实如此,因为忙等待一直在消耗CPU资源。
    但特殊情况下,忙等待能立即响应请求完成(条件判断结束),对于性能要求较高的应用有好处;而阻塞等待还需等OS调度才能继续执行下面的工作。
\end{analysis}

\subsubsection{软件方法}
Dekker算法:避免无原则礼让,规定各进程进入临界区的顺序;逻辑复杂,正确性难以证明,存在轮流问题,存在忙等待;初始化\verb'flag'都为\verb'false',\verb'turn'为1
% https://www.cnblogs.com/zhengruin/p/4994188.html
\begin{lstlisting}
void P0()
{
	while(true)
	{
		flag[0] = true; // `P0想使用关键区'
		while(flag[1]) // `检查P1是不是也想用?'
		{
			if(turn == 1) // `如果P1想用,则查看P1是否具有访问权限?'
			{
				flag[0] = false; // `如果有,则P0放弃'
				while(turn == 1); // `检查turn是否属于P1'
				flag[0] = true; // `P0想使用'
			}
		}
		visit(0); // `访问Critical Partition'
		turn = 1; // `访问完成,将权限给P1'
		flag[0] = false; // `P0结束使用'
	}
}

void P1()
{
	while(true)
	{
		flag[1] = true; // `P1想使用关键区'
		while(flag[0]) // `检查P0是不是也想用?'
		{
			if(turn == 0) // `如果P0想用,则查看P0是否具有访问权限?'
			{
				flag[1] = false; // `如果有,则P1放弃'
				while(turn == 0); // `检查turn是否属于P1'
				flag[1] = true; // ` P1想使用'
			}

		}
		  visit(1); // `访问Critical Partition'
		turn = 0; // `访问完成,将权限给P0'
		flag[1] = false; // `P1结束使用'
	}
}
\end{lstlisting}

Peterson算法:flag和turn的含义同Dekker的,但先设turn=别人,且只有flag[别人]和turn=别人同时为真时才循环等待
\begin{lstlisting}
void P0()
{
        while(true)
        {
                flag[0] = true;
                turn = 1;
                while(flag[1] && turn == 1)
                // `退出while循环的条件就是,要么另一个线程'
                // `不想要使用关键区,要么此线程拥有访问权限'
                {
                        sleep(1);
                        printf("procedure0 is waiting!\n");
                }
                //critical section
                flag[0] = false;
        }
}

void P1()
{
        while(true)
        {
                flag[1] = true;
                turn = 0;
                while(flag[0] && turn == 0)
                {
                        sleep(1);
                        printf("procedure1 is waiting!\n");
                }
                //critical section
                flag[1] = false;
        }
}
\end{lstlisting}

\subsubsection{硬件方法}
\begin{itemize}
    \item 关中断:限制处理器交替执行各进程的能力,不能用于多核
    \item 专用指令:\textbf{比较并交换},原子指令,一个指令周期内完成,不会被中断
    \begin{itemize}
        \item TestSet(TS)指令,比较并交换的bool形式
\begin{lstlisting}
int compare_and_swap (int *word, int testval, int newval)
bool testset (int i)
\end{lstlisting}
        \item Exchange/swap指令(x86\verb'xchg'指令):同上,适用于单核多核,多变量多临界区,但需要忙等待(busy waiting)/自旋等待(spin waiting),可能饥饿或死锁
\begin{lstlisting}
void exchange (int register, int memory)
\end{lstlisting}
    \end{itemize}
\end{itemize}

机器指令方法优点
\begin{itemize}
    \item 适用于单处理器或共享主存多[核]处理器系统,进程数目任意
    \item 简单且易于证明
    \item 可以使用多个变量支持多个临界区
\end{itemize}
缺点
\begin{itemize}
    \item 忙等待/自旋等待
    \item 可能饥饿或死锁
\end{itemize}

\begin{example}
    利用\verb'xchg'实现一套互斥机制并给出使用该机制的框架
\end{example}
\begin{analysis}
    由于\verb'xchg'可以交换两个变量的内容,且为原子操作,故如果临界区未被占用,经过\verb'xchg'操作后,\verb'lock_var'被置为1;而其他进程再要访问临界区时,\verb'lock_var'和\verb'ax'均为1,交换后不会发生改变,进而不断进行\verb'lock_loop'循环。
\begin{lstlisting}[language={[x86masm]Assembler}]
    lock_var db 0 ; not used critical section
lock:
    mov ax, 1
lock_loop:
    xchg [lock_var], ax
    cmp ax, 0
    jnz lock_loop
\end{lstlisting}
    解锁操作则只需将\verb'lock_var'置0即可
\begin{lstlisting}[language={[x86masm]Assembler}]
unlock:
    mov ax, 0
    xchg [lock_var], ax
\end{lstlisting}
\end{analysis}

\subsection{信号量(semaphore)}
解决RC问题一种简单高效的方法

\subsubsection{基本操作}
记录信号量
\begin{itemize}
    \item 整数:可用资源数($\geq 0$),需要初始化
    \item P操作(proberen,\verb'semWait'):信号量的值\textbf{减1}(申请一个单位的资源),若信号量变为\textbf{负数},则执行P操作进程阻塞,\textbf{让权等待}
    \item V操作(verhogen,\verb'semSignal'):信号量的值\textbf{加1}(释放一个单位的资源),若信号量\textbf{不是正数}(绝对值=现被阻塞的进程数/等待队列的长度),则使一个因P操作被阻塞的进程解除阻塞(唤醒)
\end{itemize}
需要保证P操作和V操作的\textbf{原子性}!
\begin{lstlisting}
struct semaphore {
    int count;
    struct process* L; // `阻塞队列'
} s;
void P(semaphore s) { // semWait
    s.count--;
    if (s.count < 0)
        Block(CurruntProcess, s.L);
        // `将当前进程插入该信号量对应的阻塞队列'
}
void V(semaphore s) { // semSignal
    s.count++;
    if (s.count <= 0)
        WakeUp(s.L);
}
\end{lstlisting}
注意P操作是\textbf{小于0},V操作\textbf{小于等于0}\footnote{理解为先判断是否小于0(阻塞队列非空),然后再++会比较好},且一个进程只会在一个信号量的阻塞队列中。

信号量的优点是简单且表达能力强,用P、V操作可解决多种类型的同步/互斥问题,但不够安全,P、V操作使用不当会产生死锁。

二元信号量省空间,不能代表资源数量,要引入全局变量代表数量。

\subsubsection{实现互斥(mutex)}
\begin{itemize}
    \item 对于每一个RC问题,设一个信号量(向系统调用/向内核调用),\textbf{初始化为$1$}
    \item 所有相关进程在进入临界区\textbf{之前}对该信号量进行\textbf{P操作}
    \item 出临界区\textbf{之后}进行\textbf{V操作}
\end{itemize}

\subsubsection{同步}
同步:后续动作必须在前驱动作执行完后才能进行
\begin{itemize}
    \item 对每一个同步关系都要设一个信号量,初值看具体问题(一般为0)
    \item 在\textbf{前驱动作之后}执行V操作(相当于资源产生了)
    \item 在\textbf{后续动作之前}执行P操作
\end{itemize}

\subsubsection{生产者-消费者问题}
\begin{lstlisting}
void producer() {
    while (true) {
        produce();
        P(e);

        P(s);
        append();
        V(s);

        V(n); // first
    }
}

void consumer() {
    while (true) {
        P(n); // after

        P(s);
        take();
        V(s);

        V(e);
        consume();
    }
}

// s = initSem(1); // mutex
// n = initSem(0); // # products
// e = initSem(12); // # empty entries in buffer
\end{lstlisting}

\subsubsection{读者写者问题}
可以有多个读者,但只有一个写者

读者优先
\begin{lstlisting}
int readcount;
semaphore x=1, wsem=1;
void reader() {
    while(true) {
        P(x);
        readcount++;
        if (readcount==1) P(wsem);
        V(x);
        READUNIT();
        P(x);
        readcount--;
        if (readcount==0) V(wsem);
        V(x);
    }
}

void writer() {
   while(true) {
       P(wsem);
       WRITEUNIT();
       V(wsem);
   }
}
\end{lstlisting}

写者优先
\begin{lstlisting}[multicols=2]
int readcount, writecount;
semaphore x=1, y=1, z=1, rsem=1, wsem=1;
void reader() {
    while(true) {
        P(z); P(rsem);
        P(x);
        readcount++;
        if (readcount==1) P(wsem);
        V(x);
        V(rsem); V(z);
        READUNIT();
        P(x);
        readcount--;
        if (readcount==0) V(wsem);
        V(x);
    }
}

void writer() {
    while(true) {
        P(y);
        writecount++;
        if (writecount==1) P(rsem);
        V(y);
        P(wsem);
        WRITEUNIT();
        V(wsem);
        P(y);
        writecount--;
        if (writecount==0) V(rsem);
        V(y);
    }
}
\end{lstlisting}

\subsection{管程(monitor)}
管程(monitor):通过集中管理(封装同步机制与同步策略)以保证安全(类似于OOP中的抽象类)

主要特点:
\begin{itemize}
    \item 本地变量只能由管程过程访问(封装)
    \item 进程通过调用管程过程进入管程(调用)
    \item 每次只能一个进程在执行相关管程的过程(互斥)
\end{itemize}
主要缺陷
\begin{itemize}
    \item 可能增加了两次多余的进程切换
    \item 对进程调度有特殊要求(不允许插队)
\end{itemize}

\subsection{消息传递}
send和receive。

同样可以实现互斥,相当于在进程间传递一个可使用临界区的令牌