% !TEX root = main.tex

\section{文件管理}
文件目录:将所有文件控制块(File Control Block, FCB)组织在一起,一个FCB称之为一个目录项

一级目录:整个目录组织是一个线性结构,系统中所有文件都建立在一张目录表中

优缺点:
\begin{itemize}
    \item 结构简单、易实现
    \item 文件多时目录检索时间长,从而平均检索时间长
    \item 有命名冲突:如多个文件有相同的文件名或一个文件有多个不同的文件名
\end{itemize}

二级目录:在根目录/第一级目录/主文件目录MFD下,每个用户对应一个第二级目录/用户目录UFD,在用户目录下是该用户的文件,而不再有下级目录

多级目录:上下级关系
\begin{itemize}
    \item 当前目录/工作目录\verb'.'
    \item 父目录\verb'..'
    \item 子目录(subdirectory)
    \item 根目录(root directory)\verb'/'
\end{itemize}

Unix文件系统
\begin{itemize}
    \item Unix磁盘文件系统结构
    \item 引导块(块0)
    \item 超级块(块1)
    \item i-索引结点表
\end{itemize}

\begin{itemize}
\item DOS文件系统:FAT12、FAT16、FAT32
\item Windows文件系统:NTFS
\item Linux文件系统:ext2/3
\end{itemize}

% FAT32文件系统,https://blog.csdn.net/u010650845/article/details/60881687
% WSL文件系统支持,https://blogs.msdn.microsoft.com/wsl/2016/06/15/wsl-file-system-support/
% WSL mount,https://blogs.msdn.microsoft.com/wsl/2017/04/18/file-system-improvements-to-the-windows-subsystem-for-linux/