% !TEX root = main.tex

\section{文件管理}
\subsection{文件系统概述}
文件是以计算机硬盘为载体存储在计算机上的信息集合。
在系统运行时,计算机以进程为基本单位进行资源调度和分配;而在用户进行的输入输出中,则以\textbf{文件}为基本单位。

文件的存取方式
\begin{itemize}
    \item 顺序存取
    \item 随机/直接存取:非顺序或按关键字,现代OS常用
\end{itemize}

文件目录:将所有文件控制块(File Control Block, FCB)组织在一起,一个FCB称之为一个目录项,提供文件名到文件内容/数据的映射,包含
\begin{itemize}
    \item \underline{基本信息}:文件名、文件类型、文件结构
    \item \underline{地址信息}:存放位置、文件长度
    \item \underline{访问控制信息}:主人、权限
    \item \underline{使用信息}:创建时间、最后一次访问时间和用户
\end{itemize}
每创建一个文件,系统都会分配一个FCB放在文件目录中,成为目录项。

UNIX采用文件名与文件描述信息分开的方法,文件描述信息单独形成一个称为索引结点的数据结构,称为i结点。

FAT12文件系统中,一个FCB大小为32B,软盘扇区大小为0.5KB,则每个扇区可以存16个目录项。
一般地,一个FCB大小为64B,盘块大小1KB,则在每个盘块中可以存放16个FCB(FCB必须连续存放)。
而在UNIX系统中一个目录项仅占16B,其中14B为文件名,2B为i结点指针,在1KB的盘块中可以存放64个目录项。

\begin{itemize}
\item 一级目录:整个目录组织是一个线性结构,系统中所有文件都建立在一张目录表中\\
优缺点:
\begin{itemize}
    \item 按名存取,结构简单、易实现
    \item 文件多时目录检索时间长,从而平均检索时间长
    \item 有命名冲突:如多个文件有相同的文件名或一个文件有多个不同的文件名
\end{itemize}

\item 二级目录:在根目录/第一级目录/主文件目录MFD下,每个用户对应一个第二级目录/用户目录UFD,在用户目录下是该用户的文件,而不再有下级目录

\item 多级目录:上下级关系,但需要按路径名逐级访问中间结点,增加磁盘访问次数,进而影响查询速度
\begin{itemize}
    \item 当前目录/工作目录\verb'.'
    \item 父目录\verb'..'
    \item 子目录(subdirectory)
    \item 根目录(root directory)\verb'/'
\end{itemize}
\end{itemize}

Unix文件系统
\begin{itemize}
    \item Unix磁盘文件系统结构
    \item 引导块(块0)
    \item 超级块(块1)
    \item i-索引结点表
\end{itemize}

现代操作系统所采用的文件系统
\begin{itemize}
\item DOS文件系统:FAT12、FAT16、FAT32
\item Windows文件系统:NTFS
\item Linux文件系统:ext2/3
\end{itemize}

文件共享的方式
\begin{itemize}
    \item 基于索引结点的共享方式(硬链接):多个指针指向一个索引结点,当还有一个指针指向索引结点时,索引结点就不能删除
    \item 利用符号链实现文件共享(软链接):把到达共享文件的路径记录起来,当要访问文件时,根据路径寻找文件
\end{itemize}

\subsection{文件系统实现}
文件系统层次结构通常分为(自顶向下):用户接口、文件目录系统、存储控制模块、逻辑文件系统与文件信息缓冲区、物理文件系统。

目录实现:
\begin{itemize}
    \item 线性列表:存储文件名和数据块指针。
    \item 哈希表:根据文件名得到一个值,并返回指向线性列表中的指针
\end{itemize}

文件的分配方式:
\begin{itemize}
    \item 连续分配:磁盘上连续的块,支持顺序访问和直接访问;文件长度不宜动态增加
    \item 链接分配:目录项给出首尾盘块号(FAT);离散分配,消除了外部碎片,显著提高磁盘空间利用率;只能依照顺序链查找,且需要多次访存,随机存储效率最差
    \item 索引分配:直接把每个文件的所有盘块号都集中在一起构成索引块(多开一个单独的扇区);多层索引使第一层引导块指向第二层引导块,第二层引导块再指向文件块;索引表增加存储空间开销;现代主流操作系统常用
\end{itemize}
\begin{example}
    某操作系统的文件物理组织方式采用三级索引分配,在FCB中,有10个直接数据块指针、1个一级间接块指针、1个二级间接块指针和1个三级间接块指针,每个索引指针占4B,磁盘块大小为 4KB。回答下列问题:
    \begin{enumerate}
        \item 该文件系统中最大的单个文件有多大?
        \item 对一个20MB大小的文件,描述其存储组织中有效指针的使用情况。
    \end{enumerate}
\end{example}
\begin{analysis}
    每个磁盘块可以存$4KB/4B=1024=1K$个索引指针。
    \begin{enumerate}
        \item 10 个直接块$10*4KB=40KB$、1个一级间接块$1K*4KB=4MB$、1个二级间接块$1K*1K*4KB=4GB$、1个三级间接块$1K*1K*1K*4KB=4TB$,最大文件为$4TB 4GB 4MB 40KB$。
        \item $20MB=20480KB$,10 个直接块全被使用($10*4KB=40KB$)、1个一级间接块也被使用($1K*4KB=4MB$)、1个二级间接块中的第0-2个一级间接索引块全部被使用($3*4MB=12MB$)以及第3个一级间接索引块中的第0-1013个索引被使用($ 1014*4KB=4056KB$)、三级间接块没有被使用,总共$40KB+4MB+12MB+4056KB =16MB+4096KB=16MB+4MB=20MB$。
    \end{enumerate}
    注意间接索引块编号从0开始!
\end{analysis}

% FAT32文件系统,https://blog.csdn.net/u010650845/article/details/60881687
% WSL文件系统支持,https://blogs.msdn.microsoft.com/wsl/2016/06/15/wsl-file-system-support/
% WSL mount,https://blogs.msdn.microsoft.com/wsl/2017/04/18/file-system-improvements-to-the-windows-subsystem-for-linux/