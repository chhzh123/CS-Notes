% !TEX root = main.tex

\section{简介} % 1.1-1.6
机器学习侧重于处理的算法,而模式识别则包括了数据预处理、实际运算和数据输出的完整过程。
\begin{itemize}
\item 模式识别:涵盖的范围广,包括特征提取、特征选择、降维、各种分类器等。
\item 机器学习:主要是讲学习,更多关于分类器如何训练模型,而不涉及特征方面的知识。
\end{itemize}

良好特征的四个特点:
\begin{itemize}
	\item 可区别性(不同类)
	\item 可靠性(同类)
	\item 独立性(特征之间)
	\item 参数少(复杂性)
\end{itemize}

一个对象的所有特征参数组成特征向量。
同样需要从高维测量空间(样本)中提取特征映射到低维特征空间。

模式识别分为两类
\begin{itemize}
	\item 结构/句法模式识别
	\item 统计/神经网络模式识别
\end{itemize}

模式识别系统过程如下
\[\text{传感器}\to\text{分割}\to\text{特征提取}\to\text{\textbf{分类器}}\to\text{后处理}\]
设计循环
\[\text{采集}\to\text{选择基本特征}\to\text{选择模型}\to\text{训练分类}\to\text{评价}\]