% !TEX root = main.tex

\section{贝叶斯决策论}
\subsection{离散变量}
处于类别$\omega_i$并具有特征值$x$,有后验概率\footnote{通常用$p(\cdot)$代表概率密度函数(连续变量),用$\pr{\cdot}$代表概率质量函数(离散变量)}
\[\prc{\omega_i}{x}=\frac{p(x\mid\omega_i)\pr{\omega_i}}{p(x)}\]
即
\[posterior=\frac{likelihood\times prior}{evidence}\]

无论什么情况,当我们观察到特定的$x$,对于二分类问题有错误率
\[\prc{error}{x}=
\begin{cases}
\prc{\omega_1}{x} & \text{决策}\omega_2\\
\prc{\omega_2}{x} & \text{决策}\omega_1
\end{cases}
=\min[\prc{\omega_1}{x},\prc{\omega_2}{x}]\]

平均错误概率可表示为
\[\prc{error}{x}=\intab{-\infty}{\infty}{\pr{error,x}}=\intab{-\infty}{\infty}{\prc{error}{x}p(x)}\]
注意$p(x)$是证据,可以看为是固定分布(常量)。

\begin{theorem}[贝叶斯决策/最小错误率准则]
若$P(\omega_1\mid x)>P(\omega_2\mid x)$,则判定类别为$\omega_1$;否则判为$\omega_2$。
依照这种准则可以获得最小错误率,即$P(error\mid x)=\min [P(\omega_1\mid x),P(\omega_2\mid x)]$
\end{theorem}


\subsection{连续变量}
考虑特征向量$\vx\in\rr^d$($\rr^d$称为特征空间),令$\{\omega_1,\ldots,\omega_c\}$表示有限的$c$个类别集,$\{\alpha_1,\ldots,\alpha_a\}$表示有限的$a$种可能采取的行为集,损失函数(loss)$\lambda(\alpha_t\mid\omega_j)$描述类别状态为$\omega_j$时采取行动$\alpha_i$的风险。
$p(\vx\mid\omega_j)$表示在真实类别为$\omega_j$的条件下$\vx$的概率密度函数,$P(\omega_j)$表示类别处于状态$\omega_j$时的先验概率,后验概率$P(\omega_j\mid \vx)$则通过贝叶斯公式
\[P(\omega_j\mid\vx)=\frac{p(\vx\mid\omega_j)P(\omega_j)}{p(\vx)}\]
计算得到,证据变为
\[p(\vx)=\sum_{j=1}^cp(\vx\mid\omega_j)P(\omega_j)\]

与行动$\alpha_i$相关联的风险(risk)为
\[R(\alpha_i\mid\vx)=\sum_{j=1}^c\lambda(\alpha_i\mid\omega_j)\prc{\omega_j}{\vx}\]

进而得到总损失
\[R=\intabu{}{}{R(\alpha(\vx)\mid\vx)p(\vx)}{\vx}\]

因此得到连续情形下的贝叶斯决策论:
\begin{theorem}
为最小化$R$,计算条件概率
\[R(\alpha_i\mid\vx)=\sum_{j=1}^c\lambda(\alpha_i\mid\omega_j)\prc{\omega_j}{\vx},\;\forall i=1,\ldots,a\]
选择$\alpha_i$使得$R(\alpha_i\mid\vx)$最小,进而最小化总的风险即称为贝叶斯风险,记为$R^*$
\end{theorem}

\subsection{二类分类}
对称损失/0-1损失
\[\lambda(\alpha_i\mid\omega_j)=
\begin{cases}
0 & i=j\\
1 & i\ne j
\end{cases}
\qquad i,j=1,2,\ldots,c\]
有条件风险
\[\begin{aligned}
R(\alpha_1\mid\vx)&=\lambda_{11}P(\omega_1\mid\vx)+\lambda_{12}P(\omega_2\mid\vx)\\
R(\alpha_2\mid\vx)&=\lambda_{21}P(\omega_1\mid\vx)+\lambda_{22}P(\omega_2\mid\vx)
\end{aligned}\]

可得贝叶斯决策
\[\frac{p(\vx\mid\omega_1)}{p(\vx\mid\omega_2)}>\frac{\lambda_{12}-\lambda_{22}}{\lambda_{21}-\lambda_{11}}\frac{P(\omega_2)}{P(\omega_1)}\]