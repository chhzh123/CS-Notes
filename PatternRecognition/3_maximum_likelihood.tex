% !TEX root = main.tex

\section{极大似然与贝叶斯参数估计} % 3.1-3.5 3.7-3.8 (3.9)
\subsection{极大似然估计}
假设样本集$\mD$中有$n$个样本$\vx_1,\ldots,\vx_n$,由于这些样本均独立抽取,故
\[p(\mD\mid\vtheta)=\prod_{k=1}^np(\vx_k\mid\vtheta)\]
这里的$\vtheta$为参数向量。

定义对数似然为
\[\ell(\vtheta)=\ln p(\mD\mid\vtheta)\]
进而
\[\hat{\vtheta}=\argmax_\vtheta\ell(\vtheta)\]
求解最大似然估计值的必要条件为
\[\nabla_\vtheta\ell = 0\]
而最大后验(maximum a posteriori, MAP)则是使$\ell(\vtheta)p(\vtheta)$取最大值的参数向量$\vtheta$。

样本协方差矩阵
\[C=\frac{1}{n-1}\sum_{k=1}^n(\vx_k-\hat{\mu})(\vx_k-\hat{\mu})^\T\]

\subsection{贝叶斯参数估计}
