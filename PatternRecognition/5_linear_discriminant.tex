% !TEX root = main.tex

\section{线性判别函数} % 5.1-5.8 (5.10-5.12)
\subsection{线性判别函数和判定面}
判别(discriminant)函数是指由$\vx$的各个分量线性组合而成的函数
\[g(\vx)=\vw^\T\vx+w_0\]
这里$\vw$是权向量,$w_0$称为阈权值(threshold)或偏置。

对于二类线性分类器来说,$g(\vx)>0$则判为$\omega_1$,否则判为$\omega_2$。
方程$g(\vx)=0$定义了一个判定面,将归类于$\omega_1$和$\omega_2$的点分开来。
当$g(\vx)$是线性的,这个平面称为超平面。

判别函数是特征空间某点$\vx$到超平面距离的代数度量(注意到垂直平行特性)
\[\vx=\vx_p+r\frac{\vw}{\norm{\vw}}\]
其中$\vx_p$是$\vx$在超平面$H$上的投影向量,$r$是相应的算术距离,为正则$\vx$在$H$正侧,否则在$H$负侧。
由于$g(\vx_p)=0$,有
\[g(\vx)=\vw^\T\vx+w_0=r\norm{\vw}\]
或
\[r=\frac{g(\vx)}{\norm{\vw}}\]
\begin{figure}[H]
\centering
\includegraphics[width=0.9\linewidth]{fig/linear_decision_boundary.png}
\end{figure}

\subsection{广义线性判别函数}
线性判别函数可写成
\[g(\vx)=w_0+\sum_{i=1}^d w_i x_i\]
系数$w_i$是权向量$\vw$的分量,通过加入另外的项($\vw$的各对分量之间的乘积),得到二次判别函数
\[g(\vx)=w_0+\sum_{i=1}^d w_ix_i+\sum_{i=1}^d\sum_{j=1}^d w_{ij}x_ix_j\]
因$x_ix_j=x_jx_i$,不失一般性假设$w_{ij}=w_{ji}$。

继续加入更高次项(如$w_{ijk}x_ix_jx_k$)可得到多项式判别函数,可看作对某一判别函数$g(x)$做级数展开,然后取截尾逼近,意味着某一广义线性判别函数
\[g(\vx)=\sum_{i=1}^d a_iy_i(\vx)\]
或
\[g(\vx)=\va^\T\vy\]

特别地,线性判别函数可写成
\[g(\vx)=w_0+\sum_{i=1}^d w_i x_i=\sum_{i=0}^dw_ix_i\]
令$\vy=\bmat{1 & \vx}^\T$为增广特征向量,$\va=\bmat{w_0 & \vw}$为增广权向量。

求解$\va^\T\vy_i>0$的解所采用的方法是:定义一个准则函数$J(\va)$,当$\va$是解向量时,$J(\va)$最小,因此可将其简化为一个标量函数极小化问题,通常用梯度下降法解决。
\[\va(k+1)=\va(k)-\eta(k)\nabla J(\va(k))\]
其中$\eta$为正的比例因子,即学习率。

假设准则函数可以通过二阶展开近似
\[J(\va)\approx J(\va(k))+\nabla J^\T(\va-\va(k))+\frac{1}{2}(\va-\va(k))^\T H(\va-\va(k))\]
即当选择
\[\eta(k)=\frac{\norm{\nabla J}^2}{\nabla J^T H\nabla J}\]
时,可使$J(\va(k+1))$最小化。

\subsection{支持向量机}
支持向量机(support vector machine, SVM)通过一个足够高维的非线性映射$\varphi(\cdot)$,将两类数据用超平面进行分割。
假设每个模式$\vx_k$变换到$\vy_k=\varphi(\vx_k)$,则问题在于选择$\varphi(\cdot)$。
对$n$个模式中的每一个$k=1,\ldots,n$,根据模式属于$\omega_1$或$\omega_2$,分别令$z_k=\pm 1$,增广空间$\vy$上的判别函数是
\[g(\vy)=\va^\T\vy\]
这里权向量和变换后的模式向量都是增广的(取$a_0=w_0,y_0=1$),则这样的分割超平面保证
\[z_kg(\vy_k)\geq 1,\;k=1,\ldots,n\]
\begin{figure}[H]
\centering
\includegraphics[width=0.9\linewidth]{fig/SVM.png}
\end{figure}

训练一个支持向量机的目标是找到一个具有最大间隔(largest margin)的分割平面,若间隔越大则得到的分类器也越好。
从超平面到变换后的模式$\vy$的距离是$|g(\vy)|/\norm{\va}$(即做投影),若正的间隔$b$存在,则推出
\[\frac{z_kg(\vy_k)}{\norm{\va}}\geq b,\;k=1,\ldots,n\]
目标即找一个使得$b$最大化的权向量$\va$。
由于解向量可以任意伸缩且保持超平面不变,故有限制条件$b\norm{a}=1$,即其确定的是$\norm{\va}$的极小值。

支持向量是使上式等号成立的模式向量,即支持向量是最靠近超平面的,也是最难分类的样本/对求解分类任务最富有信息的模式。

目标为极小化$\norm{\va}$,构造拉格朗日函数
\[L(\va,\valpha)=\frac{1}{2}\norm{\va}^2\sum_{k=1}^n\alpha_k[z_k\va^\T\vy_k-1]\]
可用KKT条件改写为
\[L(\valpha)=\sum_{k=1}^n\alpha_i-\frac{1}{2}\sum_{k,j}^n\alpha_k\alpha_jz_kz_j\vy^\T\vy_k\]
结合训练数据的约束条件
\[\sum_{k=1}^nz_k\alpha_k=0,\;\alpha_k\geq 0,\;k=1,\ldots,n\]
进行求解。